\documentclass{article}

% Language setting
\usepackage[english]{babel}

% Set page size and margins
% Replace `letterpaper' with `a4paper' for UK/EU standard size
\usepackage[a4paper,top=2cm,bottom=2cm,left=3cm,right=3cm,marginparwidth=1.75cm]{geometry}

% Useful packages
\usepackage{amsmath}
\usepackage{graphicx}
\usepackage[colorlinks=true, allcolors=blue]{hyperref}
\usepackage{lineno} % For line numbering

\title{Summary of Zucchini’s lessons}
\author{Marta Barbieri, Stefano Doria, Rossella Fioralli, Giuseppe Luciano}

\begin{document}

\maketitle
\begin{abstract}
    Ammetto che possa sembrare già incasinato ma così è strutturato in modo da essere cliccabile, le sezioni ci sono già l'unica cosa che dovete fare per iniziare il lavoro è scrivere nella regione che vi serve come nel miniesempio che sparirà. sarebbe bello fare tutto cliccabile ma solo se proprio ci annoiamo a morte da tutto il tempo che abbiamo a disposizione. se diventa troppo lungo probabilmente ci toccherà spezzettarlo però
    Regole:
    \begin{itemize}
\item \textbf{1} 
 si ha un tot di tempo dopo la lezione per aggiungere il riassunto così tutti possono studiare e non si accumula tutto alla fine. 
 
\item \textbf{2} 
Si ha l'obbligo di seguire, tranne causa di forza maggiore, la lezione da riassumere e si riassumono i pezzi spiegati nella specifica lezione seguendo le linee guida del prof 
\item \textbf{3}
Bisogna firmarsi nel capitolo riassunto così potete darmi la colpa
\item \textbf{4}
fare sempre prima un pull e poi un push sulla repository
\item \textbf{5}
altre regole?
\end{itemize}
   
\end{abstract}

\tableofcontents
\linenumbers % Line numbering starts here

%----------------------------------------------------------
\section{From Classical to Quantum Physics}
\subsection{Classical wave theory of light}
se scrivete qui in automatico scrive le cose dove serve
\subsection{Monochromatic waves}
\subsection{Wave propagation and Huygens' principle}
\subsection{Light interference}
\subsection{Classical interferometry}
\subsection{Classical diffractometry}
\subsection{Classical wave theory at atomic level: Bragg diffraction}
\subsection{Statistical mechanics}
\subsection{Elementary applications of statistical mechanics}
\subsection{The black body radiation and the beginning of quantum physics}
\subsection{Toward quantum physics: particles and waves}
\subsection{The photoelectric effect}
\subsection{The Compton effect}
\subsection{Davisson and Germer experiment}
\subsection{Wave–particle duality and the complementarity principle}
\subsection{The probabilistic nature of quantum physics}
\subsection{The uncertainty principle}

%----------------------------------------------------------
\section{Quantum Theory and Atoms}
\subsection{Atomic spectra}
\subsection{The Bohr theory}
\subsection{The Franck and Hertz experiment}
\subsection{The Bohr atomic model of hydrogen–like atoms}
\subsection{The correspondence principle}
\subsection{Adiabatic invariants and Wilson–Sommerfeld quantization}
\subsection{Elementary applications of Wilson–Sommerfeld quantization}
\subsection{The Sommerfeld atomic theory}
\subsection{The Stern and Gerlach experiment}
\subsection{Spin angular momentum}

%----------------------------------------------------------
\section{Wave Mechanics}
\subsection{Classical wave theory for material media and geometrical optics}
\subsection{Hamiltonian formulation of light propagation}
\subsection{The statistical ensemble}
\subsection{The Schroedinger equation}
\subsection{The statistical interpretation of the wave function}
\subsection{Mathematical modelling of the ensemble}
\subsection{The Moyal product and bracket}
\subsection{Single particle interpretation of wave mechanics}
\subsection{Momentum space formulation}
\subsection{Position–momentum uncertainty relation}
\subsection{Stationary states}
\subsection{Quasistationarity and time–energy uncertainty relation}
\subsection{Regularity and boundary conditions of the wave function}
\subsection{The time independent Schroedinger problem}
\subsection{A basic example: the potential box}
\subsection{A basic example: the free particle}
\subsection{Energy eigenvalues and eigenfunctions and their properties}
\subsection{Time evolution of the wave function}
\subsection{Time evolution of simple quantum systems}
\subsection{The Schroedinger equation for systems of many particles}
\subsection{The Schroedinger equation for a spinning particle}
\subsection{Spinning particle in a magnetic field}

%----------------------------------------------------------
\section{The Schroedinger Equation}
\subsection{Generalities on the 1–dimensional Schroedinger problem}
\subsection{Properties of the energy eigenfunctions in 1–dimension}
\subsection{Piecewise constant 1–dimensional potentials}
\subsection{1–dimensional potential box}
\subsection{Square 1–dimensional potential barrier/well}
\subsection{1–dimensional potential Dirac wall/sink}
\subsection{Semi-infinite 1–dimensional potential well and barrier}
\subsection{The harmonic oscillator}
\subsection{Periodic potentials}
\subsection{The Kroenig–Penney model}
\subsection{The Schroedinger equation for planar potentials}
\subsection{The Schroedinger equation for axial potentials}
\subsection{The Schroedinger equation for central potentials}
\subsection{The radial Schroedinger equation}
\subsection{The spherical Bessel functions}
\subsection{The free particle}
\subsection{Spherical potential box}
\subsection{Spherical potential barrier/well}
\subsection{Spherical potential Dirac wall/sink}
\subsection{The hydrogenlike atom}
\subsection{The hydrogenlike atom in parabolic coordinates}
%----------------------------------------------------------
\section{Collision Theory}
\subsection{The importance of collision experiments}
\subsection{Scattering and scattering cross section}
\subsection{Collision in one dimension}
\subsection{Collision against a 1–dimensional potential step}
\subsection{Collision against a 1–dimensional potential barrier/well}
\subsection{Collision against a Dirac wall/sink}
\subsection{Collisions in three dimensions}
\subsection{Scattering by a central potential}
\subsection{The Born approximation}
\subsection{The Born approximation for a central potential}
\subsection{Computations of cross sections by the Born approximation}
\subsection{Coulombic scattering}
\subsection{Central potential scattering and phase shifts}
\subsection{Computations of cross sections by the phase shift method}
\subsection{Phase shifts for a finite range potential}
\subsection{Low energy scattering by a finite range central potential}
\subsection{Resonant scattering by a finite range central potential}
\subsection{Low energy scattering by a spherical potential barrier/well}
\subsection{Resonances and bound states}
\subsection{The Breit–Wigner formula of resonant scattering}
\subsection{The hard sphere limit}
\subsection{The Born approximation in the phase shift method}
\subsection{Coulomb scattering revisited}
\subsection{Coulomb phase shifts}

%----------------------------------------------------------
\section{Operator Formulation of Quantum Theory}
\subsection{Physical phenomena and their mathematical modelization}
\subsection{The Hilbert space formulation of wave mechanics}
\subsection{Introducing Dirac’s bra-ket notation}
\subsection{From finite to infinite dimension}
\subsection{The Hilbert space and its operator algebra}
\subsection{Non commutativity and commutator}
\subsection{Orthonormal bases}
\subsection{Selfadjoint linear operators and representations}
\subsection{Orthogonal projectors}
\subsection{Unitary operators}
\subsection{Direct product in Dirac’s framework}
\subsection{Translating from Schroedinger to Dirac notation}
\subsection{The main representations}
\subsection{Canonical commutation relations}
\subsection{Unitary operators associated to relevant selfadjoint operators}
\subsection{The problem of quantization}
\subsection{The Ehrenfest theorem and quantum integrals of motion}
\subsection{Many particle systems and ket, bra and operator direct products}

%----------------------------------------------------------
-----------------------------------------------
\section{Foundations of Quantum Theory}
\subsection{Basic quantum experiments}
\subsection{States, observables and measurement}
\subsection{Definedness and eigenstates in quantum physics}
\subsection{Measurement and state reduction in quantum physics}
\subsection{Spectrum of a quantum observable}
\subsection{Superposition, completeness and probability}
\subsection{The Schroedinger cat}
\subsection{n–ary measurements and spectrum}
\subsection{Expectation value and uncertainty of an observable}
\subsection{Interference of measurements of observables}
\subsection{Simultaneous measurements and simultaneous eigenstates}
\subsection{Functional relationship of observables}
\subsection{The formal structure of quantum theory}
\subsection{Sample applications of quantum formalism}
\subsection{Indetermination and operator non commutativity}

%----------------------------------------------------------
\section{Elementary Applications of Quantum Theory}
\subsection{Some basic facts on the energy eigenstates}
\subsection{Virtual bound states and metastability}
\subsection{1–dimensional rigid rotator}
\subsection{Electrons in a crystal lattice}
\subsection{Particle in an electromagnetic field}
\subsection{The Aharonov–Bohm effect}
\subsection{2–state systems and Pauli formalism}
\subsection{Energy spectrum of a 2–state system}
\subsection{Examples of 2–state systems}
\subsection{Annihilation and creation operators}
\subsection{The harmonic oscillator}
\subsection{Coherent states}
\subsection{A particle in a uniform static magnetic field}

%----------------------------------------------------------
\section{Angular Momentum}
\subsection{Quantum theory of angular momentum}
\subsection{Angular momentum commutation relations}
\subsection{Angular momentum spectral theory}
\subsection{Examples of angular momenta}
\subsection{Complete sets of commuting operators and angular momentum}
\subsection{Orbital angular momentum}
\subsection{Spin 1/2 angular momentum}
\subsection{Orbital–spin factorization}
\subsection{Addition of angular momenta}
\subsection{Clebsch–Gordan coefficients}
\subsection{Examples of addition of angular momenta}
\subsection{Angular momentum and rotations}
\subsection{Rotationally covariant states}
\subsection{Rotationally covariant operators}
\subsection{Rotations and angular momentum eigenstates}
\subsection{Tensor operators}
\subsection{Tensor operators and rotationally covariant operators}
\subsection{The Wigner–Eckart theorem}
\subsection{The projection theorem}
\subsection{Applications of the projection theorem}
\subsection{Particle in a central potential revisited}
\subsection{The Kramers relations for hydrogenlike atoms}
\subsection{The quantum Runge–Lenz vector}
\subsection{Pauli theory of the spinning electron}
\subsection{Nucleon–nucleon interaction potential}
\subsection{The meaning of parity}
\subsection{Generalities on parity}
\subsection{Parity and rotationally covariant operators}

%----------------------------------------------------------
\section{Identical Particles}
\subsection{Systems of many particles}
\subsection{Identical and indistinguishable particles}
\subsection{Exchange degeneracy and statistical physics}
\subsection{Identical particles as bosons and fermions}
\subsection{Composite states and the Pauli exclusion principle}
\subsection{Spin–statistics theorem}
\subsection{Identical particles: spin and orbit sectors}

%----------------------------------------------------------
\section{Stationary Perturbation Theory}
\subsection{Stationary perturbation theory: generalities}
\subsection{Stationary perturbation theory: implementation}
\subsection{Non degenerate stationary perturbation theory}
\subsection{Symmetry preserving stationary perturbation theory}
\subsection{Perturbation of a 2–state system}
\subsection{Unidimensional potential near a local minimum}
\subsection{Stark effect of a 1–dimensional rigid rotator}
\subsection{Effect of nuclear finite size}
\subsection{Paschen–Back and Zeeman effects}
\subsection{The hydrogenlike atoms’ fine structure}

%----------------------------------------------------------
\section{Variational Theory and Its Applications}
\subsection{The variational method}
\subsection{Elementary applications of the variational method}
\subsection{The helium atom}
\subsection{The Hartree–Fock method}
\subsection{The Hartree–Fock equation}

%----------------------------------------------------------
\section{Time Evolution}
\subsection{The evolution operator}
\subsection{Time evolution and pictures}
\subsection{Dynamics of a 2–state systems}
\subsection{Nuclear magnetic resonance}
\subsection{Atom in a laser electromagnetic field}
\subsection{Time dependent perturbation theory}
\subsection{Transitions by pulse perturbations}
\subsection{Time dependent perturbation of a 2–state system}
\subsection{Ion induced alkali atomic transitions}
\subsection{Transitions by a periodic perturbation}
\subsection{Fermi’s golden rule}
\subsection{The photoelectric effect}


\end{document}

