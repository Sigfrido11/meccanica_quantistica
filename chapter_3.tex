\documentclass{article}

% Language setting
\usepackage[english]{babel}

% Set page size and margins
% Replace `letterpaper' with `a4paper' for UK/EU standard size
\usepackage[a4paper,top=2cm,bottom=2cm,left=3cm,right=3cm,marginparwidth=1.75cm]{geometry}

% Useful packages
\usepackage{amsmath}
\usepackage{graphicx}
\usepackage[colorlinks=true, allcolors=blue]{hyperref}
\usepackage{lineno} % For line numbering

\title{Summary of Zucchini’s lessons}
\author{Marta Barbieri, Stefano Doria, Rossella Fioralli, Giuseppe Luciano}

\begin{document}

\maketitle
\begin{abstract}
    Ammetto che possa sembrare già incasinato ma così è strutturato in modo da essere cliccabile, le sezioni ci sono già l'unica cosa che dovete fare per iniziare il lavoro è scrivere nella regione che vi serve come nel miniesempio che sparirà. sarebbe bello fare tutto cliccabile ma solo se proprio ci annoiamo a morte da tutto il tempo che abbiamo a disposizione. se diventa troppo lungo probabilmente ci toccherà spezzettarlo però
    Regole:
    \begin{itemize}
\item \textbf{1} 
 si ha un tot di tempo dopo la lezione per aggiungere il riassunto così tutti possono studiare e non si accumula tutto alla fine. 
 
\item \textbf{2} 
Si ha l'obbligo di seguire, tranne causa di forza maggiore, la lezione da riassumere e si riassumono i pezzi spiegati nella specifica lezione seguendo le linee guida del prof 
\item \textbf{3}
Bisogna firmarsi nel capitolo riassunto così potete darmi la colpa
\item \textbf{4}
fare sempre prima un pull e poi un push sulla repository
\item \textbf{5}
altre regole?
\end{itemize}
   
\end{abstract}

\tableofcontents
\section*{Wawe mechanics}
\subsection*{3.1. Classical wave theory for material media and geometrical optics}

Electromagnetic waves propagate not only in vacuum but also in material media. In the latter case, a wave is still characterized by a scalar wave field $\psi$, but the wave equation this obeys non longer has the simple form (1.2.1), but one that depends on the nature of the medium. For a large class of weakly inhomogeneous isotropic media, the wave equation reads
$$
\begin{equation*}
\nabla^{2} \psi-\frac{n^{2}}{c^{2}} \frac{\partial^{2} \psi}{\partial t^{2}}=0 \tag{3.1.1}
\end{equation*}
$$
where $c$ is the speed of light in vacuum (cf. eq. (1.2.2)) and $n$ is the refractive index of the medium. $n$ depends in general on the observation point. The wave equation, so, has the same form as that for the vacuum, but the wave propagation speed in the medium in $c / n$ rather than $c$ and varies throughout the medium. Note that $n>1$, since the vacuum speed of light $c$ cannot be exceeded.

The complex wave field $\psi$ can be resolved into two real fields, the phase $\varphi$ and the amplitude $a$, defined by the relation
$$
\begin{equation*}
\psi=a \exp (i \varphi) \tag{3.1.2}
\end{equation*}
$$

This decomposition, which we have already encountered in a particular case in sect. 1.2 .2 , is very natural from a physical point of view, since it answers to the intuitive picture of a wave. As we have seen in sects. 1.3, 1.5, $\varphi$ and $a$ determine the propagation and the intensity of the light associated with the wave. These properties continue to hold also in a material medium.

The wave equation (3.1.1) can be reexpressed equivalently as a pair of coupled partial differential equations in $\varphi$ and $a$,
$$
\begin{align*}
& (\boldsymbol{\nabla} \varphi)^{2}-\left(\frac{n}{c} \frac{\partial \varphi}{\partial t}\right)^{2}-\frac{1}{a}\left(\boldsymbol{\nabla}^{2} a-\frac{n^{2}}{c^{2}} \frac{\partial^{2} a}{\partial t^{2}}\right)=0  \tag{3.1.3}\\
& \boldsymbol{\nabla} \cdot\left(a^{2} \boldsymbol{\nabla} \varphi\right)-\frac{n^{2}}{c^{2}} \frac{\partial}{\partial t}\left(a^{2} \frac{\partial \varphi}{\partial t}\right)=0 \tag{3.1.4}
\end{align*}
$$

Proof. Using (3.1.2), we find
$$
\begin{align*}
\boldsymbol{\nabla}^{2} \psi- & \frac{n^{2}}{c^{2}} \frac{\partial^{2} \psi}{\partial t^{2}}  \tag{3.1.5}\\
=\{ & \boldsymbol{\nabla}^{2} a+2 i \boldsymbol{\nabla} a \cdot \boldsymbol{\nabla} \varphi+a\left(i \boldsymbol{\nabla}^{2} \varphi-(\boldsymbol{\nabla} \varphi)^{2}\right) \\
& \left.-\frac{n^{2}}{c^{2}}\left[\frac{\partial^{2} a}{\partial t^{2}}+2 i \frac{\partial a}{\partial t} \frac{\partial \varphi}{\partial t}+a\left(i \frac{\partial^{2} \varphi}{\partial t^{2}}-\left(\frac{\partial \varphi}{\partial t}\right)^{2}\right)\right]\right\} \exp (i \varphi) \\
=\left\{-(\boldsymbol{\nabla} \varphi)^{2}+\left(\frac{n}{c} \frac{\partial \varphi}{\partial t}\right)^{2}+\right. & \frac{1}{a}\left(\boldsymbol{\nabla}^{2} a-\frac{n^{2}}{c^{2}} \frac{\partial^{2} a}{\partial t^{2}}\right) \\
& \left.\left.+\frac{i}{a^{2}}\left(\boldsymbol{\nabla} \cdot\left(a^{2} \boldsymbol{\nabla} \varphi\right)-\frac{n^{2}}{c^{2}} \frac{\partial}{\partial t}\left(a^{2} \frac{\partial \varphi}{\partial t}\right)\right)\right]\right\} a \exp (i \varphi)
\end{align*}
$$

The real and imaginary part of the expression within square brackets must vanish separately. (3.1.3), (3.1.4) thus follow.

Eqs. (3.1.3), (3.1.4) are very complicated. Considerable simplification can be obtained restricting as allowed to monochromatic waves, already considered in sect. 1.2.2, for which the wave field has the simpler form
$$
\begin{equation*}
\psi_{t}=\psi_{0} \exp (-i \omega t) \tag{3.1.6}
\end{equation*}
$$
where $\omega$ is a fixed angular frequency (cf. eq. (1.3.1)). Comparing (3.1.2) and (3.1.6), we find that the time dependence of $\varphi$ and $a$ is
$$
\begin{align*}
\varphi_{t} & =\varphi_{0}-\omega t  \tag{3.1.7}\\
a_{t} & =a_{0} \tag{3.1.8}
\end{align*}
$$

Inserting (3.1.7), (3.1.8) into (3.1.3), (3.1.4), we obtain then
$$
\begin{align*}
& \left(\boldsymbol{\nabla} \varphi_{0}\right)^{2}-\kappa^{2} n^{2}-\frac{1}{a_{0}} \boldsymbol{\nabla}^{2} a_{0}=0  \tag{3.1.9}\\
& \boldsymbol{\nabla} \cdot\left(a_{0}^{2} \boldsymbol{\nabla} \varphi_{0}\right)=0 \tag{3.1.10}
\end{align*}
$$
where $\kappa=\omega / c$ (cf. eq. (1.3.1)).
The nonlinear and coupled nature of the differential equations (3.1.3), (3.1.4)
even in their monochromatic version (3.1.9), (3.1.10) makes their solution still very difficult. The equations assume a more manageable form in the regime, known as geometric optics, in which the relative time and space variations of the amplitude a are much smaller that those of the phase $\varphi$. In geometric optics, one can truncate those equations keeping only the terms containing the highest powers of the time and space derivatives of $\varphi$ of any order without compromising their accuracy. This yields the geometric optics' approximation. For the sake conceptual clarity and later convenience, we shall distinguish the exact phase and amplitude $\varphi$ and $a$ from their geometric optics' approximants $\varphi_{c}$ and $a_{c}$ through an index $c$.

Implementing the geometric optics' approximation as indicated in the previous paragraph, eqs. (3.1.3), (3.1.4) simplify into
$$
\begin{align*}
& \left(\boldsymbol{\nabla} \varphi_{c}\right)^{2}-\left(\frac{n}{c} \frac{\partial \varphi_{c}}{\partial t}\right)^{2}=0  \tag{3.1.11}\\
& \boldsymbol{\nabla} \cdot\left(a_{c}^{2} \boldsymbol{\nabla} \varphi_{c}\right)-\frac{n^{2}}{c^{2}} \frac{\partial}{\partial t}\left(a_{c}^{2} \frac{\partial \varphi_{c}}{\partial t}\right)=0 \tag{3.1.12}
\end{align*}
$$
which are called the geometric optics' equations. In the monochromatic case, $\phi_{c}$ and $a_{c}$ have a time dependence of the simple form shown in (3.1.7), (3.1.8),
$$
\begin{align*}
\varphi_{c t} & =\varphi_{c 0}-\omega t  \tag{3.1.13}\\
a_{c t} & =a_{c 0} \tag{3.1.14}
\end{align*}
$$

Eqs. (3.1.9), (3.1.10) are then approximated by
$$
\begin{align*}
& \left(\nabla \varphi_{c 0}\right)^{2}-\kappa^{2} n^{2}=0  \tag{3.1.15}\\
& \nabla \cdot\left(a_{c 0}^{2} \nabla \varphi_{c 0}\right)=0 \tag{3.1.16}
\end{align*}
$$
which are called respectively eikonal equation and transport equation ${ }^{1}$. These
equations can be demonstrated alternatively by inserting (3.1.13), (3.1.14) into (3.1.11), (3.1.12).

While we do not have a precise quantitative criterion to verify the accuracy of the approximate equations (3.1.11), (3.1.12), we do for (3.1.15), (3.1.16). Eq. (3.1.15) follows from eq. (3.1.9) by neglecting the third term in the right hand side of this latter. This approximation is accurate when $\varphi_{0}$ varies appreciably and $a_{0}$ negligibly on a distance of the order of the vacuum wave length $\lambda \sim 1 / \kappa$,
$$
\begin{equation*}
\kappa^{-1}\left|\nabla \varphi_{0}\right| \sim n \gg \kappa^{-1}\left|a_{0}^{-1} \nabla a_{0}\right| \tag{3.1.17}
\end{equation*}
$$

The equations of geometric optics can be deduced systematically by the following formal procedure. Suppose that in (3.1.2) we rescale the phase $\varphi$ into $\varphi / \xi$, where $\xi$ is a dimensionless parameter. Then, (3.1.2) becomes
$$
\begin{equation*}
\psi=a \exp (i \varphi / \xi) \tag{3.1.18}
\end{equation*}
$$
$\xi$ is introduced only as a formal bookkeeping device aimed at hierarchically grouping the various terms of the relevant equations according to the powers of $\xi^{-1}$, but eventually $\xi$ will be assigned the correct value 1 . Proceeding in this way, eqs. (3.1.3), (3.1.4) take the form
$$
\begin{align*}
& \xi^{-2}\left[(\boldsymbol{\nabla} \varphi)^{2}-\left(\frac{n}{c} \frac{\partial \varphi}{\partial t}\right)^{2}\right]-\frac{1}{a}\left(\boldsymbol{\nabla}^{2} a-\frac{n^{2}}{c^{2}} \frac{\partial^{2} a}{\partial t^{2}}\right)=0  \tag{3.1.19}\\
& \xi^{-1}\left[\boldsymbol{\nabla} \cdot\left(a^{2} \boldsymbol{\nabla} \varphi\right)-\frac{n^{2}}{c^{2}} \frac{\partial}{\partial t}\left(a^{2} \frac{\partial \varphi}{\partial t}\right)\right]=0 \tag{3.1.20}
\end{align*}
$$

Comparing the exact equations (3.1.19), (3.1.20) with the approximate geometric optics' equations (3.1.11), (3.1.12), it appears that the latter stems from the former by keeping in (3.1.19), (3.1.20) only the terms that would be more singular in a fictitious limit $\xi \rightarrow 0$. Proceeding in the same way in the monochromatic case, eqs. (3.1.9), (3.1.10) become

\footnotetext{
${ }^{1}$ The eikonal equation is usually written as $(\nabla \sigma)^{2}-n^{2}=0$, where $\sigma=\varphi_{c 0} / \kappa$ is called eikonal field. In this form, the independence of the of wave front geometry from frequency is apparent. Here, we shall leave the equation in the form (3.1.15) for later convenience.
}
$$
\begin{align*}
& \xi^{-2}\left[\left(\boldsymbol{\nabla} \varphi_{0}\right)^{2}-\kappa^{2} n^{2}\right]-\frac{1}{a_{0}} \nabla^{2} a_{0}=0  \tag{3.1.21}\\
& \xi^{-1} \boldsymbol{\nabla} \cdot\left(a_{0}^{2} \boldsymbol{\nabla} \varphi_{0}\right)=0 \tag{3.1.22}
\end{align*}
$$

Comparing the exact equations (3.1.21), (3.1.22) with the approximate eikonal and transport equations (3.1.15), (3.1.16), one sees again the latter stem from the former by keeping only the most singular terms in a fake limit $\xi \rightarrow 0$. The geometric optics' approximation, so, can be generated formally by performing the limit $\xi \rightarrow 0$. Intuitively, as $\xi \rightarrow 0$, the wave field $\psi$ becomes very sensitive to the variations of the value of its phase. Therefore, the limit corresponds to the situation where the relative time and space variations of $a$ are negligible compared with those of $\varphi$, which is precisely the geometric optics regime. Though the formal procedure we have illustrated above is not strictly necessary, it will turn out to be very useful in the study of the classical regime in wave mechanics.

\subsection*{3.2. Hamiltonian formulation of light propagation}

Recall that a wave front $\mathcal{S}$ of a wave is the locus of the points of space having a fixed phase, that is the set of the points $\boldsymbol{x}$ satisfying an equation of the form
$$
\begin{equation*}
\varphi_{t}(\boldsymbol{x})=\alpha \tag{3.2.1}
\end{equation*}
$$
where $\alpha$ is a fixed constant angle (cf. sect. 1.3). Geometrically, so, the wave front is a surface $\mathcal{S}_{t}$ depending on the time $t$ (cf. fig. 3.2.1). As $t$ elapses, the location of the wave front shifts in space leading to the wave's propagation.

The motion of wave fronts is much more easily understood for monochromatic waves. By (3.1.7), a wave front $\mathcal{S}$ for such a wave is defined by the equation
$$
\begin{equation*}
\varphi_{0}(\boldsymbol{x})-\omega t=\alpha \tag{3.2.2}
\end{equation*}
$$

Henceforth, we shall deal only with monochromatic waves.

![](https://cdn.mathpix.com/cropped/2024_09_22_5d1e855547710648961eg-0211.jpg?height=662&width=659&top_left_y=1588&top_left_x=668)

Figure 3.2.1. Propagation of a wave. The position of a wave front is shown in three successive instants $t_{1}<t_{2}<t_{3}$.

![](https://cdn.mathpix.com/cropped/2024_09_22_5d1e855547710648961eg-0212.jpg?height=668&width=732&top_left_y=577&top_left_x=664)

Figure 3.2.2. For each time instant $\tau$ there is a wave front $\mathcal{S}^{(\tau)}{ }_{t}$ such that $\mathcal{S}^{(\tau)}{ }_{t=\tau}$ contains a given point $\boldsymbol{\xi}$. The wave surface $\mathcal{S}^{(\boldsymbol{\xi})}=$ $\mathcal{S}^{(\tau)}{ }_{t=\tau}$ is independent from $\tau$. A ray through $\boldsymbol{\xi}$ is normal to $\mathcal{S}^{(\boldsymbol{\xi})}$.

For any instant $\tau$ of time and point $\boldsymbol{\xi}$ of space, there is a unique wave front $\mathcal{S}^{(\tau)}$ such that the surface $\mathcal{S}^{(\tau)}{ }_{t=\tau}$ contains $\boldsymbol{\xi}$. Furthermore, $\mathcal{S}^{(\tau)}{ }_{t=\tau}$ is independent from $\tau$. A constant wave surface $\mathcal{S}^{(\boldsymbol{\xi})}=\mathcal{S}^{(\tau)}{ }_{t=\tau}$ through $\boldsymbol{\xi}$ thus exists (cf. fig. 3.2.2). Explicitly, $\mathcal{S}^{(\boldsymbol{\xi})}$ is the locus of the points $\boldsymbol{x}$ satisfying
$$
\begin{equation*}
\varphi_{0}(\boldsymbol{x})-\varphi_{0}(\boldsymbol{\xi})=0 \tag{3.2.3}
\end{equation*}
$$

Proof. Let $\mathcal{S}^{(\tau)}$ be the wave front defined by eq. (3.2.2) with
$$
\begin{equation*}
\alpha=\varphi_{0}(\boldsymbol{\xi})-\omega t \tag{3.2.4}
\end{equation*}
$$
$\mathcal{S}^{(\tau)}$ is hence the locus of all points $\boldsymbol{x}$ satisfying
$$
\begin{equation*}
\varphi_{0}(\boldsymbol{x})-\varphi_{0}(\boldsymbol{\xi})-\omega(t-\tau)=0 \tag{3.2.5}
\end{equation*}
$$

The surface $\mathcal{S}^{(\tau)}{ }_{t=\tau}$, therefore, contains $\boldsymbol{\xi}$.

Suppose that $\mathcal{S}$ is another wave front such that $\mathcal{S}_{t=\tau}$ contains $\boldsymbol{\xi}$. Now, $\mathcal{S}$ is defined by eq. (3.2.2) for a certain value of $\alpha$. Since $\boldsymbol{\xi}$ lies on $\mathcal{S}_{t=\tau}, \alpha$ is given precisely by (3.2.4) It follows that $\mathcal{S}=\mathcal{S}^{(\tau)}$.

From (3.2.5), it follows that the surface $\mathcal{S}^{(\tau)}{ }_{t=\tau}$ is specified by eq. (3.2.3). $\mathcal{S}^{(\tau)}{ }_{t=\tau}$ is so independent from $\tau$ and thus $\mathcal{S}^{(\tau)}{ }_{t=\tau}$ is a surface $\mathcal{S}^{(\boldsymbol{\xi})}$ determined by the point $\boldsymbol{\xi}$ only.

A ray is any line in space that at each point $\boldsymbol{\xi}$ is normal to the wave surface $\mathcal{S}^{(\boldsymbol{\xi})}$ through $\boldsymbol{\xi}$ (cf. fig. 3.2.2). The location of a ray in space does not depend on time, since the surfaces $\mathcal{S}^{(\boldsymbol{\xi})}$ do not.

At each point $\boldsymbol{\xi}$ of the ray, there is defined the ray's wave vector,
$$
\begin{equation*}
\varkappa=\boldsymbol{\nabla} \varphi_{0}(\boldsymbol{\xi}) . \tag{3.2.6}
\end{equation*}
$$
$\boldsymbol{\varkappa}$ is tangent to the ray at $\boldsymbol{\xi}$.

Proof. If $\boldsymbol{\eta}$ is a vector tangent to $\mathcal{S}^{(\boldsymbol{\xi})}$ at $\boldsymbol{\xi}$, there exists a parametrized curve $\boldsymbol{\zeta}(s)$ entirely contained in $\mathcal{S}^{(\boldsymbol{\xi})}$ such that $\boldsymbol{\zeta}(0)=\boldsymbol{\xi}$ and $d \boldsymbol{\zeta}(0) / d s=\boldsymbol{\eta}$ as is evident inspecting fig. 3.2.3. Since $\varphi_{0}(\boldsymbol{\zeta}(s))-\varphi_{0}(\boldsymbol{\xi})=0$ identically by (3.2.3), we have
$$
\begin{equation*}
0=\left.\frac{d}{d s}\left(\varphi_{0}(\boldsymbol{\zeta}(s))-\varphi_{0}(\boldsymbol{\xi})\right)\right|_{s=0}=\frac{d \boldsymbol{\zeta}(0)}{d s} \cdot \boldsymbol{\nabla} \varphi_{0}(\boldsymbol{\zeta}(0))=\boldsymbol{\eta} \cdot \boldsymbol{\nabla} \varphi_{0}(\boldsymbol{\xi}) \tag{3.2.7}
\end{equation*}
$$

By (3.2.6), then, the wave vector $\boldsymbol{\kappa}$ at a point $\boldsymbol{\xi}$ is normal to every vector $\boldsymbol{\eta}$ tangent to $\mathcal{S}^{(\boldsymbol{\xi})}$ at $\boldsymbol{\xi}$ and thus normal to $\mathcal{S}^{(\boldsymbol{\xi})}$ at $\boldsymbol{\xi}$. Since the ray through $\boldsymbol{\xi}$ is normal to $\mathcal{S}^{(\boldsymbol{\xi})}, \boldsymbol{\varkappa}$ is tangent to the ray at $\boldsymbol{\xi}$.

Clearly, the vectors $\varkappa$ induce an orientation on the ray as a whole that indicated pointwise by the vectors.

To study a ray from a geometrical point of view, it is necessary to resort to a parametrization $\boldsymbol{\xi}(s)$ of the ray. It is natural to take the parameter $s$ to be the ray's arc distance defined as the signed length of the arc of the ray connecting

![](https://cdn.mathpix.com/cropped/2024_09_22_5d1e855547710648961eg-0214.jpg?height=522&width=679&top_left_y=533&top_left_x=666)

Figure 3.2.3. Any vector $\boldsymbol{\eta}$ tangent to the surface $\mathcal{S}^{(\boldsymbol{\xi})}$ at $\boldsymbol{\xi}$ arises from a parametrized curve $\boldsymbol{\zeta}(s)$ contained in $\mathcal{S}^{(\boldsymbol{\xi})}$ such that $\boldsymbol{\zeta}(0)=\boldsymbol{\xi}$ and $d \boldsymbol{\zeta}(0) / d s=\boldsymbol{\eta}$. All such tangent vectors $\boldsymbol{\eta}$ are orthogonal to $\nabla \varphi_{0}(\boldsymbol{\xi})$
a point of the ray $\boldsymbol{\xi}$ to a fixed reference point $\boldsymbol{\xi}_{0}$ (cf. fig. 3.2.4). With this parametrization choice, $\boldsymbol{\xi}$ satisfies
$$
\begin{equation*}
\frac{d \boldsymbol{\xi}}{d s}=\frac{\varkappa}{|\varkappa|} \tag{3.2.8}
\end{equation*}
$$

Proof. When $s$ varies of an amount $d s, \boldsymbol{\xi}$ varies of an amount $d \boldsymbol{\xi}$. As $s$ is the ray arc distance, $d s=|d \boldsymbol{\xi}|$. Thus, $d \boldsymbol{\xi} / d s$ is a unit vector. Furthermore, $d \boldsymbol{\xi} / d s$ is tangent to the ray at the point $\boldsymbol{\xi}$. Since the wave vector $\boldsymbol{\varkappa}$ at $\boldsymbol{\xi}$ also is, $d \boldsymbol{\xi} / d s$ is proportional to $\varkappa$. (3.2.8) follows.

The bending of a ray is measured by the second derivative $d^{2} \boldsymbol{\xi} / d s^{2}$. In fact, if $d^{2} \boldsymbol{\xi} / d s^{2}=\mathbf{0}$, the ray would be a straight line $\boldsymbol{\xi}(s)=\boldsymbol{\xi}_{0}+s \boldsymbol{\eta}_{0}$. We hace
$$
\begin{equation*}
\frac{d^{2} \boldsymbol{\xi}}{d s^{2}}=\frac{1}{|\varkappa|}\left(1-\frac{\varkappa \varkappa}{|\varkappa|^{2}}\right) \cdot \frac{d \varkappa}{d s} \tag{3.2.9}
\end{equation*}
$$

Here, $\mathbf{1}$ is the unit dyad and, for any two vectors $\boldsymbol{a}, \boldsymbol{b}, \boldsymbol{a} \boldsymbol{b}$ denotes the dyad defined by the property that $(\boldsymbol{a b}) \cdot \boldsymbol{v}=(\boldsymbol{b} \cdot \boldsymbol{v}) \boldsymbol{a}$.

![](https://cdn.mathpix.com/cropped/2024_09_22_5d1e855547710648961eg-0215.jpg?height=571&width=654&top_left_y=487&top_left_x=752)

Figure 3.2.4. The arc distance of a point $\boldsymbol{\xi}$ of a ray from a reference point $\boldsymbol{\xi}_{0}$ of it is defined as the length $s$ of the $\operatorname{arc}$ joining $\boldsymbol{\xi}$ to $\boldsymbol{\xi}_{0}$ with a positive or negative sign depending on whether $\boldsymbol{\xi}$ follows or precedes $\boldsymbol{\xi}_{0}$. It parametrizes the ray as setting $\boldsymbol{\xi}=\boldsymbol{\xi}(s)$.

Proof. From (3.2.8), using the standard vector calculus identity $\boldsymbol{\nabla}_{\boldsymbol{\varkappa}}(\boldsymbol{\varkappa} /|\boldsymbol{\varkappa}|)=$ $(1 /|\boldsymbol{\varkappa}|)\left(\mathbf{1} /-\boldsymbol{\varkappa} \boldsymbol{\varkappa} /|\boldsymbol{\varkappa}|^{2}\right)$, we have
$$
\begin{equation*}
\frac{d^{2} \boldsymbol{\xi}}{d s^{2}}=\frac{d}{d s} \frac{\varkappa}{|\varkappa|}=\frac{d \varkappa}{d s} \cdot \nabla_{\varkappa} \frac{\varkappa}{|\varkappa|}=\frac{d \varkappa}{d s} \cdot \frac{1}{|\varkappa|}\left(1-\frac{\varkappa \varkappa}{|\varkappa|^{2}}\right) . \tag{3.2.10}
\end{equation*}
$$

By the symmetry of the dyad $\mathbf{1}-\varkappa \varkappa /|\varkappa|^{2}$, this can be written equivalently in the form $(3.2 .9)$.

So far, we have studied rays from a purely geometric point of view. We now assume that $\varphi_{0}=\varphi_{c 0}$ satisfies the eikonal equation (3.1.15). We then find
$$
\begin{equation*}
\frac{d \varkappa}{d s}=|\varkappa| \frac{\boldsymbol{\nabla} n(\boldsymbol{\xi})}{n(\boldsymbol{\xi})} \tag{3.2.11}
\end{equation*}
$$

Proof. Substituting (3.2.6) into (3.2.8), we have
$$
\begin{equation*}
\frac{d \boldsymbol{\xi}}{d s}=\frac{\boldsymbol{\nabla} \varphi_{c 0}(\boldsymbol{\xi})}{\left|\boldsymbol{\nabla} \varphi_{c 0}(\boldsymbol{\xi})\right|} \tag{3.2.12}
\end{equation*}
$$

From (3.2.6), using (3.2.12), we find then
$$
\begin{align*}
& \frac{d \varkappa}{d s}=\frac{d \boldsymbol{\xi}}{d s} \cdot \boldsymbol{\nabla} \boldsymbol{\nabla} \varphi_{c 0}(\boldsymbol{\xi})=\frac{\boldsymbol{\nabla} \varphi_{c 0}(\boldsymbol{\xi})}{\left|\boldsymbol{\nabla} \varphi_{c 0}(\boldsymbol{\xi})\right|} \cdot \boldsymbol{\nabla} \boldsymbol{\nabla} \varphi_{c 0}(\boldsymbol{\xi})  \tag{3.2.13}\\
&=\frac{\boldsymbol{\nabla}\left(\boldsymbol{\nabla} \varphi_{c 0}(\boldsymbol{\xi})\right)^{2}}{2\left|\boldsymbol{\nabla} \varphi_{c 0}(\boldsymbol{\xi})\right|}=\frac{\boldsymbol{\nabla}\left|\boldsymbol{\nabla} \varphi_{c 0}(\boldsymbol{\xi})\right|^{2}}{2\left|\boldsymbol{\nabla} \varphi_{c 0}(\boldsymbol{\xi})\right|}=\boldsymbol{\nabla}\left|\boldsymbol{\nabla} \varphi_{c 0}(\boldsymbol{\xi})\right|
\end{align*}
$$

From the eikonal equation (3.1.15), we have
$$
\begin{equation*}
\left|\boldsymbol{\nabla} \varphi_{c 0}(\boldsymbol{\xi})\right|=\kappa n(\boldsymbol{\xi}) \tag{3.2.14}
\end{equation*}
$$
with $\kappa=\omega / c$. From(3.2.6), we have further
$$
\begin{equation*}
\kappa=\frac{|\varkappa|}{n(\boldsymbol{\xi})} \tag{3.2.15}
\end{equation*}
$$

Inserting $(3.2 .14),(3.2 .15)$ in (3.2.13), we get (3.2.11).

Inserting (3.2.8), (3.2.11) into (3.2.9), we obtain the ray equation
$$
\begin{equation*}
\frac{d^{2} \boldsymbol{\xi}}{d s^{2}}-\frac{\boldsymbol{\nabla} n(\boldsymbol{\xi})}{n(\boldsymbol{\xi})}+\frac{\boldsymbol{\nabla} n(\boldsymbol{\xi})}{n(\boldsymbol{\xi})} \cdot \frac{d \boldsymbol{\xi}}{d s} \frac{d \boldsymbol{\xi}}{d s}=\mathbf{0} \tag{3.2.16}
\end{equation*}
$$

This governs the bending of the rays caused by the inhomogeneity of the medium.
The ray equation can be obtained also from the Fermat principle, which has the following statement. Among all paths $\boldsymbol{\xi}$ connecting two fixed points $\boldsymbol{x}_{1}, \boldsymbol{x}_{2}$ of a medium, the one followed by a ray extremizes the optical path length
$$
\begin{equation*}
I[\boldsymbol{\xi}]=\int d s n(\boldsymbol{\xi}) \tag{3.2.17}
\end{equation*}
$$

Here, $d s=|d \boldsymbol{\xi}|$ and the integration is extended to the path $\boldsymbol{\xi}$ (cf. fig. 3.2.5).

Proof. Varying $I$ with respect to the path $\boldsymbol{\xi}$, we have
$$
\begin{equation*}
\delta I[\boldsymbol{\xi}]=\int[\delta d s n(\boldsymbol{\xi})+d s \boldsymbol{\nabla} n(\boldsymbol{\xi}) \cdot \delta \boldsymbol{\xi}] \tag{3.2.18}
\end{equation*}
$$

The variation $\delta d s$ is given by
$$
\begin{equation*}
\delta d s=\delta|d \boldsymbol{\xi}|=\frac{d \boldsymbol{\xi}}{|d \boldsymbol{\xi}|} \cdot d \delta \boldsymbol{\xi}=d s \frac{d \boldsymbol{\xi}}{d s} \cdot \frac{d \delta \boldsymbol{\xi}}{d s} \tag{3.2.19}
\end{equation*}
$$

![](https://cdn.mathpix.com/cropped/2024_09_22_5d1e855547710648961eg-0217.jpg?height=527&width=719&top_left_y=517&top_left_x=730)

Figure 3.2.5. The paths $\boldsymbol{\xi}, \boldsymbol{\xi}^{\prime}, \boldsymbol{\xi}^{\prime \prime}$ joining two fixed points $\boldsymbol{x}_{1}, \boldsymbol{x}_{2}$ are characterized by different values of the optical path length $I$. According to the Fermat principle, the path $\boldsymbol{\xi}$ actually followed by a ray passing through $\boldsymbol{x}_{1}, \boldsymbol{x}_{2}$ is the one that extremizes $I$.

Inserting (3.2.19) into (3.2.18) and integrating by parts, we find
$$
\begin{align*}
\delta I[\boldsymbol{\xi}] & =\int\left[d s \frac{d \boldsymbol{\xi}}{d s} \cdot \frac{d \delta \boldsymbol{\xi}}{d s} n(\boldsymbol{\xi})+d s \boldsymbol{\nabla} n(\boldsymbol{\xi}) \cdot \delta \boldsymbol{\xi}\right]  \tag{3.2.20}\\
& =\left.\frac{d \boldsymbol{\xi}}{d s} \cdot \delta \boldsymbol{\xi} n(\boldsymbol{\xi})\right|_{\boldsymbol{x}_{1}} ^{\boldsymbol{x}_{2}}-\int d s\left[n(\boldsymbol{\xi}) \frac{d^{2} \boldsymbol{\xi}}{d s^{2}}+\nabla n(\boldsymbol{\xi}) \cdot \frac{d \boldsymbol{\xi}}{d s} \frac{d \boldsymbol{\xi}}{d s}-\nabla n(\boldsymbol{\xi})\right] \cdot \delta \boldsymbol{\xi}
\end{align*}
$$

The integrated term vanishes as we restrict to paths connecting the fixed points $\boldsymbol{x}_{1}$, $\boldsymbol{x}_{2}$, so that $\left.\delta \boldsymbol{\xi}\right|_{\boldsymbol{x}_{1}}=\left.\delta \boldsymbol{\xi}\right|_{\boldsymbol{x}_{2}}=\mathbf{0}$. By (3.2.20), then, $\delta I$ vanishes precisely when $\boldsymbol{\xi}$ satisfies the ray equation (3.2.16).

The analysis of wave front propagation requires considering time evolution.
The arc distance speed $d s / d t$ of a ray is the arc distance swept by a wave front intersecting the ray per unit time (cf. fig. 3.2.6). $d s / d t$ is given by
$$
\begin{equation*}
\frac{d s}{d t}=\frac{c}{n(\boldsymbol{\xi})} \tag{3.2.21}
\end{equation*}
$$
where here $\left.\boldsymbol{\xi}(t) \equiv \boldsymbol{\xi}(s)\right|_{s=s(t)}$.

![](https://cdn.mathpix.com/cropped/2024_09_22_5d1e855547710648961eg-0218.jpg?height=551&width=435&top_left_y=565&top_left_x=864)

Figure 3.2.6. When a wave front shifts from its position at time $t$ to that at time $t+d t$, it sweeps an arc of a given ray of length $d s=d s / d t \cdot d t$. So, $d s / d t$ is the light arc distance speed on the ray.

Proof. For every time $t, \xi(s)$ lies on the wave front $\mathcal{S}_{t}$, so that $\boldsymbol{\xi}(s)$ satisfies eq. (3.2.2) for some fixed $\alpha$. Consequently,
$$
\begin{align*}
& 0=\frac{d}{d t}\left(\varphi_{c 0}(\boldsymbol{\xi}(s))-\omega t-\alpha\right)=\frac{d s}{d t} \frac{d \boldsymbol{\xi}(s)}{d s} \cdot \boldsymbol{\nabla} \varphi_{c 0}(\boldsymbol{\xi}(s))-\omega  \tag{3.2.22}\\
& \quad=\frac{d s}{d t} \frac{\boldsymbol{\nabla} \varphi_{c 0}(\boldsymbol{\xi}(s))}{\left|\boldsymbol{\nabla} \varphi_{c 0}(\boldsymbol{\xi}(s))\right|} \cdot \boldsymbol{\nabla} \varphi_{c 0}(\boldsymbol{\xi}(s))-\omega=\frac{d s}{d t}\left|\boldsymbol{\nabla} \varphi_{c 0}(\boldsymbol{\xi}(s))\right|-\omega,
\end{align*}
$$
where we used (3.2.12). From this relation, it follows that
$$
\begin{equation*}
\frac{d s}{d t}=\frac{\omega}{\left|\boldsymbol{\nabla} \varphi_{c 0}(\boldsymbol{\xi})\right|} \tag{3.2.23}
\end{equation*}
$$

Inserting (3.2.14) into (3.2.23) and recalling that $\kappa=\omega / c$, we obtain (3.2.21).

By (3.2.21), eqs. (3.2.8), (3.2.11) imply that $\boldsymbol{\xi}$ and $\boldsymbol{\varkappa}$ satisfy
$$
\begin{align*}
\frac{d \boldsymbol{\xi}}{d t} & =\frac{c \boldsymbol{\varkappa}}{n(\boldsymbol{\xi})|\boldsymbol{\varkappa}|}  \tag{3.2.24a}\\
\frac{d \boldsymbol{\varkappa}}{d t} & =\frac{c|\boldsymbol{\varkappa}| \boldsymbol{\nabla} n(\boldsymbol{\xi})}{n(\boldsymbol{\xi})^{2}} \tag{3.2.24b}
\end{align*}
$$

Proof. Combining (3.2.8), (3.2.21), we find
$$
\begin{equation*}
\frac{d \boldsymbol{\xi}}{d t}=\frac{d s}{d t} \frac{d \boldsymbol{\xi}}{d s}=\frac{c}{n(\boldsymbol{\xi})} \frac{\varkappa}{|\varkappa|} \tag{3.2.25}
\end{equation*}
$$
which is (3.2.24a). Similarly, combining (3.2.11), (3.2.21), we have
$$
\begin{equation*}
\frac{d \varkappa}{d t}=\frac{d s}{d t} \frac{d \varkappa}{d s}=\frac{c}{n(\boldsymbol{\xi})}|\varkappa| \frac{\boldsymbol{\nabla} n(\boldsymbol{\xi})}{n(\boldsymbol{\xi})} \tag{3.2.26}
\end{equation*}
$$
which is $(3.2 .24 b)$.

The remarkable fact about eqs. (3.2.24), known already in the 19-th century, is that the (3.2.24) are the Hamilton equations of a fictitious particle of position $\boldsymbol{\xi}$ and momentum $\boldsymbol{\varkappa}$ with Hamiltonian
$$
\begin{equation*}
\Omega(\boldsymbol{\xi}, \boldsymbol{\varkappa})=\frac{c|\boldsymbol{\varkappa}|}{n(\boldsymbol{\xi})} \tag{3.2.27}
\end{equation*}
$$

Indeed, they can be cast as
$$
\begin{align*}
\frac{d \boldsymbol{\xi}}{d t} & =\nabla_{\varkappa} \Omega(\boldsymbol{\xi}, \boldsymbol{\varkappa})  \tag{3.2.28a}\\
\frac{d \varkappa}{d t} & =-\nabla_{\boldsymbol{\xi}} \Omega(\boldsymbol{\xi}, \boldsymbol{\varkappa}) \tag{3.2.28b}
\end{align*}
$$

The rays are then the possible trajectories of the particle. In this perspective, the phase field $\varphi_{c 0}$ is nothing but the Hamilton characteristic function. In fact, the eikonal equation (3.1.15) can be cast as
$$
\begin{equation*}
\Omega\left(\boldsymbol{\xi}, \boldsymbol{\nabla} \varphi_{c 0}\right)=\omega \tag{3.2.29}
\end{equation*}
$$
which is the Hamilton-Jacobi equation for the particle.

Proof. Indeed, the eikonal equation (3.1.15) can be written in the form (3.2.14) with $\kappa=\omega / c$, which, on account of (3.2.27), is just (3.2.29).

The Fermat principle itself is a consequence of the Maupertuis least action
principle. In classical mechanics, the principle states that among all phase trajectories $(\boldsymbol{q}, \boldsymbol{p})$ with assigned configuration space ends $\boldsymbol{x}_{1}, \boldsymbol{x}_{2}$ and energy $w$, the one which solves the Hamilton equations extremizes the reduced action $J=\int \boldsymbol{p} \cdot d \boldsymbol{q}$. In the present case, as the role of energy is played by angular frequency, the principle says that among all phase trajectories $(\boldsymbol{\xi}, \boldsymbol{\varkappa})$ joining two fixed configuration points $\boldsymbol{x}_{1}, \boldsymbol{x}_{2}$ and having frequency $\omega$, the one which solves the Hamilton equations extremizes the reduced action
$$
\begin{equation*}
J=\int \varkappa \cdot d \boldsymbol{\xi} \tag{3.2.30}
\end{equation*}
$$

Proof. We can assume that (3.2.8) holds, as momentum along any path must always be tangent to it. Then, the integrand in (3.2.30) can be cast as
$$
\begin{equation*}
\varkappa \cdot d \boldsymbol{\xi}=d s \boldsymbol{\varkappa} \cdot \frac{d \boldsymbol{\xi}}{d s}=d s \boldsymbol{\varkappa} \cdot \frac{\boldsymbol{\varkappa}}{|\boldsymbol{\varkappa}|}=d s|\boldsymbol{\varkappa}| . \tag{3.2.31}
\end{equation*}
$$

By (3.2.27), the condition $\Omega(\boldsymbol{\xi}, \boldsymbol{\varkappa})=\omega$ is readily solved yielding
$$
\begin{equation*}
|\varkappa|=\frac{\omega n(\boldsymbol{\xi})}{c}=\kappa n(\boldsymbol{\xi}) \tag{3.2.32}
\end{equation*}
$$

Inserting (3.2.32) into (3.2.31) yields
$$
\begin{equation*}
\varkappa \cdot d \boldsymbol{\xi}=\kappa d s n(\boldsymbol{\xi}) . \tag{3.2.33}
\end{equation*}
$$

From (3.2.30), it follows then that
$$
\begin{equation*}
J=\int \kappa d s n(\boldsymbol{\xi})=\kappa I \tag{3.2.34}
\end{equation*}
$$

Note that $\boldsymbol{\varkappa}$ has been eliminated using the frequency assignment. Thus, extremizing $J$ in the set of all phase trajectories $(\boldsymbol{\xi}, \boldsymbol{\varkappa})$ joining $\boldsymbol{x}_{1}, \boldsymbol{x}_{2}$ and having frequency $\omega$ reduces to extremizing $I$ in the of all configuration trajectories $\boldsymbol{\xi}$ joining $\boldsymbol{x}_{1}, \boldsymbol{x}_{2}$. This proves that the Maupertuis least action principle reduces to the Fermat principle.

The above discussion shows that the front propagation of a wave in the geometrical optical limit is formally equivalent to the dynamics of a fictitious particle
in Hamilton-Jacobi theory ${ }^{2}$. Reversing this perspective, this suggests that the dynamics of a true particle in Hamilton-Jacobi theory describes the front propagation of a wave in the geometrical optical limit. This attitude will result in the Schroedinger wave equation. The waves involved here are however of a rather peculiar kind: they are probability waves.

\footnotetext{
${ }^{2}$ This is the reason why Newton's particle theory of light can also explain many features of light propagation.
}

\subsection*{3.3. The statistical ensemble}

Consider a classical particle subject to a conservative force field with potential energy $U$. A trajectory $\boldsymbol{q}(\tau)$ of the particle satisfies the equation of motion
$$
\begin{equation*}
m \frac{d^{2} \boldsymbol{q}}{d \tau^{2}}=-\nabla U(\boldsymbol{q}) \tag{3.3.1}
\end{equation*}
$$
(We use here the unconventional notation $\tau$ rather than $t$ for time for convenience.) There are infinitely many trajectories $\boldsymbol{q}(\tau)$, one for each assignment of the initial values $\boldsymbol{x}_{0}=\boldsymbol{q}(0)$ and $\boldsymbol{v}_{0}=d \boldsymbol{q}(0) / d \tau$, since the solution of the dynamical equation (3.3.1) is uniquely determined only once $\boldsymbol{x}_{0}, \boldsymbol{v}_{0}$ are specified, by a well-known property of ordinary second order differential equations, and these can specified arbitrarily. The collection of all trajectories $\boldsymbol{q}(\tau)$ constitute in this way a parametrized family $\boldsymbol{q}\left(\tau ; \boldsymbol{x}_{0}, \boldsymbol{v}_{0}\right)$ with $\boldsymbol{x}_{0}, \boldsymbol{v}_{0}$ as parameters. Let us now restrict to the trajectories $\boldsymbol{q}\left(\tau ; \boldsymbol{x}_{0}, \boldsymbol{v}_{0}\right)$ of the particle with fixed $\boldsymbol{x}_{0}$. We shall call such trajectories allowed. As $\boldsymbol{v}_{0}$ is not assigned, there are infinitely many of them forming a bundle in configuration space (cf. fig. 3.3.1). For later study of the allowed dynamics from a statistical point of view, we introduce now the concept of statistical ensemble.

The statistical ensemble of the particle is a collection of a large number $N$ of copies of the particle. The ensemble is a purely conceptual construct and should not be viewed as a physical system on its own. In particular the copies do not interact and are therefore independent.

Each copy of the particle in the ensemble moves along an allowed trajectory $\boldsymbol{q}(\tau)$. So, each begins its motion at the point $\boldsymbol{x}_{0}$. However, each does so with a speed $\boldsymbol{v}_{0}$ that is generally different from that of other copies. Therefore, the copies trace generally distinct trajectories, although the ones which share the same value of $\boldsymbol{v}_{0}$ do proceed along the same trajectory.

The instantaneous position of the copies in space form a dense set of points

![](https://cdn.mathpix.com/cropped/2024_09_22_5d1e855547710648961eg-0223.jpg?height=526&width=703&top_left_y=496&top_left_x=792)

Figure 3.3.1. The bundle of allowed trajectories. Here, only the segment of each of them trodden in a given time $\tau$ is shown.
which can be viewed as a statistical fluid (E. Madelung, 1926; cf. fig. 3.3.2). If we assign a statistical mass $1 / N$ to each copy, not to be confused with the particle's physical mass $m$, the statistical mass density $\rho_{c}(t, \boldsymbol{x})$ is defined: $d^{3} x \rho_{c}(t, \boldsymbol{x})$ is the statistical mass of the fluid contained in a spacial region around $\boldsymbol{x}$ of small

![](https://cdn.mathpix.com/cropped/2024_09_22_5d1e855547710648961eg-0223.jpg?height=519&width=654&top_left_y=1605&top_left_x=345)
(a)

![](https://cdn.mathpix.com/cropped/2024_09_22_5d1e855547710648961eg-0223.jpg?height=519&width=641&top_left_y=1605&top_left_x=1081)
(b)

Figure 3.3.2. A single copy occupies just a single point in configuration space (a). A large number of copies, instead, fill densely configurations space and, so, can be modelled as a fluid (b).
volume $d^{3} x$ at time $t$.
For any assigned time $t$ and position $\boldsymbol{x}$, there is precisely one allowed trajectory $\boldsymbol{q}(\tau)$ such that $\left.\boldsymbol{q}(\tau)\right|_{\tau=t}=\boldsymbol{x}$ (cf. fig. 3.3.3). Indeed, the condition $\left.\boldsymbol{q}\left(\tau ; \boldsymbol{x}_{0}, \boldsymbol{v}_{0}\right)\right|_{\tau=t}=\boldsymbol{x}$ is an equation that determines $\boldsymbol{v}_{0}$ in terms of $\boldsymbol{x}$ and $t$, since $\boldsymbol{x}_{0}$ is fixed. So, imposing that $\left.\boldsymbol{q}(\tau)\right|_{\tau=t}=\boldsymbol{x}$ determines $\boldsymbol{q}(\tau)$ as a whole, i.e for any time $\tau$ preceding, equal and following $t$. The velocity $d \boldsymbol{q}(\tau) /\left.d \tau\right|_{\tau=t}$ is thus determined by the choice of $t$ and $\boldsymbol{x}$ made. The statistical velocity field $\boldsymbol{v}_{c}$ is defined by $\boldsymbol{v}_{c}(t, \boldsymbol{x})=d \boldsymbol{q}(\tau) /\left.d \tau\right|_{\tau=t}$, where $\boldsymbol{q}$ is the allowed trajectory such that $\left.\boldsymbol{q}(\tau)\right|_{\tau=t}=\boldsymbol{x} . \quad \boldsymbol{v}_{c}(t, \boldsymbol{x})$ is the velocity of all copies which happen to transit in $\boldsymbol{x}$ at the time $t$.

Since the copies of the particle forming the ensemble are neither created nor destroyed, the mass of the statistical fluid is conserved. The statistical mass conservation is encoded in the equation
$$
\begin{equation*}
\frac{\partial \rho_{c}}{\partial t}+\boldsymbol{\nabla} \cdot\left(\rho_{c} \boldsymbol{v}_{c}\right)=0 \tag{3.3.2}
\end{equation*}
$$

Proof. The conservation of any additive physical quantity $q$ is expressed by a

![](https://cdn.mathpix.com/cropped/2024_09_22_5d1e855547710648961eg-0224.jpg?height=537&width=701&top_left_y=1713&top_left_x=731)

Figure 3.3.3. Only one of the infinitely many allowed trajectories reaches the point $\boldsymbol{x}$ at the time $\tau=t$.
differential equation of the universal form
$$
\begin{equation*}
\frac{\partial \rho}{\partial t}+\boldsymbol{\nabla} \cdot \boldsymbol{j}=0 \tag{3.3.3}
\end{equation*}
$$
where $\rho$ and $\boldsymbol{j}$ are the $q$ density and current density, respectively. Though this result is well-known, let us recall its derivation. Since $q$ is conserved, the rate of change of the amount of $q$ contained in a space region $\mathcal{V}$ must be equal to minus the outgoing flux of $q$ through the boundary $\partial \mathcal{V}$ of $\mathcal{V}$. Hence,
$$
\begin{equation*}
\frac{d}{d t} \int_{\mathcal{V}} d^{3} x \rho+\oint_{\partial \mathcal{V}} d^{2} \boldsymbol{x} \cdot \boldsymbol{j}=0 \tag{3.3.4}
\end{equation*}
$$

Using the Gauss theorem, (3.3.5) can be written as
$$
\begin{equation*}
\int_{\mathcal{V}} d^{3} x\left[\frac{\partial \rho}{\partial t}+\boldsymbol{\nabla} \cdot \boldsymbol{j}\right]=0 \tag{3.3.5}
\end{equation*}
$$
from which, being the region $\mathcal{V}$ arbitrary, (3.3.4) follows.
In general, the current density $\boldsymbol{j}$ is the sum of a convective contribution $\rho \boldsymbol{v}$, caused by the drifting effect of the motion of matter with velocity field $\boldsymbol{v}$, and a conductive contribution $\boldsymbol{j}^{*}$, which is the rest,
$$
\begin{equation*}
\boldsymbol{j}=\rho \boldsymbol{v}+\boldsymbol{j}^{*} \tag{3.3.6}
\end{equation*}
$$

Inserting (3.3.6) in (3.3.3), we get
$$
\begin{equation*}
\frac{\partial \rho}{\partial t}+\boldsymbol{\nabla} \cdot\left(\rho \boldsymbol{v}+\boldsymbol{j}^{*}\right)=0 \tag{3.3.7}
\end{equation*}
$$

In the case of statistical mass, the flux in entirely due to convection and there is no conduction. Setting $\rho=\rho_{c}, \boldsymbol{v}=\boldsymbol{v}_{c}$ and $\boldsymbol{j}^{*}=\mathbf{0}$ in eq. (3.3.7), we obtain (3.3.2) immediately.

The study the dynamical evolution of the statistical ensemble of the particle requires the use of the canonical formalism and a statistical reinterpretation of Hamilton-Jacobi theory.

The momentum of a copy of the particle, whose allowed trajectory is $\boldsymbol{q}(t)$, is
$$
\begin{equation*}
\boldsymbol{p}=m \frac{d \boldsymbol{q}}{d \tau} \tag{3.3.8}
\end{equation*}
$$

As it is well-known, the equations of motion (3.3.1) can be written as
$$
\begin{equation*}
\frac{d \boldsymbol{p}}{d \tau}=-\boldsymbol{\nabla} U(\boldsymbol{q}) \tag{3.3.9}
\end{equation*}
$$

The statistical momentum field is defined by the relation $\boldsymbol{p}_{c}(t, \boldsymbol{x})=m d \boldsymbol{q}(t) / d t$, where $\boldsymbol{q}$ is the unique allowed trajectory such that $\boldsymbol{q}(t)=\boldsymbol{x}$, as usual. $\boldsymbol{p}_{c}$ is related to the velocity field $\boldsymbol{v}_{c}$ in the expected way
$$
\begin{equation*}
\boldsymbol{p}_{c}=m \boldsymbol{v}_{c} \tag{3.3.10}
\end{equation*}
$$

However, formulating the theory in terms of $\boldsymbol{p}_{c}$ is more natural, as it will appear momentarily.

The Hamilton principal function is defined by
$$
\begin{equation*}
S_{c}(t, \boldsymbol{x})=\int_{0}^{t} d \tau\left[\boldsymbol{p} \cdot \frac{d \boldsymbol{q}}{d \tau}-\frac{\boldsymbol{p}^{2}}{2 m}-U(\boldsymbol{q})\right] \tag{3.3.11}
\end{equation*}
$$
where in the right hand side the trajectory $\boldsymbol{q}$ is the unique allowed trajectory such that $\left.\boldsymbol{q}(\tau)\right|_{\tau=t}=\boldsymbol{x}, \boldsymbol{p}$ refers to $\boldsymbol{q}$. $S_{c}$ has two basic properties. First, $S_{c}$ satisfies the Hamilton-Jacobi equation
$$
\begin{equation*}
\frac{\partial S_{c}}{\partial t}+\frac{\left(\boldsymbol{\nabla} S_{c}\right)^{2}}{2 m}+U=0 \tag{3.3.12}
\end{equation*}
$$

Second, $S_{c}$ is a potential for the momentum field $\boldsymbol{p}_{c}$, so that $\boldsymbol{p}_{c}$ is irrotational.
$$
\begin{equation*}
\boldsymbol{p}_{c}=\boldsymbol{\nabla} S_{c} \tag{3.3.13}
\end{equation*}
$$

Proof. Using the equations of motion (3.3.8), (3.3.9), we find
$$
\begin{align*}
& \delta S_{c}(t, \boldsymbol{x})=\int_{0}^{t} d \tau {\left[\delta \boldsymbol{p} \cdot\left(\frac{d \boldsymbol{q}}{d \tau}-\frac{\boldsymbol{p}}{m}\right)-\delta \boldsymbol{q} \cdot\left(\frac{d \boldsymbol{p}}{d \tau}+\nabla U(\boldsymbol{q})\right)\right.}  \tag{3.3.14}\\
&\left.+\frac{d}{d \tau}(\delta \boldsymbol{q} \cdot \boldsymbol{p})\right]+\delta t\left[\boldsymbol{p}(\tau) \cdot \frac{d \boldsymbol{q}(\tau)}{d \tau}-\frac{\boldsymbol{p}(\tau)^{2}}{2 m}-U(\boldsymbol{q}(\tau))\right]_{\tau=t} \\
&=\left.\delta \boldsymbol{q}(\tau) \cdot \boldsymbol{p}(\tau)\right|_{\tau=t}+\delta t\left[\boldsymbol{p}(\tau) \cdot \frac{d \boldsymbol{q}(\tau)}{d \tau}-\frac{\boldsymbol{p}(\tau)^{2}}{2 m}-U(\boldsymbol{q}(\tau))\right]_{\tau=t}
\end{align*}
$$

![](https://cdn.mathpix.com/cropped/2024_09_22_5d1e855547710648961eg-0227.jpg?height=573&width=950&top_left_y=483&top_left_x=552)

Figure 3.3.4. The allowed trajectories $\boldsymbol{q}$ and $\boldsymbol{q}+\delta \boldsymbol{q}$ such that $\left.\boldsymbol{q}(\tau)\right|_{\tau=t}=\boldsymbol{x}$ and $\left.(\boldsymbol{q}+\delta \boldsymbol{q})(\tau)\right|_{\tau=t+\delta t}=\boldsymbol{x}+\delta \boldsymbol{x}$.

Here, $\left.\boldsymbol{q}(\tau)\right|_{\tau=t}=\boldsymbol{x}, d \boldsymbol{q}(\tau) /\left.d \tau\right|_{\tau=t}=\boldsymbol{v}_{c}(t, \boldsymbol{x})$ and $\left.\boldsymbol{p}(\tau)\right|_{\tau=t}=\boldsymbol{p}_{c}(t, \boldsymbol{x})$. Further, since
$$
\begin{align*}
\boldsymbol{x}+\delta \boldsymbol{x} & =\left.(\boldsymbol{q}+\delta \boldsymbol{q})(\tau)\right|_{\tau=t+\delta t}  \tag{3.3.15}\\
& =\left.\boldsymbol{q}(\tau)\right|_{\tau=t}+\left.\delta \boldsymbol{q}(\tau)\right|_{\tau=t}+\left.\delta t \frac{d \boldsymbol{q}(\tau)}{d \tau}\right|_{\tau=t}=\boldsymbol{x}+\left.\delta \boldsymbol{q}(\tau)\right|_{\tau=t}+\delta t \boldsymbol{v}_{c}(t, \boldsymbol{x})
\end{align*}
$$
(cf. fig. 3.3.4), we have
$$
\begin{equation*}
\left.\delta \boldsymbol{q}(\tau)\right|_{\tau=t}=\delta \boldsymbol{x}-\delta t \boldsymbol{v}_{c}(t, \boldsymbol{x}) \tag{3.3.16}
\end{equation*}
$$

Using all these identities in (3.3.14), we obtain
$$
\begin{equation*}
\delta S_{c}(t, \boldsymbol{x})=-\delta t\left[\frac{\boldsymbol{p}_{c}(t, \boldsymbol{x})^{2}}{2 m}+U(\boldsymbol{x})\right]+\delta \boldsymbol{x} \cdot \boldsymbol{p}_{c}(t, \boldsymbol{x}) \tag{3.3.17}
\end{equation*}
$$
(3.3.12) and (3.3.13) follow.

Combining (3.3.10), (3.3.13), we can eliminate $\boldsymbol{v}_{c}$ for $S_{c}$. The statistical mass conservation equation (3.3.2) takes then form
$$
\begin{equation*}
\frac{\partial \rho_{c}}{\partial t}+\frac{1}{m} \boldsymbol{\nabla} \cdot\left(\rho_{c} \boldsymbol{\nabla} S_{c}\right)=0 \tag{3.3.18}
\end{equation*}
$$

Eqs. (3.3.12), (3.3.13), (3.3.18) must be supplemented with the appropriate initial
conditions. The form of these becomes immediately clear noting that, at time 0 all copies of the particle must be located at $\boldsymbol{x}_{0}$ regardless the trajectories they follow. Thus, the Hamilton principal function must satisfy
$$
\begin{equation*}
\lim _{t \rightarrow 0, \boldsymbol{x} \rightarrow x_{0}} S_{c}(t, \boldsymbol{x})=0 \tag{3.3.19}
\end{equation*}
$$
${ }^{3}$. Further, the statistical mass density $\rho_{c}(0, \boldsymbol{x})$ must have the form
$$
\begin{equation*}
\rho_{c}(0, \boldsymbol{x})=\delta\left(\boldsymbol{x}-\boldsymbol{x}_{0}\right) \tag{3.3.20}
\end{equation*}
$$
where $\delta(\boldsymbol{x})$ is a generalized function defined by the properties
$$
\begin{equation*}
\delta(\boldsymbol{x})=0 \quad \text { for } \boldsymbol{x} \neq \mathbf{0}, \quad \delta(\boldsymbol{x})=\infty \quad \text { for } \boldsymbol{x}=\mathbf{0} \tag{3.3.21}
\end{equation*}
$$
the strength of the singularity at $\boldsymbol{x}=\mathbf{0}$ being such that
$$
\begin{equation*}
\int d^{3} x \delta(\boldsymbol{x})=1 \tag{3.3.22}
\end{equation*}
$$
$\delta(\boldsymbol{x})$ is the Dirac delta function, about which we shall have more to say later ${ }^{4}$.
The Hamilton-Jacobi equation (3.3.12), the momentum field equation (3.3.13) and statistical mass conservation equation (3.3.18) are the basic equations governing the dynamics of the statistical ensemble of the classical particle. Their quantum generalization paves the way to basic equation of quantum theory, Schroedinger equation, studied next.

\footnotetext{
${ }^{3}$ The allowed trajectory $\boldsymbol{q}$ such that $\left.\boldsymbol{q}(\tau)\right|_{\tau=t}=\boldsymbol{x}$ is unambiguously defined only if $t>0$, Thus, rigorously speaking, in virtue of the definition of $S_{c}$ we have given, condition (3.3.19) cannot be cast simply as $S_{c}\left(0, \boldsymbol{x}_{0}\right)=0$.
${ }^{4} \delta(\boldsymbol{x})$ can be intuitively pictured as a function which vanishes outside a ball centerd in $\mathbf{0}$ of small radius $R$ and takes the large value $3 / 4 \pi R^{3}$ inside.
}

\subsection*{3.4. The Schroedinger equation}

As we have seen, the statistical ensemble of a particle can be treated as a statistical fluid described by the Hamilton principal function $S_{c}$ and the statistical mass density $\rho_{c}$, whose dynamics is governed by the Hamilton-Jacobi equation (3.3.12) and the statistical mass conservation equation (3.3.18),
$$
\begin{align*}
& \frac{\partial S_{c}}{\partial t}+\frac{\left(\boldsymbol{\nabla} S_{c}\right)^{2}}{2 m}+U=0  \tag{3.4.1}\\
& \frac{\partial \rho_{c}}{\partial t}+\frac{1}{m} \boldsymbol{\nabla} \cdot\left(\rho_{c} \boldsymbol{\nabla} S_{c}\right)=0 \tag{3.4.2}
\end{align*}
$$

Even though this is not immediately obvious, the statistical ensemble is in disguised form an undulatory system. Namely,
the statistical ensemble is an undulatory system in a certain form of geometrical optical regime.

More specifically, eqs. (3.4.1), (3.4.2) play the same role as the geometrical optics' equations (3.1.11), (3.1.12) with $S_{c}$ and $\rho_{c}$ being the counterparts of the geometric optical approximants $\varphi_{c}$ and $a_{c}$ of the phase and amplitude of the wave field as summarized by the following correspondence table
$$
\begin{align*}
& S_{c} \longleftrightarrow \varphi_{c}  \tag{3.4.3}\\
& \rho_{c}^{1 / 2} \longleftrightarrow a_{c} \tag{3.4.4}
\end{align*}
$$

In a classical perspective, the above hypothesis seems to be quite arbitrary and without any grounding. In a quantum one, it is not so bizarre at it may look at first glance, considering that wave-particle duality is a recurring quantum feature (cf. subsect. 1.16). We shall make the claim plausible with the following argument.

If we ignore temporarily the initial conditions, eqs. (3.4.1), (3.4.2) have a broad variety of solutions. We now look for solutions whose time dependence has
the following simple special form
$$
\begin{align*}
& S_{c t}=S_{c 0}-w t  \tag{3.4.5}\\
& \rho_{c t}=\rho_{c 0} \tag{3.4.6}
\end{align*}
$$
where $w$ is a fixed energy value. Inserting (3.4.5), (3.4.6) into (3.4.1), (3.4.2), we find readily that $S_{c}, \rho_{c}$ solve (3.4.1), (3.4.2) provided $S_{c 0}, \rho_{c 0}$ satisfy the following pair of equations
$$
\begin{align*}
& \left(\boldsymbol{\nabla} S_{c 0}\right)^{2}=2 m(w-U)  \tag{3.4.7}\\
& \boldsymbol{\nabla} \cdot\left(\rho_{c 0} \boldsymbol{\nabla} S_{c 0}\right)=0 \tag{3.4.8}
\end{align*}
$$

Now, the time dependence of $S_{c}, \rho_{c}$ shown in (3.4.5), (3.4.6) is of the same type as that of $\varphi_{c}$ and $a_{c}$ for a monochromatic wave of angular frequency $\omega$ given in eqs. (3.1.13), (3.1.14). Furthermore, eqs. (3.4.7), (3.4.8) obeyed by $S_{c 0}, \rho_{c 0}$ are similar in form to the eikonal equation (3.1.15) and transport equation (3.1.16) satisfied by $\varphi_{c 0}, a_{c 0}$. The following table,
$$
\begin{align*}
& S_{c 0} \longleftrightarrow \varphi_{c 0}  \tag{3.4.9}\\
& \rho_{c 0}^{1 / 2} \longleftrightarrow a_{c 0}  \tag{3.4.10}\\
& w \longleftrightarrow \omega  \tag{3.4.11}\\
& 2 m(w-U) \longleftrightarrow \kappa^{2} n^{2} \tag{3.4.12}
\end{align*}
$$
summarizes explicitly the formal correspondence matching the equation pairs (3.4.5), (3.4.6) and (3.1.13), (3.1.14) and likewise (3.4.7), (3.4.8) and (3.1.15), (3.1.16). In the light of the this observations our claim made seems now more convincing, notwithstanding the fact that eqs. (3.4.1), (3.4.2) and (3.1.11), (3.1.12) differ in important respects.

The assertion that the statistical ensemble has an undulatory nature opens however a number of deep conceptual questions requiring a thorough investigation. The main ones are the following.
(1) The physical characterization of the undulatory system underlying the ensemble.
(2) The identification of the wave field $\psi$ that describes it.
(3) The determination of the wave equation obeyed by $\psi$.

As it turns out, it is not possible to answer the first question directly without having answered the second and the third beforehand. For this reason we address these latter two first. This may appear counterintuitive. Fortunately, the fact that the statistical ensemble of the particle is an undulatory system in the geometrical optics' regime furnishes by itself enough clues.

Let us tackle the second question so, identifying the wave fields $\psi$. We write $\psi$ in terms of its phase and amplitude analogously to (3.1.2). The correspondences (3.4.3), (3.4.4) suggest that these quantities should be of the form $S$ and $\rho^{1 / 2}$, respectively, where $S$ and $\rho$ are certain scalar fields. $S$ and $\rho$ should turn respectively into $S_{c}$ and $\rho_{c}$ in the geometrical optics regime. Further, $S$ and $\rho$ should satisfy differential equations reducing to (3.4.1), (3.4.2) in that limit. However, since $S$ has the dimensions of an action, $S$ it cannot by directly assimilated to the phase, which, being an angle, is dimensionless. $S$ must so be divided by a constant $\hbar$ with the dimensions of an action in order to be correctly equated to the phase. As suggested by the notation, $\hbar$ will be identified eventually with the reduced Planck constant. We have therefore
$$
\begin{equation*}
\psi=\rho^{1 / 2} \exp (i S / \hbar) \tag{3.4.13}
\end{equation*}
$$

Complying with common usage in quantum theory, the wave field $\psi$ will be called wave function.

Next, we tackle the second question, determining the wave equation. To this end, we rely on the following clues.
(1) Eqs. (3.4.1), (3.4.2) should emerge from the wave equation in the formal limit $\hbar \rightarrow 0$.
(2) The wave equation obeyed by $\psi$ must be of 1 st order in time $t$.

The first point follows from the way we have obtained the geometrical optics' equations (3.1.11), (3.1.12), namely by rescaling the phase $\varphi$ by a dimensionless parameter $\xi$ and keeping the terms of the exact equations of leading order in $\xi$ in the fictitious limit $\xi \rightarrow 0$. By (3.4.13), doing the same thing in the present case is evidently tantamount to replacing $\hbar$ by $\xi \hbar$, so that the fictitious limit $\xi \rightarrow 0$ is equivalent to the formal limit $\hbar \rightarrow 0$. Thus, eqs. (3.4.1), (3.4.2), the counterparts of eqs. (3.1.11), (3.1.12) for the statistical ensemble, should emerge as $\hbar \rightarrow 0$. The second point follows from the fact (3.4.1), (3.4.2) are of 1st order in $t$. Since the number of $t$ derivations presumably does not change as $\hbar \rightarrow 0$, the wave equation also shlould have the same property. Using this information, we can guess the expression of the wave equation. It is the Schroedinger equation (E. Schroedinger, 1926),
$$
\begin{equation*}
i \hbar \frac{\partial \psi}{\partial t}=-\frac{\hbar^{2}}{2 m} \nabla^{2} \psi+U \psi \tag{3.4.14}
\end{equation*}
$$

In fact, substitution of the expression (3.4.13) of $\psi$ into (3.4.14) results in the pair of equations
$$
\begin{align*}
& \frac{\partial S}{\partial t}+\frac{1}{2 m}(\boldsymbol{\nabla} S)^{2}+U-\frac{\hbar^{2}}{2 m}\left[\frac{1}{2 \rho} \boldsymbol{\nabla}^{2} \rho-\frac{1}{4 \rho^{2}}(\boldsymbol{\nabla} \rho)^{2}\right]=0  \tag{3.4.15}\\
& \hbar\left[\frac{\partial \rho}{\partial t}+\frac{1}{m} \boldsymbol{\nabla} \cdot(\rho \boldsymbol{\nabla} S)\right]=0 \tag{3.4.16}
\end{align*}
$$

The geometrical optics approximation of these equations, obtained by keeping the contributions of leading order in $\hbar$ and replacing $S$ and $\rho$ with their geometrical optics approximants $S_{c}$ and $\rho_{c}$, yields precisely the ensemble equations (3.4.1), (3.4.2), as required.

Proof. Using (3.4.13), we find
$$
\begin{align*}
& i \hbar \frac{\partial \psi}{\partial t}+\frac{\hbar^{2}}{2 m} \nabla^{2} \psi-U \psi  \tag{3.4.17}\\
& =\left\{i \hbar\left[\frac{1}{2 \rho^{1 / 2}} \frac{\partial \rho}{\partial t}+\frac{i}{\hbar} \rho^{1 / 2} \frac{\partial S}{\partial t}\right]+\frac{\hbar^{2}}{2 m}\left[\frac{1}{2 \rho^{1 / 2}} \nabla^{2} \rho-\frac{1}{4 \rho^{3 / 2}}(\boldsymbol{\nabla} \rho)^{2}\right.\right. \\
& \left.\left.+\frac{i}{\hbar \rho^{1 / 2}} \boldsymbol{\nabla} \rho \cdot \boldsymbol{\nabla} S+\rho^{1 / 2}\left(\frac{i}{\hbar} \boldsymbol{\nabla}^{2} S-\frac{1}{\hbar^{2}}(\boldsymbol{\nabla} S)^{2}\right)\right]-U \rho^{1 / 2}\right\} \exp \left(\frac{i S}{\hbar}\right) \\
& =\left\{\frac{i \hbar}{2 \rho}\left[\frac{\partial \rho}{\partial t}+\frac{1}{m} \boldsymbol{\nabla} \cdot(\rho \boldsymbol{\nabla} S)\right]-\left[\frac{\partial S}{\partial t}+\frac{1}{2 m}(\boldsymbol{\nabla} S)^{2}+U\right.\right. \\
& \left.\left.-\frac{\hbar^{2}}{2 m}\left(\frac{1}{2 \rho} \boldsymbol{\nabla}^{2} \rho-\frac{1}{4 \rho^{2}}(\boldsymbol{\nabla} \rho)^{2}\right)\right]\right\} \rho^{1 / 2} \exp \left(\frac{i S}{\hbar}\right) .
\end{align*}
$$

Thus, (3.4.14) is equivalent to (3.4.15), (3.4.16).

In view of the above findings, we are now ready to offer a reasonable physically motivated answer to the first question we stated above but left unaddressed, the physical characterization of the undulatory system underlying the statistical ensemble, henceforth named Schroedinger undulatory system. This will lead to a radical reexamination of the basic tenets of the theory.

In the geometrical optical regime, the Schroedinger undulatory system appears in the guise of the statistical ensemble of a classical particle. Its being a collection of a large number of copies of a given physical system in that regime is a very basic property unlikely to emerge after but be absent before its attainment. Thus, at some level of description, our undulatory system should also be a collection of copies of some physical entity. It is natural to view the collection as the statistical ensemble of the entity. The entity cannot be simply the classical particle itself, else the ensemble of the entity would be necessarily equal to the ensemble of the particle, while we know it is so only in a certain limit. The entity must however be related to the particle in a special way. Determining the nature of the entity is then the key to the solution of the problem of characterizing the undulatory system.

The answer to the issues posed in the previous paragraph rests the identification of the action scale $\hbar$ with the reduced Planck constant anticipated earlier. We assume again that we are in the geometric optics regime. By (3.4.13), the formal limit $\hbar \rightarrow 0$ leading to that regime corresponds physically to the situation where the wave function $\psi$ becomes very sensitive to the value of the action $S$ and, so, to its time and space variations. Fix a time $t$ and a point $\boldsymbol{x}$. Then, by what just observed, for $t^{\prime}, \boldsymbol{x}^{\prime}$ sufficiently close to $t, \boldsymbol{x}$, the geometrical optics' wave function $\psi_{c}$ is given approximately by
$$
\begin{align*}
\psi_{c}\left(t^{\prime}, \boldsymbol{x}^{\prime}\right) & =\rho_{c}^{1 / 2}\left(t^{\prime}, \boldsymbol{x}^{\prime}\right) \exp \left(\frac{i}{\hbar} S_{c}\left(t^{\prime}, \boldsymbol{x}^{\prime}\right)\right)  \tag{3.4.18}\\
\simeq & \rho_{c}^{1 / 2}(t, \boldsymbol{x}) \exp \left[\frac{i}{\hbar}\left(S_{c}(t, \boldsymbol{x})+\frac{\partial S_{c}(t, \boldsymbol{x})}{\partial t}\left(t^{\prime}-t\right)+\boldsymbol{\nabla} S_{c}(t, \boldsymbol{x}) \cdot\left(\boldsymbol{x}^{\prime}-\boldsymbol{x}\right)\right)\right]
\end{align*}
$$
since $\rho_{c}$ can be considered approximately constant while $S_{c}$ still varies appreciably. Comparing (3.4.18) and (1.3.20), we realize that, instantaneously in time and locally in space, $\psi$ behaves as a plane wave of angular frequency and wave vector
$$
\begin{align*}
\omega & =-\frac{1}{\hbar} \frac{\partial S_{c}}{\partial t}  \tag{3.4.19a}\\
\boldsymbol{\kappa} & =\frac{1}{\hbar} \nabla S_{c} \tag{3.4.19b}
\end{align*}
$$
respectively. Now recall that the momentum field of the statistical ensemble is given by eq. (3.3.13), which we report here for convenience
$$
\begin{equation*}
\boldsymbol{p}_{c}=\nabla S_{c} \tag{3.4.20}
\end{equation*}
$$

Expressing $\partial S_{c} / \partial t$ using eq. (3.4.1) and using (3.4.20), we obtain the relations
$$
\begin{align*}
& \frac{\boldsymbol{p}_{c}{ }^{2}}{2 m}+U=\hbar \omega  \tag{3.4.21a}\\
& \boldsymbol{p}_{c}=\hbar \boldsymbol{\kappa} \tag{3.4.21b}
\end{align*}
$$
(3.4.21a), (3.4.21b) are evidently analogous to and generalize the basic PlanckEinstein relations (1.12.1a), (1.12.1b). We are led in this way to the hypothesis that
the Schroedinger undulatory system which we are considering belongs to the realm of quantum physics
and that, as suggested by notation,
the action scale $\hbar$ is the reduced Planck constant.
Let us assume that the constituents of the quantum world all have a classical analogue in the spirit of Bohr's correspondence principle (cf. sect. 2.5). Then, the simplest theory one can reasonably formulate is that
the entity of which the Schroedinger undulatory system is the statistical ensemble is a quantum particle whose classical analogue is precisely the classical particle we started with.

It is important to understand the implications of the conclusion which we have reached and confront their apparently paradoxical nature.

The Schroedinger undulatory system can be decoded as quantum ensemble, a collection of very many copies of a quantum particle.

The ensemble we started with gets identified as the classical ensemble and it is the classical analogue of the quantum one just as the classical particle it describes is the classical analogue of the quantum particle the quantum ensemble does. So, the geometric optical limit turning the quantum ensemble into the classical one is just the semiclassical regime in which quantum mechanics reproduces classical one.

The results of the analysis we have carried out up to this point are summarized in the following points.
(1) The Schroedinger undulatory system is the statistical ensemble of a quantum particle.
(2) As such, it is described by a wave function $\psi$ and its dynamics is governed by the Schroedinger equation (3.4.14) obeyed by $\psi$.
(3) The geometrical optical regime of the undulatory system is equivalent to the semiclassical regime in which the ensemble of the quantum particle reduces to that of the classical particle that is its classical analogue and corresponds to the formal limit $\hbar \rightarrow 0$.

The reducibility of the undulatory system to a statistical ensemble is a rather counterintuitive property. The familiar waves of classical physics, such as the acoustic and the electromagnetic waves, do not lend themselves to an representation of this kind. The conundrum will be solved is sect. 3.5 through the probabilistic interpretation of the wave function.

The identification of the geometric optical and semiclassical regimes allows one to elucidate the conditions under which quantum dynamics turns into classical one. By (3.4.13), for the geometric optical regime to set in, it is necessary that the amplitude $\rho^{1 / 2}$ of the wave function $\psi$ varies little on a distance on which the phase $S / \hbar$ of $\psi$ varies appreciably. Such distance corresponds to a length scale
$$
\begin{equation*}
\lambda_{d B}=\frac{2 \pi \hbar}{|\boldsymbol{\nabla} S|} \tag{3.4.22}
\end{equation*}
$$

By (3.4.20), being $S$ approximated by $S_{c}, \lambda_{d B}$ is nothing but the de Broglie wave length of the particle defined in 1.12.2. The regime, so, sets in when
$$
\begin{equation*}
\lambda_{d B}\left|\rho^{-1} \nabla \rho\right| \ll 1 \tag{3.4.23}
\end{equation*}
$$

This is then also the condition under which quantum dynamics becomes amenable to a classical description and so reduces to classical one.

The solution of the Schroedinger equation requires that an initial condition for the wave function $\psi$ be given,
$$
\begin{equation*}
\psi(0, \boldsymbol{x})=\psi_{0}(\boldsymbol{x}) \tag{3.4.24}
\end{equation*}
$$

Once $\psi_{0}$ is known, it is principle possible to solve the equation and obtain $\psi$. General methods of solution of the Schroedinger problem will be illustrated in later sections.

Writing $\psi$ as in (3.4.13) and similarly $\psi_{0}$ as
$$
\begin{equation*}
\psi_{0}=\rho_{0}{ }^{1 / 2} \exp \left(i S_{0} / \hbar\right) \tag{3.4.25}
\end{equation*}
$$
the initial condition (3.4.23) for $\psi$ yields initial conditions for $S$ and $\rho$,
$$
\begin{align*}
& S(0, \boldsymbol{x})=S_{0}(\boldsymbol{x})  \tag{3.4.26}\\
& \rho(0, \boldsymbol{x})=\rho_{0}(\boldsymbol{x}) \tag{3.4.27}
\end{align*}
$$
which are required for the solution of eqs. (3.4.15), (3.4.16). Now, consider $\hbar$ formally as a parameter. $S_{0}, \rho_{0}$ may depend on $\hbar$. Then, the solutions $S$ and $\rho$ depend on $\hbar$ in a non trivial way as the equations they satisfy as well as their initial data do. For this reason, the formal limit $\hbar \rightarrow 0$ corresponding to the semiclassical regime may yield solutions of eqs. (3.4.1), (3.4.2) more general than those studied in sect. 3.3 characterized by the special initial condition (3.3.19), (3.3.20). Thus, the Schroedinger equation covers a range of situations broader than that we originally started with.

\subsection*{3.5. The statistical interpretation of the wave function}

As anticipated in sect. 3.3, the classical statistical ensemble serves the purpose of studying the allowed dynamics of a classical particle from a statistical point of view. We are now going to see in detail how.

The ensemble is a collection of a very large number $N$ of copies of the particle intuitively thought of as a statistical fluid of copies each endowed with a fictitious statistical mass $1 / N . \rho_{c}$ is the statistical mass density of the ensemble. By definition, for any space region $\mathcal{V}$
$$
\begin{equation*}
p_{c}(\mathcal{V})=\int_{\mathcal{V}} d^{3} x \rho_{c} \tag{3.5.1}
\end{equation*}
$$
is the statistical mass of the portion of fluid contained in $\mathcal{V}$. Clearly, $p_{c}(\mathcal{V})=$ $N_{c}(\mathcal{V}) / N$, where $N_{c}(\mathcal{V})$ is the number of copies lying in $\mathcal{V}$. Thus, $p_{c}(\mathcal{V})$ is the fraction of copies present in $\mathcal{V}$. As $N$ is very large,
$p_{c}(\mathcal{V})$ is the probability of finding a randomly chosen copy within $\mathcal{V}$. Consequently, $\rho_{c}$ is the probability density of the statistical ensemble.

More can be said. If we set
$$
\begin{equation*}
\boldsymbol{j}_{c}=\frac{1}{m} \rho_{c} \boldsymbol{\nabla} S_{c}, \tag{3.5.2}
\end{equation*}
$$
the statistical mass conservation equation (3.3.18) becomes
$$
\begin{equation*}
\frac{\partial \rho_{c}}{\partial t}+\boldsymbol{\nabla} \cdot \boldsymbol{j}_{c}=0 \tag{3.5.3}
\end{equation*}
$$
(3.5.3) has the universal form of a conservation equation, whereby $\boldsymbol{j}_{c}$ is the statistical mass current density. By definition, for any oriented space surface $\mathcal{A}$,
$$
\begin{equation*}
\Phi_{c}(\mathcal{A})=\int_{\mathcal{A}} d^{2} \boldsymbol{x} \cdot \boldsymbol{j}_{c} \tag{3.5.4}
\end{equation*}
$$
is the statistical mass flowing through $\mathcal{A}$ per unit time. As $\Phi_{c}(\mathcal{A})=F_{c}(\mathcal{A}) / N$, where $F_{c}(\mathcal{A})$ is the number of copies traversing $\mathcal{A}$ per unit time, $\Phi_{c}(\mathcal{A})$ is the fraction of copies crossing $\mathcal{A}$ per unit time. Since $N$ is very large,
$\Phi_{c}(\mathcal{A})$ is the probability of finding a randomly chosen copy flowing through $\mathcal{A}$ per unit time. Consequently, $\boldsymbol{j}_{c}$ is the probability current density of the statistical ensemble.

In view of the above reinterpretation of $\rho_{c}$ and $\boldsymbol{j}_{c}$, eq. (3.5.3) can be regarded as the probability conservation equation.

Having a probabilistic framework, we can compute averages. If $Q_{f}$ is the physical quantity corresponding to a given phase function $f$, then the mean value $\left\langle Q_{f}\right\rangle_{c}$ of $Q_{f}$ of a copy is given by
$$
\begin{equation*}
\left\langle Q_{f}\right\rangle_{c}=\int d^{3} x \rho_{c} f\left(\boldsymbol{x}, \boldsymbol{\nabla} S_{c}\right) \tag{3.5.5}
\end{equation*}
$$

In particular, the mean position, momentum, angular momentum, energy of a copy, $\langle\boldsymbol{q}\rangle_{c},\langle\boldsymbol{p}\rangle_{c},\langle\boldsymbol{l}\rangle_{c},\langle H\rangle_{c}$, can be expressed in terms of $\rho_{c}$ and $S_{c}$, as
$$
\begin{align*}
\langle\boldsymbol{q}\rangle_{c} & =\int d^{3} x \rho_{c} \boldsymbol{x}  \tag{3.5.6}\\
\langle\boldsymbol{p}\rangle_{c} & =\int d^{3} x \rho_{c} \boldsymbol{\nabla} S_{c}  \tag{3.5.7}\\
\langle\boldsymbol{l}\rangle_{c} & =\int d^{3} x \rho_{c} \boldsymbol{x} \times \boldsymbol{\nabla} S_{c}  \tag{3.5.8}\\
\langle H\rangle_{c} & =\int d^{3} x \rho_{c}\left(\frac{\left(\boldsymbol{\nabla} S_{c}\right)^{2}}{2 m}+U\right) \tag{3.5.9}
\end{align*}
$$

All the above mean values depend in general on time since $\rho_{c}$ and $S_{c}$ do.

Proof. As we have seen above, $\rho_{c}$ is the probability density of the statistical ensemble. $d^{3} x \rho_{c}(t, \boldsymbol{x})$ is so the fraction of copies lying in a small region of volume $d^{3} x$ around $\boldsymbol{x}$ at time $t$. Thus, the mean value of a quantity $Q$ of the copies is given by
$$
\begin{equation*}
\langle Q\rangle_{c}=\int d^{3} x \rho_{c} Q_{c} \tag{3.5.10}
\end{equation*}
$$
where $Q_{c}$ is the field such that $Q_{c}(t, \boldsymbol{x})$ is the value of $Q$ of all copies located at $\boldsymbol{x}$ at time $t$. If $Q$ is the physical quantity $Q_{f}$ corresponding to a phase function $f$, then
$$
\begin{equation*}
Q_{f c}=f\left(\boldsymbol{q}_{c}, \boldsymbol{p}_{c}\right) \tag{3.5.11}
\end{equation*}
$$
where $\boldsymbol{q}_{c}, \boldsymbol{p}_{c}$ are the position and momentum of all copies placed at $\boldsymbol{x}$ at time $t$, respectively. We obviously have
$$
\begin{equation*}
\boldsymbol{q}_{c}=\boldsymbol{x} \tag{3.5.12}
\end{equation*}
$$
$\boldsymbol{p}_{c}$ is nothing but the momentum field of the ensemble introduced earlier in sect. 3.3, which by (3.3.13) is given by the relation
$$
\begin{equation*}
\boldsymbol{p}_{c}=\nabla S_{c} \tag{3.5.13}
\end{equation*}
$$

By substituting (3.5.12), (3.5.13) into (3.5.11), we find
$$
\begin{equation*}
Q_{f c}=f\left(\boldsymbol{x}, \boldsymbol{\nabla} S_{c}\right) \tag{3.5.14}
\end{equation*}
$$

Inserting (3.5.14) into (3.5.10), we reach (3.5.5).
For the position $\boldsymbol{q}, \boldsymbol{q}_{c}$ is given by (3.5.12) By (3.5.5), the mean position of the copies is then given by (3.5.6). For the momentum $\boldsymbol{p}, \boldsymbol{p}_{c}$ is given by (3.5.13) By (3.5.5) again, so, the mean momentum of the copies is given by (3.5.7). For the angular momentum $l$, we obtain similarly that
$$
\begin{equation*}
\boldsymbol{l}_{c}=\boldsymbol{q}_{c} \times \boldsymbol{p}_{c}=\boldsymbol{x} \times \nabla S_{c} \tag{3.5.15}
\end{equation*}
$$
so that, by (3.5.5) again, the mean angular momentum of the copies is given by (3.5.8). Finally, for the energy $H$, we have
$$
\begin{equation*}
H_{c}=\frac{\boldsymbol{p}_{c}^{2}}{2 m}+U\left(\boldsymbol{q}_{c}\right)=\frac{\left(\boldsymbol{\nabla} S_{c}\right)^{2}}{2 m}+U \tag{3.5.16}
\end{equation*}
$$
so that the mean energy of the copies is given by (3.5.9), by (3.5.5) once more.

In sect. 3.4 above, we have found that a quantum statistical ensemble characterizing the mechanics of a quantum particle also exists, that the quantum ensemble is an undulatory system described by a wave function $\psi$ and evolving according to the Schroedinger equation (3.4.14) obeyed by $\psi$ and that the semiclassical regime in which the quantum ensemble reduces to the classical one is equivalent to the geometric optical regime of the underlying undulatory system.

According to wave mechanics, the probabilistic description of the classical
statistical ensemble extends to the quantum one. Indeed, the wave function $\psi$ has the decomposition (3.4.13) in terms of the amplitude and phase fields $\rho$ and $S$. When $\psi$ satisfies the Schroedinger equation (3.4.14), $\rho$ and $S$ satisfy eqs. (3.4.15), (3.4.16). In the semiclassical limit $\hbar \rightarrow 0, \rho$ and $S$ become the statistical mass density $\rho_{c}$ and the Hamilton principal function $S_{c}$ of the classical ensemble and eqs. (3.4.15), (3.4.16) reduce to to eqs. (3.4.1), (3.4.2), respectively. As $\rho_{c}$ and $S_{c}$ via $\boldsymbol{j}_{c}$ have probabilistic interpretations, as seen above, it is plausible that $\rho$ and $S$ also do. Also the wave function $\psi$, so, acquires a probabilistic meaning. This eventually throws light on the nature of the undulatory system underlying the ensemble. The waves it features and $\psi$ describes are probability waves.

The first postulate of wave mechanics can be formulated as follows
(1) The probability density of the quantum ensemble is $\rho$.

This assertion is justified by the fact that $\rho$ reduces in the semiclassical regime to the probability density $\rho_{c}$ of the classical ensemble. So, for any space region $\mathcal{V}$,
$$
\begin{equation*}
p(\mathcal{V})=\int_{\mathcal{V}} d^{3} x \rho \tag{3.5.17}
\end{equation*}
$$
is the probability finding a randomly chosen copy located in $\mathcal{V}$. Eq. (3.5.17) clearly answers to (3.5.1).

The second postulate of wave mechanics runs as follows.
(2) The probability current density of the quantum ensemble is
$$
\begin{equation*}
\boldsymbol{j}=\frac{1}{m} \rho \boldsymbol{\nabla} S \tag{3.5.18}
\end{equation*}
$$

This claim is grounded on the fact that $\boldsymbol{j}$ becomes in the semiclassical regime the the probability current density $\boldsymbol{j}_{c}$ of the classical ensemble, by (3.5.2). The assertion is further reinforced by the observation that, by (3.4.16), $\rho$ and $\boldsymbol{j}$ satisfy together the probability conservation equation
$$
\begin{equation*}
\frac{\partial \rho}{\partial t}+\boldsymbol{\nabla} \cdot \boldsymbol{j}=0 \tag{3.5.19}
\end{equation*}
$$
extending (3.5.3). Therefore, for any oriented space surface $\mathcal{A}$
$$
\begin{equation*}
\Phi(\mathcal{A})=\int_{\mathcal{A}} d^{2} \boldsymbol{x} \cdot \boldsymbol{j} \tag{3.5.20}
\end{equation*}
$$
is the probability of finding a copy flowing through $\mathcal{A}$ per unit time. Eq. (3.5.20) clearly answers to (3.5.4).

The third postulate of wave mechanics runs as follows.
(3) For any phase function $f$, the quantum mean $\left\langle Q_{f}\right\rangle$ of the physical quantity $Q_{f}$ of a copy is given by
$$
\begin{equation*}
\left\langle Q_{f}\right\rangle=\int d^{3} x \rho f(\boldsymbol{x}, \nabla S) \tag{3.5.21}
\end{equation*}
$$

According to (3.5.21), the quantum mean position, momentum, angular momentum, energy of a copy, $\langle\boldsymbol{q}\rangle,\langle\boldsymbol{p}\rangle,\langle\boldsymbol{l}\rangle,\langle H\rangle$ then are given by
$$
\begin{align*}
& \langle\boldsymbol{q}\rangle=\int d^{3} x \rho \boldsymbol{x}  \tag{3.5.22}\\
& \langle\boldsymbol{p}\rangle=\int d^{3} x \rho \boldsymbol{\nabla} S  \tag{3.5.23}\\
& \langle\boldsymbol{l}\rangle=\int d^{3} x \rho \boldsymbol{x} \times \boldsymbol{\nabla} S  \tag{3.5.24}\\
& \langle H\rangle=\int d^{3} x \rho\left(\frac{(\boldsymbol{\nabla} S)^{2}}{2 m}+U\right) \tag{3.5.25}
\end{align*}
$$

The expressions (3.5.21) and (3.5.22)-(3.5.25) of the quantum ensemble mean values follow from extrapolating the identities (3.5.5) and (3.5.6)-(3.5.9) giving the corresponding classical ensemble quantities to the quantum regime by replacing $\rho_{c}, S_{c}$ with $\rho, S$ throughout.

Though the above way of thinking is reasonable enough, it is based on a procedure which we may define as "classical to quantum extrapolation" which has the drawback of being affected by an intrinsic ambiguity: the expression this method provides us with could miss in principle terms that vanish in the semiclassical limit $\hbar \rightarrow 0$. Therefore, we need some criterion to ascertain their quantum completeness. This is provided by a fourth postulate we shall formulate
shortly. As we shall see, while the results yielded by the first and second postulate are indeed correct, those following from the third need to be revised.

The fourth postulate of wave mechanics runs as follows.
(4) All quantities relating to the quantum ensemble depend on the fields $\rho$, $S$ through expressions polynomial in the wave function $\psi$, its conjugate $\psi^{*}$ and their space derivatives.

The requirement is suggested by our experience with other more conventional wave systems, in which the main physical quantities are given by identities involving combinations of $\psi, \psi^{*}$ of the type indicated. Our next task is checking that the probability density and current density $\rho, \boldsymbol{j}$ and the quantum means of all mechanical quantities are expressible in this way.

The quantum probability density and current density $\rho$ and $\boldsymbol{j}$, expressed in terms of the wave function $\psi$, are given by
$$
\begin{align*}
\rho & =|\psi|^{2}  \tag{3.5.26}\\
\boldsymbol{j} & =\frac{\hbar}{m} \operatorname{Im}\left(\psi^{*} \nabla \psi\right) \tag{3.5.27}
\end{align*}
$$

The fourth postulate is evidently verified.

Proof. Relation (3.5.26) follows immediately from relation (3.4.13).
Next, using (3.4.13) again, we find
$$
\begin{equation*}
\psi^{-1}(-i \hbar \nabla \psi)=-\frac{i \hbar}{2 \rho} \nabla \rho+\nabla S \tag{3.5.28}
\end{equation*}
$$
which, using (3.5.26), furnishes
$$
\begin{equation*}
\rho \boldsymbol{\nabla} S=\psi^{*}(-i \hbar \boldsymbol{\nabla} \psi)+\frac{i \hbar}{2} \boldsymbol{\nabla}|\psi|^{2} \tag{3.5.29}
\end{equation*}
$$

By (3.5.18), taking the real part of both sides of (3.5.29), we get readily (3.5.27).

Combining (3.5.17) and (3.5.26), we find that the probability that a copy randomly chosen in the ensemble be found within a space region $\mathcal{V}$ is
$$
\begin{equation*}
p(\mathcal{V})=\int_{\mathcal{V}} d^{3} x|\psi|^{2} \tag{3.5.30}
\end{equation*}
$$

Likewise, combining (3.5.20) and and (3.5.27), we find that the probability of finding a copy in the ensemble flowing through an oriented space surface $\mathcal{A}$ per unit time is given by
$$
\begin{equation*}
\Phi(\mathcal{A})=\frac{\hbar}{m} \int_{\mathcal{A}} d^{2} \boldsymbol{x} \cdot \operatorname{Im}\left(\psi^{*} \boldsymbol{\nabla} \psi\right) \tag{3.5.31}
\end{equation*}
$$

Eqs. (3.5.30), (3.5.32) provide the probabilistic interpretation of the wave function $\psi$. It is one of the most fundamental predictions of quantum theory (M. Born, 1926).

When $\mathcal{V}$ is the whole space $\mathbb{E}^{3}$, we obviously have $p(\mathcal{V})=1$. From (3.5.30), it follows then that the wave function $\psi$ must satisfy the normalization condition
$$
\begin{equation*}
\int d^{3} x|\psi|^{2}=1 \tag{3.5.32}
\end{equation*}
$$

A wave function $\psi$ obeying (3.5.32) is said normalized.
It is not obvious at all that the expression (3.5.21) of the quantum mean of a general mechanical quantity can be recast in such a way to satisfy the fourth postulate. Doing so may require in an essential way the use of eqs. (3.4.15), (3.4.16) to reshape it in the desired way. (3.4.15), (3.4.16), however, involve special combinations of $\rho$ and $S$ which are unlikely to work for a general phase function $f$. At best, they can achieve this goal in particular cases such as those of eqs. (3.5.22)-(3.5.25) and even so only up to contributions vanishing in the semiclassical limit $\hbar \rightarrow 0$, as it might be expected based on their origin from straightforward classical to quantum extrapolation. This is can be checked easily enough through a few explicit calculations.

Proof. We have to check whether expressing $\rho$ and $S$ through $\psi$ via (3.4.13) after using (3.4.15), (3.4.16), one can bring (3.5.22)-(3.5.25) in a form compatible with the fourth postulate.

Inserting (3.5.26) into (3.5.22), one finds immediately
$$
\begin{equation*}
\langle\boldsymbol{q}\rangle=\int d^{3} x \psi^{*} \boldsymbol{x} \psi \tag{3.5.33}
\end{equation*}
$$

The expression for $\langle\boldsymbol{q}\rangle$ given in (3.5.33) is clearly compatible with fourth postulate. Next, inserting (3.5.29) into (3.5.23), we get
$$
\begin{equation*}
\langle\boldsymbol{p}\rangle=\int d^{3} x\left(\psi^{*}(-i \hbar \boldsymbol{\nabla} \psi)+\frac{i \hbar}{2} \boldsymbol{\nabla}|\psi|^{2}\right)=\int d^{3} x \psi^{*}(-i \hbar \boldsymbol{\nabla} \psi) \tag{3.5.34}
\end{equation*}
$$
where in the last step we used the relation
$$
\begin{equation*}
\int d^{3} x \nabla|\psi|^{2}=0 \tag{3.5.35}
\end{equation*}
$$
which follows from one of the versions of Gauss' theorem and the fact that $|\psi|^{2}$ falls off rapidly at infinity in virtue of (3.5.32). The expression for $\langle\boldsymbol{p}\rangle$ shown in (3.5.35) is also compatible with fourth postulate. Next, putting (3.5.29) into (3.5.24), we get
$$
\begin{equation*}
\langle\boldsymbol{l}\rangle=\int d^{3} x\left(\psi^{*}(-i \hbar \boldsymbol{x} \times \boldsymbol{\nabla} \psi)+\frac{i \hbar}{2} \boldsymbol{x} \times \boldsymbol{\nabla}|\psi|^{2}\right)=\int d^{3} x \psi^{*}(-i \hbar \boldsymbol{x} \times \boldsymbol{\nabla} \psi) \tag{3.5.36}
\end{equation*}
$$
where in the last step we used the relation
$$
\begin{equation*}
\int d^{3} x \boldsymbol{x} \times \boldsymbol{\nabla}|\psi|^{2}=-\int d^{3} x \boldsymbol{\nabla} \times\left(\boldsymbol{x}|\psi|^{2}\right)=\mathbf{0} \tag{3.5.37}
\end{equation*}
$$
which follows from another of the versions of Gauss' theorem and the rapid fall off of $|\psi|^{2}$. Also the expression for $\langle\boldsymbol{l}\rangle$ in (3.5.37) satisfies the fourth postulate too. Things are more involved for the mean of $H$. From (3.5.25), using eq. (3.4.15), we get
$$
\begin{equation*}
\langle H\rangle=-\int d^{3} x \rho \frac{\partial S}{\partial t}+O\left(\hbar^{2}\right) \tag{3.5.38}
\end{equation*}
$$

We now express the right hand side of (3.5.38) in terms of $\psi$. From (3.4.13), we find
$$
\begin{equation*}
\psi^{-1} i \hbar \frac{\partial \psi}{\partial t}=\frac{i \hbar}{2 \rho} \frac{\partial \rho}{\partial t}-\frac{\partial S}{\partial t} \tag{3.5.39}
\end{equation*}
$$
which, using (3.5.26), furnishes
$$
\begin{equation*}
-\rho \frac{\partial S}{\partial t}=\psi^{*} i \hbar \frac{\partial \psi}{\partial t}-\frac{i \hbar}{2} \frac{\partial|\psi|^{2}}{\partial t} \tag{3.5.40}
\end{equation*}
$$

Inserting (3.5.40) into (3.5.38), we get
$$
\begin{equation*}
\langle H\rangle=\int d^{3} x\left(\psi^{*} i \hbar \frac{\partial \psi}{\partial t}-\frac{i \hbar}{2} \frac{\partial|\psi|^{2}}{\partial t}\right)+O\left(\hbar^{2}\right)=\int d^{3} x \psi^{*} i \hbar \frac{\partial \psi}{\partial t}+O\left(\hbar^{2}\right) \tag{3.5.41}
\end{equation*}
$$
since, by the normalization condition (3.5.32)
$$
\begin{equation*}
\int d^{3} x \frac{\partial|\psi|^{2}}{\partial t}=\frac{d}{d t} \int d^{3} x|\psi|^{2}=0 \tag{3.5.42}
\end{equation*}
$$

Using the Schroedinger equation (3.4.14), (3.5.41) can be immediately put in the form
$$
\begin{equation*}
\langle H\rangle=\int d^{3} x \psi^{*}\left(-\frac{\hbar^{2}}{2 m} \nabla^{2} \psi+U \psi\right)+O\left(\hbar^{2}\right) \tag{3.5.43}
\end{equation*}
$$

The expression of $\langle H\rangle$ given in (3.5.43), so, is compatible with the fourth postulate only up $O\left(\hbar^{2}\right)$ corrections, which in a full quantum theory cannot be neglected in general. The problem is expected to show up in even more acute form when computing the quantum means of the physical quantities $Q_{f}$ corresponding to more complicated phase functions $f$.

It is clear that the prescription for the computation of the quantum means given by the third postulate, albeit not completely wrong, it is not satisfactory in its present form if we adhere rigorously to the principle stated by the fourth postulate. To solve this problem, we have to reconsider classical means from an alternative point of view.

We begin with introducing the phase function
$$
\begin{equation*}
\varpi_{c}(t, \boldsymbol{x}, \boldsymbol{y})=\rho_{c}(t, \boldsymbol{x}) \delta\left(\boldsymbol{y}-\boldsymbol{\nabla} S_{c}(t, \boldsymbol{x})\right) \tag{3.5.44}
\end{equation*}
$$
where $\delta(\boldsymbol{y})$ is the momentum space Dirac delta function. This is defined analogously to its configuration space counterpart (cf. eqs. (3.3.21), (3.3.22)),
$$
\begin{equation*}
\delta(\boldsymbol{y})=0 \quad \text { for } \boldsymbol{y} \neq \mathbf{0}, \quad \delta(\boldsymbol{y})=\infty \quad \text { for } \boldsymbol{y}=\mathbf{0} \tag{3.5.45}
\end{equation*}
$$
the strength of the singularity at $\boldsymbol{y}=\mathbf{0}$ being such that
$$
\begin{equation*}
\int d^{3} y \delta(\boldsymbol{y})=1 \tag{3.5.46}
\end{equation*}
$$
$\varpi_{c}$ turns out to be the phase space probability distribution of the statistical ensemble. $\varpi_{c}(t, \boldsymbol{x}, \boldsymbol{y}) d^{3} x d^{3} y$ is the fraction of copies of the particle in the ensemble whose position and momentum lie in small ranges $d^{3} x, d^{3} y$ around $\boldsymbol{x}, \boldsymbol{y}$ at a given time $t$. This interpretation is supported by two properties of $\varpi_{c}$. First, the configuration space marginal distribution induced by $\varpi_{c}$ is $\rho_{c}$,
$$
\begin{equation*}
\int d^{3} y \varpi_{c}=\rho_{c} \tag{3.5.47}
\end{equation*}
$$

Second, if $Q_{f}$ is the physical quantity corresponding to a given phase function $f$, the mean of $Q_{f}$ of a copy can be expressed in terms of $\varpi_{c}$ as
$$
\begin{equation*}
\left\langle Q_{f}\right\rangle_{c}=\int d^{3} x d^{3} y f \varpi_{c} \tag{3.5.48}
\end{equation*}
$$

Proof. Below, we leave $t$ dependence understood for simplicity. From (3.5.44), we clearly have $\varpi_{c}(\boldsymbol{x}, \boldsymbol{y}) \geq 0$. From (3.5.44) again, the fact that $\int d^{3} x \rho_{c}=1$ and (3.5.46), we have
$$
\begin{equation*}
\int d^{3} x d^{3} y \varpi_{c}(\boldsymbol{x}, \boldsymbol{y})=\int d^{3} x \rho_{c}(\boldsymbol{x}) \int d^{3} y \delta\left(\boldsymbol{y}-\boldsymbol{\nabla} S_{c}(\boldsymbol{x})\right)=1 \tag{3.5.49}
\end{equation*}
$$

It follows that $\varpi_{c}$ is a phase space probability distribution. Using (3.5.44) and (3.5.46) once more, we find further
$$
\begin{equation*}
\int d^{3} y \varpi_{c}(\boldsymbol{x}, \boldsymbol{y})=\rho_{c}(\boldsymbol{x}) \int d^{3} y \delta\left(\boldsymbol{y}-\boldsymbol{\nabla} S_{c}(\boldsymbol{x})\right)=\rho_{c}(\boldsymbol{x}) \tag{3.5.50}
\end{equation*}
$$
showing property $(3.5 .47)$.
From relation (3.5.5), using (3.5.46), the mean value $\left\langle Q_{f}\right\rangle_{c}$ of $Q_{f}$ is
$$
\begin{equation*}
\left\langle Q_{f}\right\rangle_{c}=\int d^{3} x d^{3} y f(\boldsymbol{x}, \boldsymbol{y}) \rho_{c}(\boldsymbol{x}) \delta\left(\boldsymbol{y}-\boldsymbol{\nabla} S_{c}(\boldsymbol{x})\right)=\int d^{3} x d^{3} y f(\boldsymbol{x}, \boldsymbol{y}) \varpi_{c}(\boldsymbol{x}, \boldsymbol{y}) \tag{3.5.51}
\end{equation*}
$$
(3.5.48) thus holds.

The analysis of the previous paragraph suggests to look for a suitable phase space probability distribution $\varpi$ reducing to $\varpi_{c}$ in the semiclassical limit $\hbar \rightarrow 0$ such that $\varpi$ can be used to compute quantum marginal distribution and means much in the same way as $\varpi_{c}$ is to compute the classical ones. Consideration of
very basic quantum features as the uncertainty relations (cf. sect. 1.18) indicates that this cannot work, unless we are willing to accept that $\varpi$ cannot enjoy all the main properties $\varpi_{c}$ does. The measurement of position and momentum of a quantum particle interfere and cannot be carried out simultaneously (see sect. 7.10). For this reason, it is operationally impossible to determine the fraction $\varpi(t, \boldsymbol{x}, \boldsymbol{y}) d^{3} x d^{3} y$ of copies of the particle in the ensemble whose position and momentum lie in small ranges $d^{3} x, d^{3} y$ around $\boldsymbol{x}, \boldsymbol{y}$ at a given time $t$, when the phase space volume $d^{3} x d^{3} y$ is smaller than roughly $\hbar^{3}$. For reasons that will become clear momentarily, $\varpi$ must be a phase space quasiprobability distribution, that is a real valued phase space function $\varpi$ that satisfies the normalization condition $\int d^{3} x d^{3} y \varpi=1$ but can possibly take negative values. Thus, unlike $\varpi_{c}, \varpi$ is not a genuine probability distribution and must be understood only as an effective quantum phase space probability distribution. This leads to adding a fifth postulate to wave mechanics.
(5) There is a phase space quasiprobability distribution $\varpi$ such that
$$
\begin{equation*}
\varpi=\varpi_{c}+O(\hbar) \tag{3.5.52}
\end{equation*}
$$
in the semiclassical limit $\hbar \rightarrow 0$, where $\varpi_{c}$ is given by (3.5.44), whose configuration space marginal distribution is $\rho$,
$$
\begin{equation*}
\int d^{3} y \varpi=\rho \tag{3.5.53}
\end{equation*}
$$
and is such that the quantum mean $\left\langle Q_{f}\right\rangle$ of $Q_{f}$ is
$$
\begin{equation*}
\left\langle Q_{f}\right\rangle=\int d^{3} x d^{3} y f \varpi \tag{3.5.54}
\end{equation*}
$$
for any phase function $f$.
Note that by (3.5.52) $\rho=\rho_{c}+O(\hbar)$ and $\left\langle Q_{f}\right\rangle=\left\langle Q_{f}\right\rangle_{c}+O(\hbar)$ as expected on general grounds.

There is a naive phase space probability distribution satisfying requirements (3.5.53), (3.5.52) of the fifth postulate, which follows from classical to quantum
extrapolation from the expression $(3.5 .44)$ of $\varpi_{c}$, namely
$$
\begin{equation*}
\varpi(t, \boldsymbol{x}, \boldsymbol{y})=\rho(t, \boldsymbol{x}) \delta(\boldsymbol{y}-\boldsymbol{\nabla} S(t, \boldsymbol{x})) . \tag{3.5.55}
\end{equation*}
$$
which happens to be non negative. Indeed, comparing (3.5.44), (3.5.55), it is apparent that $\varpi$ is totally analogous to $\varpi_{c}$ in its mathematical form, the former being obtained from the latter by substituting the fields $\rho_{c}, S_{c}$ with their quantum counterparts $\rho, S$. Since $\rho, S$ reduce to $\rho_{c}, S_{c}$ in the semiclassical regime $\hbar \rightarrow 0, \varpi$ converges to $\varpi_{c}$ in the same limit, leading to (3.5.52). Further, the same arguments we used to prove that $\varpi_{c}$ is a phase space probability distribution satisfying $(3.5 .47),(3.5 .48)$ formally apply to show that $\varpi$ is a phase space probability distribution satisfying (3.5.53), (3.5.54). However, the expression of the quantum mean of a general mechanical quantity obtained in this way is precisely (3.5.21), which as we have seen is not compatible with the fourth postulate. We therefore need to find an alternative expression of $\varpi$ that is consistent with this latter.

A priori, it is not clear that a solution of this problem exists and, in case it does, that the solution is unique. The most successful proposals has been put forward by Wigner (E. P. Wigner, 1932) Wigner quasiprobability distribution reads
$$
\begin{equation*}
\varpi(t, \boldsymbol{x}, \boldsymbol{y})=\int \frac{d^{3} u}{(2 \pi \hbar)^{3}} \exp (i \boldsymbol{u} \cdot \boldsymbol{y} / \hbar) \psi^{*}(t, \boldsymbol{x}+\boldsymbol{u} / 2) \psi(t, \boldsymbol{x}-\boldsymbol{u} / 2) \tag{3.5.56}
\end{equation*}
$$
where $\boldsymbol{u}$ is a configuration space variable. It is not possible to justify Wigner's formula (3.5.56) a priori in simple intuitive terms. It is possible instead validate it a posteriori. The expression is designed in such a way to have the properties which one wants it to enjoy. To show this requires however a detailed technical analysis. This notwithstanding, a few preliminary remarks can already be made. To begin with, we observe that $\varpi$ is linear in both $\psi$ and $\psi^{*}$. Therefore, $\varpi$ is bound to yield a result which is linear in $\psi$ and $\psi^{*}$ and possibly their derivatives
in any relation where it enters linearly, such as those which we are concerned with. For the same reason, on account of relation (3.5.26), it also has a good chance to reproduce the density $\rho$ upon $\boldsymbol{y}$ integration. Due to (3.4.13), it may also reproduce the classical distribution $\varpi_{c}$ in the semiclassical limit $\hbar \rightarrow 0$, since for almost vanishing $\hbar$ the integration is dominated by small values of $\boldsymbol{u}$ allowing to approximate the phase of the integrand by $-\boldsymbol{u} \cdot \nabla S_{c}(\boldsymbol{x})$.
$\varpi$ is a quasiprobability distribution.

Proof. We divide the proof in a number of steps for the sake of clarity.
Step 1. The following analysis uses an integral representation of the Dirac delta function that is extensively used in wave mechanics. This is perhaps the right moment where to prove it.

For a fixed vector $\boldsymbol{w}$, we have
$$
\begin{equation*}
\delta(\boldsymbol{w})=\int \frac{d^{3} v}{(2 \pi)^{3}} \exp (i \boldsymbol{w} \cdot \boldsymbol{v}) \tag{3.5.57}
\end{equation*}
$$

To show (3.5.57), consider the integral
$$
\begin{equation*}
\bar{\delta}(\boldsymbol{w})=\int \frac{d^{3} v}{(2 \pi)^{3}} \exp (i \boldsymbol{w} \cdot \boldsymbol{v}) \tag{3.5.58}
\end{equation*}
$$

As it is, this integral is not absolutely convergent, but only conditionally so. Its value can be computed introducing a small "cut-off" $\sigma$ which makes the integration absolutely convergent and then removing the cut-off by taking the limit as $\sigma \rightarrow 0$,
$$
\begin{equation*}
\bar{\delta}(\boldsymbol{w})=\lim _{\sigma \rightarrow 0} \int \frac{d^{3} v}{(2 \pi)^{3}} \exp \left(i \boldsymbol{w} \cdot \boldsymbol{v}-\frac{\sigma^{2} \boldsymbol{v}^{2}}{2}\right) \tag{3.5.59}
\end{equation*}
$$

Here, there appears the integral
$$
\begin{equation*}
J(\boldsymbol{z}, \zeta)=\int \frac{d^{3} v}{(2 \pi)^{3}} \exp \left(\boldsymbol{z} \cdot \boldsymbol{v}-\frac{\zeta^{2} \boldsymbol{v}^{2}}{2}\right) \tag{3.5.60}
\end{equation*}
$$
where $\zeta$ is a complex parameter and $\boldsymbol{z}$ is a complex vector. As long as $\operatorname{Re} \zeta^{2}>0$, $J(\boldsymbol{z}, \zeta)$ is absolutely convergent and analytic in $\boldsymbol{z}$ and $\zeta$. To compute $J(\boldsymbol{z}, \zeta)$, we first conveniently restrict to real $\boldsymbol{z}$ and $\zeta$ and then extend the expression found to complex $\boldsymbol{z}$ and $\zeta$ using analiticity. Completing the square in the exponential turns $J(\boldsymbol{z}, \zeta)$ into
a Gaussian integral,
$$
\begin{align*}
J(\boldsymbol{z}, \zeta)=\int & \frac{d^{3} v}{(2 \pi)^{3}} \exp \left[-\frac{\zeta^{2}}{2}\left(\boldsymbol{v}-\frac{\boldsymbol{z}}{\zeta^{2}}\right)^{2}+\frac{\boldsymbol{z}^{2}}{2 \zeta^{2}}\right]  \tag{3.5.61}\\
& =\int \frac{d^{3} v^{\prime}}{(2 \pi)^{3}} \exp \left(-\frac{\zeta^{2} \boldsymbol{v}^{\prime 2}}{2}+\frac{\boldsymbol{z}^{2}}{2 \zeta^{2}}\right)=\frac{1}{(2 \pi)^{3 / 2} \zeta^{3}} \exp \left(\frac{\boldsymbol{z}^{2}}{2 \zeta^{2}}\right)
\end{align*}
$$
where we have set $\boldsymbol{v}^{\prime}=\boldsymbol{v}-\boldsymbol{z} / \zeta^{2}$ and used the well-known Gaussian integral formula $\int d^{3} v \exp \left(-\alpha \boldsymbol{v}^{2}\right)=(\pi / \alpha)^{3 / 2}$. Using (3.5.61) into (3.5.59), we get
$$
\begin{equation*}
\bar{\delta}(\boldsymbol{w})=\lim _{\sigma \rightarrow 0} J(i \boldsymbol{w}, \sigma)=\lim _{\sigma \rightarrow 0} \frac{1}{(2 \pi)^{3 / 2} \sigma^{3}} \exp \left(-\frac{\boldsymbol{w}^{2}}{2 \sigma^{2}}\right) \tag{3.5.62}
\end{equation*}
$$

Clearly, $\bar{\delta}(\boldsymbol{w})=\infty$ for $\boldsymbol{w}=\mathbf{0}$ and $\bar{\delta}(\boldsymbol{w})=0$ for $\boldsymbol{w} \neq \mathbf{0}$. Furthermore, for any regular function $f(\boldsymbol{w})$, we have
$$
\begin{align*}
\int d^{3} w^{\prime} \bar{\delta}\left(\boldsymbol{w}^{\prime}-\boldsymbol{w}\right) f\left(\boldsymbol{w}^{\prime}\right)=\lim _{\sigma \rightarrow 0} & \int \frac{d^{3} w^{\prime}}{(2 \pi)^{3 / 2} \sigma^{3}} \exp \left(-\frac{\left(\boldsymbol{w}^{\prime}-\boldsymbol{w}\right)^{2}}{2 \sigma^{2}}\right) f\left(\boldsymbol{w}^{\prime}\right)  \tag{3.5.63}\\
& =\lim _{\sigma \rightarrow 0} \int \frac{d^{3} \rho}{(2 \pi)^{3 / 2}} \exp \left(-\frac{\rho^{2}}{2}\right) f(\boldsymbol{w}+\sigma \boldsymbol{\rho})=f(\boldsymbol{w})
\end{align*}
$$
where we have set $\boldsymbol{\rho}=\left(\boldsymbol{w}^{\prime}-\boldsymbol{w}\right) / \sigma$. We conclude that $\bar{\delta}(\boldsymbol{w})=\delta(\boldsymbol{w})$, the $\boldsymbol{w}$ space Dirac delta function obtaining (3.5.57).

There exists a slightly more general version of (3.5.57), viz
$$
\begin{equation*}
\delta(\boldsymbol{w})=\int \frac{|\xi|^{3} d^{3} v}{(2 \pi)^{3}} \exp (i \xi \boldsymbol{w} \cdot \boldsymbol{v}) \tag{3.5.64}
\end{equation*}
$$
where $\xi \neq 0$ is any real number. Indeed, setting $\boldsymbol{v}^{\prime}=\xi \boldsymbol{v}$ we find
$$
\begin{equation*}
\int \frac{|\xi|^{3} d^{3} v}{(2 \pi)^{3}} \exp (i \xi \boldsymbol{w} \cdot \boldsymbol{v})=\int \frac{d^{3} v^{\prime}}{(2 \pi)^{3}} \exp \left(i \boldsymbol{w} \cdot \boldsymbol{v}^{\prime}\right)=\delta(\boldsymbol{w}) \tag{3.5.65}
\end{equation*}
$$

Step 2. We show next that $\varpi$ is a real valued function. Indeed
$$
\begin{align*}
\varpi(\boldsymbol{x}, \boldsymbol{y})^{*} & =\left[\int \frac{d^{3} u}{(2 \pi \hbar)^{3}} \exp (i \boldsymbol{u} \cdot \boldsymbol{y} / \hbar) \psi^{*}(\boldsymbol{x}+\boldsymbol{u} / 2) \psi(\boldsymbol{x}-\boldsymbol{u} / 2)\right]^{*}  \tag{3.5.66}\\
& =\int \frac{d^{3} u}{(2 \pi \hbar)^{3}} \exp (-i \boldsymbol{u} \cdot \boldsymbol{y} / \hbar) \psi(\boldsymbol{x}+\boldsymbol{u} / 2) \psi^{*}(\boldsymbol{x}-\boldsymbol{u} / 2) \\
& =\int \frac{d^{3} u^{\prime}}{(2 \pi \hbar)^{3}} \exp \left(i \boldsymbol{u}^{\prime} \cdot \boldsymbol{y} / \hbar\right) \psi^{*}\left(\boldsymbol{x}+\boldsymbol{u}^{\prime} / 2\right) \psi\left(\boldsymbol{x}-\boldsymbol{u}^{\prime} / 2\right)=\varpi(\boldsymbol{x}, \boldsymbol{y})
\end{align*}
$$
where we left $t$ dependence understood and set $\boldsymbol{u}^{\prime}=-\boldsymbol{u}$.
Step 3. We show finally that $\varpi$ is normalized. We exploit here the integral representation of the configuration space Dirac delta function provided by (3.5.64),
$$
\begin{equation*}
\delta(\boldsymbol{x})=\int \frac{d^{3} y}{(2 \pi \hbar)^{3}} \exp (i \boldsymbol{x} \cdot \boldsymbol{y} / \hbar) \tag{3.5.67}
\end{equation*}
$$

Using (3.5.67) and leaving $t$ dependence understood again, (3.5.56) gives
$$
\begin{array}{r}
\int d^{3} y \varpi(\boldsymbol{x}, \boldsymbol{y})=\int d^{3} u\left[\frac{d^{3} y}{(2 \pi \hbar)^{3}} \exp (i \boldsymbol{u} \cdot \boldsymbol{y} / \hbar)\right] \psi^{*}(\boldsymbol{x}+\boldsymbol{u} / 2) \psi(\boldsymbol{x}-\boldsymbol{u} / 2)  \tag{3.5.68}\\
=\int d^{3} u \delta(\boldsymbol{u}) \psi^{*}(\boldsymbol{x}+\boldsymbol{u} / 2) \psi(\boldsymbol{x}-\boldsymbol{u} / 2)=\psi^{*}(\boldsymbol{x}) \psi(\boldsymbol{x})=|\psi(\boldsymbol{x})|^{2}
\end{array}
$$

From here, by the normalization relation (3.5.32) obeyed by the wave function $\psi$,
$$
\begin{equation*}
\int d^{3} x d^{3} y \varpi(\boldsymbol{x}, \boldsymbol{y})=\int d^{3} x|\psi(\boldsymbol{x})|^{2}=1 \tag{3.5.69}
\end{equation*}
$$
as required.

Wigner's quasiprobability distribution $\varpi$ satisfies the semiclassical requirement (3.5.52) and the reduction relation (3.5.53).

Proof. Showing the semiclassical requirement (3.5.52) requires some work. Setting $\boldsymbol{\xi}=\boldsymbol{u} / \hbar$ in (3.5.56) and leaving $t$ dependence understood as before, we find
$$
\begin{equation*}
\varpi(\boldsymbol{x}, \boldsymbol{y})=\int \frac{d^{3} \xi}{(2 \pi)^{3}} \exp (i \boldsymbol{\xi} \cdot \boldsymbol{y}) \psi^{*}(\boldsymbol{x}+\hbar \boldsymbol{\xi} / 2) \psi(\boldsymbol{x}-\hbar \boldsymbol{\xi} / 2) \tag{3.5.70}
\end{equation*}
$$

We analyze the integrand of (3.5.70). By (3.4.13), we have
$$
\begin{align*}
& \psi^{*}(\boldsymbol{x}+\hbar \boldsymbol{\xi} / 2) \psi(\boldsymbol{x}-\hbar \boldsymbol{\xi} / 2)  \tag{3.5.71}\\
&=\rho^{1 / 2}(\boldsymbol{x}+\hbar \boldsymbol{\xi} / 2) \rho^{1 / 2}(\boldsymbol{x}-\hbar \boldsymbol{\xi} / 2) \exp (-i S(\boldsymbol{x}+\hbar \boldsymbol{\xi} / 2) / \hbar+i S(\boldsymbol{x}-\hbar \boldsymbol{\xi} / 2) / \hbar)
\end{align*}
$$

By expanding the various terms in powers of $\hbar$, we have
$$
\begin{equation*}
\rho^{1 / 2}(\boldsymbol{x} \pm \hbar \boldsymbol{\xi} / 2)=\rho^{1 / 2}(\boldsymbol{x})+O(\hbar)=\rho_{c}^{1 / 2}(\boldsymbol{x})+O(\hbar) \tag{3.5.72}
\end{equation*}
$$
and analogously
$$
\begin{align*}
-i S(\boldsymbol{x}+\hbar \boldsymbol{\xi} / 2) / \hbar & +i S(\boldsymbol{x}-\hbar \boldsymbol{\xi} / 2) / \hbar  \tag{3.5.73}\\
& =-i S(\boldsymbol{x}) / \hbar-i \boldsymbol{\xi} \cdot \boldsymbol{\nabla} S(\boldsymbol{x}) / 2+i S(\boldsymbol{x}) / \hbar-i \boldsymbol{\xi} \cdot \nabla S(\boldsymbol{x}) / 2+O(\hbar) \\
& =-i \boldsymbol{\xi} \cdot \boldsymbol{\nabla} S_{c}(\boldsymbol{x})+O(\hbar)
\end{align*}
$$

In the last step, we used that in the formal limit $\hbar \rightarrow 0 \rho$ and $S$ reduce to their classical counterparts $\rho_{c}$ and $S_{c}$, respectively. It follows that
$$
\begin{equation*}
\psi^{*}(\boldsymbol{x}+\hbar \boldsymbol{\xi} / 2) \psi(\boldsymbol{x}-\hbar \boldsymbol{\xi} / 2)=\rho_{c}(\boldsymbol{x}) \exp \left(-i \boldsymbol{\xi} \cdot \boldsymbol{\nabla} S_{c}(\boldsymbol{x})\right)+O(\hbar) \tag{3.5.74}
\end{equation*}
$$

Inserting this expansion into (3.5.70) and recalling (3.5.44), we find
$$
\begin{align*}
\varpi(\boldsymbol{x}, \boldsymbol{y}) & =\rho_{c}(\boldsymbol{x}) \int \frac{d^{3} \xi}{(2 \pi)^{3}} \exp \left(i \boldsymbol{\xi} \cdot\left(\boldsymbol{y}-\boldsymbol{\nabla} S_{c}(\boldsymbol{x})\right)\right)+O(\hbar)  \tag{3.5.75}\\
& =\rho_{c}(\boldsymbol{x}) \delta\left(\boldsymbol{y}-\boldsymbol{\nabla} S_{c}(\boldsymbol{x})\right)+O(\hbar)=\varpi_{c}(\boldsymbol{x}, \boldsymbol{y})+O(\hbar)
\end{align*}
$$
as was to be shown.
(3.5.53) is an immediate consequence of relations (3.5.68) obtained earlier and $(3.5 .26)$.

Wigner's quasiprobability distribution can possibly take negative values. However, it must satisfy the bound
$$
\begin{equation*}
|\varpi(t, \boldsymbol{x}, \boldsymbol{y})| \leq 1 /(\pi \hbar)^{3} \tag{3.5.76}
\end{equation*}
$$

Proof. The Cauchy-Schwarz inequality $\left|\int d^{3} x \phi^{*} \chi\right| \leq\left(\int d^{3} x|\phi|^{2}\right)^{1 / 2}\left(\int d^{3} x|\chi|^{2}\right)^{1 / 2}$ is the key to the proof. With $t$ dependence understood as usual, we have
$$
\begin{align*}
|\varpi(\boldsymbol{x}, \boldsymbol{y})|= & \left\lvert\, \int \frac{d^{3} u^{\prime}}{(\pi \hbar)^{3}} \exp \left(i\left(\boldsymbol{x}+\boldsymbol{u}^{\prime}\right) \cdot \boldsymbol{y} / \hbar\right) \psi^{*}\left(\boldsymbol{x}+\boldsymbol{u}^{\prime}\right)\right.  \tag{3.5.77}\\
\leq & \exp \left(-i\left(\boldsymbol{x}-\boldsymbol{u}^{\prime}\right) \cdot \boldsymbol{y} / \hbar\right) \psi\left(\boldsymbol{x}-\boldsymbol{u}^{\prime}\right) \mid \\
& \frac{1}{(\pi \hbar)^{3}}\left(\int d^{3} u^{\prime}\left|\exp (-i(\boldsymbol{x}+\boldsymbol{u}) \cdot \boldsymbol{y} / \hbar) \psi\left(\boldsymbol{x}+\boldsymbol{u}^{\prime}\right)\right|^{2}\right)^{1 / 2} \\
& \quad \times\left(\int d^{3} u^{\prime}\left|\exp (-i(\boldsymbol{x}-\boldsymbol{u}) \cdot \boldsymbol{y} / \hbar) \psi\left(\boldsymbol{x}-\boldsymbol{u}^{\prime}\right)\right|^{2}\right)^{1 / 2} \\
= & \frac{1}{(\pi \hbar)^{3}}\left(\int d^{3} u^{\prime \prime}\left|\psi\left(\boldsymbol{u}^{\prime \prime}\right)\right|^{2}\right)^{1 / 2}\left(\int d^{3} u^{\prime \prime \prime}\left|\psi\left(\boldsymbol{u}^{\prime \prime \prime}\right)\right|^{2}\right)^{1 / 2}=\frac{1}{(\pi \hbar)^{3}}
\end{align*}
$$
where we set successively $\boldsymbol{u}^{\prime}=\boldsymbol{u} / 2$ and $\boldsymbol{u}^{\prime \prime}=\boldsymbol{u}^{\prime}+\boldsymbol{x}, \boldsymbol{u}^{\prime \prime \prime}=-\boldsymbol{u}^{\prime}+\boldsymbol{x}$ and used the normlaization condition (3.5.32) in the last step.

The occurrence of negative values for Wigner's quasiprobability distribution can be verified in specific examples. We consider here an ideal system, a one dimensional harmonic oscillator, to allow for a graphic representation of the distribution.
Fig. 3.5.1 shows the plot of Wigner's quasiprobability distributions $\varpi_{0}, \varpi_{1}$ for the the wave functions $\psi_{0}, \psi_{1}$ describing the two lowest energy stationary states of the ensemble of the oscillator.

The expression (3.5.56) of Wigner's quasiprobability distribution is sufficiently explicit to allow for obtaining a general expression of a quantum mean $\left\langle Q_{f}\right\rangle$,

![](https://cdn.mathpix.com/cropped/2024_09_22_5d1e855547710648961eg-0254.jpg?height=887&width=659&top_left_y=1080&top_left_x=706)

Figure 3.5.1. Wigner's quasiprobability distributions $\varpi_{0}, \varpi_{1}$ for the the wave functions $\psi_{0}, \psi_{1}$ describing the lowest energy stationary states of the ensemble of a one dimensional harmonic oscillator of mass $m$ and frequency $\omega$. Above, $\tilde{x}=x / \ell, \tilde{y}=\ell y / \hbar$ where $\ell=(\hbar / m \omega)^{1 / 2}$ and $\tilde{\varpi}_{i}=\pi \hbar \varpi_{i} . \varpi_{0}$ is non negative while $\varpi_{1}$ can take values of both signs.
namely
$$
\begin{equation*}
\left\langle Q_{f}\right\rangle=\int d^{3} x \psi^{*} \mathrm{Q}_{f} \psi \tag{3.5.78}
\end{equation*}
$$
where $\mathrm{Q}_{f} \psi$ is the wave function
$$
\begin{equation*}
\mathrm{Q}_{f} \psi(t, \boldsymbol{x})=\int d^{3} u K_{f}(\boldsymbol{x}, \boldsymbol{u}) \psi(t, \boldsymbol{u}) \tag{3.5.79}
\end{equation*}
$$
the integral kernel $K_{f}$ in the right hand side being given by
$$
\begin{equation*}
K_{f}(\boldsymbol{x}, \boldsymbol{u})=\int \frac{d^{3} y}{(2 \pi \hbar)^{3}} \exp (i(\boldsymbol{x}-\boldsymbol{u}) \cdot \boldsymbol{y} / \hbar) f((\boldsymbol{x}+\boldsymbol{u}) / 2, \boldsymbol{y}) \tag{3.5.80}
\end{equation*}
$$
$K_{f}$ is called the Weyl-Wigner transform of the phase function $f$ (H. Weyl, 1927).

Proof.We compute $\left\langle Q_{f}\right\rangle$ inserting expression (3.5.56) of Wigner's quasiprobability distribution into the general relation (3.5.54),
$$
\begin{align*}
\left\langle Q_{f}\right\rangle & =\int \frac{d^{3} x d^{3} y d^{3} u}{(2 \pi \hbar)^{3}} f(\boldsymbol{x}, \boldsymbol{y}) \exp (i \boldsymbol{u} \cdot \boldsymbol{y} / \hbar) \psi^{*}(\boldsymbol{x}+\boldsymbol{u} / 2) \psi(\boldsymbol{x}-\boldsymbol{u} / 2)  \tag{3.5.81}\\
& =\int \frac{d^{3} x^{\prime} d^{3} u d^{3} y}{(2 \pi \hbar)^{3}} f\left(\boldsymbol{x}^{\prime}-\boldsymbol{u} / 2, \boldsymbol{y}\right) \psi^{*}\left(\boldsymbol{x}^{\prime}\right) \exp (i \boldsymbol{u} \cdot \boldsymbol{y} / \hbar) \psi\left(\boldsymbol{x}^{\prime}-\boldsymbol{u}\right) \\
& =\int \frac{d^{3} x^{\prime} d^{3} u^{\prime} d^{3} y}{(2 \pi \hbar)^{3}} \psi^{*}\left(\boldsymbol{x}^{\prime}\right) \exp \left(i\left(\boldsymbol{x}^{\prime}-\boldsymbol{u}^{\prime}\right) \cdot \boldsymbol{y} / \hbar\right) f\left(\left(\boldsymbol{x}^{\prime}+\boldsymbol{u}^{\prime}\right) / 2, \boldsymbol{y}\right) \psi\left(\boldsymbol{u}^{\prime}\right) \\
& =\int d^{3} x^{\prime} \psi^{*}\left(\boldsymbol{x}^{\prime}\right) \int d^{3} u^{\prime} K_{f}\left(\boldsymbol{x}^{\prime}, \boldsymbol{u}^{\prime}\right) \psi\left(\boldsymbol{u}^{\prime}\right)=\int d^{3} x^{\prime} \psi^{*}\left(\boldsymbol{x}^{\prime}\right) \mathrm{Q}_{f} \psi\left(\boldsymbol{x}^{\prime}\right)
\end{align*}
$$
where we set first replaced $\boldsymbol{x}$ by $\boldsymbol{x}^{\prime}$ defined by $\boldsymbol{x}=\boldsymbol{x}^{\prime}-\boldsymbol{u} / 2$ and then $\boldsymbol{u}$ by $\boldsymbol{u}^{\prime}$ defined by $\boldsymbol{u}=\boldsymbol{x}^{\prime}-\boldsymbol{u}^{\prime}$. (3.5.78)-(3.5.80) are so verified.

By (3.5.86), (3.5.80), we can express the quantum mean position, momentum, angular momentum, energy of a copy, $\langle\boldsymbol{q}\rangle,\langle\boldsymbol{p}\rangle,\langle\boldsymbol{l}\rangle,\langle H\rangle$ in terms of $\psi$,
$$
\begin{align*}
\langle\boldsymbol{q}\rangle & =\int d^{3} x \psi^{*} \boldsymbol{x} \psi  \tag{3.5.82}\\
\langle\boldsymbol{p}\rangle & =\int d^{3} x \psi^{*}(-i \hbar \boldsymbol{\nabla} \psi) \tag{3.5.83}
\end{align*}
$$
$$
\begin{align*}
& \langle\boldsymbol{l}\rangle=\int d^{3} x \psi^{*}(-i \hbar \boldsymbol{x} \times \boldsymbol{\nabla} \psi)  \tag{3.5.84}\\
& \langle H\rangle=\int d^{3} x \psi^{*}\left(-\frac{\hbar^{2}}{2 m} \boldsymbol{\nabla}^{2} \psi+U \psi\right) \tag{3.5.85}
\end{align*}
$$

These relations are clearly compatible with the fourth postulate.

Proof. We divide the proof in a number of steps for the sake of clarity.
Step 1. Let $f$ be any phase function. Inserting (3.5.79) into (3.5.78)
$$
\begin{equation*}
\left\langle Q_{f}\right\rangle=\int d^{3} x d^{3} u \psi^{*}(\boldsymbol{x}) K_{f}(\boldsymbol{x}, \boldsymbol{u}) \psi(\boldsymbol{u}) \tag{3.5.86}
\end{equation*}
$$
where we left $t$ dependence understood. Now, we are going to calculate the kernel $K_{f}$ for a special choice of $f$, namely
$$
\begin{equation*}
f(\boldsymbol{x}, \boldsymbol{y})=g(\boldsymbol{x}) h(\boldsymbol{y}) \tag{3.5.87}
\end{equation*}
$$
where $g$ is arbitrary while $h$ is required to be polynomial for reasons which will become clear momentarily. Using (3.5.80) and exploiting (3.5.64) and noticing that $\boldsymbol{y} \exp (-i \boldsymbol{u} \cdot \boldsymbol{y} / \hbar)=i \hbar \boldsymbol{\nabla}_{\boldsymbol{u}} \exp (-i \boldsymbol{u} \cdot \boldsymbol{y} / \hbar)$, we have
$$
\begin{align*}
K_{f}(\boldsymbol{x}, \boldsymbol{u}) & =\int \frac{d^{3} y}{(2 \pi \hbar)^{3}} \exp (i(\boldsymbol{x}-\boldsymbol{u}) \cdot \boldsymbol{y} / \hbar) g((\boldsymbol{x}+\boldsymbol{u}) / 2) h(\boldsymbol{y})  \tag{3.5.88}\\
& =\int \frac{d^{3} y}{(2 \pi \hbar)^{3}} h\left(i \hbar \boldsymbol{\nabla}_{\boldsymbol{u}}\right) \exp (i(\boldsymbol{x}-\boldsymbol{u}) \cdot \boldsymbol{y} / \hbar) g((\boldsymbol{x}+\boldsymbol{u}) / 2) \\
& =h\left(i \hbar \boldsymbol{\nabla}_{\boldsymbol{u}}\right) \delta(\boldsymbol{u}-\boldsymbol{x}) g((\boldsymbol{x}+\boldsymbol{u}) / 2)
\end{align*}
$$

Inserting (3.5.88) into the general mean expression (3.5.86), we find
$$
\begin{align*}
\left\langle Q_{f}\right\rangle & =\int d^{3} x \int d^{3} u \psi^{*}(\boldsymbol{x}) h\left(i \hbar \boldsymbol{\nabla}_{\boldsymbol{u}}\right) \delta(\boldsymbol{u}-\boldsymbol{x}) g((\boldsymbol{x}+\boldsymbol{u}) / 2) \psi(\boldsymbol{u})  \tag{3.5.89}\\
& =\int d^{3} x \int d^{3} u \psi^{*}(\boldsymbol{x}) \delta(\boldsymbol{u}-\boldsymbol{x}) h\left(-i \hbar \boldsymbol{\nabla}_{\boldsymbol{u}}\right)[g((\boldsymbol{x}+\boldsymbol{u}) / 2) \psi(\boldsymbol{u})] \\
& =\int d^{3} x \psi^{*}(\boldsymbol{x}) h\left(-i \hbar \boldsymbol{\nabla}_{\boldsymbol{u}}\right)[g((\boldsymbol{x}+\boldsymbol{u}) / 2) \psi(\boldsymbol{u})]_{\boldsymbol{u}=\boldsymbol{x}}
\end{align*}
$$
where in passing from the first to the second line we performed integrations by parts transforming the differential polynomial $h\left(i \hbar \boldsymbol{\nabla}_{\boldsymbol{u}}\right)$ into $h\left(-i \hbar \boldsymbol{\nabla}_{\boldsymbol{u}}\right)$.

Step 2. We now apply formula (3.5.89) to some relevant special cases. Suppose first that $g(\boldsymbol{x})$ is a component of $\boldsymbol{x}$ and $h(\boldsymbol{y})=1$. Then,
$$
\begin{equation*}
\langle\boldsymbol{q}\rangle=\int d^{3} x \psi^{*}(\boldsymbol{x})[(\boldsymbol{x}+\boldsymbol{u}) / 2 \psi(\boldsymbol{u})]_{\boldsymbol{u}=\boldsymbol{x}}=\int d^{3} x \psi^{*}(\boldsymbol{x}) \boldsymbol{x} \psi(\boldsymbol{x}) \tag{3.5.90}
\end{equation*}
$$
proving (3.5.82). Next, suppose that $g(\boldsymbol{x})=1$ and $h(\boldsymbol{y})$ is a component of $\boldsymbol{y}$. Then,
$$
\begin{equation*}
\langle\boldsymbol{p}\rangle=\int d^{3} x \psi^{*}(\boldsymbol{x})\left(-i \hbar \boldsymbol{\nabla}_{\boldsymbol{u}}\right) \psi(\boldsymbol{u})_{\boldsymbol{u}=\boldsymbol{x}}=\int d^{3} x \psi^{*}(\boldsymbol{x})(-i \hbar) \boldsymbol{\nabla}_{\boldsymbol{x}} \psi(\boldsymbol{x}) \tag{3.5.91}
\end{equation*}
$$
showing (3.5.83). Next, suppose that $g(\boldsymbol{x})$ and $h(\boldsymbol{y})$ are a component of $\boldsymbol{x}$ and $\boldsymbol{y}$, respectively. Using dyadic notation, we find
$$
\begin{align*}
\langle\boldsymbol{q} \boldsymbol{p}\rangle & =\int d^{3} x \psi^{*}(\boldsymbol{x}) \frac{1}{2}\left\{\left(-i \hbar \boldsymbol{\nabla}_{\boldsymbol{u}}\right)[(\boldsymbol{x}+\boldsymbol{u}) \psi(\boldsymbol{u})]_{\boldsymbol{u}=\boldsymbol{x}}\right\}^{t}  \tag{3.5.92}\\
& =\int d^{3} x \psi^{*}(\boldsymbol{x}) \frac{1}{2}\left[-i \hbar \mathbf{1} \psi(\boldsymbol{u})+(\boldsymbol{x}+\boldsymbol{u})\left(-i \hbar \boldsymbol{\nabla}_{\boldsymbol{u}}\right) \psi(\boldsymbol{u})\right]_{\boldsymbol{u}=\boldsymbol{x}} \\
& =\int d^{3} x \psi^{*}(\boldsymbol{x})\left[-\frac{i}{2} \hbar \mathbf{1} \psi(\boldsymbol{x})+\left(-i \hbar \boldsymbol{x} \boldsymbol{\nabla}_{\boldsymbol{x}}\right) \psi(\boldsymbol{x})\right]
\end{align*}
$$

Recalling that the unit dyad $\mathbf{1}$ is symmetric and therefore vanishes under antisymmetrization, we obtain then
$$
\begin{equation*}
\langle\boldsymbol{l}\rangle=\langle\boldsymbol{q} \times \boldsymbol{p}\rangle=\int d^{3} x \psi^{*}(\boldsymbol{x})(-i \hbar) \boldsymbol{x} \times \boldsymbol{\nabla}_{\boldsymbol{x}} \psi(\boldsymbol{x}) \tag{3.5.93}
\end{equation*}
$$
proving (3.5.84). Finally, we consider the cases where $g(\boldsymbol{x})=1$ and $h(\boldsymbol{y})=\boldsymbol{y}^{2} / 2$ and $g(\boldsymbol{x})=U(\boldsymbol{x})$ and $h(\boldsymbol{y})=1$ adding which we get the Hamilton function $H(\boldsymbol{x}, \boldsymbol{y})$.
$$
\begin{align*}
\langle H\rangle & =\int d^{3} x \psi^{*}(\boldsymbol{x})\left[\frac{1}{2 m}\left(-i \hbar \boldsymbol{\nabla}_{\boldsymbol{u}}\right)^{2} \psi(\boldsymbol{u})+U((\boldsymbol{x}+\boldsymbol{u}) / 2) \psi(\boldsymbol{u})\right]_{\boldsymbol{u}=\boldsymbol{x}}  \tag{3.5.94}\\
& =\int d^{3} x \psi^{*}(\boldsymbol{x})\left[-\frac{\hbar^{2}}{2 m} \boldsymbol{\nabla}_{\boldsymbol{x}}{ }^{2} \psi(\boldsymbol{x})+U(\boldsymbol{x}) \psi(\boldsymbol{x})\right]
\end{align*}
$$
proving (3.5.85).

\subsection*{3.6. Mathematical modelling of the ensemble}

When we view the ensemble of a quantum particle as a physical system, we have to provide a description of it in terms of states and observables. In turn, these latter require an appropriate mathematical modellization.

The states of the ensemble are its statistical modes as a collection of copies of the particle. The observables of the ensembles are the probability distributions and mean values of the main mechanical quantities.

The normalized wave function $\psi$ determines the probability density and current density $\rho, \boldsymbol{j}$ of the ensemble and the mean values of all quantities $Q_{f}$ associated with phase functions $f$ of a copy of the particle in the ensemble via (3.5.26), (3.5.27) and (3.5.78)-(3.5.80). Since all statistical quantities characterizing the ensemble are ultimately specified by $\psi$, we are led to the following conclusion.

The state of the ensemble is encoded by a normalized wave functions $\psi$.
Inspecting (3.5.78), it appears that the mean $\left\langle Q_{f}\right\rangle$ of $Q_{f}$ is naturally expressed through the wave function $\mathrm{Q}_{f} \psi$ given by (3.5.79). $\mathrm{Q}_{f}$ must be thought of as an operation assigning to any given wave function $\phi$, not necessarily satisfying the Schroedinger equation, another wave function $\phi^{\prime}$ determined by $\phi$ according to a fixed prescription and hence denoted as $\mathrm{Q}_{f} \phi$. Relation (3.5.79) already provides the expression of the assignment,
$$
\begin{equation*}
\mathrm{Q}_{f} \phi(\boldsymbol{x})=\int d^{3} u K_{f}(\boldsymbol{x}, \boldsymbol{u}) \phi(\boldsymbol{u}) \tag{3.6.1}
\end{equation*}
$$
where the integral kernel $K_{f}$ is the Weyl-Wigner transform of $f$ defined in (3.5.80). $\mathrm{Q}_{f}$ is called the $Q_{f}$ operator since it is associated with the quantity $Q_{f}$. Indeed, $\mathrm{Q}_{f}$ is a Hermitian linear operator. Linearity is the property that
$$
\begin{equation*}
\mathrm{Q}_{f}\left(c_{1} \phi_{1}+c_{2} \phi_{2}\right)=c_{1} \mathrm{Q}_{f} \phi_{1}+c_{2} \mathrm{Q}_{f} \phi_{2} \tag{3.6.2}
\end{equation*}
$$
for any wave functions $\phi_{1}, \phi_{2}$ and complex scalars $c_{1}, c_{2}$. Hermiticity ensures that
$$
\begin{equation*}
\left[\int d^{3} x \phi^{*} \mathrm{Q}_{f} \phi\right]^{*}=\int d^{3} x \phi^{*} \mathrm{Q}_{f} \phi \tag{3.6.3}
\end{equation*}
$$
for any wave function $\phi$.

Proof. Linearity of $\mathrm{Q}_{f}$ is immediately verified. Indeed,
$$
\begin{align*}
\mathrm{Q}_{f}\left(c_{1} \phi_{1}+c_{2} \phi_{2}\right)(\boldsymbol{x}) & =\int d^{3} u K_{f}(\boldsymbol{x}, \boldsymbol{u})\left(c_{1} \phi_{1}(\boldsymbol{u})+c_{2} \phi_{2}(\boldsymbol{u})\right)  \tag{3.6.4}\\
& =c_{1} \int d^{3} u K_{f}(\boldsymbol{x}, \boldsymbol{u}) \phi_{1}(\boldsymbol{u})+c_{2} \int d^{3} u K_{f}(\boldsymbol{x}, \boldsymbol{u}) \phi_{2}(\boldsymbol{u}) \\
& =c_{1} \mathrm{Q}_{f} \phi_{1}(\boldsymbol{x})+c_{2} \mathrm{Q}_{f} \phi_{2}(\boldsymbol{x})
\end{align*}
$$
leading to $(3.6 .2)$.
The kernel $K_{f}$ enjoys the basic property of being Hermitian,
$$
\begin{equation*}
K_{f}(\boldsymbol{u}, \boldsymbol{x})^{*}=K_{f}(\boldsymbol{x}, \boldsymbol{u}) \tag{3.6.5}
\end{equation*}
$$

Indeed, using the defining relation (3.5.80), we have
$$
\begin{align*}
K_{f}(\boldsymbol{u}, \boldsymbol{x})^{*} & =\left[\int \frac{d^{3} y}{(2 \pi \hbar)^{3}} \exp (i(\boldsymbol{u}-\boldsymbol{x}) \cdot \boldsymbol{y} / \hbar) f((\boldsymbol{u}+\boldsymbol{x}) / 2, \boldsymbol{y})\right]^{*}  \tag{3.6.6}\\
& =\int \frac{d^{3} y}{(2 \pi \hbar)^{3}} \exp (i(\boldsymbol{x}-\boldsymbol{u}) \cdot \boldsymbol{y} / \hbar) f((\boldsymbol{x}+\boldsymbol{u}) / 2, \boldsymbol{y})=K_{f}(\boldsymbol{x}, \boldsymbol{u})
\end{align*}
$$
as claimed above. Using (3.6.5), we find
$$
\begin{align*}
{\left[\int d^{3} x \phi^{*} \mathrm{Q}_{f} \phi\right]^{*} } & =\left[\int d^{3} x \int d^{3} u \phi^{*}(\boldsymbol{x}) K_{f}(\boldsymbol{x}, \boldsymbol{u}) \phi(\boldsymbol{u})\right]^{*}  \tag{3.6.7}\\
& =\int d^{3} x \int d^{3} u \phi(\boldsymbol{x}) K_{f}(\boldsymbol{x}, \boldsymbol{u})^{*} \phi^{*}(\boldsymbol{u}) \\
& =\int d^{3} u \int d^{3} x \phi^{*}(\boldsymbol{u}) K_{f}(\boldsymbol{u}, \boldsymbol{x}) \phi(\boldsymbol{x})=\int d^{3} x \phi^{*} \mathrm{Q}_{f} \phi
\end{align*}
$$
(3.6.3) is so verified.

In view of relation (3.5.78), the Hermiticity of the operator $\mathrm{Q}_{f}$ ensures of the reality of the mean $\left\langle Q_{f}\right\rangle$.

The simple and elegant structure exhibited by eq. (3.5.78) suggests to us a general rule with regard to the mathematical modellization of mechanical quantities in wave mechanics.

For any phase function $f$, the mechanical quantity $Q_{f}$ is represented mathematically by the Hermitian linear operator $\mathrm{Q}_{f}$.

Inspecting the expressions (3.5.82)-(3.5.85) of the means $\langle\boldsymbol{q}\rangle,\langle\boldsymbol{p}\rangle,\langle\boldsymbol{l}\rangle$ and $\langle H\rangle$ of position, momentum, angular momentum and energy of a copy, it is immediate to read off the expressions of the operators $\mathbf{q}, \mathbf{p}, \mathbf{l}$ and $H$,
$$
\begin{align*}
& \mathbf{q}=\boldsymbol{x}  \tag{3.6.8}\\
& \mathbf{p}=-i \hbar \boldsymbol{\nabla}  \tag{3.6.9}\\
& \mathbf{l}=-i \hbar \boldsymbol{x} \times \boldsymbol{\nabla}  \tag{3.6.10}\\
& \mathrm{H}=-\frac{\hbar^{2}}{2 m} \boldsymbol{\nabla}^{2}+U \tag{3.6.11}
\end{align*}
$$

The singular delta-like nature of the Weyl-Wigner transform for these special phase function has yielded straightforward differential operators.

This mechanical quantity/Hermitian operator correspondence $f \mapsto \mathrm{Q}_{f}$ is linear, that is for any two phase functions $f, g$ and real scalars $a, b$,
$$
\begin{equation*}
\mathrm{Q}_{a f+b g}=a \mathrm{Q}_{f}+b \mathrm{Q}_{g} \tag{3.6.12}
\end{equation*}
$$
as it is evident from the linearity of Weyl-Wigner transform $K_{f}$ in $f$, see eq. (3.5.80). Unfortunately, the correspondence does not behave nicely under multiplication of phase functions $f, g$ : there is no simple relation between the operators $\mathrm{Q}_{f g}, \mathrm{Q}_{f}, \mathrm{Q}_{g}$. As it turns out, although Weyl-Wigner transformation provides the correct expressions of the operators of the main quantities, it does not for more complicated ones. For these, the more advanced quantum theory expounded later is required.

\subsection*{3.7. The Moyal product and bracket}

We have seen in sect. 3.6 that with any phase function $f$ we can associate an operator $\mathrm{Q}_{f}$ expressible by means of the Weyl-Wigner transform $K_{f}$ of $f$ given by eq. (3.5.80) through relation (3.6.1). The operators of this form constitute a rather special class of operators. It turns out to be convenient to study operators from a somewhat more general point of view.

In general, an operator is a mapping A that associates with any wave function $\phi$ another wave function $\mathrm{A} \phi$ depending on $\phi$ according to a certain prescription. Here, we shall consider only linear operators. An operator A is linear if it preserves linear combinations of wave functions, i.e.
$$
\begin{equation*}
\mathrm{A}\left(c_{1} \phi_{1}+c_{2} \phi_{2}\right)=c_{1} \mathrm{~A} \phi_{1}+c_{2} \mathrm{~A} \phi_{2} \tag{3.7.1}
\end{equation*}
$$
for any wave functions $\phi_{1}, \phi_{2}$ and complex scalars $c_{1}, c_{2}$. Given linear operators A, B and complex scalars $a, b$, it is possible to construct the operator linear combination $a \mathrm{~A}+b \mathrm{~B}$. This is the linear operator acting on a wave function $\phi$ as
$$
\begin{equation*}
(a \mathrm{~A}+b \mathrm{~B}) \phi=a \mathrm{~A} \phi+b \mathrm{~B} \phi \tag{3.7.2}
\end{equation*}
$$

Given operators $\mathrm{A}, \mathrm{B}$, it is also possible to construct the operator product AB . This is the linear operator whose action on a wave function $\phi$ is given by the sequential action of first B and then A on $\phi$,
$$
\begin{equation*}
(\mathrm{AB}) \phi=\mathrm{A}(\mathrm{B} \phi) \tag{3.7.3}
\end{equation*}
$$

Linear operators form therefore an algebra. This algebra is associative as
$$
\begin{equation*}
(\mathrm{AB}) \mathrm{C}=\mathrm{A}(\mathrm{BC}) \tag{3.7.4}
\end{equation*}
$$
for all operator triples $\mathrm{A}, \mathrm{B}, \mathrm{C}$, but it is not commutative as
$$
\begin{equation*}
(\mathrm{AB}) \neq \mathrm{BA} \tag{3.7.5}
\end{equation*}
$$
for a generic operator pairs $\mathrm{A}, \mathrm{B}$.

As already recalled, the operator $\mathrm{Q}_{f}$ associated with a phase function $f$ is expressible by means of the integral kernel $K_{f}$. Actually, under rather general assumptions, for any linear operator A there is an integral kernel $K_{\mathrm{A}}$ such that
$$
\begin{equation*}
\mathrm{A} \phi(\boldsymbol{x})=\int d^{3} u K_{\mathrm{A}}(\boldsymbol{x}, \boldsymbol{u}) \phi(\boldsymbol{u}) \tag{3.7.6}
\end{equation*}
$$
for any wave function $\phi . K_{\mathrm{A}}$ is uniquely determined by A . In fact
$$
\begin{equation*}
K_{\mathrm{A}}(\boldsymbol{x}, \boldsymbol{u})=\mathrm{A} \epsilon_{\boldsymbol{u}}(\boldsymbol{x}) \tag{3.7.7}
\end{equation*}
$$
where $\epsilon_{\boldsymbol{u}}$ is the wave function localized at the configuration space point $\boldsymbol{u}$,
$$
\begin{equation*}
\epsilon_{\boldsymbol{u}}(\boldsymbol{x})=\delta(\boldsymbol{x}-\boldsymbol{u}) \tag{3.7.8}
\end{equation*}
$$

Proof. Let $\phi$ be any phase function. Then,
$$
\begin{equation*}
\phi(\boldsymbol{x})=\int d^{3} u \delta(\boldsymbol{x}-\boldsymbol{u}) \phi(\boldsymbol{u})=\int d^{3} u \epsilon_{\boldsymbol{u}}(\boldsymbol{x}) \phi(\boldsymbol{u}) \tag{3.7.9}
\end{equation*}
$$
so that $\phi$ can be expressed as a linear combination of the wave functions $\epsilon_{\boldsymbol{u}}$ with coefficients given by the values $\phi(\boldsymbol{u})$. Therefore,
$$
\begin{equation*}
\mathrm{A} \phi(\boldsymbol{x})=\int d^{3} u \mathrm{~A} \epsilon_{\boldsymbol{u}}(\boldsymbol{x}) \phi(\boldsymbol{u})=\int d^{3} u K_{\mathrm{A}}(\boldsymbol{x}, \boldsymbol{u}) \phi(\boldsymbol{u}) \tag{3.7.10}
\end{equation*}
$$
where $K_{\mathrm{A}}(\boldsymbol{x}, \boldsymbol{u})$ is defined by (3.7.7). Hence, the operator A has an integral kernel expression of the form (3.7.6).

Suppose conversely that A can be expressed through some integral kernel $\Lambda$ so that
$$
\begin{equation*}
\mathrm{A} \phi(\boldsymbol{x})=\int d^{3} u \Lambda(\boldsymbol{x}, \boldsymbol{u}) \phi(\boldsymbol{u}) \tag{3.7.11}
\end{equation*}
$$

Then, we have
$$
\begin{align*}
& \Lambda(\boldsymbol{x}, \boldsymbol{u})=\int d^{3} w \Lambda(\boldsymbol{x}, \boldsymbol{w}) \delta(\boldsymbol{w}-\boldsymbol{u})  \tag{3.7.12}\\
& \quad=\int d^{3} w \Lambda(\boldsymbol{x}, \boldsymbol{w}) \epsilon_{\boldsymbol{u}}(\boldsymbol{w})=\mathrm{A} \epsilon_{\boldsymbol{u}}(\boldsymbol{x})=K_{\mathrm{A}}(\boldsymbol{x}, \boldsymbol{u})
\end{align*}
$$

In this way, (3.7.6) is the only integral kernel representation A has.

Conversely, any integral kernel $\Lambda$ determines an operator $\Omega_{\Lambda}$, viz
$$
\begin{equation*}
\Omega_{\Lambda} \phi(\boldsymbol{x})=\int d^{3} u \Lambda(\boldsymbol{x}, \boldsymbol{u}) \phi(\boldsymbol{u}) \tag{3.7.13}
\end{equation*}
$$

The linearity of $\Omega_{\Lambda}$ follows from the linearity of the integral in the right hand side with respect to the wave function $\phi$.

Consistency requires that
$$
\begin{equation*}
\Omega_{K_{\mathrm{A}}}=\mathrm{A} \tag{3.7.14}
\end{equation*}
$$
for any linear operator A and that
$$
\begin{equation*}
K_{\Omega_{\Lambda}}=\Lambda \tag{3.7.15}
\end{equation*}
$$
for any integral kernel $\Lambda$. This is indeed the case.

Proof. $\Omega_{K_{\mathrm{A}}}$ is given by the right hand side of (3.7.13) with $\Lambda=K_{\mathrm{A}}$. The expression of $\Omega_{K_{\mathrm{A}}}$ obtained through this substitution equals the right hand side (3.7.6). Hence, (3.7.14) holds.

By expression (3.7.7), we have
$$
\begin{align*}
K_{\Omega_{\Lambda}}(\boldsymbol{x}, \boldsymbol{u})=\Omega_{\Lambda} \epsilon_{\boldsymbol{u}}(\boldsymbol{x})=\int d^{3} w & \Lambda(\boldsymbol{x}, \boldsymbol{w}) \epsilon_{\boldsymbol{u}}(\boldsymbol{w})  \tag{3.7.16}\\
& =\int d^{3} w \Lambda(\boldsymbol{x}, \boldsymbol{w}) \delta(\boldsymbol{w}-\boldsymbol{u})=\Lambda(\boldsymbol{x}, \boldsymbol{u})
\end{align*}
$$

Hence, (3.7.15) holds.

The kernel $K_{a \mathrm{~A}+b \mathrm{~B}}$ of the linear combination $a \mathrm{~A}+b \mathrm{~B}$ of linear operators A , B with complex scalars $a, b$ is given by
$$
\begin{equation*}
K_{a \mathrm{~A}+b \mathrm{~B}}=a K_{\mathrm{A}}+b K_{\mathrm{B}} \tag{3.7.17}
\end{equation*}
$$

The kernel $K_{\mathrm{AB}}$ of the product AB of two operators $\mathrm{A}, \mathrm{B}$,
$$
\begin{equation*}
K_{\mathrm{AB}}=K_{\mathrm{A}} * K_{\mathrm{B}} \tag{3.7.18}
\end{equation*}
$$
where for any two integral kernels $\Lambda, \Xi, \Lambda * \Xi$ is the kernel
$$
\begin{equation*}
\Lambda * \Xi(\boldsymbol{x}, \boldsymbol{u})=\int d^{3} w \Lambda(\boldsymbol{x}, \boldsymbol{w}) \Xi(\boldsymbol{w}, \boldsymbol{u}) \tag{3.7.19}
\end{equation*}
$$

Proof. By relations (3.7.2) and (3.7.7), we have
$$
\begin{align*}
K_{a \mathrm{~A}+b \mathrm{~B}}(\boldsymbol{x}, \boldsymbol{u})=(a \mathrm{~A} & +b \mathrm{~B}) \epsilon_{\boldsymbol{u}}(\boldsymbol{x}) \\
& =a \mathrm{~A} \epsilon_{\boldsymbol{u}}(\boldsymbol{x})+b \mathrm{~B} \epsilon_{\boldsymbol{u}}(\boldsymbol{x})=a K_{\mathrm{A}}(\boldsymbol{x}, \boldsymbol{u})+b K_{\mathrm{B}}(\boldsymbol{x}, \boldsymbol{u})
\end{align*}
$$

This proves (3.7.17).
By relations (3.7.3), (3.7.6) and (3.7.7), we have
$$
\begin{align*}
K_{\mathrm{AB}}(\boldsymbol{x}, \boldsymbol{u}) & =(\mathrm{AB}) \epsilon_{\boldsymbol{u}}(\boldsymbol{x})=\mathrm{A}\left(\mathrm{B} \epsilon_{\boldsymbol{u}}\right)(\boldsymbol{x})  \tag{3.7.21}\\
& =\int d^{3} w K_{\mathrm{A}}(\boldsymbol{x}, \boldsymbol{w}) \mathrm{B} \epsilon_{\boldsymbol{u}}(\boldsymbol{w}) \\
& =\int d^{3} w K_{\mathrm{A}}(\boldsymbol{x}, \boldsymbol{w}) K_{\mathrm{B}}(\boldsymbol{w}, \boldsymbol{u})=K_{\mathrm{A}} * K_{\mathrm{B}}(\boldsymbol{x}, \boldsymbol{u})
\end{align*}
$$

This proves (3.7.18), (3.7.19).

The kernel $\Lambda * \Xi$ is called the convolution of $\Lambda, \Xi$. We shall have more to say about it momentarily.

Therefore, there is a one-to-one correspondence between linear operators and integral kernels that such that the algebraic properties of the former are reflected faithfully in those of the latter. We can in this way phrase the analysis of the properties of operators entirely in terms of those of the kernels.

In particular, linear operator product is represented by integral kernel convolution. The properties of the latter mirror accurately those of the former. The non commutativity of operator product, see eq. (3.7.3), is paralleled by the non commutativity of kernel convolution. If ( $\Lambda, \Xi$ are integral kernels, then
$$
\begin{equation*}
\Lambda * \Xi \neq \Xi * \Lambda \tag{3.7.22}
\end{equation*}
$$
in general.

Proof. By (3.7.3), using (3.7.15), (3.7.18), we have
$$
\begin{equation*}
\Lambda * \Xi=K_{\Omega_{\Lambda}} * K_{\Omega_{\Xi}}=K_{\Omega_{\Lambda} \Omega_{\Xi}} \neq K_{\Omega_{\Xi} \Omega_{\Lambda}}=K_{\Omega_{\Xi}} * K_{\Omega_{\Lambda}}=\Xi * \Lambda \tag{3.7.23}
\end{equation*}
$$
showing (3.7.22)

Similarly, the associativity of operator product, see eq. (3.7.4), is paralleled by the associativity of kernel convolution, as or any three integral kernels $(\Lambda, \Xi, \Sigma$,
$$
\begin{equation*}
(\Lambda * \Xi) * \Sigma=\Lambda *(\Xi * \Sigma) \tag{3.7.24}
\end{equation*}
$$

Proof. By (3.7.4), using (3.7.15), (3.7.18), we have
$$
\begin{align*}
& (\Lambda * \Xi) * \Sigma=\left(K_{\Omega_{\Lambda}} * K_{\Omega_{\Xi}}\right) * K_{\Omega_{\Sigma}}=K_{\left(\Omega_{\Lambda} \Omega_{\Xi}\right) \Omega_{\Sigma}}  \tag{3.7.25}\\
& =K_{\Omega_{\Lambda}\left(\Omega_{\Xi} \Omega_{\Sigma}\right)}=K_{\Omega_{\Lambda}} *\left(K_{\Omega_{\Xi}} * K_{\Omega_{\Sigma}}\right)=\Lambda *(\Xi * \Sigma),
\end{align*}
$$
proving $(3.7 .24)$.

By (3.6.1) and (3.7.6), the integral kernel $K_{\mathrm{Q}_{f}}$ of the linear operator $\mathrm{Q}_{f}$ associated with a phase function $f$ is just the Weyl-Wigner transform $K_{f}$ of $f$ of eq. (3.5.80). As there is an integral kernel $K_{\mathrm{A}}$ expressing every linear operator A and every integral kernel $\Lambda$ expresses a linear operator $\Omega_{\Lambda}$, the question arises about whether for any integral kernel $\Lambda$ there is a phase function $\chi_{A}$ such that
$$
\begin{equation*}
K_{\chi_{\Lambda}}=\Lambda \tag{3.7.26}
\end{equation*}
$$

To be meaningful, further, the phase function $\chi_{\Lambda}$, if it does exist, should have the following property. If $\Lambda=K_{f}$, then
$$
\begin{equation*}
\chi_{K_{f}}=f \tag{3.7.27}
\end{equation*}
$$

Let us tackle next this issue.
As it appears form inspection of relation (3.5.80), the integral kernel $K_{f}$ corresponding to a phase function $f$ is given by an integral expression with a complex $\hbar$-dependent integrand. We expect so the function $\chi_{A}$ to be complex and $\hbar$-dependent in general. so, the algebra of ordinary $\hbar$-independent real valued
phase functions is to small for the problem we want to solve. We have to enlarge it to include $\hbar$-dependent complex valued phase functions. We do so below.

We can now verify that for the kernel $\Lambda$, the function $\chi_{A}$ is given by
$$
\begin{equation*}
\chi_{\Lambda}(\boldsymbol{x}, \boldsymbol{y})=\int d^{3} u \exp (-i \boldsymbol{u} \cdot \boldsymbol{y} / \hbar) \Lambda(\boldsymbol{x}+\boldsymbol{u} / 2, \boldsymbol{x}-\boldsymbol{u} / 2) \tag{3.7.28}
\end{equation*}
$$

Conditions (3.7.26), (3.7.27) are satisfied.

Proof. Setting $f=\chi_{A}$ in (3.5.80) and expressing $\chi_{A}$ using (3.7.28), we have
$$
\begin{align*}
& K_{\chi_{\Lambda}}(\boldsymbol{x}, \boldsymbol{u})= \int \frac{d^{3} p}{(2 \pi \hbar)^{3}} \exp (i(\boldsymbol{x}-\boldsymbol{u}) \cdot \boldsymbol{p} / \hbar) \chi_{\Lambda}((\boldsymbol{x}+\boldsymbol{u}) / 2, \boldsymbol{p})  \tag{3.7.29}\\
&= \int \frac{d^{3} p}{(2 \pi \hbar)^{3}} \exp (i(\boldsymbol{x}-\boldsymbol{u}) \cdot \boldsymbol{p} / \hbar) \\
& \int d^{3} w \exp (-i \boldsymbol{w} \cdot \boldsymbol{p} / \hbar) \Lambda((\boldsymbol{x}+\boldsymbol{u}) / 2+\boldsymbol{w} / 2,(\boldsymbol{x}+\boldsymbol{u}) / 2-\boldsymbol{w} / 2) \\
&= \int d^{3} w \Lambda((\boldsymbol{x}+\boldsymbol{u}+\boldsymbol{w}) / 2,(\boldsymbol{x}+\boldsymbol{u}-\boldsymbol{w}) / 2) \\
& \quad \int \frac{d^{3} p}{(2 \pi \hbar)^{3}} \exp (i(\boldsymbol{x}-\boldsymbol{u}-\boldsymbol{w}) \cdot \boldsymbol{p} / \hbar) \\
&= \int d^{3} w \Lambda((\boldsymbol{x}+\boldsymbol{u}+\boldsymbol{w}) / 2,(\boldsymbol{x}+\boldsymbol{u}-\boldsymbol{w}) / 2) \delta(\boldsymbol{x}-\boldsymbol{u}-\boldsymbol{w})=\Lambda(\boldsymbol{x}, \boldsymbol{u})
\end{align*}
$$
where (3.5.67) was used. This shows (3.7.26).
Setting $\Lambda=K_{f}$ in eq. (3.7.28) and using expression (3.5.80), we find similarly
$$
\begin{align*}
\chi_{K_{f}}(\boldsymbol{x}, \boldsymbol{y}) & =\int d^{3} u \exp (-i \boldsymbol{u} \cdot \boldsymbol{y} / \hbar) K_{f}(\boldsymbol{x}+\boldsymbol{u} / 2, \boldsymbol{x}-\boldsymbol{u} / 2)  \tag{3.7.30}\\
& =\int d^{3} u \exp (-i \boldsymbol{u} \cdot \boldsymbol{y} / \hbar) \int \frac{d^{3} p}{(2 \pi \hbar)^{3}} \exp (i \boldsymbol{u} \cdot \boldsymbol{p} / \hbar) f(\boldsymbol{x}, \boldsymbol{p}) \\
& =\int d^{3} p \int \frac{d^{3} u}{(2 \pi \hbar)^{3}} \exp (i \boldsymbol{u} \cdot(\boldsymbol{p}-\boldsymbol{y}) / \hbar) f(\boldsymbol{x}, \boldsymbol{p}) \\
& =\int d^{3} p \delta(\boldsymbol{p}-\boldsymbol{y}) f(\boldsymbol{x}, \boldsymbol{p})=f(\boldsymbol{x}, \boldsymbol{y})
\end{align*}
$$
where (3.5.67) was employed again. This verifies (3.7.27).

The convolution of the integral kernels, defined in eq.. (3.7.19), is mirrored
by a special relation among the corresponding $\chi$ functions,
$$
\begin{equation*}
\chi_{A * \Xi}=\chi_{A} * \chi_{\Xi} \tag{3.7.31}
\end{equation*}
$$

Here, for any two phase functions $f, g, f * g$ is given by
$$
\begin{align*}
& f * g(\boldsymbol{x}, \boldsymbol{y})  \tag{3.7.32}\\
& \quad=\left.\exp \left(i \hbar\left(\boldsymbol{\nabla}_{\boldsymbol{x}} \cdot \boldsymbol{\nabla}_{\boldsymbol{y}^{\prime}}-\boldsymbol{\nabla}_{\boldsymbol{x}^{\prime}} \cdot \boldsymbol{\nabla}_{\boldsymbol{y}}\right) / 2\right) f(\boldsymbol{x}, \boldsymbol{y}) g\left(\boldsymbol{x}^{\prime}, \boldsymbol{y}^{\prime}\right)\right|_{\boldsymbol{x}^{\prime}=\boldsymbol{x}, \boldsymbol{y}^{\prime}=\boldsymbol{y}}
\end{align*}
$$
where $\exp \left(i \hbar\left(\boldsymbol{\nabla}_{\boldsymbol{x}} \cdot \boldsymbol{\nabla}_{\boldsymbol{y}^{\prime}}-\boldsymbol{\nabla}_{\boldsymbol{x}^{\prime}} \cdot \boldsymbol{\nabla}_{\boldsymbol{y}}\right) / 2\right)$ denotes the formal operator expression
$$
\begin{align*}
& \exp \left(i \hbar\left(\boldsymbol{\nabla}_{\boldsymbol{x}} \cdot \boldsymbol{\nabla}_{\boldsymbol{y}^{\prime}}-\boldsymbol{\nabla}_{\boldsymbol{x}^{\prime}} \cdot \boldsymbol{\nabla}_{\boldsymbol{y}}\right) / 2\right)  \tag{3.7.33}\\
&=\sum_{n=0}^{\infty} \frac{(i \hbar / 2)^{n}}{n!}\left(\boldsymbol{\nabla}_{\boldsymbol{x}} \cdot \boldsymbol{\nabla}_{\boldsymbol{y}^{\prime}}-\boldsymbol{\nabla}_{\boldsymbol{x}^{\prime}} \cdot \boldsymbol{\nabla}_{\boldsymbol{y}}\right)^{n}
\end{align*}
$$
$D^{n}$ denoting the product of $n$ copies of an operator $D . f * g$ is called the star product or Moyal product of $f, g$ (J. Moyal, 1949). It plays a fundamental role in the analysis of the relation between calssical and quantum observables.

Proof. The phase space Fourier transform of a fase function $f$ is
$$
\begin{equation*}
\tilde{f}(\boldsymbol{s}, \boldsymbol{t})=\int \frac{d^{3} x}{(2 \pi \hbar)^{3}} \int \frac{d^{3} y}{(2 \pi \hbar)^{3}} \exp (-i(\boldsymbol{s} \cdot \boldsymbol{x}+\boldsymbol{t} \cdot \boldsymbol{y}) / \hbar) f(\boldsymbol{x}, \boldsymbol{y}) \tag{3.7.34}
\end{equation*}
$$
$f$ can be expressed as the inverse Fourier expansion of $\tilde{f}$,
$$
\begin{equation*}
f(\boldsymbol{x}, \boldsymbol{y})=\int d^{3} s \int d^{3} t \exp (i(\boldsymbol{x} \cdot \boldsymbol{s}+\boldsymbol{y} \cdot \boldsymbol{t}) / \hbar) \tilde{f}(s, \boldsymbol{t}) \tag{3.7.35}
\end{equation*}
$$

Indeed, using relation (3.5.67), we find
$$
\begin{align*}
\int d^{3} s \int d^{3} t & \exp (i(\boldsymbol{x} \cdot \boldsymbol{s}+\boldsymbol{y} \cdot \boldsymbol{t}) / \hbar) \tilde{f}(s, \boldsymbol{t})  \tag{3.7.36}\\
= & \int d^{3} s \int d^{3} t \exp (i(\boldsymbol{x} \cdot \boldsymbol{s}+\boldsymbol{y} \cdot \boldsymbol{t}) / \hbar) \\
& \int \frac{d^{3} x^{\prime}}{(2 \pi \hbar)^{3}} \int \frac{d^{3} y^{\prime}}{(2 \pi \hbar)^{3}} \exp \left(-i\left(\boldsymbol{s} \cdot \boldsymbol{x}^{\prime}+\boldsymbol{t} \cdot \boldsymbol{y}^{\prime}\right) / \hbar\right) f\left(\boldsymbol{x}^{\prime}, \boldsymbol{y}^{\prime}\right) \\
= & \int d^{3} x^{\prime} \int d^{3} y^{\prime} f\left(\boldsymbol{x}^{\prime}, \boldsymbol{y}^{\prime}\right)
\end{align*}
$$
$$
\begin{array}{r}
\int \frac{d^{3} s}{(2 \pi \hbar)^{3}} \int \frac{d^{3} t}{(2 \pi \hbar)^{3}} \exp \left(i\left(\boldsymbol{s} \cdot\left(\boldsymbol{x}-\boldsymbol{x}^{\prime}\right)+\boldsymbol{t} \cdot\left(\boldsymbol{y}-\boldsymbol{y}^{\prime}\right)\right) / \hbar\right) \\
=\int d^{3} x^{\prime} \int d^{3} y^{\prime} f\left(\boldsymbol{x}^{\prime}, \boldsymbol{y}^{\prime}\right) \delta\left(\boldsymbol{x}-\boldsymbol{x}^{\prime}\right) \delta\left(\boldsymbol{y}-\boldsymbol{y}^{\prime}\right)=f(\boldsymbol{x}, \boldsymbol{y})
\end{array}
$$
proving (3.7.35).
Using the defining expression (3.5.80), expression (3.7.35) and relation (3.5.67), the Weyl-Wigner transform $K_{f}$ of a phase function $f$ can be cast as
$$
\begin{align*}
K_{f}(\boldsymbol{x}, \boldsymbol{u})= & \int \frac{d^{3} y}{(2 \pi \hbar)^{3}} \exp (i(\boldsymbol{x}-\boldsymbol{u}) \cdot \boldsymbol{y} / \hbar) f((\boldsymbol{x}+\boldsymbol{u}) / 2, \boldsymbol{y})  \tag{3.7.37}\\
= & \int \frac{d^{3} y}{(2 \pi \hbar)^{3}} \exp (i(\boldsymbol{x}-\boldsymbol{u}) \cdot \boldsymbol{y} / \hbar) \\
& \int d^{3} s \int d^{3} t \exp (i((\boldsymbol{x}+\boldsymbol{u}) \cdot \boldsymbol{s} / 2+\boldsymbol{y} \cdot \boldsymbol{t}) / \hbar) \tilde{f}(\boldsymbol{s}, \boldsymbol{t}) \\
= & \int d^{3} s \int d^{3} t \exp (i(\boldsymbol{x}+\boldsymbol{u}) \cdot \boldsymbol{s} / 2 \hbar) \tilde{f}(\boldsymbol{s}, \boldsymbol{t}) \\
& \int \frac{d^{3} y}{(2 \pi \hbar)^{3}} \exp (i(\boldsymbol{x}-\boldsymbol{u}+\boldsymbol{t}) \cdot \boldsymbol{y} / \hbar) \\
= & \int d^{3} s \int d^{3} t \exp (i(\boldsymbol{x}+\boldsymbol{u}) \cdot \boldsymbol{s} / 2 \hbar) \tilde{f}(\boldsymbol{s}, \boldsymbol{t}) \delta(\boldsymbol{x}-\boldsymbol{u}+\boldsymbol{t}) \\
= & \int d^{3} s \exp (i(\boldsymbol{x}+\boldsymbol{u}) \cdot \boldsymbol{s} / 2 \hbar) \tilde{f}(\boldsymbol{s}, \boldsymbol{u}-\boldsymbol{x})
\end{align*}
$$

Let $\Lambda$ be an integral kernel. Then, $\Lambda=K_{\chi_{A}}$ by (3.7.26). By (3.7.37), so,
$$
\begin{equation*}
\Lambda(\boldsymbol{x}, \boldsymbol{u})=\int d^{3} s \exp (i(\boldsymbol{x}+\boldsymbol{u}) \cdot \boldsymbol{s} / 2 \hbar) \tilde{\chi}_{\Lambda}(\boldsymbol{s}, \boldsymbol{u}-\boldsymbol{x}) \tag{3.7.38}
\end{equation*}
$$

Consider the product $\Lambda * \Xi$ of two integral kernels $\Lambda, \Xi$ given by (3.7.19). By (3.7.28) and (3.7.38), the associated function $\chi_{A * \Xi}(\boldsymbol{x}, \boldsymbol{y})$ is
$$
\begin{align*}
\chi_{A * \Xi}(\boldsymbol{x}, \boldsymbol{y})= & \int d^{3} q \exp (-i \boldsymbol{q} \cdot \boldsymbol{y} / \hbar) \Lambda * \Xi(\boldsymbol{x}+\boldsymbol{q} / 2, \boldsymbol{x}-\boldsymbol{q} / 2)  \tag{3.7.39}\\
= & \int d^{3} q \exp (-i \boldsymbol{q} \cdot \boldsymbol{y} / \hbar) \int d^{3} r \Lambda(\boldsymbol{x}+\boldsymbol{q} / 2, \boldsymbol{r}) \Xi(\boldsymbol{r}, \boldsymbol{x}-\boldsymbol{q} / 2) \\
= & \int d^{3} q \exp (-i \boldsymbol{q} \cdot \boldsymbol{y} / \hbar) \int d^{3} r \\
& \int d^{3} v \exp (i(\boldsymbol{x}+\boldsymbol{q} / 2+\boldsymbol{r}) \cdot \boldsymbol{v} / 2 \hbar) \tilde{\chi}_{\Lambda}(\boldsymbol{v}, \boldsymbol{r}-\boldsymbol{x}-\boldsymbol{q} / 2) \\
& \int d^{3} z \exp (i(\boldsymbol{x}-\boldsymbol{q} / 2+\boldsymbol{r}) \cdot \boldsymbol{z} / 2 \hbar) \tilde{\chi}(\boldsymbol{z},-\boldsymbol{r}+\boldsymbol{x}-\boldsymbol{q} / 2)
\end{align*}
$$

We now set $\boldsymbol{u}=\boldsymbol{r}-\boldsymbol{x}-\boldsymbol{q} / 2, \boldsymbol{t}=-\boldsymbol{r}+\boldsymbol{x}-\boldsymbol{q} / 2$ from which $\boldsymbol{q}=-(\boldsymbol{u}+\boldsymbol{t})$ and $\boldsymbol{r}=(\boldsymbol{u}-\boldsymbol{t}) / 2+\boldsymbol{x}$. The Jacobian of the change of variables $(\boldsymbol{q}, \boldsymbol{r}) \rightarrow(\boldsymbol{u}, \boldsymbol{t})$ is
$$
\left[\operatorname{det}\left(\begin{array}{cc}
-1, & -1 \\
1 / 2, & -1 / 2
\end{array}\right)\right]^{3}=1
$$
where the third power is due to the 3-dimensional nature of the transformation carried out. Therefore,
$$
\begin{align*}
\chi_{A * \Xi}(\boldsymbol{x}, \boldsymbol{y})= & \int d^{3} v \int d^{3} z \int d^{3} u \int d^{3} t \tilde{\chi}_{A}(\boldsymbol{v}, \boldsymbol{u}) \tilde{\chi} \Xi(\boldsymbol{z}, \boldsymbol{t})  \tag{3.7.40}\\
& \exp (i(\boldsymbol{u}+\boldsymbol{t}) \cdot \boldsymbol{y} / \hbar+i(2 \boldsymbol{x}-\boldsymbol{t}) \cdot \boldsymbol{v} / 2 \hbar+i(2 \boldsymbol{x}+\boldsymbol{u}) \cdot \boldsymbol{z} / 2 \hbar) \\
= & \int d^{3} v \int d^{3} z \int d^{3} u \int d^{3} t \tilde{\chi}_{A}(\boldsymbol{v}, \boldsymbol{u}) \tilde{\chi}(\boldsymbol{z}, \boldsymbol{t}) \\
& \exp (i(\boldsymbol{x} \cdot(\boldsymbol{v}+\boldsymbol{z})+\boldsymbol{y} \cdot(\boldsymbol{u}+\boldsymbol{t})) / \hbar+i(\boldsymbol{u} \cdot \boldsymbol{z}-\boldsymbol{t} \cdot \boldsymbol{v}) / 2 \hbar)
\end{align*}
$$

We consider the exponential in the integrand. We have,
$$
\begin{align*}
& \exp (i(\boldsymbol{x} \cdot(\boldsymbol{v}+\boldsymbol{z})+\boldsymbol{y} \cdot(\boldsymbol{u}+\boldsymbol{t})) / \hbar+i(\boldsymbol{u} \cdot \boldsymbol{z}-\boldsymbol{t} \cdot \boldsymbol{v}) / 2 \hbar)  \tag{3.7.41}\\
& \quad=\left.\sum_{n=0}^{\infty} \frac{(i / 2 \hbar)^{n}}{n!}(\boldsymbol{u} \cdot \boldsymbol{z}-\boldsymbol{t} \cdot \boldsymbol{v})^{n} \exp \left(i\left(\boldsymbol{x} \cdot \boldsymbol{v}+\boldsymbol{x}^{\prime} \cdot \boldsymbol{z}+\boldsymbol{y} \cdot \boldsymbol{u}+\boldsymbol{y}^{\prime} \cdot \boldsymbol{t}\right) / \hbar\right)\right|_{\boldsymbol{x}^{\prime}=\boldsymbol{x}, \boldsymbol{y}^{\prime}=\boldsymbol{y}}
\end{align*}
$$

Since $-i \hbar \boldsymbol{\nabla}_{\boldsymbol{x}} \exp (i \boldsymbol{x} \cdot \boldsymbol{v} / \hbar)=\boldsymbol{v} \exp (i \boldsymbol{x} \cdot \boldsymbol{v} / \hbar),-i \hbar \boldsymbol{\nabla}_{\boldsymbol{x}^{\prime}} \exp \left(i \boldsymbol{x}^{\prime} \cdot \boldsymbol{z} / \hbar\right)=\boldsymbol{z} \exp \left(i \boldsymbol{x}^{\prime} \cdot \boldsymbol{z} / \hbar\right)$, $-i \hbar \boldsymbol{\nabla}_{\boldsymbol{y}} \exp (i \boldsymbol{y} \cdot \boldsymbol{u} / \hbar)=\boldsymbol{u} \exp (i \boldsymbol{y} \cdot \boldsymbol{u} / \hbar),-i \hbar \boldsymbol{\nabla}_{\boldsymbol{y}^{\prime}} \exp \left(i \boldsymbol{y}^{\prime} \cdot \boldsymbol{t} / \hbar\right)=\boldsymbol{t} \exp \left(i \boldsymbol{y}^{\prime} \cdot \boldsymbol{t} / \hbar\right)$, this relation can be cast as
$$
\begin{align*}
& \exp (i(\boldsymbol{x} \cdot(\boldsymbol{v}+\boldsymbol{z})+\boldsymbol{y} \cdot(\boldsymbol{u}+\boldsymbol{t})) / \hbar+i(\boldsymbol{u} \cdot \boldsymbol{z}-\boldsymbol{t} \cdot \boldsymbol{v}) / 2 \hbar)  \tag{3.7.42}\\
& =\sum_{n=0}^{\infty} \frac{(i \hbar / 2)^{n}}{n!}\left(\boldsymbol{\nabla}_{\boldsymbol{x}} \cdot \boldsymbol{\nabla}_{\boldsymbol{y}^{\prime}}-\boldsymbol{\nabla}_{\boldsymbol{x}^{\prime}} \cdot \boldsymbol{\nabla}_{\boldsymbol{y}}\right)^{n} \\
& \left.\quad \exp \left(i\left(\boldsymbol{x} \cdot \boldsymbol{v}+\boldsymbol{x}^{\prime} \cdot \boldsymbol{z}+\boldsymbol{y} \cdot \boldsymbol{u}+\boldsymbol{y}^{\prime} \cdot \boldsymbol{t}\right) / \hbar\right)\right|_{\boldsymbol{x}^{\prime}=\boldsymbol{x}, \boldsymbol{y}^{\prime}=\boldsymbol{y}} \\
& \quad=\exp \left(i \hbar\left(\boldsymbol{\nabla}_{\boldsymbol{x}} \cdot \boldsymbol{\nabla}_{\boldsymbol{y}^{\prime}}-\boldsymbol{\nabla}_{\boldsymbol{x}^{\prime}} \cdot \boldsymbol{\nabla}_{\boldsymbol{y}}\right) / 2\right) \\
& \left.\quad \exp \left(i\left(\boldsymbol{x} \cdot \boldsymbol{v}+\boldsymbol{x}^{\prime} \cdot \boldsymbol{z}+\boldsymbol{y} \cdot \boldsymbol{u}+\boldsymbol{y}^{\prime} \cdot \boldsymbol{t}\right) / \hbar\right)\right|_{\boldsymbol{x}^{\prime}=\boldsymbol{x}, \boldsymbol{y}^{\prime}=\boldsymbol{y}}
\end{align*}
$$

Substituting (3.7.42) into (3.7.40) and using (3.7.35), we get
$$
\begin{align*}
& \chi_{\Lambda * \Xi}(\boldsymbol{x}, \boldsymbol{y})=\exp \left(i \hbar\left(\boldsymbol{\nabla}_{\boldsymbol{x}} \cdot \boldsymbol{\nabla}_{\boldsymbol{y}^{\prime}}-\boldsymbol{\nabla}_{\boldsymbol{x}^{\prime}} \cdot \boldsymbol{\nabla}_{\boldsymbol{y}}\right) / 2\right)  \tag{3.7.43}\\
& \qquad \int d^{3} v \int d^{3} z \int d^{3} u \int d^{3} t \tilde{\chi}_{\Lambda}(\boldsymbol{v}, \boldsymbol{u}) \tilde{\chi}_{\Xi}(\boldsymbol{z}, \boldsymbol{t})
\end{align*}
$$
$$
\begin{aligned}
& \left.\quad \exp \left(i\left(\boldsymbol{x} \cdot \boldsymbol{v}+\boldsymbol{x}^{\prime} \cdot \boldsymbol{z}+\boldsymbol{y} \cdot \boldsymbol{u}+\boldsymbol{y}^{\prime} \cdot \boldsymbol{t}\right) / \hbar\right)\right|_{\boldsymbol{x}^{\prime}=\boldsymbol{x}, \boldsymbol{y}^{\prime}=\boldsymbol{y}} \\
& =\left.\exp \left(i \hbar\left(\boldsymbol{\nabla}_{\boldsymbol{x}} \cdot \boldsymbol{\nabla}_{\boldsymbol{y}^{\prime}}-\boldsymbol{\nabla}_{\boldsymbol{x}^{\prime}} \cdot \boldsymbol{\nabla}_{\boldsymbol{y}}\right) / 2\right) \chi_{\Lambda}(\boldsymbol{x}, \boldsymbol{y}) \chi_{\Xi}\left(\boldsymbol{x}^{\prime}, \boldsymbol{y}^{\prime}\right)\right|_{\boldsymbol{x}^{\prime}=\boldsymbol{x}, \boldsymbol{y}^{\prime}=\boldsymbol{y}} \\
& =\chi_{\Lambda} * \chi_{\Xi}(\boldsymbol{x}, \boldsymbol{y}) .
\end{aligned}
$$

This shows (3.7.31).

The phase function Moyal product has formal properties which answers to those of the integral kernel convolution. As convolution, Moyal product is non commutative: if $f, g$ are phase functions, then
$$
\begin{equation*}
f * g \neq g * f \tag{3.7.44}
\end{equation*}
$$
in general.

Proof. By (3.7.22), using (3.7.27), (3.7.31), we have
$$
\begin{equation*}
f * g=\chi_{K_{f}} * \chi_{K_{g}}=\chi_{K_{f} * K_{g}} \neq \chi_{K_{g} * K_{f}}=\chi_{K_{g}} * \chi_{K_{f}}=g * f . \tag{3.7.45}
\end{equation*}
$$

This proves (3.7.44).

In similar fashion, just as convolution, Moyal product is associative: for any phase functions $f, g$, $h$, one has
$$
\begin{equation*}
(f * g) * h=f *(g * h) \tag{3.7.46}
\end{equation*}
$$

Proof. By (3.7.24), using (3.7.27), (3.7.31), we have
$$
\begin{align*}
& (f * g) * h=\left(\chi_{K_{f}} * \chi_{K_{g}}\right) * \chi_{K_{h}}=\chi_{\left(K_{f} * K_{g}\right) * K_{h}}  \tag{3.7.47}\\
& \quad=\chi_{K_{f} *\left(K_{g} * K_{h}\right)}=\chi_{K_{f}} *\left(\chi_{K_{g}} * \chi_{K_{h}}\right)=f *(g * h)
\end{align*}
$$
showing $(3.7 .46)$

The Moyal product of phase functions therefore differs in a basic way from the
ordinary product. Like the later, the former is associative, but unlike the latter, the former is non commutative. This non commutativity is a pure quantum effect. In fact, by writing the Moyal product of two phase functions $f, g$ to lowest non trivial order in the Planck constant $\hbar$. From (3.7.32), we find that
$$
\begin{equation*}
f * g=f g+i \hbar\{f, g\} / 2+O\left(\hbar^{2}\right) \tag{3.7.48}
\end{equation*}
$$
where $\{f, g\}$ denotes the classical Poisson bracket of $f, g$,
$$
\begin{equation*}
\{f, g\}=\boldsymbol{\nabla}_{\boldsymbol{x}} f \cdot \boldsymbol{\nabla}_{\boldsymbol{y}} g-\boldsymbol{\nabla}_{\boldsymbol{x}} g \cdot \boldsymbol{\nabla}_{\boldsymbol{y}} f \tag{3.7.49}
\end{equation*}
$$

Proof. From (3.7.32), truncating the operator exponential series (3.7.33) to first order in $\hbar$, we obtain
$$
\begin{align*}
f * g(\boldsymbol{x}, \boldsymbol{y})= & \left.f(\boldsymbol{x}, \boldsymbol{y}) g\left(\boldsymbol{x}^{\prime}, \boldsymbol{y}^{\prime}\right)\right|_{\boldsymbol{x}^{\prime}=\boldsymbol{x}, \boldsymbol{y}^{\prime}=\boldsymbol{y}}  \tag{3.7.50}\\
& \quad+\left.(i \hbar / 2)\left(\boldsymbol{\nabla}_{\boldsymbol{x}} \cdot \boldsymbol{\nabla}_{\boldsymbol{y}^{\prime}}-\boldsymbol{\nabla}_{\boldsymbol{x}^{\prime}} \cdot \boldsymbol{\nabla}_{\boldsymbol{y}}\right) f(\boldsymbol{x}, \boldsymbol{y}) g\left(\boldsymbol{x}^{\prime}, \boldsymbol{y}^{\prime}\right)\right|_{\boldsymbol{x}^{\prime}=\boldsymbol{x}, \boldsymbol{y}^{\prime}=\boldsymbol{y}}+O\left(\hbar^{2}\right) \\
= & f(\boldsymbol{x}, \boldsymbol{y}) g(\boldsymbol{x}, \boldsymbol{y}) \\
& +(i \hbar / 2)\left(\boldsymbol{\nabla}_{\boldsymbol{x}} f(\boldsymbol{x}, \boldsymbol{y}) \cdot \boldsymbol{\nabla}_{\boldsymbol{y}} g(\boldsymbol{x}, \boldsymbol{y})-\boldsymbol{\nabla}_{\boldsymbol{x}} g(\boldsymbol{x}, \boldsymbol{y}) \cdot \boldsymbol{\nabla}_{\boldsymbol{y}} f(\boldsymbol{x}, \boldsymbol{y})\right)+O\left(\hbar^{2}\right) \\
= & f g(\boldsymbol{x}, \boldsymbol{y})+(i \hbar / 2)\{f, g\}(\boldsymbol{x}, \boldsymbol{y})+O\left(\hbar^{2}\right)
\end{align*}
$$

Relation (3.7.48) follows.

In this way, a well-known classical object, the Poisson bracket, governs the Moyal non commutativity of phase functions. This furnishes a novel interpretation of the Poisson bracket. It furthermore naturally suggests introducing the Moyal bracket
$$
\begin{equation*}
\{f, g\}_{*}=(f * g-g * f) / i \hbar \tag{3.7.51}
\end{equation*}
$$

The Moyal bracket, so, measures the non commutativity of the Moyal product. It is also the quantum analog of the Poisson bracket, since by (3.7.49) we have
$$
\begin{equation*}
\{f, g\}_{*}=\{f, g\}+O(\hbar) \tag{3.7.52}
\end{equation*}
$$

The following identity shows the relationship between the Moyal product of two phase functions $f, g$ and the product of the corresponding operators $\mathrm{Q}_{f}, \mathrm{Q}_{g}$,
$$
\begin{equation*}
f * g=\chi_{K_{Q_{f} Q_{g}}} \tag{3.7.53}
\end{equation*}
$$

Proof. One has indeed
$$
\begin{equation*}
f * g=\chi_{K_{f}} * \chi_{K_{g}}=\chi_{K_{f} * K_{g}}=\chi_{K_{Q_{f}}} * K_{Q_{g}}=\chi_{K_{Q_{f} Q_{g}}} \tag{3.7.54}
\end{equation*}
$$

In the first step we used relation (3.7.27). In the second step, we applied relation (3.7.31) with $\Lambda=K_{f}, \Xi=k_{g}$. In the third step we used that $K_{f}$ is the integral kernel of the operator $\mathrm{Q}_{f}$ so that $K_{f}=K_{Q_{f}}$ and similarly for $g$. Finally, in the fourth step, we employed relation (3.7.18) with $\mathrm{A}=\mathrm{Q}_{f}, \mathrm{~B}=\mathrm{Q}_{g}$.

The following conclusion is reached.
The algebra of phase functions under Moyal multiplication provides a faithful image of the algebra of the associated operators. All the algebraic properties of the operators $Q_{f}$ are mirrored in those of the underlying phase functions $f$ if Moyal rather than ordinary multiplication is used. The non commutativity of operators is then found to be an essentially quantum effect disappearing in the semiclassical limit $\hbar \rightarrow 0$.

Since $f * f \neq f^{2}$, we have $K_{f * f} \neq K_{f^{2}}$ in general. By (3.7.26) and (3.7.53), then, $K_{\mathrm{Q}_{f}{ }^{2}}=K_{f * f} \neq K_{f^{2}}=K_{\mathrm{Q}_{f^{2}}}$. It follows that $\mathrm{Q}_{f}{ }^{2} \neq \mathrm{Q}_{f^{2}}$ in general in the quantization framework based on the Wigner quasiprobability distribution and Weyl-Wigner transform of sect. 3.5. It is believed instead that
$$
\begin{equation*}
\mathrm{Q}_{f}^{p}=\mathrm{Q}_{f^{p}} \tag{3.7.55}
\end{equation*}
$$
should hold for any integer $p>0$ in any viable quantum theory. This property does in indeed hold in the modern quantum theory.

\subsection*{3.8. Single particle interpretation of wave mechanics}

It is now time to pause and appraise the import of the results which we have obtained.

In classical mechanics, a particle is characterized by its states and observables. The state of the particle determines uniquely the value of each of its observables. Indeed, if we prepare the particle in a given state and measure the value of an assigned observable repeatedly, we obtain at each iteration of the experiment always the same result specific of the state, once due allowance for statistical experimental error is made. This well-known fact of classical physics is based on a body of evidence so large that it is can be hardly questioned.

In wave mechanics, a particle is also characterized by its states and observables. However, the state of the particle does not determine the value of each of its observables the way it does classically. In fact, if we prepare the particle in a given state and measure the value of a chosen observable repeatedly, we find at each reprise of the procedure generally different results even after taking into due account statistical experimental error. It is not possible to predict which value will be obtained in the next iteration, but it is possible to measure the probabilities of the possible values. This surprising conclusion is forced upon us by a careful examination of basic quantum experiences.

Consider for instance Taylor's experiment (cf. sect. 1.16). The photons are produced exactly in the same way and so they are all prepared presumably in the same state. However, observation of the photons position on a screen by means of a plate shows that the photons reach the screen at different points. One cannot foretell at which point the next photon will arrive, but observing many photons one can measure the probability of a photon arriving at a certain site of the screen. This shows that the state of a photon does not determine its position on the screen but only the probabilities of its possible positions.

The evidence of experiments therefore yields the following puzzling conclusion.
Classical states are deterministic while quantum states are probabilistic.
This imples also that classical and quantum observables are different in the following respect.

While in classical mechanics an observable has always a definite value uniquely determined by the particle's state, in wave mechanics it does not. The state of the particle determines only the probalilities of the possible measured values of the observable.

The Hamilton formulation of classical mechanics adequately accounts for the deterministic nature of the states of a classical particle, since observables are conceived as functions of the particle's state, making their value determined by this latter.

The Schroedinger formulation of wave mechanics describes a quantum particle through the associated ensemble. The statistical nature of the ensemble renders it in principle capable of accounting for the probabilistic nature of the particle's states. However, the way this comes about is not immediately obvious. The ensemble is distinct from the particle and the states and observables of the former appear to be unlike those of the latter. The natural question arises whether a single particle formulation of wave mechanics based on ensemble theory is possible at all. Happily, it is.

Since the copies of the particle in the ensemble are all equal to it, we can identify the particle with any arbitrarily chosen copy. Hence, when we measure the value of an observable of the particle, it is as if we were measuring the value of the corresponding observable of a copy of the particle randomly extracted from the ensemble. According to wave mechanics, the state of the ensemble does not determine the value that is found but only the probabilities of the possible values. The only conclusion that can be drawn from this analysis is the following.

The states and observables of the ensemble can be identified with the probabilistic states and observables of the underlying quantum particle.

The mathematical modellization of the ensemble, which we outlined in sect. 3.6 , in this way directly provides a mathematical description also of the wave mechanics of a single particle.

The probalistic state of the quantum particle is encoded by a normalized wave function. The observables of the a particle are represented by suitable linear operators.

Now, we can understand which problems Schroedinger's theory actually solves. The single particle formulation of wave mechanics does not obviously parallel the classical Hamiltonian formulation we recalled above and reduce to it under appropriate conditions as described by Bohr's correspondence principle (cf. sect. 2.5). It seems indeed paradoxical that wave mechanics, which is probabilistic, reproduces in the semiclassical regime classical mechanics, which is instead deterministic. The paradox is solved by ensemble theory. By its statistical nature, the classical ensemble is a perfectly conceivable candidate for being the semiclassical limit of the quantum one. As we saw, Schroedinger's wave mechanics provides a formulation of quantum theory describing a quantum ensemble that semiclassically reproduces the classical ensemble as required.

The probabilistic features of wave mechanics are inextricably linked to its undulatory aspects. In Taylor's experiment, for instance, the statistical distribution of the photons' detection points on the plate faithfully reproduces the modulation of light intensity and is therefore correlated with the interference maxima and minima of the classic Young experience (cf. sect. 1.6). Schroedinger's wave mechanics, by being a wave theory, has the potential of explaining also the interplay of undulatory and probabilistic aspects of quantum physics.

\subsection*{3.9. Momentum space formulation}

In Hamiltonian mechanics, position $\boldsymbol{q}$ and momentum $\boldsymbol{p}$ are interchangeable: they can be switched one to the other by the canonical transformation $(\boldsymbol{q}, \boldsymbol{p}) \rightarrow(\boldsymbol{p},-\boldsymbol{q})$. Since wave mechanics is rooted in Hamiltonian mechanics, as the derivation of the Schroedinger equation we have illustrated in sects. 3.3, 3.4 shows, we expect that this dual role of position and momentum should emerge somehow in wave mechanics too. If we accept that the distinction between configuration and momentum space is only a conventional one, then we should be able to formulate wave mechanics through a momentum space wave function in terms of which the momentum space probability density and current density and the means of all mechanical quantities can be expressed. This is indeed so, as we shall show in due course in this section.

Every configuration space wave function $\phi$ has the expansion
$$
\begin{equation*}
\phi(\boldsymbol{x})=\int \frac{d^{3} y}{(2 \pi \hbar)^{3 / 2}} \exp (i \boldsymbol{x} \cdot \boldsymbol{y} / \hbar) \tilde{\phi}(\boldsymbol{y}) \tag{3.9.1}
\end{equation*}
$$
where $\tilde{\phi}$ is the associated momentum space wave function
$$
\begin{equation*}
\tilde{\phi}(\boldsymbol{y})=\int \frac{d^{3} x}{(2 \pi \hbar)^{3 / 2}} \exp (-i \boldsymbol{y} \cdot \boldsymbol{x} / \hbar) \phi(\boldsymbol{x}) \tag{3.9.2}
\end{equation*}
$$

This is just the form Fourier theorem takes in wave mechanics.

Proof. Expressing the configuration space delta function using (3.5.67), we have
$$
\begin{align*}
\phi(\boldsymbol{x}) & =\int d^{3} u \delta(\boldsymbol{x}-\boldsymbol{u}) \phi(\boldsymbol{u})  \tag{3.9.3}\\
& =\int d^{3} u\left(\int \frac{d^{3} y}{(2 \pi \hbar)^{3}} \exp (i(\boldsymbol{x}-\boldsymbol{u}) \cdot \boldsymbol{y} / \hbar)\right) \phi(\boldsymbol{u}) \\
& =\int \frac{d^{3} y}{(2 \pi \hbar)^{3 / 2}} \exp (i \boldsymbol{x} \cdot \boldsymbol{y} / \hbar) \int \frac{d^{3} u}{(2 \pi \hbar)^{3 / 2}} \exp (-i \boldsymbol{y} \cdot \boldsymbol{u} / \hbar) \phi(\boldsymbol{u}) \\
& =\int \frac{d^{3} y}{(2 \pi \hbar)^{3 / 2}} \exp (i \boldsymbol{x} \cdot \boldsymbol{y} / \hbar) \tilde{\phi}(\boldsymbol{y})
\end{align*}
$$
where $\tilde{\phi}$ is given by (3.9.2), as claimed.

By the above result, there is a one-to-one correspondence between configuration and momentum space wave functions. In particular, the configuration space wave function $\psi$ of the ensemble enjoys the expansion
$$
\begin{equation*}
\psi(t, \boldsymbol{x})=\int \frac{d^{3} y}{(2 \pi \hbar)^{3 / 2}} \exp (i \boldsymbol{x} \cdot \boldsymbol{y} / \hbar) \tilde{\psi}(t, \boldsymbol{y}) \tag{3.9.4}
\end{equation*}
$$
where $\tilde{\psi}$ is the ensemble's associated momentum space wave function
$$
\begin{equation*}
\tilde{\psi}(t, \boldsymbol{y})=\int \frac{d^{3} x}{(2 \pi \hbar)^{3 / 2}} \exp (-i \boldsymbol{y} \cdot \boldsymbol{x} / \hbar) \psi(t, \boldsymbol{x}) \tag{3.9.5}
\end{equation*}
$$

Since by virtue of (3.9.4), (3.9.5), $\tilde{\psi}$ both determines and is determined by $\psi$. $\tilde{\psi}$ should obey an equation equivalent to the Schroedinger equation (3.4.14) obeyed by $\psi$. This is the momentum space Schroedinger equation,
$$
\begin{equation*}
i \hbar \frac{\partial \tilde{\psi}}{\partial t}=\frac{\boldsymbol{y}^{2}}{2 m} \tilde{\psi}+\mathrm{U} \tilde{\psi} \tag{3.9.6}
\end{equation*}
$$
where $\mathrm{U} \tilde{\psi}$ denotes the momentum space wave function
$$
\begin{equation*}
\mathrm{U} \tilde{\psi}(t, \boldsymbol{y})=\int d^{3} v \tilde{U}(\boldsymbol{y}-\boldsymbol{v}) \tilde{\psi}(t, \boldsymbol{v}) \tag{3.9.7}
\end{equation*}
$$
$\tilde{U}(\boldsymbol{v})$ being momentum space potential energy function,
$$
\begin{equation*}
\tilde{U}(\boldsymbol{y})=\int \frac{d^{3} x}{(2 \pi \hbar)^{3}} \exp (-i \boldsymbol{y} \cdot \boldsymbol{x} / \hbar) U(\boldsymbol{x}) \tag{3.9.8}
\end{equation*}
$$
defined in analogy to (3.9.2) except for an extra $(2 \pi \hbar)^{-3 / 2}$ normalization factor. The structural similarity of (3.4.14) and (3.9.6) is evident.

Proof. By the Schroedinger equation (3.4.14), we have
$$
\begin{equation*}
0=\int \frac{d^{3} x}{(2 \pi \hbar)^{3 / 2}} \exp (-i \boldsymbol{y} \cdot \boldsymbol{x} / \hbar) \tag{3.9.9}
\end{equation*}
$$
$$
\begin{array}{r}
\left(i \hbar \frac{\partial \psi(t, \boldsymbol{x})}{\partial t}+\frac{\hbar^{2}}{2 m} \boldsymbol{\nabla}_{\boldsymbol{x}}^{2} \psi(t, \boldsymbol{x})-U(\boldsymbol{x}) \psi(t, \boldsymbol{x})\right) \\
=i \hbar \frac{\partial}{\partial t} \int \frac{d^{3} x}{(2 \pi \hbar)^{3 / 2}} \exp (-i \boldsymbol{y} \cdot \boldsymbol{x} / \hbar) \psi(t, \boldsymbol{x}) \\
+\frac{\hbar^{2}}{2 m} \int \frac{d^{3} x}{(2 \pi \hbar)^{3 / 2}} \boldsymbol{\nabla}_{\boldsymbol{x}}^{2} \exp (-i \boldsymbol{y} \cdot \boldsymbol{x} / \hbar) \psi(t, \boldsymbol{x}) \\
-\int \frac{d^{3} x}{(2 \pi \hbar)^{3 / 2}} \exp (-i \boldsymbol{y} \cdot \boldsymbol{x} / \hbar) U(\boldsymbol{x}) \psi(t, \boldsymbol{x})
\end{array}
$$

By (3.9.5), the first term in the right hand side of (3.9.9) is
$$
\begin{equation*}
i \hbar \frac{\partial}{\partial t} \int \frac{d^{3} x}{(2 \pi \hbar)^{3 / 2}} \exp (-i \boldsymbol{y} \cdot \boldsymbol{x} / \hbar) \psi(t, \boldsymbol{x})=i \hbar \frac{\partial \tilde{\psi}(t, \boldsymbol{y})}{\partial t} \tag{3.9.10}
\end{equation*}
$$

By (3.9.5) again, using that $\boldsymbol{\nabla}_{\boldsymbol{x}}{ }^{2} \exp (-i \boldsymbol{y} \cdot \boldsymbol{x} / \hbar)=-\hbar^{-2} \boldsymbol{y}^{2} \exp (-i \boldsymbol{y} \cdot \boldsymbol{x} / \hbar)$, the second term in the right hand side of (3.9.9) is
$$
\begin{equation*}
\frac{\hbar^{2}}{2 m} \int \frac{d^{3} x}{(2 \pi \hbar)^{3 / 2}} \boldsymbol{\nabla}_{\boldsymbol{x}}^{2} \exp (-i \boldsymbol{y} \cdot \boldsymbol{x} / \hbar) \psi(t, \boldsymbol{x})=-\frac{\boldsymbol{y}^{2}}{2 m} \tilde{\psi}(t, \boldsymbol{y}) \tag{3.9.11}
\end{equation*}
$$

Further, by (3.9.5) once more, using (3.5.64), we have
$$
\begin{align*}
& \int \frac{d^{3} x}{(2 \pi \hbar)^{3 / 2}} \exp (-i \boldsymbol{y} \cdot \boldsymbol{x} / \hbar) U(\boldsymbol{x}) \psi(t, \boldsymbol{x})  \tag{3.9.12}\\
& =\int \frac{d^{3} x}{(2 \pi \hbar)^{3 / 2}} \exp (-i \boldsymbol{y} \cdot \boldsymbol{x} / \hbar) U(\boldsymbol{x}) \int d^{3} u \delta(\boldsymbol{x}-\boldsymbol{u}) \psi(t, \boldsymbol{u}) \\
& =\int \frac{d^{3} x}{(2 \pi \hbar)^{3 / 2}} \exp (-i \boldsymbol{y} \cdot \boldsymbol{x} / \hbar) U(\boldsymbol{x}) \int d^{3} u \int \frac{d^{3} v}{(2 \pi \hbar)^{3}} \exp (i(\boldsymbol{x}-\boldsymbol{u}) \cdot \boldsymbol{v} / \hbar) \psi(t, \boldsymbol{u}) \\
& =\int d^{3} v\left(\int \frac{d^{3} x}{(2 \pi \hbar)^{3}} \exp (-i(\boldsymbol{y}-\boldsymbol{v}) \cdot \boldsymbol{x} / \hbar) U(\boldsymbol{x})\right) \int \frac{d^{3} u}{(2 \pi \hbar)^{3 / 2}} \exp (-i \boldsymbol{v} \cdot \boldsymbol{u}) \psi(t, \boldsymbol{u}) \\
& =\int d^{3} v \tilde{U}(\boldsymbol{y}-\boldsymbol{v}) \tilde{\psi}(t, \boldsymbol{v})=\mathrm{U} \tilde{\psi}(t, \boldsymbol{y})
\end{align*}
$$
where $\tilde{U}$ is given by (3.9.7). Substituting (3.9.10)-(3.9.12) into (3.9.9), we obtain (3.9.6) immediately.

Using the momentum space wave function $\tilde{\psi}$, we can construct the function
$$
\begin{equation*}
\tilde{\rho}=|\tilde{\psi}|^{2} \tag{3.9.13}
\end{equation*}
$$

The natural question arises whether $\tilde{\rho}$ can be interpreted as the momentum space probability density of the ensemble by virtue of the analogy of its expression to that of the configuration space probability density $\rho$ given in eq. (3.5.26). The answer to this question requires a detailed analysis.

Dimensional analysis of relation (3.9.5) reveals that the momentum space wave function $\tilde{\psi}$ has the dimension of momentum to the power $-3 / 2$. Hence, $\tilde{\rho}$ has the dimension of momentum to the power -3 , the appropriate one for a momentum space probability density. Moreover, $\tilde{\rho}$ is non negative and normalized as required for a probability density.

Proof. We have to show that $\tilde{\rho} \geq 0$ and that
$$
\begin{equation*}
\int d^{3} y \tilde{\rho}=1 \tag{3.9.14}
\end{equation*}
$$

From (3.9.13), it is immediately evident that $\tilde{\rho} \geq 0$. To show (3.9.14), we proceed as follows. Below, we leave the $t$ dependence of $\psi$ and $\tilde{\psi}$ understood for brevity. as usual. From (3.5.32), using (3.5.64), we have
$$
\begin{align*}
1 & =\int d^{3} x|\psi(\boldsymbol{x})|^{2}=\int d^{3} x d^{3} u \delta(\boldsymbol{x}-\boldsymbol{u}) \psi^{*}(\boldsymbol{x}) \psi(\boldsymbol{u})  \tag{3.9.15}\\
& =\int d^{3} x d^{3} u \int \frac{d^{3} y}{(2 \pi \hbar)^{3}} \exp (i(\boldsymbol{x}-\boldsymbol{u}) \cdot \boldsymbol{y} / \hbar) \psi^{*}(\boldsymbol{x}) \psi(\boldsymbol{u}) \\
& =\int d^{3} y\left(\int \frac{d^{3} x}{(2 \pi \hbar)^{3 / 2}} \exp (-i \boldsymbol{y} \cdot \boldsymbol{x} / \hbar) \psi(\boldsymbol{x})\right)^{*} \int \frac{d^{3} u}{(2 \pi \hbar)^{3 / 2}} \exp (-i \boldsymbol{y} \cdot \boldsymbol{u} / \hbar) \psi(\boldsymbol{u}) \\
& =\int d^{3} y|\tilde{\psi}(\boldsymbol{y})|^{2}=\int d^{3} y \tilde{\rho}(\boldsymbol{y})
\end{align*}
$$
(3.9.14) is in this way verified.

So, $\tilde{\rho}$ satisfies the minimal requirements for being a momentum space probability distribution.

Another property that must be fulfilled is conservation of probability. There must exist a momentum space probability current density $\tilde{\boldsymbol{J}}$ such that
$$
\begin{equation*}
\frac{\partial \tilde{\rho}}{\partial t}+\nabla \cdot \tilde{\boldsymbol{\jmath}}=0 \tag{3.9.16}
\end{equation*}
$$
holds. An expression of $\tilde{\boldsymbol{J}}$ can in fact be found,
$$
\begin{equation*}
\tilde{\boldsymbol{\jmath}}(t, \boldsymbol{y})=\frac{1}{2 \pi \hbar} \boldsymbol{\nabla}_{\boldsymbol{y}} \int d^{3} v \frac{1}{|\boldsymbol{y}-\boldsymbol{v}|} \operatorname{Im}\left(\tilde{\psi}^{*}(t, \boldsymbol{v}) \mathrm{U} \tilde{\psi}(t, \boldsymbol{v})\right) \tag{3.9.17}
\end{equation*}
$$

Proof. We divide the argument in two steps.
Step 1. We recall first a basic result of vector calculus. For any scalar field $f$, there is a vector field $\boldsymbol{k}$ whose divergence is $f$,
$$
\begin{equation*}
\boldsymbol{\nabla} \cdot \boldsymbol{k}=f \tag{3.9.18}
\end{equation*}
$$
$\boldsymbol{k}$ in turn enjoys the decomposition of the form
$$
\begin{equation*}
k=\nabla g+\boldsymbol{\nabla} \times \boldsymbol{l} \tag{3.9.19}
\end{equation*}
$$
where $g$ is the scalar field is given by the formula
$$
\begin{equation*}
g(\boldsymbol{z})=-\frac{1}{4 \pi} \int d^{3} w \frac{1}{|\boldsymbol{z}-\boldsymbol{w}|} f(\boldsymbol{w}) \tag{3.9.20}
\end{equation*}
$$
and $\boldsymbol{l}$ is a vector field. Since $\boldsymbol{\nabla} \cdot \boldsymbol{\nabla} \times \boldsymbol{l}=\mathbf{0}$ identically, the actual form of $\boldsymbol{l}$ is immaterial for the validity of (3.9.18).

As $\boldsymbol{\nabla} \cdot \boldsymbol{\nabla} \times \boldsymbol{l}=\mathbf{0}$ trivially, to show (3.9.18) it is sufficient to check that the scalar field $g$ given by eq. (3.9.20) $\boldsymbol{\nabla} \cdot \boldsymbol{\nabla} g=\boldsymbol{\nabla}^{2} g=f$. To begin with, we note that upon changing the integration variable from $\boldsymbol{z}$ to $\boldsymbol{r}=\boldsymbol{w}-\boldsymbol{z} g$ can be expressed as
$$
\begin{equation*}
g(\boldsymbol{z})=-\frac{1}{4 \pi} \int d^{3} r \frac{1}{|\boldsymbol{r}|} f(\boldsymbol{r}+\boldsymbol{z}) \tag{3.9.21}
\end{equation*}
$$

Hence,
$$
\begin{align*}
\boldsymbol{\nabla}_{\boldsymbol{z}}^{2} g(\boldsymbol{z})= & -\frac{1}{4 \pi} \int d^{3} r \frac{1}{|\boldsymbol{r}|} \nabla_{z}^{2} f(\boldsymbol{r}+\boldsymbol{z})  \tag{3.9.22}\\
= & -\frac{1}{4 \pi} \lim _{\epsilon \rightarrow 0+} \int_{|\boldsymbol{r}|>\epsilon} d^{3} r \frac{1}{|\boldsymbol{r}|} \nabla_{\boldsymbol{r}}^{2} f(\boldsymbol{r}+\boldsymbol{z}) \\
= & \frac{1}{4 \pi} \lim _{\epsilon \rightarrow 0+}\left(\int_{|\boldsymbol{r}|>\epsilon} d^{3} r \nabla_{\boldsymbol{r}} \frac{1}{|\boldsymbol{r}|} \cdot \nabla_{\boldsymbol{r}} f(\boldsymbol{r}+\boldsymbol{z})+\oint_{|\boldsymbol{r}|=\epsilon} d^{2} \boldsymbol{r} \cdot \frac{1}{|\boldsymbol{r}|} \nabla_{\boldsymbol{r}} f(\boldsymbol{r}+\boldsymbol{z})\right) \\
= & \frac{1}{4 \pi} \lim _{\epsilon \rightarrow 0+}\left(-\int_{|\boldsymbol{r}|>\epsilon} d^{3} r \boldsymbol{\nabla}_{\boldsymbol{r}}{ }^{2} \frac{1}{|\boldsymbol{r}|} f(\boldsymbol{r}+\boldsymbol{z})\right. \\
& \left.\quad-\oint_{|\boldsymbol{r}|=\epsilon} d^{2} \boldsymbol{r} \cdot \nabla_{\boldsymbol{r}} \frac{1}{|\boldsymbol{r}|} f(\boldsymbol{r}+\boldsymbol{z})+\oint_{|\boldsymbol{r}|=\epsilon} d^{2} \boldsymbol{r} \cdot \frac{1}{|\boldsymbol{r}|} \nabla_{\boldsymbol{r}} f(\boldsymbol{r}+\boldsymbol{z})\right)
\end{align*}
$$

To proceed further, we notice that for $\boldsymbol{r} \neq \mathbf{0}$
$$
\begin{align*}
& \nabla_{\boldsymbol{r}} \frac{1}{|\boldsymbol{r}|}=-\frac{\boldsymbol{r}}{|\boldsymbol{r}|^{3}}  \tag{3.9.23}\\
& \nabla_{\boldsymbol{r}}^{2} \frac{1}{|\boldsymbol{r}|}=\nabla_{\boldsymbol{r}} \cdot \nabla_{\boldsymbol{r}} \frac{1}{|\boldsymbol{r}|}=-\nabla_{\boldsymbol{r}} \cdot \frac{\boldsymbol{r}}{|\boldsymbol{r}|^{3}}=-\frac{\nabla_{r} \cdot \boldsymbol{r}}{|\boldsymbol{r}|^{3}}+\boldsymbol{r} \cdot \frac{3 \boldsymbol{r}}{|\boldsymbol{r}|^{5}}=0 \tag{3.9.24}
\end{align*}
$$
as $\boldsymbol{\nabla}_{\boldsymbol{r}} \cdot \boldsymbol{r}=3$. Setting also $\boldsymbol{r}=\epsilon \boldsymbol{n}$, we have
$$
\begin{align*}
\oint_{|\boldsymbol{r}|=\epsilon} d^{2} \boldsymbol{r} \cdot \nabla_{\boldsymbol{r}} \frac{1}{|\boldsymbol{r}|} f(\boldsymbol{r}+\boldsymbol{z}) & =-\oint_{|\boldsymbol{r}|=\epsilon} d^{2} \boldsymbol{r} \cdot \frac{\boldsymbol{r}}{|\boldsymbol{r}|^{3}} f(\boldsymbol{r}+\boldsymbol{z})  \tag{3.9.25}\\
& =-\oint_{|\boldsymbol{n}|=1} \epsilon^{2} d^{2} \boldsymbol{n} \cdot \frac{\epsilon \boldsymbol{n}}{|\epsilon \boldsymbol{n}|^{3}} f(\epsilon \boldsymbol{n}+\boldsymbol{z}) \\
& =-\oint_{|\boldsymbol{n}|=1} d^{2} n f(\boldsymbol{z})+O(\epsilon)=-4 \pi f(\boldsymbol{z})+O(\epsilon), \\
\oint_{|\boldsymbol{r}|=\epsilon} d^{2} \boldsymbol{r} \cdot \frac{1}{|\boldsymbol{r}|} \nabla_{r} f(\boldsymbol{r}+\boldsymbol{z}) & =\oint_{|\boldsymbol{n}|=1} \epsilon^{2} d^{2} \boldsymbol{n} \cdot \frac{1}{|\epsilon \boldsymbol{n}|} \nabla_{\boldsymbol{z}} f(\epsilon \boldsymbol{n}+\boldsymbol{z})=O(\epsilon) \tag{3.9.26}
\end{align*}
$$

Inserting (3.9.24)-(3.9.26) into (3.9.22), we find
$$
\begin{equation*}
\boldsymbol{\nabla}_{\boldsymbol{z}} \cdot \boldsymbol{\nabla}_{\boldsymbol{z}} g(\boldsymbol{z})=\boldsymbol{\nabla}_{\boldsymbol{z}}^{2} g(\boldsymbol{z})=f(\boldsymbol{z}) \tag{3.9.27}
\end{equation*}
$$
(3.9.18) is in this way shown.

Step 2. Using the momentum space Schroedinger equation (3.9.6), we have
$$
\begin{align*}
\frac{\partial \tilde{\rho}}{\partial t} & =\frac{\partial\left(\tilde{\psi}^{*} \tilde{\psi}\right)}{\partial t}=\frac{\partial \tilde{\psi}^{*}}{\partial t} \psi+\tilde{\psi}^{*} \frac{\partial \tilde{\psi}}{\partial t}  \tag{3.9.28}\\
& =-\frac{1}{i \hbar}\left(\frac{\boldsymbol{y}^{2}}{2 m} \tilde{\psi}+\mathrm{U} \tilde{\psi}\right)^{*} \tilde{\psi}+\frac{1}{i \hbar} \tilde{\psi}^{*}\left(\frac{\boldsymbol{y}^{2}}{2 m} \tilde{\psi}+\mathrm{U} \tilde{\psi}\right)=\frac{2}{\hbar} \operatorname{Im}\left(\tilde{\psi}^{*} \mathrm{U} \tilde{\psi}\right)
\end{align*}
$$

We now leave $\tilde{\boldsymbol{\jmath}}$ be the current density such that
$$
\begin{equation*}
\boldsymbol{\nabla} \cdot \tilde{\boldsymbol{J}}=-\frac{2}{\hbar} \operatorname{Im}\left(\tilde{\psi}^{*} U \tilde{\psi}\right) \tag{3.9.29}
\end{equation*}
$$

By (3.9.18)-(3.9.20), $\tilde{\boldsymbol{J}}$ is given in its minimal form minimal by $\tilde{\boldsymbol{J}}=\boldsymbol{\nabla} \tilde{h}$
$$
\begin{equation*}
h(t, \boldsymbol{y})=\frac{1}{2 \pi \hbar} \int d^{3} v \frac{1}{|\boldsymbol{y}-\boldsymbol{v}|} \operatorname{Im}\left(\tilde{\psi}^{*}(t, \boldsymbol{v}) \mathrm{U} \tilde{\psi}(t, \boldsymbol{v})\right) \tag{3.9.30}
\end{equation*}
$$

In this way, (3.9.16) is satisfied with $\tilde{\boldsymbol{J}}$ given by (3.9.17).

We remark that the expression of $\boldsymbol{\mathcal { J }}$ given by (3.9.17) may be modified by adding a term of the form $\boldsymbol{\nabla} \times \tilde{\boldsymbol{e}}$ for some vector field $\tilde{\boldsymbol{e}}$ without spoiling (3.9.16). From a theoretical point of view, the minimal form is preferable as it makes for the
simplest most predictive theory.
Further evidence for the correctness of the interpretation of $\tilde{\rho}$ as momentum space probability density will come from a momentum space analysis of Wigner quasi probability distribution (cf. sect. 3.5, eq. (3.5.56)) $\varpi . \varpi$ can be expressed in terms of the momentum space wave function
$$
\begin{equation*}
\varpi(t, \boldsymbol{x}, \boldsymbol{y})=\int \frac{d^{3} v}{(2 \pi \hbar)^{3}} \exp (-i \boldsymbol{v} \cdot \boldsymbol{x} / \hbar) \tilde{\psi}^{*}(t, \boldsymbol{y}+\boldsymbol{v} / 2) \tilde{\psi}(t, \boldsymbol{y}-\boldsymbol{v} / 2) \tag{3.9.31}
\end{equation*}
$$

Proof. Using the expansion (3.9.4), we can reshape the expression (3.5.56) of Wigner's distribution as
$$
\begin{align*}
& \varpi(\boldsymbol{x}, \boldsymbol{y})=\int \frac{d^{3} u}{(2 \pi \hbar)^{3}}\left(\int \frac{d^{3} w}{(2 \pi \hbar)^{3 / 2}} \exp (i(\boldsymbol{x}+\boldsymbol{u} / 2) \cdot \boldsymbol{w} / \hbar) \tilde{\psi}(\boldsymbol{w})\right)^{*}  \tag{3.9.32}\\
& \times \int \frac{d^{3} w^{\prime}}{(2 \pi \hbar)^{3 / 2}} \exp \left(i(\boldsymbol{x}-\boldsymbol{u} / 2) \cdot \boldsymbol{w}^{\prime} / \hbar\right) \tilde{\psi}\left(\boldsymbol{w}^{\prime}\right) \exp (i \boldsymbol{u} \cdot \boldsymbol{y} / \hbar) \\
& =\int \frac{d^{3} w}{(2 \pi \hbar)^{3 / 2}} \int \frac{d^{3} w^{\prime}}{(2 \pi \hbar)^{3 / 2}} \tilde{\psi}^{*}(\boldsymbol{w}) \tilde{\psi}\left(\boldsymbol{w}^{\prime}\right) \exp \left(i\left(\boldsymbol{w}^{\prime}-\boldsymbol{w}\right) \cdot \boldsymbol{x} / \hbar\right) \\
& \times \int \frac{d^{3} u}{(2 \pi \hbar)^{3}} \exp \left(i \boldsymbol{u} \cdot\left(\boldsymbol{y}-\left(\boldsymbol{w}+\boldsymbol{w}^{\prime}\right) / 2\right) \hbar\right) \\
& =\int \frac{d^{3} w}{(2 \pi \hbar)^{3 / 2}} \int \frac{d^{3} w^{\prime}}{(2 \pi \hbar)^{3 / 2}} \tilde{\psi}^{*}(\boldsymbol{w}) \tilde{\psi}\left(\boldsymbol{w}^{\prime}\right) \\
& \times \exp \left(i\left(\boldsymbol{w}^{\prime}-\boldsymbol{w}\right) \cdot \boldsymbol{x} / \hbar\right) \delta\left(\boldsymbol{y}-\left(\boldsymbol{w}+\boldsymbol{w}^{\prime}\right) / 2\right),
\end{align*}
$$
where we employed (3.5.67). Now, we carry out the transformation of integration variables $\boldsymbol{w}=\boldsymbol{v}^{\prime}+\boldsymbol{v} / 2, \boldsymbol{w}^{\prime}=\boldsymbol{v}^{\prime}-\boldsymbol{v} / 2$. Its Jacobian is 1 . This can be readily seen as follows. In one dimension, the Jacobian of the transformation $w=v^{\prime}+v / 2$, $w^{\prime}=v^{\prime}-v / 2$ is clearly 1 . As in three dimensions, the transformation does not mix vector components belonging to different dimensions, its Jacobian is $1^{3}=1$. We get so
$$
\begin{align*}
\varpi(\boldsymbol{x}, \boldsymbol{y})= & \int \frac{d^{3} v}{(2 \pi \hbar)^{3}} \exp (-i \boldsymbol{v} \cdot \boldsymbol{x} / \hbar)  \tag{3.9.33}\\
& \times \int d^{3} v^{\prime} \tilde{\psi}^{*}\left(\boldsymbol{v}^{\prime}+\boldsymbol{v} / 2\right) \tilde{\psi}\left(\boldsymbol{v}^{\prime}-\boldsymbol{v} / 2\right) \delta\left(\boldsymbol{y}-\boldsymbol{v}^{\prime}\right) \\
= & \int \frac{d^{3} v}{(2 \pi \hbar)^{3}} \exp (-i \boldsymbol{v} \cdot \boldsymbol{x} / \hbar) \tilde{\psi}^{*}(\boldsymbol{y}+\boldsymbol{v} / 2) \tilde{\psi}(\boldsymbol{y}-\boldsymbol{v} / 2)
\end{align*}
$$

This shows (3.9.31).

Notice that the mathematical structure of expressions (3.5.56) and (3.9.31) is essentially the same, one being obtained from the other by interchanging configuration and momentum space and the corresponding wave function. This reflects the canonical duality of position and momentum that was recalled at the beginning of our discussion.

The momentum space probability density $\tilde{\rho}$ is the momentum space reduction of Wigner's distribution
$$
\begin{equation*}
\int d^{3} x \varpi=\tilde{\rho} \tag{3.9.34}
\end{equation*}
$$

Proof. As usual, we leave $t$ dependence understood. By a calculation totally analogous to that carried out in (3.5.68), we find
$$
\begin{equation*}
\int d^{3} x \varpi(\boldsymbol{x}, \boldsymbol{y})=|\tilde{\psi}(\boldsymbol{y})|^{2} \tag{3.9.35}
\end{equation*}
$$

By (3.9.13), so, $\tilde{\rho}$ is the momentum space reduction of Wigner's distribution.

This finding is an important consistency test further that further strengthens the interpretation of $\tilde{\rho}$ we proposed earlier in this section.

The momentum space expression (3.9.31) of Wigner's quasiprobability distribution allows to obtain general momentum space expressions of the quantum mean $\left\langle Q_{f}\right\rangle$ of the mechanical quantity associated with a given phase function $f$,
$$
\begin{equation*}
\left\langle Q_{f}\right\rangle=\int d^{3} y \tilde{\psi}^{*} \tilde{Q}_{f} \tilde{\psi} \tag{3.9.36}
\end{equation*}
$$
where $\tilde{Q}_{f} \tilde{\psi}$ is the wave function
$$
\begin{equation*}
\tilde{Q}_{f} \tilde{\psi}(t, \boldsymbol{y})=\int d^{3} v \tilde{K}_{f}(\boldsymbol{y}, \boldsymbol{v}) \tilde{\psi}(t, \boldsymbol{v}) \tag{3.9.37}
\end{equation*}
$$
the integral kernel $\tilde{K}_{f}$ in the right hand side being given by
$$
\begin{equation*}
\tilde{K}_{f}(\boldsymbol{y}, \boldsymbol{v})=\int \frac{d^{3} x}{(2 \pi \hbar)^{3}} \exp (-i \boldsymbol{x} \cdot(\boldsymbol{y}-\boldsymbol{v}) / \hbar) f(\boldsymbol{x},(\boldsymbol{y}+\boldsymbol{v}) / 2) \tag{3.9.38}
\end{equation*}
$$
$\tilde{K}_{f}$ is called the momentum space Weyl-Wigner transform of $f$.

Proof. We compute $\left\langle Q_{f}\right\rangle$ inserting expression (3.9.31) of Wigner's quasiprobability distribution into the general relation (3.5.54) and proceeding in a way completely analogous to that followed in the computation in (3.5.81),

Note that the mathematical structure of relations (3.5.78)-(3.5.80) and (3.9.36)(3.9.38) is essentially the same, one being obtained from the other by interchanging configuration and momentum space. This reflects again the canonical duality of position and momentum.

If our hypothesis about $\tilde{\psi}$ is correct, the quantum mean position, momentum, angular momentum, energy of a copy, $\langle\boldsymbol{q}\rangle,\langle\boldsymbol{p}\rangle,\langle\boldsymbol{l}\rangle,\langle H\rangle$, should be given in terms of $\tilde{\psi}$ by expressions analogous to $(3.5 .82)-(3.5 .85)$ with the roles of position and momentum interchanged. In fact, we have
$$
\begin{align*}
\langle\boldsymbol{q}\rangle & =\int d^{3} y \tilde{\psi}^{*} i \hbar \boldsymbol{\nabla} \tilde{\psi}  \tag{3.9.39}\\
\langle\boldsymbol{p}\rangle & =\int d^{3} y \tilde{\psi}^{*} \boldsymbol{y} \tilde{\psi}  \tag{3.9.40}\\
\langle\boldsymbol{l}\rangle & =\int d^{3} y \tilde{\psi}^{*}(-i \hbar \boldsymbol{y} \times \boldsymbol{\nabla} \tilde{\psi})  \tag{3.9.41}\\
\langle H\rangle & =\int d^{3} y \tilde{\psi}^{*} \frac{\boldsymbol{y}^{2}}{2 m} \tilde{\psi}+\int d^{3} y \tilde{\psi}^{*} \mathrm{U} \tilde{\psi} \tag{3.9.42}
\end{align*}
$$

The evident formal similarity of (3.5.82)-(3.5.85) and (3.9.39)-(3.9.42) confirms our guess.

Proof. The calculation proceeds along the same lines as the computation (3.5.82)(3.5.85) in configuration space.

Let $f$ be any phase function. Inserting (3.9.37) into (3.9.36)
$$
\begin{equation*}
\left\langle Q_{f}\right\rangle=\int d^{3} y d^{3} v \tilde{\psi}^{*}(\boldsymbol{y}) \tilde{K}_{f}(\boldsymbol{y}, \boldsymbol{v}) \tilde{\psi}(\boldsymbol{v}) \tag{3.9.43}
\end{equation*}
$$
where we left $t$ dependence understood as usual. We are now going to calculate the kernel $\tilde{K}_{f}$ when $f$ is of the special form
$$
\begin{equation*}
f(\boldsymbol{x}, \boldsymbol{y})=g(\boldsymbol{x}) h(\boldsymbol{y}) \tag{3.9.44}
\end{equation*}
$$
where $g$ is required to be polynomial while $h$ is arbitrary. Using (3.9.38) and exploiting (3.5.64) and noticing that $\boldsymbol{x} \exp (i \boldsymbol{v} \cdot \boldsymbol{x} / \hbar)=-i \hbar \boldsymbol{\nabla}_{\boldsymbol{v}} \exp (i \boldsymbol{v} \cdot \boldsymbol{x} / \hbar)$, by a calculation essentially identical to that in eq. (3.5.88), we find
$$
\begin{equation*}
\tilde{K}_{f}(\boldsymbol{y}, \boldsymbol{v})=g\left(-i \hbar \boldsymbol{\nabla}_{\boldsymbol{v}}\right) \delta(\boldsymbol{v}-\boldsymbol{y}) h((\boldsymbol{y}+\boldsymbol{v}) / 2) \tag{3.9.45}
\end{equation*}
$$

Inserting (3.9.45) into the general mean expression (3.9.43), we obtain
$$
\begin{equation*}
\left\langle Q_{f}\right\rangle=\int d^{3} y \tilde{\psi}^{*}(\boldsymbol{y}) g\left(i \hbar \boldsymbol{\nabla}_{\boldsymbol{v}}\right)[h((\boldsymbol{y}+\boldsymbol{v}) / 2) \tilde{\psi}(\boldsymbol{v})]_{\boldsymbol{v}=\boldsymbol{y}} \tag{3.9.46}
\end{equation*}
$$
proceeding just as in (3.5.89),
We now apply formula (3.9.46) to some relevant special cases. For $g(\boldsymbol{x})$ a component of $\boldsymbol{x}$ and $h(\boldsymbol{y})=1$, we obtain (3.9.39) by a calculation analogous to that in (3.5.91). For $g(\boldsymbol{x})=1$ and $h(\boldsymbol{y})$ a component of $\boldsymbol{y}$, we reach (3.9.40) by a calculation similar to that in (3.5.90). With $g(\boldsymbol{x})$ and $h(\boldsymbol{y})$ respectively a component of $\boldsymbol{x}$ and $\boldsymbol{y}$, (3.9.41) proceeding just as in (3.5.92), (3.5.93). Obtaining (3.9.42) requires a bit more work. Here, we have to consider the functions $g(\boldsymbol{x})=1$ and $h(\boldsymbol{y})=\boldsymbol{y}^{2} / 2$ and $g(\boldsymbol{x})=U(\boldsymbol{x})$ and $h(\boldsymbol{y})=1$ adding which we get the Hamilton function $H(\boldsymbol{x}, \boldsymbol{y})$. In the first case, use of (3.9.46) yields immediately the first term in the right hand side of (3.9.42). In the second case, $g$ is generally non polynomial and (3.9.46) cannot be applied. A direct application of the general formula (3.9.38) with $f(\boldsymbol{x}, \boldsymbol{y})=U(\boldsymbol{x})$ furnishes however immediately the expression
$$
\begin{equation*}
K_{U}(\boldsymbol{y}, \boldsymbol{v})=\tilde{U}(\boldsymbol{y}-\boldsymbol{v}) \tag{3.9.47}
\end{equation*}
$$
where $\tilde{U}$ is defined in (3.9.8). Relation (3.9.36) then gives
$$
\begin{equation*}
\langle U\rangle=\int d^{3} y d^{3} v \tilde{\psi}^{*}(\boldsymbol{y}) \tilde{K}_{U}(\boldsymbol{y}, \boldsymbol{v}) \tilde{\psi}(\boldsymbol{v}) \tag{3.9.48}
\end{equation*}
$$
$$
=\int d^{3} y \tilde{\psi}^{*}(\boldsymbol{y}) \int d^{3} v \tilde{U}(\boldsymbol{y}-\boldsymbol{v}) \tilde{\psi}(\boldsymbol{v})=\int d^{3} y \tilde{\psi}^{*}(\boldsymbol{y}) \mathrm{U} \tilde{\psi}(\boldsymbol{y})
$$
which is the second term in the right hand side of (3.9.42).

In sect. (3.6), we have seen that in the configuration space formulation of ensemble theory states and observables enjoy a mathematical modellization whereby a) the state of the ensemble is encoded by a normalized configuration space wave functions $\psi$ and $b$ ) for any phase function $f$ the mechanical quantity $Q_{f}$ is represented by a configuration space Hermitian linear operator $\mathrm{Q}_{f}$. The analysis carried out in this section shows that the in the momentum space formulation of ensemble theory states and observables undergo a corresponding mathematical modellization according to which $a$ ) the state of the ensemble is encoded by a normalized momentum space wave functions $\psi$ and $b$ ) for any phase function $f$ the mechanical quantities $Q_{f}$ is represented by a momentum space Hermitian linear operator $\tilde{Q}_{f}$. As we shall see in later chapters, there are deep reasons for this.

As it turns out, while the possibility of a dual momentum space formulation of ensemble theory is of great theoretical salience, the configuration space formulation is technically more advantageous. The reason for this is not difficult to see. In the configuration space approach, the Schroedinger equation is a pure partial differential equation. For this type of differential problems, there exist powerful solution techniques which can be usefully employed. Conversely, in the momentum space approach, by virtue of the non locality of the potential term (see eq. (3.9.7)), the Schroedinger equation is an integral differential equation, an arduous type of analytical problem.

\subsection*{3.10. Position-momentum uncertainty relation}

The position-momentum uncertainty relation expresses formally a typical quantum phenomenon and epitomizes the departure of quantum physiscs from classical. Whenever, for a certain state of the ensemble, copies are sharply localized in configuration space, they are loosely so in momentum space and viceversa. This property can be formulated more quantitatively in the language of statistics as follows.

The mean value $\bar{q}_{i}$ and the standard deviation $\Delta q_{i}$ of the position coordinate $q_{i}$ of the copies of the particle are expressible in terms of the configuration space probability density $\rho$ of the ensemble according to
$$
\begin{align*}
& \bar{q}_{i}=\left\langle q_{i}\right\rangle=\int d^{3} x \rho x_{i}  \tag{3.10.1}\\
& \Delta q_{i}=\left\langle\left(q_{i}-\left\langle q_{i}\right\rangle\right)^{2}\right\rangle^{1 / 2}=\left(\int d^{3} x \rho\left(x_{i}-\left\langle q_{i}\right\rangle\right)^{2}\right)^{1 / 2} \tag{3.10.2}
\end{align*}
$$

Similarly, the mean value $\bar{p}_{i}$ and the standard deviation $\Delta p_{i}$ of the momentum component $p_{i}$ of the copies are expressible in terms of the ensemble's momentum space probability density $\tilde{\rho}$ according to
$$
\begin{align*}
& \bar{p}_{i}=\left\langle p_{i}\right\rangle=\int d^{3} y \tilde{\rho} y_{i}  \tag{3.10.3}\\
& \Delta p_{i}=\left\langle\left(p_{i}-\left\langle p_{i}\right\rangle\right)^{2}\right\rangle^{1 / 2}=\left(\int d^{3} y \tilde{\rho}\left(y_{i}-\left\langle p_{i}\right\rangle\right)^{2}\right)^{1 / 2} \tag{3.10.4}
\end{align*}
$$

In quantum physics, standard deviations are commonly called uncertainties. The position-momentum uncertainty relation states that
$$
\begin{equation*}
\Delta q_{i} \Delta p_{j} \geq \frac{\hbar \delta_{i j}}{2} \tag{3.10.5}
\end{equation*}
$$
(W. Heisenberg, 1927, and E. H. Kennard, 1927).

Proof. The proof is based on the classic Cauchy-Schwarz inequality, the use of which requires suitably expressing $\Delta q_{i}, \Delta p_{i}$ as integrals on the same space, e. g configuration space.

We use relation (3.5.89) which furnishes the mean $\left\langle Q_{f}\right\rangle$ of the mechanical quantity $Q_{f}$ associated with a phase function $f$ factorized as the product of a configuration space function $g$ and a polynomial momentum space function $h$. To compute the uncertainty $\Delta q_{i}{ }^{2}$ of $q_{i}$, we take $g(\boldsymbol{x})=\left(x_{i}-\left\langle q_{i}\right\rangle\right)^{2}$ and $h(\boldsymbol{y})=1$. We find then that $\Delta q_{i}{ }^{2}$ can be put in the form
$$
\begin{align*}
\Delta q_{i}{ }^{2} & =\int d^{3} x \psi^{*}\left(x_{i}-\left\langle q_{i}\right\rangle\right)^{2} \psi  \tag{3.10.6}\\
& =\int d^{3} x\left(\left(x_{i}-\left\langle q_{i}\right\rangle\right) \psi\right)^{*}\left(x_{i}-\left\langle q_{i}\right\rangle\right) \psi=\int d^{3} x\left|x_{i} \psi-\left\langle q_{i}\right\rangle \psi\right|^{2}
\end{align*}
$$

To compute the uncertainty $\Delta p_{i}{ }^{2}$ of $p_{i}$, we take $g(\boldsymbol{x})=1$ and $h(\boldsymbol{y})=\left(y_{i}-\left\langle p_{i}\right\rangle\right)^{2} \cdot \Delta p_{i}{ }^{2}$ is then found to be given by
$$
\begin{align*}
\Delta p_{i}{ }^{2} & =\int d^{3} x \psi^{*}\left(-i \hbar \nabla_{i}-\left\langle p_{i}\right\rangle\right)^{2} \psi  \tag{3.10.7}\\
& =\int d^{3} x\left(\left(-i \hbar \nabla_{i}-\left\langle p_{i}\right\rangle\right) \psi\right)^{*}\left(-i \hbar \nabla_{i}-\left\langle p_{i}\right\rangle\right) \psi=\int d^{3} x\left|-i \hbar \nabla_{i} \psi-\left\langle p_{i}\right\rangle \psi\right|^{2}
\end{align*}
$$
where in the second step we performed an integration by parts. From (3.10.6), (3.10.7), using the Cauchy-Schwarz inequality, we have then
$$
\begin{align*}
\Delta q_{i} \Delta p_{j} & =\left(\int d^{3} x\left|x_{i} \psi-\left\langle q_{i}\right\rangle \psi\right|^{2}\right)^{1 / 2}\left(\int d^{3} x\left|-i \hbar \nabla_{j} \psi-\left\langle p_{j}\right\rangle \psi\right|^{2}\right)^{1 / 2}  \tag{3.10.8}\\
& \geq\left|\int d^{3} x\left(x_{i} \psi-\left\langle q_{i}\right\rangle \psi\right)^{*}\left(-i \hbar \nabla_{j} \psi-\left\langle p_{j}\right\rangle \psi\right)\right| \\
& \geq\left|\operatorname{Im} \int d^{3} x\left(x_{i} \psi-\left\langle q_{i}\right\rangle \psi\right)^{*}\left(-i \hbar \nabla_{j} \psi-\left\langle p_{j}\right\rangle \psi\right)\right| \\
& =\hbar\left|\operatorname{Re} \int d^{3} x\left(x_{i}-\left\langle q_{i}\right\rangle\right) \psi^{*} \nabla_{j} \psi\right|=\frac{\hbar}{2}\left|\int d^{3} x\left(x_{i}-\left\langle q_{i}\right\rangle\right)\left(\psi^{*} \nabla_{j} \psi+\nabla_{j} \psi^{*} \psi\right)\right| \\
& \left.=\frac{\hbar}{2}\left|\int d^{3} x \nabla_{j}\left(x_{i}-\left\langle q_{i}\right\rangle\right) \psi^{*} \psi\right|=\left.\frac{\hbar}{2}\left|\int d^{3} x \delta_{i j}\right| \psi\right|^{2} \right\rvert\,=\frac{\hbar \delta_{i j}}{2}
\end{align*}
$$
showing (3.10.5).

Let us now examine the import of the uncertainty relation (3.10.5). The
uncertainties $\Delta q_{i}$ and $\Delta p_{i}$ describe how the values of the position and momentum components $q_{i}$ and $p_{i}$ of the copies of the particle in the ensemble are distributed around their mean values $\bar{q}_{i}$ and $\bar{p}_{i}$, respectively. The smaller $\Delta q_{i}$, the closer the copies' values of $q_{i}$ are to $\bar{q}_{i}$. The smaller $\Delta p_{i}$, the closer the copies' values of $p_{i}$ are to $\bar{p}_{i}$ as shown in fig. 3.10.1. According to (3.10.5), $\Delta q_{i}, \Delta p_{i}$ cannot be simultaneously arbitrarily small, their product being bounded below by $\hbar / 2$. In particular, when $\Delta q_{i} \Delta p_{i} \simeq \hbar$, saturating the bound, the smaller $\Delta q_{i}$, the larger $\Delta p_{i}$ and viceversa. This fact is intimately related to the undulatory nature of the ensemble discovered by Schroedinger. Using the second Planck-Einstein relation (1.12.1b), we can cast (3.10.5) in the form $\Delta q_{i} \Delta \kappa_{i} \geq 1 / 2$. A relation of this form emerged in the wave theory of light long before the advent of quantum physics. It relates the width $\Delta q_{i}$ of a wave packet and the spread $\Delta \kappa_{i}$ of its monochromatic components in wave vector space and indicates that a compact wave packet cannot be monochromatic even approximately.

Recall that the ensemble formulation of quantum physics is just a way of describing the dynamics of a quantum particle (cf. sect. 3.5). The state of the ensemble corresponds to a state of the particle. In Schroedinger's theory, the state is encoded in the configuration space wave function $\psi$ and has the basic property that repeated measurements of the particle's position do not yield in general one and the same value each time, but a range of values occurring with probabilities determined by $\psi$. The analysis of the sect. 3.9 reveals us that the state is also encoded in the momentum space wave function $\tilde{\psi}$ and has the equally basic property that repeated measurements of the particle's momentum do not yield in general one and the same value each time, but a range of values occurring with probabilities determined by $\tilde{\psi}$. The uncertainties $\Delta q_{i}, \Delta p_{i}$ of the position and momentum coordinates $q_{i}$ and $p_{i}$ measure precisely the degree of certainty of the values of $q_{i}$ and $p_{i}$ in the state. The uncertainty relation (3.10.5) states that there is no quantum state where $q_{i}$ and $p_{i}$ have simultaneously certain values. The

![](https://cdn.mathpix.com/cropped/2024_09_22_5d1e855547710648961eg-0291.jpg?height=1186&width=1354&top_left_y=540&top_left_x=383)

Figure 3.10.1. The uncertainties $\Delta q_{a i}, \Delta q_{b i}$ of the configurazion space Guassian wave packets $(a)$ and (b) satisfy $\Delta q_{a i}>\Delta q_{b i}$. The uncertainties $\Delta p_{c i}, \Delta p_{d i}$ of the associated momentum space Guassian wave packets $(c)$ and $(d)$ satisfy correspondingly $\Delta p_{c i}<\Delta p_{d i}$ as expected by the uncertainty relations.
more certain the value of $q_{i}$, the more uncertain that of $p_{i}$ and viceversa. This impossibility also rules out that a quantum particle has a well defined trajectory, since the existence of this requires simultaneously certain values of $q_{i}$ and $p_{i}$.

\subsection*{3.11. Stationary states}

As we remarked at the end of sect. 3.5, since the wave function $\psi$ determines the probability density and current density $\rho, \boldsymbol{j}$ of the ensemble as well the mean values $\left\langle Q_{f}\right\rangle$ of all mechanical quantities $Q_{f}$ through (3.5.26), (3.5.27) and (3.5.78), respectively, $\psi$ encodes the state of the statistical ensemble. So,
the time dependence of $\psi$ determines the ensemble's dynamics.

Understanding the time evolution of $\psi$ is therefore a necessary condition for an effective explanation of the ensemble's dynamics.

It is natural to wonder whether the Schroedinger equation (3.4.14) has solutions analogous to the monochromatic waves of undulatory optics (cf. sect. 1.3), that is with a harmonic time dependence
$$
\begin{equation*}
\psi_{t}=\exp (-i w t / \hbar) \psi_{0} \tag{3.11.1}
\end{equation*}
$$
$w$ is a constant with the dimension of energy. The frequency of the wave described by $\psi$ is $\omega=w / \hbar$ is agreement with the Planck-Einstein relation (1.12.1a). In order $\psi$ to obey (3.4.14), $\psi_{0}$ must satisfy
$$
\begin{equation*}
-\frac{\hbar^{2}}{2 m} \nabla^{2} \psi_{0}+U \psi_{0}=w \psi_{0} \tag{3.11.2}
\end{equation*}
$$

Proof. Demanding that $\psi_{t}$, as given by (3.11.1), obeys the Schroedinger equation (3.4.14), we find
$$
\begin{align*}
0 & =i \hbar \frac{\partial \psi_{t}}{\partial t}+\frac{\hbar^{2}}{2 m} \nabla^{2} \psi_{t}-U \psi_{t}  \tag{3.11.3}\\
& =i \hbar \frac{d}{d t} \exp (-i w t / \hbar) \psi_{0}+\exp (-i w t / \hbar)\left(\frac{\hbar^{2}}{2 m} \nabla^{2} \psi_{0}-U \psi_{0}\right) \\
& =\exp (-i w t / \hbar)\left(w \psi_{0}+\frac{\hbar^{2}}{2 m} \nabla^{2} \psi_{0}-U \psi_{0}\right)
\end{align*}
$$

It follows that $\psi_{0}$ obeys (3.11.2) as claimed.

Eq. (3.11.2) is called time independent Schroedinger equation to distinguish it from the time dependent Schroedinger equation (3.4.14) from which it derives. Just as the wave function $\psi$ given in (3.11.1) describes a quantum analog of a monochromatic wave of classical optics, the Schroedinger equation (3.11.2) is the counterpart of the Helmholtz wave equation (1.3.2).

When $\psi$ has the simple harmonic time dependence shown in (3.11.1), the probability density and current density $\rho, \boldsymbol{j}$ and all the mean values $\langle Q\rangle$ turn out to be time independent.

Proof. By (3.5.26), (3.5.27) and (3.5.78), the densities $\rho, \boldsymbol{j}$ and any mean value $\langle Q\rangle$ depend on $t$ through the wave function $\psi_{t}$, which enters in combinations of the form $\psi_{t}{ }^{*} \mathrm{~A} \psi_{t}$ with A some linear operator acting on wave functions. When $\psi_{t}$ is of the form (3.11.1), we have
$$
\begin{equation*}
\psi_{t}^{*} \mathrm{~A} \psi_{t}=\psi_{0}^{*} \exp \left(\frac{i w t}{\hbar}\right) \mathrm{A} \exp \left(-\frac{i w t}{\hbar}\right) \psi_{0}=\psi_{0}^{*} \mathrm{~A} \psi_{0} \tag{3.11.4}
\end{equation*}
$$
$\psi_{t}{ }^{*} \mathrm{~A} \psi_{t}$ is so time independent. Thus, $\rho, \boldsymbol{j}$ and all mean values $\langle Q\rangle$ are all time independent, as claimed.

This shows that the time dependence of $\psi$ is physically inconsequential and thus fictitious. We conclude that
the wave function $\psi$ of the form (3.11.1) with $\psi_{0}$ satisfying the time independent Schroedinger equation (3.11.2) effectively describes a stationary state of the ensemble, that is a state which does not evolve as time elapses.

For the stationary state described by the wave function $\psi$ of eq. (3.11.1) the mean energy of a copy is given by
$$
\begin{equation*}
\langle H\rangle=w \tag{3.11.5}
\end{equation*}
$$

Furthermore, the mean quadratic energy deviation of a copy vanishes,
$$
\begin{equation*}
\left\langle(H-\langle H\rangle)^{2}\right\rangle=0 \tag{3.11.6}
\end{equation*}
$$

Proof. The mean value $\langle Q\rangle$ of any quantity $Q$ is given by eq. (3.5.78), where Q is a suitable operator depending on $Q$ and $\psi$ is the wave function of the ensemble. In the present case, since the ensemble's wave function $\psi$ is of the form (3.11.1), $\langle Q\rangle$ is time independent, as shown earlier. So, we can compute the mean value inserting in (3.5.78) the initial wave function $\psi_{0}$.

For the energy $H$, the operator H is given by (3.6.11). By (3.11.2), we have then
$$
\begin{equation*}
\mathrm{H} \psi_{0}=-\frac{\hbar^{2}}{2 m} \nabla^{2} \psi_{0}+U \psi_{0}=w \psi_{0} \tag{3.11.7}
\end{equation*}
$$

Let us compute the mean energy $\langle H\rangle$ of a copy in the ensemble. By what said above, using (3.11.7), we find that
$$
\begin{equation*}
\langle H\rangle=\int d^{3} x \psi_{0}{ }^{*} \mathrm{H} \psi_{0}=\int d^{3} x \psi_{0}{ }^{*} w \psi_{0}=w \int d^{3} x\left|\psi_{0}\right|^{2}=w \tag{3.11.8}
\end{equation*}
$$

This shows (3.11.5).
Next, let us compute the mean quadratic energy deviation $\left\langle(\Delta H)^{2}\right\rangle$ of a copy in the ensemble. To begin with, we note that
$$
\begin{equation*}
\left\langle(H-\langle H\rangle)^{2}\right\rangle=\left\langle H^{2}\right\rangle-\langle H\rangle^{2} \tag{3.11.9}
\end{equation*}
$$
as follows from the calculation
$$
\begin{align*}
\left\langle(H-\langle H\rangle)^{2}\right\rangle=\left\langle H^{2}-\right. & \left.2\langle H\rangle H+\langle H\rangle^{2}\right\rangle  \tag{3.11.10}\\
& =\left\langle H^{2}\right\rangle-2\langle H\rangle\langle H\rangle+\langle H\rangle^{2}=\left\langle H^{2}\right\rangle-\langle H\rangle^{2}
\end{align*}
$$

We thus need the mean values $\langle H\rangle,\left\langle H^{2}\right\rangle$. We have already computed $\langle H\rangle$ in (3.11.8). To compute $\left\langle H^{2}\right\rangle$, we need the operator corresponding to the square energy $H^{2}$. It is reasonable to assume that this is given by the square $\mathrm{H}^{2}$ of the operator H . Therefore,
$$
\begin{align*}
\left\langle H^{2}\right\rangle=\int d^{3} x \psi_{0}{ }^{*} \mathrm{HH} \psi_{0} & =\int d^{3} x \psi_{0}{ }^{*} \mathrm{H} w \psi_{0}  \tag{3.11.11}\\
= & \int d^{3} x \psi_{0}^{*} w w \psi_{0}=w^{2} \int d^{3} x\left|\psi_{0}\right|^{2}=w^{2}
\end{align*}
$$

Inserting (3.11.8), (3.11.11) into (3.11.9), we obtain (3.11.6).

The fact that the mean energy $\langle H\rangle$ equals the energy value $w$ and the mean quadratic energy deviation $\left\langle(H-\langle H\rangle)^{2}\right\rangle$ vanishes means that
in the stationary state of the ensemble corresponding to a wave function $\psi$ of the form (3.11.1) the energy of every copy of the particle is precisely $w$.

In this respect, energy and position have therefore sharply different behaviors: while the copies have different positions distributed according to a certain probability density, they all have the same energy.

In sect. 3.5, we have seen that the statistical ensemble is a theoretical construct describing a quantum particle and its distinctly quantum features. The state of the ensemble, which is is encoded in the wave function $\psi$, corresponds to the state of the particle. Unlike in classical physics, this state is probabilistic: in the state, a generic observable does not have a unique and predictable value, but multiple and unpredictable ones the probability of each of which is determined by the state. The above discussion leads us to the following conclusions.

A wave function $\psi$ of the form (3.11.1) encodes a stationary quantum state of the particle. In this state, the value of the particle's energy is certain and equal to $w$.

In isolated atoms and molecules as well as of nuclei, the mean values of all relevant quantities are independent from time indicating that the states these systems are in are stationary. Their wave functions, so, are of the form (3.18.11). Further, the possible values of their energy are precisely their discrete energy levels. The resolution of the Schroedinger problem is therefore crucial in atomic and molecular and in nuclear physics.

In the general case, where the time dependence of $\psi$ is not harmonic, the ensemble is instead in a non stationary state, because the densities $\rho, \boldsymbol{j}$ as well
as any mean value $\left\langle Q_{f}\right\rangle$ do depend on time. This case will be studied in greater detail in the next section.

\subsection*{3.12. Quasistationarity and time-energy uncertainty relation}

It is possible to analyze the time evolution of the wave function from a more basic point of view, that does not rely on the assumption of a simple harmonic time dependence as in sect. 3.11.

Let $\psi$ be any time dependent wave function. Then, $\psi$ can be expressed as a superposition of harmonic time dependent wave functions all possible frequency $w / \hbar$, where $w$ is an energy parameter. Explicitly,
$$
\begin{equation*}
\psi_{t}=\int_{-\infty}^{\infty} \frac{d w}{2 \pi \hbar} \zeta_{w} \exp \left(-\frac{i w t}{\hbar}\right) \tag{3.12.1}
\end{equation*}
$$
where $\zeta_{w}$ is the time independent wave function
$$
\begin{equation*}
\zeta_{w}=\int_{-\infty}^{\infty} d t \psi_{t} \exp \left(\frac{i w t}{\hbar}\right) \tag{3.12.2}
\end{equation*}
$$

Proof. The delta function $\delta(t)$ has the integral representation
$$
\begin{equation*}
\int_{-\infty}^{\infty} \frac{d w}{2 \pi \hbar} \exp \left(\frac{i w t}{\hbar}\right)=\delta(t) \tag{3.12.3}
\end{equation*}
$$
by an argument analogous to that leading to (3.5.57). Using (3.12.3), we find
$$
\begin{align*}
\int_{-\infty}^{\infty} \frac{d w}{2 \pi \hbar} \zeta_{w} \exp \left(-\frac{i w t}{\hbar}\right) & =\int_{-\infty}^{\infty} \frac{d w}{2 \pi \hbar}\left[\int_{-\infty}^{\infty} d t^{\prime} \psi_{t^{\prime}} \exp \left(\frac{i w t^{\prime}}{\hbar}\right)\right] \exp \left(-\frac{i w t}{\hbar}\right) \\
& =\int_{-\infty}^{\infty} d t^{\prime} \psi_{t^{\prime}} \int_{-\infty}^{\infty} \frac{d w}{2 \pi \hbar} \exp \left(\frac{i w\left(t^{\prime}-t\right)}{\hbar}\right)  \tag{3.12.4}\\
& =\int_{-\infty}^{\infty} d t^{\prime} \psi_{t^{\prime}} \delta\left(t^{\prime}-t\right)=\psi_{t}
\end{align*}
$$
showing (3.12.1).

Suppose now that the wave function $\psi$ satisfies the Schroedinger equation (3.4.14). Then, the $\zeta_{w}$ satisfy the time independent Schroedinger equation (3.11.2)
$$
\begin{equation*}
-\frac{\hbar^{2}}{2 m} \nabla^{2} \zeta_{w}+U \zeta_{w}=w \zeta_{w} \tag{3.12.5}
\end{equation*}
$$

Further, the $\zeta_{w}$ satisfies the generalized orthonormality relation
$$
\begin{equation*}
\int d^{3} x \zeta_{w^{\prime}}{ }^{*} \zeta_{w}=(2 \pi \hbar)^{2} \varpi(w) \delta\left(w^{\prime}-w\right) \tag{3.12.6}
\end{equation*}
$$
where $\varpi(w)$ is an energy function such that $\varpi(w) \geq 0$ and obeying the sum rule
$$
\begin{equation*}
\int_{-\infty}^{\infty} d w \varpi(w)=1 \tag{3.12.7}
\end{equation*}
$$

Proof. Indeed, from (3.12.2), by the Schroedinger equation (3.12.3), we have
$$
\begin{align*}
-\frac{\hbar^{2}}{2 m} \nabla^{2} \zeta_{w}+U \zeta_{w} & =-\frac{\hbar^{2}}{2 m} \nabla^{2} \int_{-\infty}^{\infty} d t \psi_{t} \exp \left(\frac{i w t}{\hbar}\right)+U \int_{-\infty}^{\infty} d t \psi_{t} \exp \left(\frac{i w t}{\hbar}\right) \\
& =\int_{-\infty}^{\infty} d t\left(-\frac{\hbar^{2}}{2 m} \nabla^{2} \psi_{t}+U \psi_{t}\right) \exp \left(\frac{i w t}{\hbar}\right)  \tag{3.12.8}\\
& =\int_{-\infty}^{\infty} d t i \hbar \frac{\partial \psi_{t}}{\partial t} \exp \left(\frac{i w t}{\hbar}\right) \\
& =-\int_{-\infty}^{\infty} d t \psi_{t} i \hbar \frac{\partial}{\partial t} \exp \left(\frac{i w t}{\hbar}\right) \\
& =\int_{-\infty}^{\infty} d t \psi_{t} \exp \left(\frac{i w t}{\hbar}\right) w=\zeta_{w} w
\end{align*}
$$
showing (3.12.5). Next, using that $\zeta_{w}$ solves (3.12.5), we compute
$$
\begin{align*}
\left(w^{\prime}-w\right) \int d^{3} x \zeta_{w^{\prime}}{ }^{*} \zeta_{w}= & \int d^{3} x\left(w^{\prime} \zeta_{w^{\prime}}\right)^{*} \zeta_{w}-\int d^{3} x \zeta_{w^{\prime}}{ }^{*} w \zeta_{w}  \tag{3.12.9}\\
= & \int d^{3} x\left(-\frac{\hbar^{2}}{2 m} \nabla^{2} \zeta_{w^{\prime}}+U \zeta_{w^{\prime}}\right)^{*} \zeta_{w} \\
& \quad-\int d^{3} x \zeta_{w^{\prime}}{ }^{*}\left(-\frac{\hbar^{2}}{2 m} \nabla^{2} \zeta_{w}+U \zeta_{w}\right) \\
= & \frac{\hbar^{2}}{2 m} \int d^{3} x\left(\zeta_{w^{\prime}}{ }^{*} \nabla^{2} \zeta_{w}-\nabla^{2} \zeta_{w^{\prime}}{ }^{*} \zeta_{w}\right) \\
= & \frac{\hbar^{2}}{2 m} \int d^{3} x \boldsymbol{\nabla} \cdot\left(\zeta_{w^{\prime}}{ }^{*} \nabla \zeta_{w}-\nabla \zeta_{w^{\prime}}{ }^{*} \zeta_{w}\right)=0
\end{align*}
$$

Therefore, for $w \neq w^{\prime}, \int d^{3} x \zeta_{w^{\prime}}{ }^{*} \zeta_{w}=0$. From the normalization relation (3.5.32), using (3.12.2) and (3.12.5), we also have
$$
\begin{align*}
1= & \int d^{3} x\left|\psi_{0}\right|^{2}  \tag{3.12.10}\\
& =\int d^{3} x\left[\int_{-\infty}^{\infty} \frac{d w^{\prime}}{2 \pi \hbar} \zeta_{w^{\prime}}\right]^{*} \int_{-\infty}^{\infty} \frac{d w}{2 \pi \hbar} \zeta_{w}=\int_{-\infty}^{\infty} \frac{d w^{\prime}}{2 \pi \hbar} \int_{-\infty}^{\infty} \frac{d w}{2 \pi \hbar} \int d^{3} x \zeta_{w^{\prime}}{ }^{*} \zeta_{w}
\end{align*}
$$

For $w^{\prime}=w$, so, $\int d^{3} x \zeta_{w^{\prime}}{ }^{*} \zeta_{w}$ must have a Dirac delta singularity, else the last integral would vanish. So, (3.12.6) holds for some function $\varpi(w) . \varpi(w) \geq 0$ as the integrand in left hand side of (3.12.6) is real non negative for $w^{\prime}=w$. Now, by (3.12.10),
$$
1=\int_{-\infty}^{\infty} \frac{d w^{\prime}}{2 \pi \hbar} \int_{-\infty}^{\infty} \frac{d w}{2 \pi \hbar}(2 \pi \hbar)^{2} \varpi(w) \delta\left(w^{\prime}-w\right)=\int_{-\infty}^{\infty} d w \varpi(w)
$$

This shows (3.12.7).

It turns out that $\varpi(w)$ is the probability distribution of energy values. In fact the mean value of the energy of the copies of the ensemble is
$$
\begin{equation*}
\bar{w}:=\langle H\rangle=\int_{-\infty}^{\infty} d w w \varpi(w) \tag{3.12.11}
\end{equation*}
$$

The mean quadratic deviation of the energy of the copies is correspondingly
$$
\begin{equation*}
\Delta w^{2}:=\left\langle(H-\langle H\rangle)^{2}\right\rangle=\int_{-\infty}^{\infty} d w(w-\bar{w})^{2} \varpi(w) \tag{3.12.12}
\end{equation*}
$$

Proof. Inserting (3.12.2) into the energy mean value expression (3.13.28), we find
$$
\begin{align*}
\langle H\rangle_{t}= & \int d^{3} x \psi_{t}^{*}\left(-\frac{\hbar^{2}}{2 m} \nabla^{2} \psi_{t}+U \psi_{t}\right)  \tag{3.12.13}\\
= & \int d^{3} x\left[\int_{-\infty}^{\infty} \frac{d w^{\prime}}{2 \pi \hbar} \zeta_{w^{\prime}} \exp \left(-\frac{i w^{\prime} t}{\hbar}\right)\right]^{*} \\
& \times\left[-\frac{\hbar^{2}}{2 m} \nabla^{2} \int_{-\infty}^{\infty} \frac{d w}{2 \pi \hbar} \zeta_{w} \exp \left(-\frac{i w t}{\hbar}\right)+U \int_{-\infty}^{\infty} \frac{d w}{2 \pi \hbar} \zeta_{w} \exp \left(-\frac{i w t}{\hbar}\right)\right] \\
= & \int_{-\infty}^{\infty} \frac{d w^{\prime}}{2 \pi \hbar} \int_{-\infty}^{\infty} \frac{d w}{2 \pi \hbar} \int d^{3} x \zeta_{w^{\prime}}{ }^{*}\left(-\frac{\hbar^{2}}{2 m} \nabla^{2} \zeta_{w}+U \zeta_{w}\right) \exp \left(\frac{i\left(w^{\prime}-w\right) t}{\hbar}\right) \\
= & \int_{-\infty}^{\infty} \frac{d w^{\prime}}{2 \pi \hbar} \int_{-\infty}^{\infty} \frac{d w}{2 \pi \hbar} \int d^{3} x \zeta_{w^{\prime}}{ }^{*} w \zeta_{w} \exp \left(\frac{i\left(w^{\prime}-w\right) t}{\hbar}\right) \\
= & \int_{-\infty}^{\infty} \frac{d w^{\prime}}{2 \pi \hbar} \int_{-\infty}^{\infty} \frac{d w}{2 \pi \hbar} w(2 \pi \hbar)^{2} \varpi(w) \delta\left(w^{\prime}-w\right) \exp \left(\frac{i\left(w^{\prime}-w\right) t}{\hbar}\right) \\
= & \int_{-\infty}^{\infty} d w w \varpi(w)
\end{align*}
$$

This shows (3.12.11) and identifies $\varpi(w)$ as the energy probability distribution function. Based on this, (3.12.12) is obvious.

We observe that $\bar{w}$ and $\Delta w^{2}$ are both time independent.
We consider now the important case where $\psi$ represents a stationary state of the ensemble. In accordance with (3.11.1), we have so
$$
\begin{equation*}
\psi_{t}=\exp \left(-\frac{i w_{0} t}{\hbar}\right) \psi_{0} \tag{3.12.14}
\end{equation*}
$$
where $w_{0}$ is some energy value. Then,
$$
\begin{equation*}
\zeta_{w}=2 \pi \hbar \delta\left(w-w_{0}\right) \psi_{0} \tag{3.12.15}
\end{equation*}
$$

Proof. Inserting (3.12.14) into (3.12.2) and using relation (3.12.3) and the further relation $\delta(\omega \tau)=\tau^{-1} \delta(\omega)$ for $\tau>0$, we have
$$
\begin{equation*}
\zeta_{w}=\int_{-\infty}^{\infty} d t \exp \left(\frac{i\left(w-w_{0}\right) t}{\hbar}\right) \psi_{0}=2 \pi \hbar \delta\left(w-w_{0}\right) \psi_{0} \tag{3.12.16}
\end{equation*}
$$

This shows (3.12.15).

The energy distribution function associated with the stationary state is given by
$$
\begin{equation*}
\varpi(w)=\delta\left(w-w_{0}\right) \tag{3.12.17}
\end{equation*}
$$

Proof. From (3.12.15), recalling that $\int d^{3} x\left|\psi_{0}\right|^{2}=1$, we find
$$
\begin{align*}
& \int d^{3} x \zeta_{w^{\prime}}{ }^{*} \zeta_{w}=\int d^{3} x 2 \pi \hbar \delta\left(w^{\prime}-w_{0}\right) \psi_{0}{ }^{*} 2 \pi \hbar \delta\left(w-w_{0}\right) \psi_{0}  \tag{3.12.18}\\
& =(2 \pi \hbar)^{2} \delta\left(w^{\prime}-w_{0}\right) \delta\left(w-w_{0}\right) \int d^{3} x\left|\psi_{0}\right|^{2}=(2 \pi \hbar)^{2} \delta\left(w-w_{0}\right) \delta\left(w^{\prime}-w\right)
\end{align*}
$$

Comparing with (3.12.6), we deduce (3.12.17).

Inserting (3.12.17) into (3.12.11), (3.12.12), we find readily that the values of the mean energy and mean quadratic energy deviation for the stationary state,
$$
\begin{align*}
& \bar{w}=w_{0}  \tag{3.12.19}\\
& \Delta w=0 \tag{3.12.20}
\end{align*}
$$

Proof. Combining (3.12.17) and (3.12.11), we get
$$
\begin{equation*}
\bar{w}=\int_{-\infty}^{\infty} d w w \delta\left(w-w_{0}\right)=w_{0} \tag{3.12.21}
\end{equation*}
$$
showing (3.12.19). Similarly, combining (3.12.17) and (3.12.12), we get
$$
\begin{equation*}
\Delta w=\int_{-\infty}^{\infty} d w\left(w-w_{0}\right)^{2} \delta\left(w-w_{0}\right)=0 \tag{3.12.22}
\end{equation*}
$$
proving $(3.12 .20)$.

By (3.12.19), (3.12.20), when the ensemble is in the stationary state corresponding to $\psi$, all copies of the ensemble have energy equal to $w_{0}$. This means that a measurement of energy on the quantum particle underlying the ensemble when it in the state encoded by $\psi$ gives the value $w_{0}$ with certainty. We recover in this way the results we obtained by another route in sect. 3.11.

Suppose now that $\psi$ does not represent a stationary state of the ensemble. We want to find a quantity that measures the extent to which the wave function $\psi$ fails to be one of a stationary state. A reasonable definition may be
$$
\begin{equation*}
D\left[\psi_{t}\right]=\int d^{3} x\left|\psi_{t}-\exp \left(\frac{-i \bar{w} t}{\hbar}\right) \psi_{0}\right|^{2} \tag{3.12.23}
\end{equation*}
$$

Indeed, when $\psi$ is stationary, $D\left[\psi_{t}\right]$ vanishes identically on account of (3.12.14), (3.12.19). Conversely, when $\psi$ is not stationary $D\left[\psi_{t}\right]$ does not vanish and is the larger the more $\psi$ deviates from the form it would have to have if it were stationary. The value 2 of the exponent is used only for convenience. $D\left[\psi_{t}\right]$ can be expressed in terms of the distribution function $\varpi(w)$,
$$
\begin{equation*}
D\left[\psi_{t}\right]=4 \int_{-\infty}^{\infty} d w \varpi(w) \sin ^{2}\left(\frac{(w-\bar{w}) t}{2 \hbar}\right) \tag{3.12.24}
\end{equation*}
$$

Proof. Inserting the expansion (3.12.2) into (3.12.23) and using (3.12.6), we get
$$
D\left[\psi_{t}\right]=\int d^{3} x\left[\int_{-\infty}^{\infty} \frac{d w^{\prime}}{2 \pi \hbar} \zeta_{w^{\prime}} \exp \left(-\frac{i w^{\prime} t}{\hbar}\right)-\exp \left(-\frac{i \bar{w} t}{\hbar}\right) \int_{-\infty}^{\infty} \frac{d w^{\prime}}{2 \pi \hbar} \zeta_{w^{\prime}}\right]^{*}
$$
$$
\begin{align*}
& \times\left[\int_{-\infty}^{\infty} \frac{d w}{2 \pi \hbar} \zeta_{w} \exp \left(-\frac{i w t}{\hbar}\right)-\exp \left(-\frac{i \bar{w} t}{\hbar}\right) \int_{-\infty}^{\infty} \frac{d w}{2 \pi \hbar} \zeta_{w}\right] \\
& =\int d^{3} x\left[\int_{-\infty}^{\infty} \frac{d w^{\prime}}{2 \pi \hbar} \zeta_{w^{\prime}}\left(\exp \left(-\frac{i\left(w^{\prime}-\bar{w}\right) t}{\hbar}\right)-1\right)\right]^{*}  \tag{3.12.25}\\
& \times \int_{-\infty}^{\infty} \frac{d w}{2 \pi \hbar} \zeta_{w}\left(\exp \left(-\frac{i(w-\bar{w}) t}{\hbar}\right)-1\right) \\
& =\int_{-\infty}^{\infty} \frac{d w^{\prime}}{2 \pi \hbar} \int_{-\infty}^{\infty} \frac{d w}{2 \pi \hbar}\left(\exp \left(\frac{i\left(w^{\prime}-\bar{w}\right) t}{\hbar}\right)-1\right) \\
& \times\left(\exp \left(-\frac{i(w-\bar{w}) t}{\hbar}\right)-1\right) \int d^{3} x \zeta_{w^{\prime}}{ }^{*} \zeta_{w} \\
& =\int_{-\infty}^{\infty} \frac{d w^{\prime}}{2 \pi \hbar} \int_{-\infty}^{\infty} \frac{d w}{2 \pi \hbar}\left(\exp \left(\frac{i\left(w^{\prime}-\bar{w}\right) t}{\hbar}\right)-1\right) \\
& \times\left(\exp \left(-\frac{i(w-\bar{w}) t}{\hbar}\right)-1\right)(2 \pi \hbar)^{2} \varpi(w) \delta\left(w^{\prime}-w\right) \\
& =\int_{-\infty}^{\infty} d w \varpi(w)\left|\exp \left(-\frac{i(w-\bar{w}) t}{\hbar}\right)-1\right|^{2}
\end{align*}
$$

Noting that $|\exp (i \alpha)-1|^{2}=(\cos \alpha-1)^{2}+\sin ^{2} \alpha=2(1-\cos \alpha)=4 \sin ^{2}(\alpha / 2)$, we finally reach $(3.12 .24)$.

In the non stationary case, the time $\Delta t$ which must elapse in order $\psi_{t}$ to deviate from $\exp (-i \bar{w} t / \hbar) \psi_{0}$ appreciably is the minimal time $\Delta t$ such that
$$
\begin{equation*}
D\left[\psi_{\Delta t}\right] \simeq 1 \tag{3.12.26}
\end{equation*}
$$

By $(3.12 .24), \Delta t$ satisfies
$$
\begin{equation*}
4 \int_{-\infty}^{\infty} d w \varpi(w) \sin ^{2}\left(\frac{(w-\bar{w}) \Delta t}{2 \hbar}\right) \simeq 1 \tag{3.12.27}
\end{equation*}
$$

Approximating the sine with its argument, this relation becomes
$$
\begin{equation*}
\int_{-\infty}^{\infty} d w \varpi(w)(w-\bar{w})^{2} \Delta t^{2} \simeq \hbar^{2} \tag{3.12.28}
\end{equation*}
$$

Recalling (3.12.12), we conclude that
$$
\begin{equation*}
\Delta t \Delta w \simeq \hbar \tag{3.12.29}
\end{equation*}
$$

This is the time-energy uncertainty relation. By (3.12.29), the smaller $\Delta w$, the larger $\Delta t$. The stationary case is recovered in the limit where $\Delta w$ vanishes and, correspondingly, $\Delta t$ becomes infinite (cf. eq. (3.12.20)). If $\Delta w$ is non zero but sufficiently small, then $\Delta t$ gets very large. In that situation, the state resembles a stationary state for a long time, but eventually evolves from its initial form. For this reason, the state may be called quasistationary.

\subsection*{3.13. Regularity and boundary conditions of the wave function}

It sometimes happens that the potential energy $U$ undergoes a very rapid variation in a subregion $\mathcal{R}$ of configuration space which is shaped as a thin layer bounded by two surfaces $\mathcal{S}_{1}, \mathcal{S}_{2}$. If the thickness of the layer is much smaller than all the relevant length scales characterizing the dynamics of the particle under study, it is convenient to idealize the layer as a surface $\mathcal{S}$ lying somewhere between $\mathcal{S}_{1}, \mathcal{S}_{2}$ and the potential energy as a function with a jump discontinuity at $\mathcal{S}$. See fig. 3.13 .1 for a pictorial illustration.

Since the Schroedinger equation (3.4.14) relates the second spacial derivatives of the wave function $\psi$ to the potential energy $U$, some of those derivatives must exhibit a jump discontinuity as well at $\mathcal{S}$. So, $\psi$ cannot be a smooth function along $\mathcal{S}$. Let us describe in more detail its degree of regularity.

Consider a point $\boldsymbol{x}$ of the discontinuity surface $\mathcal{S}$. A generic point $\boldsymbol{x}^{\prime}$ in a small neighborhood of $\boldsymbol{x}$ can be written as
$$
\begin{equation*}
\boldsymbol{x}^{\prime}=\boldsymbol{x}+\boldsymbol{z}_{\perp}+z_{\|} \boldsymbol{n} \tag{3.13.1}
\end{equation*}
$$
where $\boldsymbol{n}$ is the normal unit vector to $\mathcal{S}$ at $\boldsymbol{x}$ and $\boldsymbol{z}_{\perp}$ with $\boldsymbol{z}_{\perp} \cdot \boldsymbol{n}=0$ and $z_{\|} \boldsymbol{n}$ are the normal and parallel deviations of $\boldsymbol{x}^{\prime}$ from $\boldsymbol{x}$ (cf. fig. 3.13.2). Locally near $\boldsymbol{x}$, we can identify $\mathcal{S}$ with its tangent plane at $\boldsymbol{x}$. In this approximation, $\boldsymbol{x}^{\prime}$ lies on $\mathcal{S}$ precisely when $z_{\|}=0$. Near $\boldsymbol{x}$, so, a function is discontinuous at $\mathcal{S}$ if it is discontinuous as a function of the coordinate $z_{\|}$at $z_{\|}=0$.

Near $\boldsymbol{x}$, the Schroedinger equation (3.4.14) can be written as
$$
\begin{equation*}
i \hbar \frac{\partial \psi}{\partial t}=-\frac{\hbar^{2}}{2 m}\left(\nabla_{z_{\perp}}^{2} \psi+\frac{\partial^{2} \psi}{\partial z_{\|}^{2}}\right)+U \psi \tag{3.13.2}
\end{equation*}
$$

Now, the potential energy $U$ has a jump discontinuity at $z_{\|}=0$. As a function of $z_{\|}$, the least regular derivative term of the wave function $\psi$ is the highest derivative one with respect to $z_{\|}, \partial^{2} \psi / \partial z_{\|}{ }^{2}$. As $U$, so, $\partial^{2} \psi / \partial z_{\|}{ }^{2}$ has a jump discontinuity at $z_{\|}=0 . \quad \boldsymbol{\nabla}_{\boldsymbol{z}_{\perp}} \boldsymbol{\nabla}_{\boldsymbol{z}_{\perp}} \psi$ is instead continuous. From the above analysis, we reach
the following conclusion.
$$
\psi \text { and } \boldsymbol{\nabla} \psi \text { are continuous across } \mathcal{S} \text {. }
$$

The derivation of the Schroedinger equation expounded in sect. 3.4, and the statistical interpretation of the wave function propounded in sect. 3.5 implicitly

![](https://cdn.mathpix.com/cropped/2024_09_22_5d1e855547710648961eg-0305.jpg?height=1329&width=747&top_left_y=875&top_left_x=643)

Figure 3.13.1. A rapid variation of the potential energy $U(\boldsymbol{x})$ in a layer shaped region $\mathcal{R}(a)$ can be modelled as a discontinuity along a surface $\mathcal{S}(b)$. Here, we consider again space as two dimensional to allow for a pictorial representation.

![](https://cdn.mathpix.com/cropped/2024_09_22_5d1e855547710648961eg-0306.jpg?height=476&width=570&top_left_y=559&top_left_x=737)

Figure 3.13.2. In a neighborhood of a point $\boldsymbol{x}$ of the surface $\mathcal{S}$, the deviation vector $\boldsymbol{x}^{\prime}-\boldsymbol{x}$ of a point $\boldsymbol{x}^{\prime}$ can be decomposed in components $z_{\|} \boldsymbol{n}, \boldsymbol{z}_{\perp}$ normal and tangent to $\mathcal{S}$ at $\boldsymbol{x}$.
assume that the particle can access any region of the configuration space $\mathbb{E}^{3}$. From a physical point of view, it is interesting to consider situations in which the particle is confined in a space region $\mathcal{D}$. The copies of the particle in the ensemble then also are. None of them can ever be found outside $\mathcal{D}$.

The confinement in $\mathcal{D}$ can be achieved subjecting the particle to very high potential barrier $U_{\mathcal{D}}$ at the boundary $\partial \mathcal{D}$ of $\mathcal{D}$,
$$
\begin{equation*}
U_{\mathcal{D}}(\boldsymbol{x})=0, \text { for } x \text { in } \mathcal{D}, \quad U_{\mathcal{D}}(\boldsymbol{x})=U_{0}, \text { for } x \text { not in } \mathcal{D} \tag{3.13.3}
\end{equation*}
$$
where $U_{0}$ is an energy scale much larger than the typical energy of the particle (cf. fig. 3.13.3). In the ensemble of the particle, taking the limit $U_{0} \rightarrow \infty$ forces the copies inside $\mathcal{D}$, since none of them will ever have enough energy to overcome the barrier. The probability of finding a copy outside $\mathcal{D}$, so, vanishes identically. As the probability of finding a copy in a arbitrary region $\mathcal{V}$ is is given by (3.5.30), we conclude that in the limit $U_{0} \rightarrow \infty \psi$ must vanish identically outside $\mathcal{D}$
$$
\begin{equation*}
\psi(t, \boldsymbol{x})=0, \text { for } \boldsymbol{x} \text { not in } \mathcal{D} \tag{3.13.4}
\end{equation*}
$$

A more detailed analysis shows further that $\psi$ satisfies in addition vanishing

![](https://cdn.mathpix.com/cropped/2024_09_22_5d1e855547710648961eg-0307.jpg?height=299&width=746&top_left_y=555&top_left_x=684)
(a)

![](https://cdn.mathpix.com/cropped/2024_09_22_5d1e855547710648961eg-0307.jpg?height=511&width=725&top_left_y=1070&top_left_x=689)
(b)

Figure 3.13.3. A particle confined in a spacial domain $\mathcal{D}(a)$ behaves as if it were subject to a very high potential barrier $U_{\mathcal{D}}$ outside $\mathcal{D}(b)$. Here, we consider again space as two dimensional to allow for a pictorial representation.

Dirichlet boundary conditions on $\mathcal{D}$, namely
$$
\begin{equation*}
\psi(t, \boldsymbol{x})=0, \text { for } \boldsymbol{x} \text { in } \partial \mathcal{D} \tag{3.13.5}
\end{equation*}
$$

The following argument make this plausible.

Proof. The potential $U_{\mathcal{D}}$ is discontinuous at $\partial \mathcal{D}$. The considerations of the first
part of this section so apply. Using the parametrization (3.13.1), in a neighborhood of a point $\boldsymbol{x}$ of $\partial \mathcal{D}$ the confining potential $U_{\mathcal{D}}$ can be modelled approximately as
$$
\begin{align*}
& U_{\mathcal{D}}(\boldsymbol{z})=0, \quad z_{\|}<0  \tag{3.13.6}\\
& U_{\mathcal{D}}(\boldsymbol{z})=U_{0}, \quad z_{\|}>0
\end{align*}
$$
with $U_{0}>0$. We shall eventually take the limit $U_{0} \rightarrow \infty$. For such simple potential, it is not too difficult to construct a solution of the Schroedinger equation (3.13.2).

Consider the wave function
$$
\begin{equation*}
\psi_{\boldsymbol{k}}(t, \boldsymbol{z})=\exp \left(-i w_{\boldsymbol{k}} t / \hbar\right) \phi_{\boldsymbol{k}}(\boldsymbol{z}) \tag{3.13.7}
\end{equation*}
$$
where $\boldsymbol{k}$ is a wave vector of normal and tangent components $k_{\|}, \boldsymbol{k}_{\perp}, w_{\boldsymbol{k}}$ is the energy
$$
\begin{equation*}
w_{k}=\frac{\hbar^{2} \boldsymbol{k}^{2}}{2 m} \tag{3.13.8}
\end{equation*}
$$
and $\phi_{\boldsymbol{k}}$ is the wave function
$$
\begin{align*}
& \phi_{\boldsymbol{k}}(\boldsymbol{z})=\exp \left(i \boldsymbol{k}_{\perp} \cdot \boldsymbol{z}_{\perp}\right)\left[\frac{\sin \left(k_{\|} z_{\|}\right)}{k_{\|}}-\frac{\cos \left(k_{\|} z_{\|}\right)}{\kappa_{k_{\|}}}\right], \quad z_{\|}<0  \tag{3.13.9}\\
& \phi_{\boldsymbol{k}}(\boldsymbol{z})=-\exp \left(i \boldsymbol{k}_{\perp} \cdot \boldsymbol{z}_{\perp}\right) \frac{\exp \left(-\kappa_{k_{\|}} z_{\|}\right)}{\kappa_{k_{\|}}}, \quad z_{\|}>0
\end{align*}
$$
$\kappa_{k_{\|}}$being the wave number
$$
\begin{equation*}
\kappa_{k_{\|}}=\left(2 m U_{0} / \hbar^{2}-k_{\|}^{2}\right)^{1 / 2} \tag{3.13.10}
\end{equation*}
$$

The restriction $k_{\|}{ }^{2} \leq 2 m U_{0} / \hbar^{2}$ is assumed to ensure the reality of $\kappa_{k_{\|}}$. Let us verify that $\psi_{\boldsymbol{k}}$ obeys the Schroedinger equation (3.13.2).

To begin with, we observe that
$$
\begin{align*}
& i \hbar \frac{\partial \psi_{\boldsymbol{k} t}}{\partial t}=w_{\boldsymbol{k}} \exp \left(-i w_{\boldsymbol{k}} t / \hbar\right) \phi_{\boldsymbol{k}}  \tag{3.13.11}\\
& -\frac{\hbar^{2}}{2 m}\left(\boldsymbol{\nabla}_{\boldsymbol{z}_{\perp}}{ }^{2} \psi_{\boldsymbol{k} t}+\frac{\partial^{2} \psi_{\boldsymbol{k} t}}{\partial z_{\|}{ }^{2}}\right)+U_{\mathcal{D}} \psi_{\boldsymbol{k} t}  \tag{3.13.12}\\
& \quad=\exp \left(-i w_{\boldsymbol{k}} t / \hbar\right)\left[-\frac{\hbar^{2}}{2 m}\left(\boldsymbol{\nabla}_{\boldsymbol{z}_{\perp}}^{2} \phi_{\boldsymbol{k}}+\frac{\partial^{2} \phi_{\boldsymbol{k}}}{\partial z_{\|}{ }^{2}}\right)+U_{\mathcal{D}} \phi_{\boldsymbol{k}}\right]
\end{align*}
$$

Hence, to verify that $\psi_{\boldsymbol{k}}$ obeys (3.13.2) it suffices to check that the equation
$$
\begin{equation*}
-\frac{\hbar^{2}}{2 m}\left(\nabla_{\boldsymbol{z}_{\perp}}^{2} \phi_{\boldsymbol{k}}+\frac{\partial^{2} \phi_{\boldsymbol{k}}}{\partial z_{\|}{ }^{2}}\right)+U_{\mathcal{D}} \phi_{\boldsymbol{k}}=w_{\boldsymbol{k}} \phi_{\boldsymbol{k}} \tag{3.13.13}
\end{equation*}
$$
is satisfied. This is readily done using the computations
$$
\begin{align*}
& -\frac{\hbar^{2}}{2 m} \nabla_{\boldsymbol{z}_{\perp}}{ }^{2} \exp \left(i \boldsymbol{k}_{\perp} \cdot \boldsymbol{z}_{\perp}\right)=\frac{\hbar^{2} \boldsymbol{k}_{\perp}{ }^{2}}{2 m} \exp \left(i \boldsymbol{k}_{\perp} \cdot \boldsymbol{z}_{\perp}\right)  \tag{3.13.14}\\
& -\frac{\hbar^{2}}{2 m} \frac{\partial^{2}}{\partial z_{\|}{ }^{2}}\left[\frac{\sin \left(k_{\|} z_{\|}\right)}{k_{\|}}-\frac{\cos \left(k_{\|} z_{\|}\right)}{\kappa_{k_{\|}}}\right]=\frac{\hbar^{2} k_{\|}{ }^{2}}{2 m}\left[\frac{\sin \left(k_{\|} z_{\|}\right)}{k_{\|}}-\frac{\cos \left(k_{\|} z_{\|}\right)}{\kappa_{k_{\|}}}\right]  \tag{3.13.15}\\
& -\frac{\hbar^{2}}{2 m} \frac{\partial^{2}}{\partial z_{\|}{ }^{2}} \frac{\exp \left(-\kappa_{k_{\|}} z_{\|}\right)}{\kappa_{k_{\|}}}+U_{0} \frac{\exp \left(-\kappa_{k_{\|}} z_{\|}\right)}{\kappa_{k_{\|}}}  \tag{3.13.16}\\
& \quad=\left[-\frac{\hbar^{2} \kappa_{k_{\|}}{ }^{2}}{2 m}+U_{0}\right] \frac{\exp \left(-\kappa_{k_{\|}} z_{\|}\right)}{\kappa_{k_{\|}}}=\frac{\hbar^{2} k_{\|}{ }^{2}}{2 m} \frac{\exp \left(-\kappa_{k_{\|}} z_{\|}\right)}{\kappa_{k_{\|}}}
\end{align*}
$$
and recalling that $\boldsymbol{k}_{\perp}{ }^{2}+k_{\|}{ }^{2}=\boldsymbol{k}^{2}$.
Next, we check that the wave function $\psi_{\boldsymbol{k}}$ as a function of $z_{\|}$is continuous and has continuous first derivative at $z_{\|}=0$. To this end, it suffices to check that $\phi_{\boldsymbol{k}}$ has these properties, since $\psi_{\boldsymbol{k}}$ depends on $\boldsymbol{z}$ through $\phi_{\boldsymbol{k}}$ by (3.13.7). Eq. (3.13.9) shows that $\phi_{\boldsymbol{k}}$ is a continuous function of $z_{\|}$at $z_{\|}=0$ as required. Eq. (3.13.9) also yields the expressions
$$
\begin{align*}
& \frac{\partial \phi_{\boldsymbol{k}}(\boldsymbol{z})}{\partial z_{\|}}=\exp \left(i \boldsymbol{k}_{\perp} \cdot \boldsymbol{z}_{\perp}\right)\left[\cos \left(k_{\|} z_{\|}\right)+\frac{k_{\|} \sin \left(k_{\|} z_{\|}\right)}{\kappa_{k_{\|}}}\right], \quad z_{\|}<0  \tag{3.13.17}\\
& \frac{\partial \phi_{\boldsymbol{k}}(\boldsymbol{z})}{\partial z_{\|}}=\exp \left(i \boldsymbol{k}_{\perp} \cdot \boldsymbol{z}_{\perp}\right) \exp \left(-\kappa_{k_{\|}} z_{\|}\right), \quad z_{\|}>0
\end{align*}
$$
which show that $\partial \phi_{\boldsymbol{k}} / \partial z_{\|}$is also a continuous function of $z_{\|}$at $z_{\|}=0$, again as required.
From the above analysis, it follows that the wave function $\psi_{\boldsymbol{k}}$ given in eq. (3.13.7) is a solution of the Schroedinger equation enjoying the required regularity properties. We now take the limit $U_{0} \rightarrow \infty$. In the limit, (3.13.7) keeps to hold. Since by (3.13.10) $\kappa_{k_{\|}} \rightarrow \infty$ as $U_{0} \rightarrow \infty$, however, $\phi_{\boldsymbol{k}}$ now reads
$$
\begin{align*}
& \phi_{\boldsymbol{k}}(\boldsymbol{z})=\exp \left(i \boldsymbol{k}_{\perp} \cdot \boldsymbol{z}_{\perp}\right) \frac{\sin \left(k_{\|} z_{\|}\right)}{k_{\|}}, \quad z_{\|}<0  \tag{3.13.18}\\
& \phi_{\boldsymbol{k}}(\boldsymbol{z})=0, \quad z_{\|}>0
\end{align*}
$$

By (3.13.6), the Schroedinger equation (3.13.2) does not contain the energy scale $U_{0}$ for $z_{\|}<0$. Therefore, in the limit, $\psi_{\boldsymbol{k}}$ solves (3.13.2) in the same space region. From (3.13.18), it follows further that $\psi_{\boldsymbol{k}}$ obeys
$$
\begin{equation*}
\psi_{\boldsymbol{k}}(t, \boldsymbol{z})=0 \quad \text { for } z_{\|}=0 \tag{3.13.19}
\end{equation*}
$$

This proves that $\psi_{\boldsymbol{k}}$ satisfies the boundary condition (3.13.4).
Remark. From (3.13.18), it follows that in the limit $U_{0} \rightarrow \infty \phi_{\boldsymbol{k}}$ is continuous in $z_{\|}=0$. It follows further that $\partial \phi_{\boldsymbol{k}} / \partial z_{\|}$instead is not, since
$$
\begin{align*}
& \frac{\partial \phi_{\boldsymbol{k}}(\boldsymbol{z})}{\partial z_{\|}}=\exp \left(i \boldsymbol{k}_{\perp} \cdot \boldsymbol{z}_{\perp}\right) \cos \left(k_{\|} z_{\|}\right), \quad z_{\|}<0  \tag{3.13.20}\\
& \frac{\partial \phi_{\boldsymbol{k}}(\boldsymbol{z})}{\partial z_{\|}}=0, \quad z_{\|}>0
\end{align*}
$$

The divergence of $U_{0}$ thus decreases the regularity of $\psi_{\boldsymbol{k}}$.

We note the case where the particle is not confined is formally a particular instance of the case where it is: it corresponds to the limit in which $\mathcal{D}$ dilates eventually filling the whole configuration space $\mathbb{E}^{3}$ with the boundary $\partial \mathcal{D}$ of $\mathcal{D}$ pushed at infinity in $\mathbb{E}^{3}$.

For a particle confined in a region $\mathcal{D}$, the probabilistic interpretation of the wave function is unchanged. The probability of finding a copy in an arbitrary region $\mathcal{V}$ contained in $\mathcal{D}$ is still given by
$$
\begin{equation*}
p(\mathcal{V})=\int_{\mathcal{V}} d^{3} x|\psi|^{2} \tag{3.13.21}
\end{equation*}
$$
in keeping with (3.5.30). In the same way, the probability of finding a copy in the ensemble flowing through an oriented space surface $\mathcal{A}$ contained in $\mathcal{D}$ per unit time is still given by
$$
\begin{equation*}
\Phi(\mathcal{A})=\frac{\hbar}{m} \int_{\mathcal{A}} d^{2} \boldsymbol{x} \cdot \operatorname{Im}\left(\psi^{*} \boldsymbol{\nabla} \psi\right) \tag{3.13.22}
\end{equation*}
$$
in keeping with (3.5.31). The normalization condition (3.5.32) of $\psi$ thus reads
$$
\begin{equation*}
\int_{\mathcal{D}} d^{3} x|\psi|^{2}=1 \tag{3.13.23}
\end{equation*}
$$
generalizing (3.5.32).
Taking the condition (3.13.4) into account, we can easily update the expression (3.5.78) of the quantum mean $\left\langle Q_{f}\right\rangle$ of the quantity $Q_{f}$ corresponding to a phase function $f$
$$
\begin{equation*}
\left\langle Q_{f}\right\rangle=\int_{\mathcal{D}} d^{3} x \psi^{*} \mathrm{Q}_{f} \psi \tag{3.13.24}
\end{equation*}
$$
where $\mathrm{Q}_{f} \psi$ is given as before by (3.5.79), (3.5.80). In particular, the quantum mean values $\langle\boldsymbol{q}\rangle,\langle\boldsymbol{p}\rangle,\langle\boldsymbol{l}\rangle$ and $\langle H\rangle$ of the position $\boldsymbol{q}$, momentum $\boldsymbol{p}$, angular momentum $\boldsymbol{l}$ and energy $H$ given in eqs. (3.5.82)-(3.5.85) take presently the form
$$
\begin{align*}
\langle\boldsymbol{q}\rangle & =\int_{\mathcal{D}} d^{3} x \psi^{*} \boldsymbol{x} \psi  \tag{3.13.25}\\
\langle\boldsymbol{p}\rangle & =\int_{\mathcal{D}} d^{3} x \psi^{*}(-i \hbar \boldsymbol{\nabla}) \psi  \tag{3.13.26}\\
\langle\boldsymbol{l}\rangle & =\int_{\mathcal{D}} d^{3} x \psi^{*}(-i \hbar \boldsymbol{x} \times \boldsymbol{\nabla}) \psi  \tag{3.13.27}\\
\langle H\rangle & =\int_{\mathcal{D}} d^{3} x \psi^{*}\left(-\frac{\hbar^{2}}{2 m} \boldsymbol{\nabla}^{2} \psi+U \psi\right) \tag{3.13.28}
\end{align*}
$$

The quantum mean values $\langle\boldsymbol{q}\rangle,\langle\boldsymbol{p}\rangle,\langle\boldsymbol{l}\rangle$ and $\langle H\rangle$ of $\boldsymbol{q}, \boldsymbol{p}, \boldsymbol{l}$ and $H$ must be real numbers. It is not immediately obvious from expressions (3.13.25)-(3.13.28) that they are, but in fact they are thanks to the boundary conditions (3.13.5).

Proof. From (3.13.25), we have
$$
\begin{equation*}
\langle\boldsymbol{q}\rangle-\langle\boldsymbol{q}\rangle^{*}=\int_{\mathcal{D}} d^{3} x \psi^{*} \boldsymbol{x} \psi-\left[\int_{\mathcal{D}} d^{3} x \psi^{*} \boldsymbol{x} \psi\right]^{*}=\mathbf{0} \tag{3.13.29}
\end{equation*}
$$

The quantum mean value of $\boldsymbol{q}$ is thus automatically real.
Next, from (3.13.26), we compute
$$
\begin{align*}
\langle\boldsymbol{p}\rangle-\langle\boldsymbol{p}\rangle^{*} & =\int_{\mathcal{D}} d^{3} x \psi^{*}(-i \hbar \boldsymbol{\nabla}) \psi-\left[\int_{\mathcal{D}} d^{3} x \psi^{*}(-i \hbar \boldsymbol{\nabla}) \psi\right]^{*}  \tag{3.13.30}\\
& =-i \hbar \int_{\mathcal{D}} d^{3} x\left(\psi^{*} \boldsymbol{\nabla} \psi+\boldsymbol{\nabla} \psi^{*} \psi\right)=-i \hbar \int_{\mathcal{D}} d^{3} x \boldsymbol{\nabla}|\psi|^{2}=-i \hbar \oint_{\partial \mathcal{D}} d^{2} \boldsymbol{x}|\psi|^{2}
\end{align*}
$$
where, in the last step, we applied one of the versions of Gauss' theorem. $\langle\boldsymbol{p}\rangle$ is therefore real if (3.13.5) holds.

Next, from (3.13.27), we compute
$$
\begin{align*}
\langle\boldsymbol{l}\rangle-\langle\boldsymbol{l}\rangle^{*} & =\int_{\mathcal{D}} d^{3} x \psi^{*}(-i \hbar \boldsymbol{x} \times \boldsymbol{\nabla}) \psi-\left[\int_{\mathcal{D}} d^{3} x \psi^{*}(-i \hbar \boldsymbol{x} \times \boldsymbol{\nabla}) \psi\right]^{*}  \tag{3.13.31}\\
& =-i \hbar \int_{\mathcal{D}} d^{3} x\left(\psi^{*} \boldsymbol{x} \times \boldsymbol{\nabla} \psi+\boldsymbol{x} \times \boldsymbol{\nabla} \psi^{*} \psi\right)=-i \hbar \int_{\mathcal{D}} d^{3} x \boldsymbol{x} \times \boldsymbol{\nabla}|\psi|^{2}
\end{align*}
$$
$$
=i \hbar \int_{\mathcal{D}} d^{3} x \boldsymbol{\nabla} \times\left(\boldsymbol{x}|\psi|^{2}\right)=i \hbar \oint_{\mathcal{D} \mathcal{D}} d^{2} \boldsymbol{x} \times \boldsymbol{x}|\psi|^{2}
$$
where, in the last step, we applied another version of Gauss' theorem. $\langle\boldsymbol{l}\rangle$ is therefore real if, again, (3.13.5) holds.

Finally, from (3.13.28), we have
$$
\begin{align*}
& \langle H\rangle-\langle H\rangle^{*}  \tag{3.13.32}\\
& =\int_{\mathcal{D}} d^{3} x \psi^{*}\left(-\frac{\hbar^{2}}{2 m} \nabla^{2} \psi+U \psi\right)-\left[\int_{\mathcal{D}} d^{3} x \psi^{*}\left(-\frac{\hbar^{2}}{2 m} \boldsymbol{\nabla}^{2} \psi+U \psi\right)\right]^{*} \\
& =-\frac{\hbar^{2}}{2 m} \int_{\mathcal{D}} d^{3} x\left(\psi^{*} \nabla^{2} \psi-\boldsymbol{\nabla}^{2} \psi^{*} \psi\right)=-\frac{\hbar^{2}}{2 m} \int_{\mathcal{D}} d^{3} x \boldsymbol{\nabla} \cdot\left(\psi^{*} \boldsymbol{\nabla} \psi-\boldsymbol{\nabla} \psi^{*} \psi\right) \\
& =-\frac{\hbar^{2}}{2 m} \oint_{\partial \mathcal{D}} d^{2} \boldsymbol{x} \cdot\left(\psi^{*} \boldsymbol{\nabla} \psi-\boldsymbol{\nabla} \psi^{*} \psi\right)
\end{align*}
$$
where we used again Gauss' theorem. $\langle H\rangle$ is real, if, once more, (3.13.5) holds.

\subsection*{3.14. The time independent Schroedinger problem}

The main problem of wave mechanics is the determination of the wave function $\psi$ describing a particle probabilistically by solving the Schroedinger equation (3.4.14) in the allowed space region with the appended boundary and normalization conditions (3.13.5) and (3.13.23). The linearity of this problem suggests that its solution may proceed by finding a set of basic solutions $\psi_{n}$ labelled by one or more discrete or continuous indices denoted collectively by $n$ such that $\psi$, regardless its explicit form, enjoys an expansion like
$$
\begin{equation*}
\psi=\sum_{n_{d}} \int d n_{c} c_{n} \psi_{n} \tag{3.14.1}
\end{equation*}
$$
where the $c_{n}$ are coefficients depending on the $n$ and $\sum_{n_{d}}, \int d n_{c}$ denote summation on the discrete and integration on the continuous indices. The determination of $\psi$ is then reduced to that of the $c_{n}$.

The basic wave functions $\psi_{n}$ must fulfil certain requirements in order an expansion of $\psi$ of the form (3.14.1) to hold. First, in order $\psi$ to obey the Schroedinger equation (3.4.14) in the relevant region $\mathcal{D}$, also all the $\psi_{n}$ should,
$$
\begin{equation*}
i \hbar \frac{\partial \psi_{n}}{\partial t}=-\frac{\hbar^{2}}{2 m} \boldsymbol{\nabla}^{2} \psi_{n}+U \psi_{n} \quad \text { in } \mathcal{D} \tag{3.14.2}
\end{equation*}
$$

Second, as $\psi$ and $\boldsymbol{\nabla} \psi$ are continuous across any discontinuity surface $\mathcal{S}$ of the potential energy $U, \psi_{n}$ and $\boldsymbol{\nabla} \psi_{n}$ should enjoy the same property. Third, in order $\psi$ to satisfy the boundary condition (3.13.5) in $\mathcal{D}$, also all the $\psi_{n}$ should,
$$
\begin{equation*}
\psi_{n}(t, \boldsymbol{x})=0, \quad \text { for } \boldsymbol{x} \text { in } \partial \mathcal{D} \tag{3.14.3}
\end{equation*}
$$

Fourth and last, the $\psi_{n}$ should have a controlled growth at spacial infinity in $\mathcal{D}$ whenever reachable to guarantee that $\psi$ satisfies the normalization condition (3.13.23), namely they should be bounded by some constant $C$,
$$
\begin{equation*}
\left|\psi_{n}(t, \boldsymbol{x})\right| \leq C, \quad \text { as }|\boldsymbol{x}| \rightarrow \infty \text { in } \mathcal{D} \tag{3.14.4}
\end{equation*}
$$

While the requirements (3.14.2) and (3.14.3) are in a certain sense expected answering as they do to analogous properties of the wave function $\psi$, requirement (3.14.4) is not obvious. (3.14.4) allows for normalizable as well as non normalizable basic wave functions $\psi_{n}$. The occurrence of non normalizable $\psi_{n}$ may appear puzzling in view of the normalizability of $\psi$ to which these $\psi_{n}$ contribute through the expansion (3.14.1), but in fact excluding the non normalizable $\psi_{n}$ in general compromises the very possibility of expressing $\psi$ as in (3.14.1). The normalizability of $\psi$ is guaranteed by appropriate requirements on the coefficients $c_{n}$ as functions of the indices $n$. Since non vanishing discrete linear combinations of non normalizable functions are also non normalizable, non normalizable wave functions $\psi_{n}$ can occur only if the indices $n$ are at least partly continuous. To have an idea about how this works out in a concrete situation, recall that in wave theory any localized wave packet can be decomposed in monochromatic plane wave components, even though the former is normalizable and the latter are not having an infinite extension.

We demand also that the basic wave functions $\psi_{n}$ are linearly independent, in order to avoid the redundancies in the expansion (3.14.1) associated with linear relation expressing some of the $\psi_{n}$ in terms of others. In particular, the $\psi_{n}$ should be non identically vanishing, simply because vanishing $\psi_{n}$ obviously cannot contribute to the expansion.

The next step of our analysis will be the construction of a suitable set of basic wave functions $\psi_{n}$. One such wave function, $\psi_{s}$, must obey to begin with the Schroedinger equation (3.14.2). This is a linear partial differential equation for $\psi_{s}$ as a function of time $t$ and position $\boldsymbol{x}$ with no mixed $t$ and $\boldsymbol{x}$ derivatives. A standard method of solution of equations of this type is by separation of variables. We look therefore for a wave function $\psi_{s}$ of the factorized form
$$
\begin{equation*}
\psi_{s}(t, \boldsymbol{x})=\alpha(t) \phi(\boldsymbol{x}) \tag{3.14.5}
\end{equation*}
$$
with $\alpha(t)$ a function of $t$ and $\phi(\boldsymbol{x})$ a function of $\boldsymbol{x}$ only. Inserting (3.14.5) into
(3.4.14), we obtain the equation
$$
\begin{equation*}
i \hbar \frac{1}{\alpha} \frac{d \alpha}{d t}=\frac{1}{\phi}\left(-\frac{\hbar^{2}}{2 m} \nabla^{2} \phi+U \phi\right) \tag{3.14.6}
\end{equation*}
$$

Proof. Substituting (3.14.5) into (3.4.14), we find
$$
\begin{align*}
0 & =i \hbar \frac{\partial \psi_{s}}{\partial t}+\frac{\hbar^{2}}{2 m} \nabla^{2} \psi_{s}-U \psi_{s}  \tag{3.14.7}\\
& =i \hbar \frac{d \alpha}{d t} \phi+\alpha\left(\frac{\hbar^{2}}{2 m} \nabla^{2} \phi-U \phi\right) \\
& =\left[i \hbar \frac{1}{\alpha} \frac{d \alpha}{d t}-\frac{1}{\phi}\left(-\frac{\hbar^{2}}{2 m} \nabla^{2} \phi+U \phi\right)\right] \alpha \phi
\end{align*}
$$
from which (3.14.6) follows readily.

The left and right hand sides of (3.14.6) being functions of $t$ and $\boldsymbol{x}$ only, respectively implies that there is a constant $w$ such that
$$
\begin{align*}
& i \hbar \frac{1}{\alpha} \frac{d \alpha}{d t}=w  \tag{3.14.8}\\
& \frac{1}{\phi}\left(-\frac{\hbar^{2}}{2 m} \nabla^{2} \phi+U \phi\right)=w \tag{3.14.9}
\end{align*}
$$

Dimensional analysis shows that $w$ is an energy.
Eq. (3.14.8) can be written in the form
$$
\begin{equation*}
i \hbar \frac{d \alpha}{d t}-w \alpha=0 \tag{3.14.10}
\end{equation*}
$$

This is immediately solved
$$
\begin{equation*}
\alpha(t)=\exp \left(-\frac{i w t}{\hbar}\right) \tag{3.14.11}
\end{equation*}
$$

Here, we have assumed that $\alpha(0)=1$ without loss of generality, since, by virtue of the form of the ansatz (3.14.5), the constant $\alpha(0)$ can be absorbed into $\phi$.

Eq. (3.14.9) can be put in the form
$$
\begin{equation*}
-\frac{\hbar^{2}}{2 m} \nabla^{2} \phi+U \phi=w \phi \quad \text { in } \mathcal{D} \tag{3.14.12}
\end{equation*}
$$
where we have specified again the space domain where the equation holds. Eq. (3.14.12) is called time independent Schroedinger equation. We have already encountered in sect. 3.11.

There are further restrictions the wave function $\phi$ must obey besides solving the Schroedinger equation (3.14.12). They all follow from corresponding requirements on the space dependence of $\psi_{s}$. We examine them next.

First, in order $\psi_{s}$ and $\boldsymbol{\nabla} \psi_{s}$ to be continuous at a discontinuity surface $\mathcal{S}$ of the potential energy $U, \phi$ and $\boldsymbol{\nabla} \phi$ also have to.
$\phi$ and $\boldsymbol{\nabla} \phi$ are continuous across $\mathcal{S}$.
Second, since $\psi_{s}$ satisfies the boundary conditions (3.14.3), $\phi$ also must. We have therefore
$$
\begin{equation*}
\phi(\boldsymbol{x})=0, \text { for } \boldsymbol{x} \text { in } \partial \mathcal{D} \tag{3.14.13}
\end{equation*}
$$

Third and last, to ensure that $\psi_{s}$ obeys the boundedness requirement (3.14.4), $\phi$ also must.
$$
\begin{equation*}
|\phi(\boldsymbol{x})| \leq C, \text { as }|\boldsymbol{x}| \rightarrow \infty \text { in } \mathcal{D} \tag{3.14.14}
\end{equation*}
$$

For reasons explained above, the special solution $\psi_{s} \equiv 0$ must be discarded. This leads to a final requirement for the wave function $\phi$.
$\phi$ is non identically vanishing.
While the Schroedinger equation (3.14.12) has generically a non zero solution $\phi$ for every energy value $w$, it does have one satisfying the regularity and boundary and boundedness conditions (3.14.13) and (3.14.14) only for certain values $w$. The admissible values of $w$ are determined precisely by (3.14.13), (3.14.14), since any wave function $\phi$ obeying (3.14.12) depends implicitly on $w$. We can therefore define the Schroedinger problem as follows.

The Schroedinger problem of the particle consists in finding the energy values $w$ such that there exist non zero wave functions $\phi$ solving the time independent Schroedinger equation (3.14.12) that are continuous and with continuous first space derivatives and satisfy the boundary conditions (3.14.13) and the boundedness requirement (3.14.14).

The unknown of the problem are thus $w$ and $\phi$ and not just $\phi$. If a pair $w, \phi$ solves the problem, then $w$ is called an energy eigenvalue and $\phi$ an energy eigenfunction belonging to $w$.

The Schroedinger problem is linear: if $\phi_{1}, \phi_{2}$ are eigenfunctions belonging to the same eigenvalue $w$, then $c_{1} \phi_{1}+c_{2} \phi_{2}$ with $c_{1}, c_{2}$ complex coefficients also is as long as it is not identically zero. Thus the eigenfunctions $\phi$ belonging to $w$ together with the identically vanishing function form a complex vector space $\mathcal{H}_{w}$, called the energy eigenspace of $w . \mathcal{H}_{w}$ is characterized by its dimension $d_{w}$. $d_{w} \geq 1$, possibly infinite. $w$ is said non degenerate if $d_{w}=1$, while it is said $d_{w}$-degenerate if $d_{w}>1$.

When solving the Schroedinger problem, it is not necessary to specify all the eigenfunctions $\phi$ of the eigenspaces $\mathcal{H}_{w}$ of all eigenvalues $w$, but only a basis $\phi_{w a}$ of eigenfunctions of each eigenspace $\mathcal{H}_{w}$, since any eigenfunction $\phi$ is just a straightforward linear combination of the basis eigenfunctions $\phi_{w a}$. The choice of the bases $\phi_{w a}$ is essentially non unique.

The Schroedinger problem admits infinitely many solutions even after allowing only for linearly independent eigenfunctions. In general, the solutions can be presented as a collection of eigenvalues $w_{n}$ and eigenfunctions $\phi_{n}$ labelled by indices $n$, which can be partly discrete and partly continuous, with the property that for each $n \phi_{n}$ belongs to $w_{n}$ and that for $n \neq m$ one has $\phi_{n} \neq \phi_{m}$. When degeneration occurs, it is possible that $w_{n}=w_{m}$ even if $n \neq m$.

The basic solution of the Schroedinger equation we introduced at the beginning of this sections then should be the wave functions $\psi_{n}$ such that
$$
\begin{equation*}
\psi_{n t}=\exp \left(-i w_{n} t / \hbar\right) \phi_{n} \tag{3.14.15}
\end{equation*}
$$

It remains to show that these $\psi_{n}$ allow for expanding any wave function $\psi$ obeying the Schroedinger equation (3.4.14) as in (3.14.1).

The collection of all distinct energy eigenvalues $w_{n}$ is called energy spectrum of the particle. The spectrum characterizes physically the particle. As we have seen in sect. 3.11, a stationary state of the particle of energy $w$ is described by a wave function $\psi$ of the form (3.11.1) such that $\psi_{0}=\left.\psi_{t}\right|_{t=0}$ is a normalized wave function obeying the time independent Schroedinger equation (3.11.2) with eigenvalue $w$. It follows that when we solve the Schroedinger problem, we are finding, among other things, also the wave functions describing the stationary states of the particle and their energies. These wave functions are the $\psi_{n}$ of the form (3.14.15) such that the associated eigenfunction $\phi_{n}$ is normalizable.

\subsection*{3.15. A basic example: the potential box}

The time independent Schroedinger equation for a quantum particle confined in a bounded space region of arbitrary shape and subject there to a sufficiently regular but otherwise arbitrary potential (cf. fig. 3.15.1 a) can be solved by approximating the region by a box shaped domain with edges of suitable lengths and the potential with a constant one which we may assume to vanish without loss of generality (cf. fig. 3.15.1 b). This leads to the potential box model. Although it is not quantitatively accurate, the model is simple and elegant, highlights basic quantum effects and captures many essential qualitative features shared by the energy eigenvalues and eigenfunctions of a wide range of quantum particles forced in a finite space region. It is moreover one of the very few models in quantum mechanics which can be solved analytically in closed form without requiring the use of sophisticated mathematics. Finally, it serves as a useful introduction to the theory of sect. 3.17 .

The potential box model describes a particle free to move in a rectangular spacial box $\mathcal{B}$, which we coordinatize by $0 \leq x_{1} \leq l_{1}, 0 \leq x_{2} \leq l_{2}, 0 \leq x_{3} \leq l_{3}$. As the particle is subject to no force within $\mathcal{B}$, the potential energy vanishes,
$$
\begin{equation*}
U\left(x_{1}, x_{2}, x_{3}\right)=0 \tag{3.15.1}
\end{equation*}
$$

The time independent Schroedinger equation (3.14.12) of the particle is then
$$
\begin{equation*}
\frac{\partial^{2} \phi}{\partial x_{1}{ }^{2}}+\frac{\partial^{2} \phi}{\partial x_{2}{ }^{2}}+\frac{\partial^{2} \phi}{\partial x_{3}{ }^{2}}+\frac{2 m w}{\hbar^{2}} \phi=0 \tag{3.15.2}
\end{equation*}
$$
with $\phi$ non identically vanishing and satisfying the boundary conditions (3.14.13)
$$
\begin{align*}
& \phi\left(0, x_{2}, x_{3}\right)=\phi\left(l_{1}, x_{2}, x_{3}\right)=0  \tag{3.15.3a}\\
& \phi\left(x_{1}, 0, x_{3}\right)=\phi\left(x_{1}, l_{2}, x_{3}\right)=0  \tag{3.15.3b}\\
& \phi\left(x_{1}, x_{2}, 0\right)=\phi\left(x_{1}, x_{2}, l_{3}\right)=0 \tag{3.15.3c}
\end{align*}
$$

![](https://cdn.mathpix.com/cropped/2024_09_22_5d1e855547710648961eg-0320.jpg?height=475&width=751&top_left_y=570&top_left_x=687)
(a)

![](https://cdn.mathpix.com/cropped/2024_09_22_5d1e855547710648961eg-0320.jpg?height=459&width=746&top_left_y=1294&top_left_x=684)
(b)

Figure 3.15.1. A general potential $U$ in a bounded domain (a) can be approximated by a potential $U_{0}=0$ in a box shaped domain. Here, we consider space as two dimensional to allow for a pictorial representation.

The form of the differential equation (3.15.2) suggests that its solution $\phi$ should be of factorized form,
$$
\begin{equation*}
\phi\left(x_{1}, x_{2}, x_{3}\right)=f_{1}\left(x_{1}\right) f_{2}\left(x_{2}\right) f_{3}\left(x_{3}\right) \tag{3.15.4}
\end{equation*}
$$

Inserting (3.15.4) into (3.15.2), the Schroedinger equation becomes
$$
\begin{equation*}
\frac{1}{f_{1}} \frac{d^{2} f_{1}}{d x_{1}{ }^{2}}+\frac{1}{f_{2}} \frac{d^{2} f_{2}}{d x_{2}^{2}}+\frac{1}{f_{3}} \frac{d^{2} f_{3}}{d x_{3}{ }^{2}}+\frac{2 m w}{\hbar^{2}}=0 \tag{3.15.5}
\end{equation*}
$$

Proof. Indeed, since $f_{1}, f_{2}, f_{3}$ are functions of $x_{1}, x_{2}, x_{3}$ only, insertion of (3.15.4) into (3.15.2) yields the equation
$$
\begin{align*}
& \frac{d^{2} f_{1}}{d x_{1}{ }^{2}} f_{2} f_{3}+f_{1} \frac{d^{2} f_{2}}{d x_{2}{ }^{2}} f_{3}+f_{1} f_{2} \frac{d^{2} f_{3}}{d x_{3}{ }^{2}}+\frac{2 m w}{\hbar^{2}} f_{1} f_{2} f_{3}  \tag{3.15.6}\\
& \quad=\left[\frac{1}{f_{1}} \frac{d^{2} f_{1}}{d x_{1}{ }^{2}}+\frac{1}{f_{2}} \frac{d^{2} f_{2}}{d x_{2}{ }^{2}}+\frac{1}{f_{3}} \frac{d^{2} f_{3}}{d x_{3}{ }^{2}}+\frac{2 m w}{\hbar^{2}}\right] f_{1} f_{2} f_{3}=0
\end{align*}
$$
from which (3.15.5) follows immediately.

Since $f_{1}, f_{2}, f_{3}$ depend respectively only on $x_{1}, x_{2}, x_{3}$, eq. (3.15.5) above can be satisfied identically only if
$$
\begin{align*}
& \frac{d^{2} f_{1}}{d x_{1}^{2}}+k_{1}^{2} f_{1}=0  \tag{3.15.7a}\\
& \frac{d^{2} f_{2}}{d x_{2}^{2}}+k_{2}^{2} f_{2}=0  \tag{3.15.7b}\\
& \frac{d^{2} f_{3}}{d x_{3}^{2}}+k_{3}^{2} f_{3}=0 \tag{3.15.7c}
\end{align*}
$$
where $k_{1}, k_{2}, k_{3}$ are wave numbers such that
$$
\begin{equation*}
k_{1}^{2}+{k_{2}}^{2}+k_{3}^{2}=\frac{2 m w}{\hbar^{2}} \tag{3.15.8}
\end{equation*}
$$

The fact that $\phi$ is non identically vanishing entails $f_{1}, f_{2}, f_{3}$ are each non identically vanishing. The boundary conditions (3.15.3) obeyed by $\phi$ imply furthermore that $f_{1}, f_{2}, f_{3}$ satisfy
$$
\begin{align*}
& f_{1}(0)=f_{1}\left(l_{1}\right)=0  \tag{3.15.9a}\\
& f_{2}(0)=f_{2}\left(l_{2}\right)=0  \tag{3.15.9b}\\
& f_{3}(0)=f_{3}\left(l_{3}\right)=0 \tag{3.15.9c}
\end{align*}
$$

The solution of the differential problem (3.15.7), (3.15.9) yields the values
$$
\begin{align*}
& k_{1 n_{1}}=\frac{\pi n_{1}}{l_{1}}  \tag{3.15.10a}\\
& k_{2 n_{2}}=\frac{\pi n_{2}}{l_{2}}  \tag{3.15.10b}\\
& k_{3 n_{3}}=\frac{\pi n_{3}}{l_{3}} \tag{3.15.10c}
\end{align*}
$$
of the constants $k_{1}, k_{2}, k_{3}$ and correspondingly the expressions
$$
\begin{align*}
& f_{1 n_{1}}\left(x_{1}\right)=\left(\frac{2}{l_{1}}\right)^{1 / 2} \sin \left(\frac{\pi n_{1} x_{1}}{l_{1}}\right)  \tag{3.15.11a}\\
& f_{2 n_{2}}\left(x_{2}\right)=\left(\frac{2}{l_{2}}\right)^{1 / 2} \sin \left(\frac{\pi n_{2} x_{2}}{l_{2}}\right)  \tag{3.15.11b}\\
& f_{3 n_{3}}\left(x_{3}\right)=\left(\frac{2}{l_{3}}\right)^{1 / 2} \sin \left(\frac{\pi n_{3} x_{3}}{l_{3}}\right) \tag{3.15.11c}
\end{align*}
$$
of the functions $f_{1}\left(x_{1}\right), f_{2}\left(x_{2}\right), f_{3}\left(x_{3}\right)$, where $n_{1}, n_{2}, n_{3}$ are positive integers and the normalization constants are conventional.

Proof. We observe that the differential problem we have to solve is formally the same for the three space directions. It consists in solving the differential equation
$$
\begin{equation*}
\frac{d^{2} f}{d x^{2}}+k^{2} f=0 \tag{3.15.12}
\end{equation*}
$$
on an interval $0<x<l$ of a certain length $l$ with $f$ non identically vanishing and satisfying the boundary conditions
$$
\begin{equation*}
f(0)=f(l)=0 \tag{3.15.13}
\end{equation*}
$$

From standard calculus, the general solution of (3.15.12) is given by
$$
\begin{align*}
& f(x)=a \sin (k x)+b \cos (k x), \quad k \neq 0  \tag{3.15.14a}\\
& f(x)=a x+b, \quad k=0 \tag{3.15.14b}
\end{align*}
$$

By (3.15.14b), if $k=0, f$ satisfies the boundary condition (3.15.13) only if $a=b=0$. i. e. only if $f \equiv 0$, which is impossible. Consequently, $k \neq 0 . \quad f$ is thus given by (3.15.14a). The boundary condition (3.15.13) at $x=0$ is fulfilled provided $b=0$. Further, the condition at $x=l$ is satisfied if either $a=0$ or $k=\pi n / l$ with $n$ an integer. But, as $b=0, a \neq 0$ and $n \neq 0$, else $f \equiv 0$, which is impossible. From (3.15.14a),
further, it is apparent that negative values of $n$ yield the same solutions as the positive ones up to sign. In conclusion, $k$ is given by
$$
\begin{equation*}
k_{n}=\frac{\pi n}{l} \tag{3.15.15}
\end{equation*}
$$
with $n$ and a positive integer. Consequently, $f$ reads as
$$
\begin{equation*}
f_{n}(x)=\left(\frac{2}{l}\right)^{1 / 2} \sin \left(\frac{\pi n x}{l}\right) \tag{3.15.16}
\end{equation*}
$$
with conventional normalization. Eqs. (3.15.10) and (3.15.11) are so shown.

We can now write down the solution of the Schroedinger problem of the particle in the box. The energy eigenvalues and eigenfunctions of the particle are labelled by a triple $\boldsymbol{n}=\left(n_{1}, n_{2}, n_{3}\right)$ of positive integers and are given by
$$
\begin{equation*}
w_{\boldsymbol{n}}=\frac{\hbar^{2} \pi^{2}}{2 m}\left(\frac{n_{1}^{2}}{l_{1}{ }^{2}}+\frac{n_{2}^{2}}{l_{2}{ }^{2}}+\frac{n_{3}^{2}}{l_{3}{ }^{2}}\right) \tag{3.15.17}
\end{equation*}
$$
and
$$
\begin{equation*}
\phi_{\boldsymbol{n}}(\boldsymbol{x})=\left(\frac{8}{l_{1} l_{2} l_{3}}\right)^{1 / 2} \sin \left(\frac{\pi n_{1} x_{1}}{l_{1}}\right) \sin \left(\frac{\pi n_{2} x_{2}}{l_{2}}\right) \sin \left(\frac{\pi n_{3} x_{3}}{l_{3}}\right) \tag{3.15.18}
\end{equation*}
$$
respectively. The $\phi_{\boldsymbol{n}}$ are orthonormal, that is they satisfy
$$
\begin{equation*}
\int_{\mathcal{B}} d^{3} x \phi_{\boldsymbol{n}^{\prime}}{ }^{*} \phi_{\boldsymbol{n}}=\delta_{\boldsymbol{n}^{\prime}, \boldsymbol{n}} \tag{3.15.19}
\end{equation*}
$$

Furthermore, the $\phi_{\boldsymbol{n}}$ constitute a complete set, so that
$$
\begin{align*}
\phi & =\sum_{\boldsymbol{n}} c_{\boldsymbol{n}} \phi_{\boldsymbol{n}}  \tag{3.15.20}\\
c_{\boldsymbol{n}} & =\int_{\mathcal{B}} d^{3} x \phi_{\boldsymbol{n}}{ }^{*} \phi \tag{3.15.21}
\end{align*}
$$
for any wave function $\phi$.

Proof. (3.15.17) follows immediately from (3.15.8) and the (3.15.10). (3.15.18) results from substituting the (3.15.11) into (3.15.4).

Next, we show the orthonormality relation (3.15.19). Since the treatment of the three factors of the eigenfunction $\phi_{\boldsymbol{n}}$ is formally identical, we suppress configuration indices throughout. The functions $f_{n}$ with $n$ a positive integer given in eq. (3.15.16)
are orthonormal, that is
$$
\begin{equation*}
\int_{0}^{l} d x f_{n^{\prime}}{ }^{*} f_{n}=\delta_{n^{\prime}, n} \tag{3.15.22}
\end{equation*}
$$

To show the relations (3.15.22), we observe that
$$
\begin{align*}
\int_{0}^{l} d x f_{n^{\prime}}{ }^{*} f_{n} & =\frac{2}{l} \int_{0}^{l} d x \sin \left(\frac{\pi n^{\prime} x}{l}\right) \sin \left(\frac{\pi n x}{l}\right)  \tag{3.15.23}\\
& =\frac{1}{l} \int_{0}^{l} d x\left[\cos \left(\frac{\left(n^{\prime}-n\right) \pi x}{l}\right)-\cos \left(\frac{\left(n^{\prime}+n\right) \pi x}{l}\right)\right] \\
& =\frac{1}{\pi}\left[\int_{0}^{\pi} d \xi \cos \left(\left(n^{\prime}-n\right) \xi\right)-\int_{0}^{\pi} d \xi \cos \left(\left(n^{\prime}+n\right) \xi\right)\right]
\end{align*}
$$
where in the last step we have set $\xi=\pi x / l$. The two integrals vanish unless $n^{\prime}-n=0$ and $n^{\prime}+n=0$, respectively. The second condition is never fulfilled by the positivity of $n, n^{\prime}$. Relation (3.15.22) so follows. From (3.15.4) and (3.15.22), we have then
$$
\begin{align*}
\int_{\mathcal{B}} d^{3} x \phi_{\boldsymbol{n}^{\prime}}{ }^{*} \phi_{\boldsymbol{n}}=\int_{0}^{l_{1}} d x_{1} f_{1 n_{1}{ }^{\prime}}{ }^{*} f_{1 n_{1}} \int_{0}^{l_{2}} d x_{2} f_{2 n_{2}}{ }^{*} & f_{2 n_{2}} \int_{0}^{l_{3}} d x_{3} f_{3 n_{3}}{ }^{*} f_{3 n_{3}}  \tag{3.15.24}\\
& =\delta_{n_{1}, n_{1}} \delta_{n_{2}, n_{2}} \delta_{n_{3}{ }^{\prime}, n_{3}}=\delta_{\boldsymbol{n}^{\prime}, \boldsymbol{n}}
\end{align*}
$$
relation (3.15.19) is so proven.
The proof of the completeness relations (3.15.20) requires more work.
Step 1. Let $f(u)$ be a function defined and regular in the interval $0<u<\pi$. The $D^{ \pm}$transform of $f$ is the function $D^{ \pm} f(u)$ on the interval $0<u<\pi$ defined by
$$
\begin{equation*}
D^{ \pm} f(u)=\lim _{T \rightarrow \infty} \int_{0}^{\pi} d v \frac{\sin (T(u \mp v))}{2 \pi \sin ((u \mp v) / 2)} f(v) \tag{3.15.25}
\end{equation*}
$$

We want to to show that the limit above exists and compute it. To this end, we change the integration variable setting $z=\mp T(u \mp v)$. Then, $\mp T u<z<\mp T(u \mp \pi)$. Further, $v= \pm u+z / T$. We have so
$$
\begin{equation*}
D^{ \pm} f(u)=\lim _{T \rightarrow \infty} \int_{\mp T u}^{\mp T(u \mp \pi)} d z f( \pm u+z / T) \frac{\sin z}{2 \pi T \sin (z / 2 T)} \tag{3.15.26}
\end{equation*}
$$

We notice that $2 T \sin (z / 2 T) \simeq z$ for large $T$. For the upper sign, we have $-T u \rightarrow-\infty$, $-T(u-\pi) \rightarrow \infty$. Hence,
$$
\begin{equation*}
D^{+} f(u)=\frac{1}{\pi} \int_{-\infty}^{\infty} d z \frac{\sin z}{z} f(u)=f(u)=\int_{0}^{\pi} d v \delta(u-v) f(v) \tag{3.15.27}
\end{equation*}
$$
where the well-known value $\int_{-\infty}^{\infty} d z \sin z / z=\pi$ of the Dirichlet integral was used. For the upper sign, we simultaneously have $T u, T(u+\pi) \rightarrow \infty$. Hence,
$$
\begin{equation*}
D^{-} f(u)=0 \tag{3.15.28}
\end{equation*}
$$

Comparing (3.15.27), (3.15.28) with (3.15.26), we conclude that
$$
\begin{align*}
& \lim _{T \rightarrow \infty} \frac{\sin (T(u-v))}{2 \pi \sin ((u-v) / 2)}=\delta(u-v)  \tag{3.15.29}\\
& \lim _{T \rightarrow \infty} \frac{\sin (T(u+v))}{2 \pi \sin ((u+v) / 2)}=0 \tag{3.15.30}
\end{align*}
$$
for $0<u, v<\pi$.
Step 2. By the algebraic identity $\sum_{k=0}^{p} z^{k}=\left(z^{p+1}-1\right) /(z-1)$, we have
$$
\begin{align*}
\sum_{n=-N}^{N} \exp (i n u) & =\exp (-i N u) \sum_{n=-N}^{N} \exp (i(N+n) u)  \tag{3.15.31}\\
& =\exp (-i N u) \sum_{k=0}^{2 N} \exp (i k u) \\
& =\exp (-i N u) \frac{\exp (i(2 N+1) u)-1}{\exp (i u)-1} \\
& =\frac{\exp (i(N+1 / 2) u)-\exp (-i(N+1 / 2) u)}{\exp (i u / 2)-\exp (-i u / 2)}=\frac{\sin ((N+1 / 2) u)}{\sin (u / 2)}
\end{align*}
$$
where we set $k=N+n$. Let $0<x, x^{\prime}<l$. Using the computation (3.15.30), we find
$$
\begin{align*}
& \frac{2}{l} \sum_{n=1}^{\infty} \sin \left(\frac{\pi n x}{l}\right) \sin \left(\frac{\pi n x^{\prime}}{l}\right)  \tag{3.15.32}\\
&=-\frac{1}{2 l} \sum_{n=1}^{\infty}\left[\exp \left(\frac{i \pi n x}{l}\right)-\exp \left(-\frac{i \pi n x}{l}\right)\right]\left[\exp \left(\frac{i \pi n x^{\prime}}{l}\right)-\exp \left(-\frac{i \pi n x^{\prime}}{l}\right)\right] \\
&=-\frac{1}{2 l} \sum_{n=1}^{\infty}\left[\exp \left(\frac{i \pi n\left(x+x^{\prime}\right)}{l}\right)+\exp \left(-\frac{i \pi n\left(x+x^{\prime}\right)}{l}\right)\right. \\
&\left.\quad-\exp \left(\frac{i \pi n\left(x-x^{\prime}\right)}{l}\right)-\exp \left(-\frac{i \pi n\left(x-x^{\prime}\right)}{l}\right)\right] \\
&= \frac{1}{2 l}\left[\sum_{n=-\infty}^{\infty} \exp \left(\frac{i \pi n\left(x-x^{\prime}\right)}{l}\right)-\sum_{n=-\infty}^{\infty} \exp \left(\frac{i \pi n\left(x+x^{\prime}\right)}{l}\right)\right] \\
&=\frac{1}{2 l} \lim _{N \rightarrow \infty}\left[\sum_{n=-N}^{N} \exp \left(\frac{i \pi n\left(x-x^{\prime}\right)}{l}\right)-\sum_{n=-N}^{N} \exp \left(\frac{i \pi n\left(x+x^{\prime}\right)}{l}\right)\right] \\
&=\frac{1}{2 l} \lim _{N \rightarrow \infty}\left[\frac{\sin \left((N+1 / 2) \pi\left(x-x^{\prime}\right) / l\right)}{\sin \left(\pi\left(x-x^{\prime}\right) / 2 l\right)}-\frac{\sin \left((N+1 / 2) \pi\left(x+x^{\prime}\right) / l\right)}{\sin \left(\pi\left(x+x^{\prime}\right) / 2 l\right)}\right]
\end{align*}
$$

Using (3.15.29), (3.15.30) with $T=N+1 / 2$ above, we find
$$
\begin{equation*}
\frac{2}{l} \sum_{n=1}^{\infty} \sin \left(\frac{\pi n x}{l}\right) \sin \left(\frac{\pi n x^{\prime}}{l}\right)=\frac{\pi}{l} \delta\left(\frac{\pi\left(x-x^{\prime}\right)}{l}\right)=\delta\left(x-x^{\prime}\right) \tag{3.15.33}
\end{equation*}
$$
for $0<x, x^{\prime}<l$, where we used the identity $\delta(a u)=a^{-1} \delta(u)$ valid for $a>0$. In the notation (3.15.16), relation (3.15.33) reads as
$$
\begin{equation*}
\sum_{n=1}^{\infty} f_{n}(x) f_{n}^{*}\left(x^{\prime}\right)=\delta\left(x-x^{\prime}\right) \tag{3.15.34}
\end{equation*}
$$
where $0<x, x^{\prime}<l$.
Step 3. We now show that, for $\boldsymbol{x}, \boldsymbol{x}^{\prime}$ in $\mathcal{B}$,
$$
\begin{equation*}
\sum_{\boldsymbol{n}} \phi_{\boldsymbol{n}}(\boldsymbol{x}) \phi_{\boldsymbol{n}}{ }^{*}\left(\boldsymbol{x}^{\prime}\right)=\delta\left(\boldsymbol{x}-\boldsymbol{x}^{\prime}\right) \tag{3.15.35}
\end{equation*}
$$

Indeed, by (3.15.4) and the identity (3.15.34),
$$
\begin{align*}
& \sum_{\boldsymbol{n}} \phi_{\boldsymbol{n}}(\boldsymbol{x}) \phi_{\boldsymbol{n}}\left(\boldsymbol{x}^{\prime}\right)  \tag{3.15.36}\\
& \quad=\sum_{n_{1}=1}^{\infty} f_{1 n_{1}}\left(x_{1}\right) f_{1 n_{1}}{ }^{*}\left(x_{1}{ }^{\prime}\right) \sum_{n_{2}=1}^{\infty} f_{2 n_{2}}\left(x_{2}\right) f_{2 n_{2}}{ }^{*}\left(x_{2}{ }^{\prime}\right) \sum_{n_{3}=1}^{\infty} f_{3 n_{3}}\left(x_{3}\right) f_{3 n_{3}}{ }^{*}\left(x_{3}{ }^{\prime}\right) \\
& \quad=\delta\left(x_{1}-x_{1}{ }^{\prime}\right) \delta\left(x_{2}-x_{2}{ }^{\prime}\right) \delta\left(x_{3}-x_{3}{ }^{\prime}\right)=\delta\left(\boldsymbol{x}-\boldsymbol{x}^{\prime}\right)
\end{align*}
$$

Consider now a wave function $\phi$ defined in $\mathcal{B}$ and let $c_{\boldsymbol{n}}$ be the coefficients given by (3.15.21). Then,
$$
\begin{align*}
\sum_{\boldsymbol{n}} c_{\boldsymbol{n}} \phi_{\boldsymbol{n}}(\boldsymbol{x}) & =\sum_{\boldsymbol{n}} \phi_{\boldsymbol{n}}(\boldsymbol{x}) \int_{\mathcal{B}} d^{3} x^{\prime} \phi_{\boldsymbol{n}}{ }^{*}\left(\boldsymbol{x}^{\prime}\right) \phi\left(\boldsymbol{x}^{\prime}\right)  \tag{3.15.37}\\
& =\int_{\mathcal{B}} d^{3} x^{\prime} \sum_{\boldsymbol{n}} \phi_{\boldsymbol{n}}(\boldsymbol{x}) \phi_{\boldsymbol{n}}{ }^{*}\left(\boldsymbol{x}^{\prime}\right) \phi\left(\boldsymbol{x}^{\prime}\right)=\int_{\mathcal{B}} d^{3} x^{\prime} \delta\left(\boldsymbol{x}-\boldsymbol{x}^{\prime}\right) \phi\left(\boldsymbol{x}^{\prime}\right)=\phi(\boldsymbol{x})
\end{align*}
$$

This shows that the orthonormal set $\phi_{\boldsymbol{n}}$ is complete.

The eigenfunctions $\phi_{\boldsymbol{n}}$ and the associated probability density $\rho_{\boldsymbol{n}}=\left|\phi_{\boldsymbol{n}}\right|^{2}$ are plotted in fig. 3.15.2. With the harmonic time dependent factor $\exp \left(-i w_{\boldsymbol{n}} t / \hbar\right)$ attached, they represent standing waves in the box.

Next, we consider a related problem: a particle in a potential box some of whose three edges stretch to infinity. More specifically, we distinguish three cases: case $i$, where $l_{1}=\infty, l_{2}, l_{3}<\infty$; case $i i$, where $l_{1}, l_{2}=\infty, l_{3}<\infty$; case $i i i$, where $l_{1}, l_{2}, l_{3}=\infty$. In all three cases, the wave function $\phi$ satisfies the Schroedinger
![](https://cdn.mathpix.com/cropped/2024_09_22_5d1e855547710648961eg-0327.jpg?height=1107&width=905&top_left_y=516&top_left_x=604)

Figure 3.15.2. The energy eigenfunction $\phi_{n_{1} n_{2}}$ and the associated probability density $\rho_{n_{1} n_{2}}=\left|\phi_{n_{1} n_{2}}\right|^{2}$ for the values $n_{1}=5, n_{2}=4$. Here again, we consider space as two dimensional to allow for a pictorial representation.
equation (3.15.2) and the the left half of each of the three boundary condition pairs (3.15.3a)-(3.15.3c),
$$
\begin{align*}
& \phi\left(0, x_{2}, x_{3}\right)=0  \tag{3.15.38a}\\
& \phi\left(x_{1}, 0, x_{3}\right)=0  \tag{3.15.38b}\\
& \phi\left(x_{1}, x_{2}, 0\right)=0 \tag{3.15.38c}
\end{align*}
$$

There are further conditions depending on which case occurs. In case $i$, we have
$$
\begin{align*}
& \left|\phi\left(x_{1}, x_{2}, x_{3}\right)\right| \leq C_{1}, \quad x_{1} \rightarrow \infty  \tag{3.15.39a}\\
& \phi\left(x_{1}, l_{2}, x_{3}\right)=0  \tag{3.15.39b}\\
& \phi\left(x_{1}, x_{2}, l_{3}\right)=0 \tag{3.15.39c}
\end{align*}
$$

Here, the boundedness condition (3.15.39a) has replaced the second half of boundary condition pair (3.15.3a), while the remaining conditions (3.15.39b), (3.15.39c) are just the second half of the boundary condition pairs (3.15.3b), (3.15.3c). In case $i i$, we have similarly
$$
\begin{align*}
& \left|\phi\left(x_{1}, x_{2}, x_{3}\right)\right| \leq C_{1}, \quad x_{1} \rightarrow \infty  \tag{3.15.40a}\\
& \left|\phi\left(x_{1}, x_{2}, x_{3}\right)\right| \leq C_{2},  \tag{3.15.40b}\\
& x_{2} \rightarrow \infty  \tag{3.15.40c}\\
& \phi\left(x_{1}, x_{2}, l_{3}\right)=0
\end{align*}
$$

Here, the boundedness conditions (3.15.40a), (3.15.40b) have replaced the second half of boundary condition pairs (3.15.3a), (3.15.3b), while the remaining condition $(3.15 .40 \mathrm{c})$ is just the second half of the boundary condition pair (3.15.3c). Finally, in case $i i i$, we have
$$
\begin{align*}
& \left|\phi\left(x_{1}, x_{2}, x_{3}\right)\right| \leq C_{1}, \quad x_{1} \rightarrow \infty,  \tag{3.15.41a}\\
& \left|\phi\left(x_{1}, x_{2}, x_{3}\right)\right| \leq C_{2}, \quad x_{2} \rightarrow \infty,  \tag{3.15.41b}\\
& \left|\phi\left(x_{1}, x_{2}, x_{3}\right)\right| \leq C_{3}, \quad x_{3} \rightarrow \infty \tag{3.15.41c}
\end{align*}
$$

The boundedness conditions (3.15.40) take the place of the second half of boundary condition pairs (3.15.3). The solution of the Schroedinger problem proceeds again through the factorization ansatz (3.15.4). The functions $f_{1}, f_{2}, f_{3}$ satisfy again the equations (3.15.7) with the wave numbers $k_{1}, k_{2}, k_{3}$ obeying the constraint (3.15.8). In all three cases, $f_{1}, f_{2}, f_{3}$ obey the boundary conditions
$$
\begin{align*}
& f_{1}(0)=0  \tag{3.15.42a}\\
& f_{2}(0)=0 \tag{3.15.42b}
\end{align*}
$$
$$
\begin{equation*}
f_{3}(0)=0 \tag{3.15.42c}
\end{equation*}
$$
and further conditions depending on which of the three cases occurs, namely
$$
\begin{align*}
& \left|f_{1}\left(x_{1}\right)\right| \leq C_{1}, \quad x_{1} \rightarrow \infty  \tag{3.15.43a}\\
& f_{2}\left(l_{2}\right)=0  \tag{3.15.43b}\\
& f_{3}\left(l_{3}\right)=0 \tag{3.15.43c}
\end{align*}
$$
in case $i$, next
$$
\begin{align*}
& \left|f_{1}\left(x_{1}\right)\right| \leq C_{1}, \quad x_{1} \rightarrow \infty  \tag{3.15.44a}\\
& \left|f_{2}\left(x_{2}\right)\right| \leq C_{2},  \tag{3.15.44b}\\
& x_{2} \rightarrow \infty  \tag{3.15.44c}\\
& f_{3}\left(l_{3}\right)=0
\end{align*}
$$
in case $i i$ and finally
$$
\begin{align*}
& \left|f_{1}\left(x_{1}\right)\right| \leq C_{1}, \quad x_{1} \rightarrow \infty,  \tag{3.15.45a}\\
& \left|f_{2}\left(x_{2}\right)\right| \leq C_{2}, \quad x_{2} \rightarrow \infty,  \tag{3.15.45b}\\
& \left|f_{3}\left(x_{3}\right)\right| \leq C_{3}, \quad x_{3} \rightarrow \infty \tag{3.15.45c}
\end{align*}
$$
in case iii. The solutions of eqs. (3.15.7) obeying case by case the above boundary and boundedness conditions are readily obtained. We find the values
$$
\begin{align*}
& k_{1 \kappa_{1}}=\kappa_{1}  \tag{3.15.46a}\\
& k_{2 n_{2}}=\frac{\pi n_{2}}{l_{2}}  \tag{3.15.46b}\\
& k_{3 n_{3}}=\frac{\pi n_{3}}{l_{3}} \tag{3.15.46c}
\end{align*}
$$
of $k_{1}, k_{2}, k_{3}$ and the expressions
$$
\begin{align*}
& f_{1 \kappa_{1}}\left(x_{1}\right)=2 \sin \left(\kappa_{1} x_{1}\right)  \tag{3.15.47a}\\
& f_{2 n_{2}}\left(x_{2}\right)=\left(\frac{2}{l_{2}}\right)^{1 / 2} \sin \left(\frac{\pi n_{2} x_{2}}{l_{2}}\right) \tag{3.15.47b}
\end{align*}
$$
$$
\begin{equation*}
f_{3 n_{3}}\left(x_{3}\right)=\left(\frac{2}{l_{3}}\right)^{1 / 2} \sin \left(\frac{\pi n_{3} x_{3}}{l_{3}}\right) \tag{3.15.47c}
\end{equation*}
$$
of $f_{1}\left(x_{1}\right), f_{2}\left(x_{2}\right), f_{3}\left(x_{3}\right)$ in case $i$, the values
$$
\begin{align*}
& k_{1 \kappa_{1}}=\kappa_{1}  \tag{3.15.48a}\\
& k_{2 \kappa_{2}}=\kappa_{2}  \tag{3.15.48b}\\
& k_{3 n_{3}}=\frac{\pi n_{3}}{l_{3}} \tag{3.15.48c}
\end{align*}
$$
of $k_{1}, k_{2}, k_{3}$ and the expressions
$$
\begin{align*}
& f_{1 \kappa_{1}}\left(x_{1}\right)=2 \sin \left(\kappa_{1} x_{1}\right)  \tag{3.15.49a}\\
& f_{2 \kappa_{2}}\left(x_{2}\right)=2 \sin \left(\kappa_{2} x_{2}\right)  \tag{3.15.49b}\\
& f_{3 n_{3}}\left(x_{3}\right)=\left(\frac{2}{l_{3}}\right)^{1 / 2} \sin \left(\frac{\pi n_{3} x_{3}}{l_{3}}\right) \tag{3.15.49c}
\end{align*}
$$
of $f_{1}\left(x_{1}\right), f_{2}\left(x_{2}\right), f_{3}\left(x_{3}\right)$ in case $i i$ and
$$
\begin{align*}
& k_{1 \kappa_{1}}=\kappa_{1},  \tag{3.15.50a}\\
& k_{2 \kappa_{2}}=\kappa_{2},  \tag{3.15.50b}\\
& k_{3 \kappa_{3}}=\kappa_{3} . \tag{3.15.50c}
\end{align*}
$$
of $k_{1}, k_{2}, k_{3}$ and the expressions
$$
\begin{align*}
& f_{1 \kappa_{1}}\left(x_{1}\right)=2 \sin \left(\kappa_{1} x_{1}\right),  \tag{3.15.51a}\\
& f_{2 \kappa_{2}}\left(x_{2}\right)=2 \sin \left(\kappa_{2} x_{2}\right),  \tag{3.15.51b}\\
& f_{3 \kappa_{3}}\left(x_{3}\right)=2 \sin \left(\kappa_{3} x_{3}\right) \tag{3.15.51c}
\end{align*}
$$
of $f_{1}\left(x_{1}\right), f_{2}\left(x_{2}\right), f_{3}\left(x_{3}\right)$ in case iii. Above, in all cases, $\kappa_{1}, \kappa_{2}, \kappa_{3}$ are positive wave numbers and $n_{2}, n_{3}$ are positive integers.

Proof. The proof is similar enough to that of relations (3.15.10) and (3.15.11), the only difference being that the boundary conditions $f(l)=0$ is replaced by the
boundedness condition $|f(x)| \leq C$ as $x \rightarrow \infty$ when $l=\infty$. We consider now this latter case. The solution $(3.15 .14 \mathrm{~b})$, again, is to be rejected, since it vanishes at $x=0$ and is bounded as $x \rightarrow \infty$ only if $a=b=0$, that is if $f$ vanishes identically. The solution (3.15.14a) satisfies the boundary condition $f(0)=0$ again if $b=0$ and, for $b=0$, it is non identically vanishing and bounded as $x \rightarrow \infty$ if $k$ is real and non zero. From (3.15.14a), further, it is apparent that negative values of $k$ yield the same solutions as the positive ones up to sign. We thus have
$$
\begin{equation*}
k_{\kappa}=\kappa \tag{3.15.52}
\end{equation*}
$$
with $\kappa>0$. Consequently, $f$ reads
$$
\begin{equation*}
f_{\kappa}(x)=2 \sin (\kappa x) \tag{3.15.53}
\end{equation*}
$$
with conventional normalization. Eqs. (3.15.46)-(3.15.51) are in this way shown.

We can now write down the solution of the Schroedinger problem of the particle in the dilated box in the three relevant cases. In case $i$, the energy eigenvalues and eigenfunctions of the particle are labelled by a positive wave number $\kappa_{1}$ and two positive integer quantum numbers $n_{2}, n_{3}$ and are given by
$$
\begin{equation*}
w_{\kappa_{1}, n_{2}, n_{3}}=\frac{\hbar^{2}}{2 m}\left(\kappa_{1}^{2}+\frac{\pi^{2} n_{2}^{2}}{l_{2}^{2}}+\frac{\pi^{2} n_{3}^{2}}{l_{3}^{2}}\right) \tag{3.15.54}
\end{equation*}
$$
and
$$
\begin{equation*}
\phi_{\kappa_{1}, n_{2}, n_{3}}(\boldsymbol{x})=2\left(\frac{4}{l_{2} l_{3}}\right)^{1 / 2} \sin \left(\kappa_{1} x_{1}\right) \sin \left(\frac{\pi n_{2} x_{2}}{l_{2}}\right) \sin \left(\frac{\pi n_{3} x_{3}}{l_{3}}\right) \tag{3.15.55}
\end{equation*}
$$
respectively. The $\phi_{\kappa_{1}, n_{2}, n_{3}}$ are generalized orthonormal, that is they satisfy
$$
\begin{equation*}
\int_{\mathcal{B}} d^{3} x \phi_{\kappa_{1}^{\prime}, n_{2}{ }^{\prime}, n_{3}}{ }^{*} \phi_{\kappa_{1}, n_{2}, n_{3}}=2 \pi \delta\left(\kappa_{1}{ }^{\prime}-\kappa_{1}\right) \delta_{n_{2}{ }^{\prime}, n_{2}} \delta_{n_{3^{\prime}}, n_{3}} \tag{3.15.56}
\end{equation*}
$$

Furthermore, the $\phi_{\kappa_{1}, n_{2}, n_{3}}$ constitute a complete set, so that
$$
\begin{align*}
& \phi=\int_{0}^{\infty} \frac{d \kappa_{1}}{2 \pi} \sum_{n_{2}, n_{3}=1}^{\infty} c_{\kappa_{1}, n_{2}, n_{3}} \phi_{\kappa_{1}, n_{2}, n_{3}}  \tag{3.15.57}\\
& c_{\kappa_{1}, n_{2}, n_{3}}=\int_{\mathcal{B}} d^{3} x \phi_{\kappa_{1}, n_{2}, n_{3}}{ }^{*} \phi \tag{3.15.58}
\end{align*}
$$
for any wave function $\phi$. In case $i i$, the energy eigenvalues and eigenfunctions of the particle are labelled by two positive wave numbers $\kappa_{1}, \kappa_{2}$ and a positive integer quantum number $n_{3}$ and are given by
$$
\begin{equation*}
w_{\kappa_{1}, \kappa_{2}, n_{3}}=\frac{\hbar^{2}}{2 m}\left(\kappa_{1}^{2}+\kappa_{2}^{2}+\frac{\pi^{2} n_{3}^{2}}{l_{3}{ }^{2}}\right) \tag{3.15.59}
\end{equation*}
$$
and
$$
\begin{equation*}
\phi_{\kappa_{1}, \kappa_{2}, n_{3}}(\boldsymbol{x})=4\left(\frac{2}{l_{3}}\right)^{1 / 2} \sin \left(\kappa_{1} x_{1}\right) \sin \left(\kappa_{2} x_{2}\right) \sin \left(\frac{\pi n_{3} x_{3}}{l_{3}}\right) \tag{3.15.60}
\end{equation*}
$$
respectively. The $\phi_{\kappa_{1}, \kappa_{2}, n_{3}}$ are also generalized orthonormal,
$$
\begin{equation*}
\int_{\mathcal{B}} d^{3} x \phi_{\kappa_{1}^{\prime}, \kappa_{2}^{\prime}, n_{3}}{ }^{*} \phi_{\kappa_{1}, \kappa_{2}, n_{3}}=(2 \pi)^{2} \delta\left(\kappa_{1}^{\prime}-\kappa_{1}\right) \delta\left(\kappa_{2}^{\prime}-\kappa_{2}\right) \delta_{n_{3}}, n_{3} \tag{3.15.61}
\end{equation*}
$$

Furthermore, the $\phi_{\kappa_{1}, \kappa_{2}, n_{3}}$ constitute a complete set, so that
$$
\begin{align*}
& \phi=\int_{0}^{\infty} \frac{d \kappa_{1}}{2 \pi} \int_{0}^{\infty} \frac{d \kappa_{2}}{2 \pi} \sum_{n_{3}=1}^{\infty} c_{\kappa_{1}, \kappa_{2}, n_{3}} \phi_{\kappa_{1}, \kappa_{2}, n_{3}}  \tag{3.15.62}\\
& c_{\kappa_{1}, \kappa_{2}, n_{3}}=\int_{\mathcal{B}} d^{3} x \phi_{\kappa_{1}, \kappa_{2}, n_{3}}^{*} \phi \tag{3.15.63}
\end{align*}
$$
for any wave function $\phi$. In case $i i i$, finally, the energy eigenvalues and eigenfunctions of the particle are labelled by a triple $\boldsymbol{\kappa}=\left(\kappa_{1}, \kappa_{2}, \kappa_{3}\right)$ of positive wave numbers and are explicitly given by
$$
\begin{equation*}
w_{\boldsymbol{\kappa}}=\frac{\hbar^{2} \boldsymbol{\kappa}^{2}}{2 m} \tag{3.15.64}
\end{equation*}
$$
and
$$
\begin{equation*}
\phi_{\boldsymbol{\kappa}}(\boldsymbol{x})=8 \sin \left(\kappa_{1} x_{1}\right) \sin \left(\kappa_{2} x_{2}\right) \sin \left(\kappa_{3} x_{3}\right) \tag{3.15.65}
\end{equation*}
$$
respectively. The $\phi_{\kappa}$ are again generalized orthonormal,
$$
\begin{equation*}
\int_{\mathcal{B}} d^{3} x \phi_{\boldsymbol{\kappa}^{\prime}}{ }^{*} \phi_{\boldsymbol{\kappa}}=(2 \pi)^{3} \delta\left(\boldsymbol{\kappa}^{\prime}-\boldsymbol{\kappa}\right) \tag{3.15.66}
\end{equation*}
$$

Furthermore, the $\phi_{\boldsymbol{\kappa}}$ constitute a complete set, so that
$$
\begin{align*}
& \phi=\int_{\mathcal{K}} \frac{d^{3} \kappa}{(2 \pi)^{3}} c_{\boldsymbol{\kappa}} \phi_{\boldsymbol{\kappa}}  \tag{3.15.67}\\
& c_{\boldsymbol{\kappa}}=\int_{\mathcal{B}} d^{3} x \phi_{\boldsymbol{\kappa}}{ }^{*} \phi \tag{3.15.68}
\end{align*}
$$
for any wave function $\phi$, where $\mathcal{K}$ is the octant $\kappa_{1}, \kappa_{2}, \kappa_{3}>0$.

Proof. (3.15.54), (3.15.59) (3.15.64) follow immediately from (3.15.8) upon using respectively (3.15.46), (3.15.48), (3.15.50). (3.15.55), (3.15.60), (3.15.65) result from substituting respectively the (3.15.47), (3.15.49), (3.15.51) into (3.15.4).

Next, we show the orthonormality relations (3.15.56), (3.15.61), (3.15.66). Since the treatment of the three factors of the eigenfunctions is formally identical, we suppress configuration indices temporarily. For finite $l$, relation (3.15.22) holds, as it was proven earlier. For $l=\infty$, we have instead
$$
\begin{equation*}
\int_{0}^{\infty} d x f_{\kappa^{\prime}}{ }^{*} f_{\kappa}=2 \pi \delta\left(\kappa^{\prime}-\kappa\right) . \tag{3.15.69}
\end{equation*}
$$

Indeed, we have
$$
\begin{align*}
\int_{0}^{\infty} d x f_{\kappa^{\prime}}{ }^{*} f_{\kappa}= & 4 \int_{0}^{\infty} d x \sin \left(\kappa^{\prime} x\right) \sin (\kappa x)  \tag{3.15.70}\\
= & -\int_{0}^{\infty} d x\left[\exp \left(i \kappa^{\prime} x\right)-\exp \left(-i \kappa^{\prime} x\right)\right][\exp (i \kappa x)-\exp (-i \kappa x)] \\
= & \int_{0}^{\infty} d x\left[\exp \left(i\left(\kappa^{\prime}-\kappa\right) x\right)+\exp \left(-i\left(\kappa^{\prime}-\kappa\right) x\right)\right. \\
& \left.\quad-\exp \left(i\left(\kappa^{\prime}+\kappa\right) x\right)-\exp \left(-i\left(\kappa^{\prime}+\kappa\right) x\right)\right] \\
= & \int_{-\infty}^{\infty} d x \exp \left(i\left(\kappa^{\prime}-\kappa\right) x\right)-\int_{-\infty}^{\infty} d x \exp \left(i\left(\kappa^{\prime}+\kappa\right) x\right) \\
= & 2 \pi\left[\delta\left(\kappa^{\prime}-\kappa\right)-\delta\left(\kappa^{\prime}+\kappa\right)\right]
\end{align*}
$$
where we used the integral representation
$$
\begin{equation*}
\int_{-\infty}^{\infty} \frac{d x}{2 \pi} \exp (i k x)=\delta(k) \tag{3.15.71}
\end{equation*}
$$
of the wave number delta function $\delta(k)$ analogous to (3.5.57). Above $\delta\left(\kappa^{\prime}+\kappa\right)=0$ identically, as $\kappa, \kappa^{\prime}>0$. Relation (3.15.69) so follows.

Using relations (3.15.22), (3.15.69), in case $i$ we find
$$
\begin{align*}
& \int_{\mathcal{B}} d^{3} x \phi_{\kappa_{1}{ }^{\prime}, n_{2}{ }^{\prime}, n_{3}}{ }^{*}{ }^{*} \phi_{\kappa_{1}, n_{2}, n_{3}}  \tag{3.15.72}\\
& =\int_{0}^{\infty} d x_{1} f_{1 \kappa_{1}{ }{ }^{*}} f_{1 \kappa_{1}} \int_{0}^{l_{2}} d x_{2} f_{2 n_{2}{ }^{\prime}}{ }^{*} f_{2 n_{2}} \int_{0}^{l_{3}} d x_{3} f_{3 n_{3}}{ }^{*}{ }^{*} f_{3 n_{3}}
\end{align*}
$$
$$
=2 \pi \delta\left(\kappa_{1}^{\prime}-\kappa_{1}\right) \delta_{n_{2}{ }^{\prime}, n_{2}} \delta_{n_{3}^{\prime}, n_{3}}
$$
showing (3.15.56). Proceeding similarly, in case $i i$ we find
$$
\begin{align*}
& \int_{\mathcal{B}} d^{3} x \phi_{\kappa_{1}}, \kappa_{2}, n_{3}{ }^{*} \phi_{\kappa_{1}, \kappa_{2}, n_{3}}  \tag{3.15.73}\\
& \quad=\int_{0}^{\infty} d x_{1} f_{1 \kappa_{1}{ }^{\prime}}{ }^{*} f_{1 \kappa_{1}} \int_{0}^{\infty} d x_{2} f_{2 \kappa_{2}}{ }^{*} f_{2 \kappa_{2}} \int_{0}^{l_{3}} d x_{3} f_{3 n_{3}{ }{ }^{*}} f_{3 n_{3}} \\
& \quad=(2 \pi)^{2} \delta\left(\kappa_{1}{ }^{\prime}-\kappa_{1}\right) \delta\left(\kappa_{2}{ }^{\prime}-\kappa_{2}\right) \delta_{n_{3}{ }^{\prime}, n_{3}}
\end{align*}
$$
proving (3.15.61). Finally, in case iii we get
$$
\begin{align*}
\int_{\mathcal{B}} d^{3} x \phi_{\boldsymbol{\kappa}^{\prime}}{ }^{*} \phi_{\boldsymbol{\kappa}}=\int_{0}^{\infty} d x_{1} f_{1 \kappa_{1}{ }^{\prime}}{ }^{*} f_{1 \kappa_{1}} \int_{0}^{\infty} d x_{2} f_{2 \kappa_{2}{ }^{\prime}}{ }^{*} f_{2 \kappa_{2}} \int_{0}^{\infty} d x_{3} f_{3 \kappa_{3}{ }^{*}}{ }^{*} f_{3 \kappa_{3}}  \tag{3.15.74}\\
=(2 \pi)^{3} \delta\left(\kappa_{1}{ }^{\prime}-\kappa_{1}\right) \delta\left(\kappa_{2}{ }^{\prime}-\kappa_{2}\right) \delta\left(\kappa_{3}{ }^{\prime}-\kappa_{3}\right)=(2 \pi)^{3} \delta\left(\boldsymbol{\kappa}^{\prime}-\boldsymbol{\kappa}\right)
\end{align*}
$$

\section*{(3.15.66) is so also shown.}

Next, we show the completeness relations (3.15.57), (3.15.62), (3.15.67). Again, since the treatment of the three factors of the eigenfunctions is formally identical, we leave configuration indices understood. For finite $l$, relation (3.15.34), which was proven earlier, holds. For $l=\infty$, we have instead
$$
\begin{equation*}
\int_{0}^{\infty} \frac{d \kappa}{2 \pi} f_{\kappa}(x) f_{\kappa}{ }^{*}\left(x^{\prime}\right)=\delta\left(x-x^{\prime}\right) \tag{3.15.75}
\end{equation*}
$$

Indeed, we have
$$
\begin{equation*}
\int_{0}^{\infty} d \kappa f_{\kappa}(x) f_{\kappa}^{*}\left(x^{\prime}\right)=4 \int_{0}^{\infty} d \kappa \sin (\kappa x) \sin \left(\kappa x^{\prime}\right)=2 \pi\left[\delta\left(x-x^{\prime}\right)-\delta\left(x+x^{\prime}\right)\right] \tag{3.15.76}
\end{equation*}
$$
by a calculation essentially identical to (3.15.70) with $x, \kappa, \kappa^{\prime}$ replaced by $\kappa, x, x^{\prime}$ respectively. Since $\delta\left(x+x^{\prime}\right)=0$ being $x, x^{\prime}>0,(3.15 .75)$ holds.

Using relations (3.15.34), (3.15.75), we can obtain in explicit form the required eigenfunction completeness relations in the three relevant cases. In case $i$ we find
$$
\begin{equation*}
\int_{0}^{\infty} \frac{d \kappa_{1}}{2 \pi} \sum_{n_{2}, n_{3}=1}^{\infty} \phi_{\kappa_{1}, n_{2}, n_{3}}(\boldsymbol{x}) \phi_{\kappa_{1}, n_{2}, n_{3}}^{*}\left(\boldsymbol{x}^{\prime}\right)=\delta\left(\boldsymbol{x}-\boldsymbol{x}^{\prime}\right) \tag{3.15.77}
\end{equation*}
$$

Indeed, we have
$$
\begin{equation*}
\int_{0}^{\infty} \frac{d \kappa_{1}}{2 \pi} \sum_{n_{2}, n_{3}=1}^{\infty} \phi_{\kappa_{1}, n_{2}, n_{3}}(\boldsymbol{x}) \phi_{\kappa_{1}, n_{2}, n_{3}}{ }^{*}\left(\boldsymbol{x}^{\prime}\right) \tag{3.15.78}
\end{equation*}
$$
$$
\begin{aligned}
& =\int_{0}^{\infty} \frac{d \kappa_{1}}{2 \pi} f_{1 \kappa_{1}}\left(x_{1}\right) f_{1 \kappa_{1}}{ }^{*}\left(x_{1}^{\prime}\right) \sum_{n_{2}=1}^{\infty} f_{2 n_{2}}\left(x_{2}\right) f_{2 n_{2}}{ }^{*}\left(x_{2}^{\prime}\right) \sum_{n_{3}=1}^{\infty} f_{3 n_{3}}\left(x_{3}\right) f_{3 n_{3}}{ }^{*}\left(x_{3}{ }^{\prime}\right) \\
& =\delta\left(x_{1}-x_{1}^{\prime}\right) \delta\left(x_{2}-x_{2} "\right) \delta\left(x_{2}-x_{2}^{\prime}\right)=\delta\left(\boldsymbol{x}-\boldsymbol{x}^{\prime}\right)
\end{aligned}
$$
as stated. Proceeding similarly, in case $i i$ we find
$$
\begin{equation*}
\int_{0}^{\infty} \frac{d \kappa_{1}}{2 \pi} \int_{0}^{\infty} \frac{d \kappa_{2}}{2 \pi} \sum_{n_{3}=1}^{\infty} \phi_{\kappa_{1}, \kappa_{2}, n_{3}}(\boldsymbol{x}) \phi_{\kappa_{1}, \kappa_{2}, n_{3}}^{*}\left(\boldsymbol{x}^{\prime}\right)=\delta\left(\boldsymbol{x}-\boldsymbol{x}^{\prime}\right) \tag{3.15.79}
\end{equation*}
$$

Indeed, we have
$$
\begin{align*}
& \int_{0}^{\infty} \frac{d \kappa_{1}}{2 \pi} \int_{0}^{\infty} \frac{d \kappa_{2}}{2 \pi} \sum_{n_{3}=1}^{\infty} \phi_{\kappa_{1}, \kappa_{2}, n_{3}}(\boldsymbol{x}) \phi_{\kappa_{1}, \kappa_{2}, n_{3}}\left(\boldsymbol{x}^{\prime}\right)  \tag{3.15.80}\\
& =\int_{0}^{\infty} \frac{d \kappa_{1}}{2 \pi} f_{1 \kappa_{1}}\left(x_{1}\right) f_{1 \kappa_{1}}{ }^{*}\left(x_{1}{ }^{\prime}\right) \int_{0}^{\infty} \frac{d \kappa_{2}}{2 \pi} f_{2 \kappa_{2}}\left(x_{2}\right) f_{2 \kappa_{2}}{ }^{*}\left(x_{2}{ }^{\prime}\right) \sum_{n_{3}=1}^{\infty} f_{3 n_{3}}\left(x_{3}\right) f_{3 n_{3}}{ }^{*}\left(x_{3}{ }^{\prime}\right) \\
& =\delta\left(x_{1}-x_{1}{ }^{\prime}\right) \delta\left(x_{2}-x_{2}{ }^{\prime}\right) \delta\left(x_{2}-x_{2}{ }^{\prime}\right)=\delta\left(\boldsymbol{x}-\boldsymbol{x}^{\prime}\right)
\end{align*}
$$

Finally, in case $i$ ii we get
$$
\begin{equation*}
\int_{\mathcal{K}} \frac{d^{3} \kappa}{(2 \pi)^{3}} \phi_{\boldsymbol{\kappa}}(\boldsymbol{x}) \phi_{\boldsymbol{\kappa}}^{*}\left(\boldsymbol{x}^{\prime}\right)=\delta\left(\boldsymbol{x}-\boldsymbol{x}^{\prime}\right) \tag{3.15.81}
\end{equation*}
$$
by the calculation
$$
\begin{align*}
& \int_{\mathcal{K}} \frac{d^{3} \kappa}{(2 \pi)^{3}} \phi_{\boldsymbol{\kappa}}(\boldsymbol{x}) \phi_{\boldsymbol{\kappa}}{ }^{*}\left(\boldsymbol{x}^{\prime}\right)  \tag{3.15.82}\\
& =\int_{0}^{\infty} \frac{d \kappa_{1}}{2 \pi} f_{1 \kappa_{1}}\left(x_{1}\right) f_{1 \kappa_{1}}{ }^{*}\left(x_{1}{ }^{\prime}\right) \int_{0}^{\infty} \frac{d \kappa_{2}}{2 \pi} f_{2 \kappa_{2}}\left(x_{2}\right) f_{2 \kappa_{2}}{ }^{*}\left(x_{2}{ }^{\prime}\right) \int_{0}^{\infty} \frac{d \kappa_{3}}{2 \pi} f_{3 \kappa_{3}}\left(x_{3}\right) f_{3 \kappa_{3}}{ }^{*}\left(x_{3}{ }^{\prime}\right) \\
& =\delta\left(x_{1}-x_{1}{ }^{\prime}\right) \delta\left(x_{2}-x_{2}{ }^{\prime}\right) \delta\left(x_{3}-x_{3}{ }^{\prime}\right)=\delta\left(\boldsymbol{x}-\boldsymbol{x}^{\prime}\right) .
\end{align*}
$$

Consider now a wave function $\phi$ defined in $\mathcal{B}$. In case $i$, assuming that the coefficients $c_{\kappa_{1}, n_{2}, n_{3}}$ given by (3.15.58), we have
$$
\begin{align*}
& \int_{0}^{\infty} \frac{d \kappa_{1}}{2 \pi} \sum_{n_{2}, n_{3}=1}^{\infty} c_{\kappa_{1}, n_{2}, n_{3}} \phi_{\kappa_{1}, n_{2}, n_{3}}(\boldsymbol{x})  \tag{3.15.83}\\
& =\int_{0}^{\infty} \frac{d \kappa_{1}}{2 \pi} \sum_{n_{2}, n_{3}=1}^{\infty} \phi_{\kappa_{1}, n_{2}, n_{3}}(\boldsymbol{x}) \int_{\mathcal{B}} d^{3} x^{\prime} \phi_{\kappa_{1}, n_{2}, n_{3}}{ }^{*}\left(\boldsymbol{x}^{\prime}\right) \phi\left(\boldsymbol{x}^{\prime}\right) \\
& =\int_{\mathcal{B}} d^{3} x^{\prime} \int_{0}^{\infty} \frac{d \kappa_{1}}{2 \pi} \sum_{n_{2}, n_{3}=1}^{\infty} \phi_{\kappa_{1}, n_{2}, n_{3}}(\boldsymbol{x}) \phi_{\kappa_{1}, n_{2}, n_{3}}{ }^{*}\left(\boldsymbol{x}^{\prime}\right) \phi\left(\boldsymbol{x}^{\prime}\right) \\
& =\int_{\mathcal{B}} d^{3} x^{\prime} \delta\left(\boldsymbol{x}-\boldsymbol{x}^{\prime}\right) \phi\left(\boldsymbol{x}^{\prime}\right)=\phi(\boldsymbol{x})
\end{align*}
$$

Identity (3.15.57) is so demonstrated. Next, in case $i i$, with the coefficients $c_{\kappa_{1}, \kappa_{2}, n_{3}}$
given by (3.15.63), we have
$$
\begin{align*}
& \int_{0}^{\infty} \frac{d \kappa_{1}}{2 \pi} \int_{0}^{\infty} \frac{d \kappa_{2}}{2 \pi} \sum_{n_{3}=1}^{\infty} c_{\kappa_{1}, \kappa_{2}, n_{3}} \phi_{\kappa_{1}, \kappa_{2}, n_{3}}(\boldsymbol{x})  \tag{3.15.84}\\
& =\int_{0}^{\infty} \frac{d \kappa_{1}}{2 \pi} \int_{0}^{\infty} \frac{d \kappa_{2}}{2 \pi} \sum_{n_{3}=1}^{\infty} \phi_{\kappa_{1}, \kappa_{2}, n_{3}}(\boldsymbol{x}) \int_{\mathcal{B}} d^{3} x^{\prime} \phi_{\kappa_{1}, \kappa_{2}, n_{3}}{ }^{*}\left(\boldsymbol{x}^{\prime}\right) \phi\left(\boldsymbol{x}^{\prime}\right) \\
& =\int_{\mathcal{B}} d^{3} x^{\prime} \int_{0}^{\infty} \frac{d \kappa_{1}}{2 \pi} \int_{0}^{\infty} \frac{d \kappa_{2}}{2 \pi} \sum_{n_{3}=1}^{\infty} \phi_{\kappa_{1}, \kappa_{2}, n_{3}}(\boldsymbol{x}) \phi_{\kappa_{1}, \kappa_{2}, n_{3}}{ }^{*}\left(\boldsymbol{x}^{\prime}\right) \phi\left(\boldsymbol{x}^{\prime}\right) \\
& =\int_{\mathcal{B}} d^{3} x^{\prime} \delta\left(\boldsymbol{x}-\boldsymbol{x}^{\prime}\right) \phi\left(\boldsymbol{x}^{\prime}\right)=\phi(\boldsymbol{x})
\end{align*}
$$

This shows relation (3.15.62). Finally, in case $i i$, with the coefficients $c_{\kappa}$ given by (3.15.68), we find
$$
\begin{align*}
\int_{\mathcal{K}} \frac{d^{3} \kappa}{(2 \pi)^{3}} c_{\boldsymbol{\kappa}} \phi_{\boldsymbol{\kappa}}(\boldsymbol{x})=\int_{\mathcal{K}} \frac{d^{3} \kappa}{(2 \pi)^{3}} \phi_{\boldsymbol{\kappa}}(\boldsymbol{x}) \int_{\mathcal{B}} d^{3} x^{\prime} \phi_{\boldsymbol{\kappa}}{ }^{*}\left(\boldsymbol{x}^{\prime}\right) \phi\left(\boldsymbol{x}^{\prime}\right)  \tag{3.15.85}\\
=\int_{\mathcal{B}} d^{3} x^{\prime} \int_{\mathcal{K}} \frac{d^{3} \kappa}{(2 \pi)^{3}} \phi_{\boldsymbol{\kappa}}(\boldsymbol{x}) \phi_{\boldsymbol{\kappa}}{ }^{*}\left(\boldsymbol{x}^{\prime}\right) \phi\left(\boldsymbol{x}^{\prime}\right)=\int_{\mathcal{B}} d^{3} x^{\prime} \delta\left(\boldsymbol{x}-\boldsymbol{x}^{\prime}\right) \phi\left(\boldsymbol{x}^{\prime}\right)=\phi(\boldsymbol{x})
\end{align*}
$$
proving also identity (3.15.67).

The case iii eigenfunctions $\phi_{\boldsymbol{\kappa}}$ are plotted in fig. 3.15.3.
We now would like to understand whether the eigenvalues and eigenfunctions of the potential box with some infinitely long edges are in a sense to be made precise a limit of the eigenvalues and eigenfunctions of the potential box with finitely long edges. To make our discussion as simple as possible, we consider a potential box characterized by a length scale $\ell$ that sets its size, in the sense that the box is identified as the region of configuration space such that $0 \leq x_{1} \leq l_{1}=$ $\ell \lambda_{1}, 0 \leq x_{2} \leq l_{2}=\ell \lambda_{2}, 0 \leq x_{3} \leq l_{3}=\ell \lambda_{3}$, where $\lambda_{1}, \lambda_{2}, \lambda_{3}$ are fixed numbers. We want to study the energy eigenvalues and eigenfunctions of the box as $\ell \rightarrow \infty$. To this end, we shall attach a suffix $\ell$ to every object that depends on $\ell$.

As we found above, the energy eigenvalues $w_{\ell n}$ and eigenfunctions $\phi_{\ell n}$ are labelled by a triple of positive integer quantum numbers $\boldsymbol{n}=\left(n_{1}, n_{2}, n_{3}\right)$. By $(3.15 .17),(3.15 .18), w_{\ell n}$ reads so as
$$
\begin{equation*}
w_{\ell n}=\frac{\hbar^{2} \pi^{2}}{2 m \ell^{2}}\left(\frac{n_{1}{ }^{2}}{\lambda_{1}{ }^{2}}+\frac{n_{2}{ }^{2}}{\lambda_{2}{ }^{2}}+\frac{n_{3}{ }^{2}}{\lambda_{3}{ }^{2}}\right) \tag{3.15.86}
\end{equation*}
$$

![](https://cdn.mathpix.com/cropped/2024_09_22_5d1e855547710648961eg-0337.jpg?height=416&width=782&top_left_y=535&top_left_x=650)

Figure 3.15.3. The energy eigenfunction $\phi_{\kappa}$ and $\kappa_{1} l_{1}=\kappa_{1} l_{2}=$ .5236 in a pictorial two dimensional representation.
while $\phi_{\ell n}$ can be cast as
$$
\begin{equation*}
\phi_{\ell \boldsymbol{n}}(\boldsymbol{x})=\left(\frac{8}{\ell^{3} \lambda_{1} \lambda_{2} \lambda_{3}}\right)^{1 / 2} \sin \left(\frac{\pi n_{1} x_{1}}{\ell \lambda_{1}}\right) \sin \left(\frac{\pi n_{2} x_{2}}{\ell \lambda_{2}}\right) \sin \left(\frac{\pi n_{3} x_{3}}{\ell \lambda_{3}}\right) \tag{3.15.87}
\end{equation*}
$$

We examine now the behavior of $w_{\ell n}$ and $\phi_{\ell n}$ as $\ell \rightarrow \infty$.
By (3.15.86), for fixed $\boldsymbol{n}, w_{\ell \boldsymbol{n}} \rightarrow 0$ as $\ell \rightarrow \infty$. However, this does not mean that the whole energy spectrum is squashed down to 0 . Since the $n_{1}, n_{2}, n_{3}$ are not bounded above, for an arbitrarily large $\ell$ there will be some sufficiently large $\boldsymbol{n}$ such that $w_{\ell \boldsymbol{n}}$ is as large as one wishes. In the limit $\ell \rightarrow \infty$, rather, the eigenvalues become denser and denser, since by (3.15.86) the difference of any two consecutive eigenvalues falls off as $1 / \ell^{2}$ (cf. fig. 3.15.4). So, the discrete spectrum turns into the energy interval $[0, \infty[$ and becomes continuous.

The analysis of the behaviour of the eigenfunctions $\phi_{\ell \boldsymbol{n}}$ as $\ell \rightarrow \infty$ is subtler. By (3.15.87), $\phi_{\ell \boldsymbol{n}}$ vanishes identically when $\ell \rightarrow \infty$ because of the suppressing factor $\ell^{-3 / 2}$. Since the Schroedinger equation is linear, however, we are allowed to redefine the normalization of the eigenfunctions. To avoid a trivial limit, we consider the renormalized eigenfunctions $\ell^{3 / 2} \phi_{\ell \boldsymbol{n}}$. By (3.15.87), for fixed $\boldsymbol{n}$, $\ell^{3 / 2} \phi_{\ell \boldsymbol{n}}(\boldsymbol{x}) \rightarrow 0$ as $\ell \rightarrow \infty$ for any point $\boldsymbol{x}$ lying in the domain $\mathcal{B}_{\ell}$ the eigenfunctions are defined on for sufficiently large $\ell$. But now this does not imply at all that all the eigenfunctions $\phi_{\ell \boldsymbol{n}}$ vanish identically in the limit. In fact, as $\mathcal{B}_{\ell}$ de-

![](https://cdn.mathpix.com/cropped/2024_09_22_5d1e855547710648961eg-0338.jpg?height=692&width=944&top_left_y=565&top_left_x=577)

Figure 3.15.4. As $\ell$ grows, the discrete energy spectrum (a) becomes denser and denser (b) and eventually turns into a continuous energy spectrum $(c)$.
pends on $\ell$ and is indeed of linear size $\ell, \ell^{3 / 2} \phi_{\ell \boldsymbol{n}}(\boldsymbol{x})$ will be appreciably different from 0 at those points $\boldsymbol{x}$ in $\mathcal{B}_{\ell}$ lying at a distance of order $\ell$ from the origin, regardless how large $\ell$ is. Further, since the $n_{1}, n_{2}, n_{3}$ are not bounded above, for arbitrarily large $\ell$ there will be some sufficiently large $\boldsymbol{n}$ such that $\ell^{3 / 2} \phi_{\ell \boldsymbol{n}}(\boldsymbol{x})$ differs appreciably from 0 at any point $\boldsymbol{x}$ in $\mathcal{B}_{\ell}$. The eigenfunctions rather form a kind of lattice in an infinite dimensional continuous eigenfunction space. For large $\ell$ and large $n_{1}, n_{2}, n_{3}, \ell^{3 / 2} \phi_{\ell \boldsymbol{n}}(\boldsymbol{x})$ changes little when $n_{1}, n_{2}, n_{3}$ change of a few units. In the limit $\ell \rightarrow \infty$, this lattice becomes denser and denser and eventually fills the whole space. The eigenfunction lattice thus turns into a continuum analogously to the eigenvalues. This statement should however be made more precise.

To properly understand the mechanism underlying the emergence of an eigenvalue and eigenfunction continuum, it is necessary to examine more closely the combined dependence of the energy eigenvalues $w_{\ell \boldsymbol{n}}$ and eigenfunctions $\phi_{\ell \boldsymbol{n}}$ on
the quantum numbers $n_{1}, n_{2}, n_{3}$ and the length scale $\ell$. In the right hand side of (3.15.86), (3.15.87), the $n_{1}, n_{2}, n_{3}$ and $\ell$ always appear in the combinations
$$
\begin{equation*}
\kappa_{1 \ell}=\frac{\pi n_{1}}{\ell \lambda_{1}}, \quad \kappa_{2 \ell}=\frac{\pi n_{2}}{\ell \lambda_{2}}, \quad \kappa_{3 \ell}=\frac{\pi n_{3}}{\ell \lambda_{3}} \tag{3.15.88}
\end{equation*}
$$
organizing in a triple $\boldsymbol{\kappa}_{\ell}=\left(\kappa_{1 \ell}, \kappa_{2 \ell}, \kappa_{3 \ell}\right)$ of rescaled quantum numbers. Indeed, in terms of these, $w_{\ell \boldsymbol{n}}$ reads as
$$
\begin{equation*}
w_{\ell \boldsymbol{n}}=\tilde{w}_{\boldsymbol{\kappa}_{\ell}}=\frac{\hbar^{2} \boldsymbol{\kappa}_{\ell}^{2}}{2 m} \tag{3.15.89}
\end{equation*}
$$
while $\phi_{\ell_{n}}$ can be expressed as
$$
\begin{equation*}
(2 \lambda \ell)^{3 / 2} \phi_{\ell \boldsymbol{n}}(\boldsymbol{x})=\tilde{\phi}_{\boldsymbol{\kappa}_{\ell}}(\boldsymbol{x})=8 \sin \left(\kappa_{1 \ell} x_{1}\right) \sin \left(\kappa_{2 \ell} x_{2}\right) \sin \left(\kappa_{3 \ell} x_{3}\right) \tag{3.15.90}
\end{equation*}
$$
where $\lambda=\left(\lambda_{1} \lambda_{2} \lambda_{3}\right)^{1 / 3}$. This observation is the clue to properly understand the limit $\ell \rightarrow \infty$.

Let us now analyse the relationship between the quantum number triples $\boldsymbol{n}$ and the rescaled quantum number triples $\boldsymbol{\kappa}_{\ell}$ both for fixed $\ell$ and in the limit $\ell \rightarrow \infty$. The triples $\boldsymbol{n}$ are made of three positive integer numbers independent from $\ell$. They thus form a discrete 3-dimensional quantum number lattice $\mathcal{Q}_{0}$ (cf. fig. 3.15.5). The rescaled triples $\boldsymbol{\kappa}_{\ell}$ instead consist of three real numbers dependent on $\ell$, each of which is positive integer multiples of $1 / \ell$. They thus are contained in a continuous 3-dimensional quantum number space $\mathcal{Q}$. For a given value of $\ell$, the triples $\boldsymbol{n}$ of $\mathcal{Q}_{0}$ are in one-to-one correspondence with the rescaled triples $\boldsymbol{\kappa}_{\ell}$ of $\mathcal{Q}$ via (3.15.88). In this way, the $\boldsymbol{\kappa}_{\ell}$ span a 3 -dimensional lattice in $\mathcal{Q}$ (cf. fig. 3.15.6 (a)). As $\ell$ grows, this lattice becomes denser and denser (cf. fig. 3.15.6 (b)) and eventually fills $\mathcal{Q}$ (cf. fig. 3.15.6 (c)). In fact, though each individual triple $\boldsymbol{\kappa}_{\ell}$ tends to $\mathbf{0}$ as $\ell \rightarrow \infty$, the whole set of triples does not collapse to $\mathbf{0}$, since in (3.15.88) the integers $n_{1}, n_{2}, n_{3}$ can be arbitrarily large.

Since the energy eigenvalues $\tilde{w}_{\ell \boldsymbol{n}}$ and eigenfunctions $\tilde{\phi}_{\ell \boldsymbol{n}}$ satisfy the Schroedinger equation (3.15.2) with the boundary conditions (3.15.3) identically for any finite value of $\ell$, they must do so also when in the limit $\ell \rightarrow \infty$. However, as we

Figure 3.15.5. The discrete 3-dimensional quantum number lattice $\mathcal{Q}_{0}$
have seen, the convergence of the $w_{\ell \boldsymbol{n}}$ and $\phi_{\ell \boldsymbol{n}}$ in the limit is a delicate matter. The discussion of the previous paragraph shows the proper way the limit must be performed. One cannot simply take the limit of $w_{\ell n}$ and $\phi_{\ell n}$ at fixed quantum number triple $\boldsymbol{n}$, since proceeding in this naive manner it turns out that all the eigenvalues vanish and all the eigenfunctions vanish at any fixed space point in the limit. Rather, one must express $w_{\ell \boldsymbol{n}}$ and $(2 \lambda \ell)^{3 / 2} \phi_{\ell \boldsymbol{n}}$ in terms of the rescaled quantum number triple $\boldsymbol{\kappa}_{\ell}$ as $\tilde{w}_{\boldsymbol{\kappa}_{\ell}}$ and $\tilde{\phi}_{\boldsymbol{\kappa}_{\ell}}$ as in (3.15.89), (3.15.90) and then take the limit as $\ell \rightarrow \infty, \boldsymbol{n} \rightarrow \infty$ with fixed value $\boldsymbol{\kappa}_{\ell}=\boldsymbol{\kappa}$ with $\boldsymbol{\kappa}=\left(\kappa_{1}, \kappa_{2}, \kappa_{3}\right)$ a triple of real quantum numbers, which means that the unscaled quantum number triple $\boldsymbol{n}$ must diverge accordingly. The limit $\ell \rightarrow \infty$, once correctly performed, provides the unbounded space energy eigenvalues and eigenfunctions $w_{\boldsymbol{\kappa}}$ and $\phi_{\boldsymbol{\kappa}}$ labelled by the triples $\boldsymbol{\kappa}=\left(\kappa_{1}, \kappa_{2}, \kappa_{3}\right)$. In terms of these $w_{\boldsymbol{\kappa}}$ is given by
$$
\begin{equation*}
w_{\boldsymbol{\kappa}}=\tilde{w}_{\boldsymbol{\kappa}}=\frac{\hbar^{2} \boldsymbol{\kappa}^{2}}{2 m} \tag{3.15.91}
\end{equation*}
$$
while $\phi_{\kappa}$ reads as
$$
\begin{equation*}
\phi_{\boldsymbol{\kappa}}(\boldsymbol{x})=\tilde{\phi}_{\boldsymbol{\kappa}}(\boldsymbol{x})=8 \sin \left(\kappa_{1} x_{1}\right) \sin \left(\kappa_{2} x_{2}\right) \sin \left(\kappa_{2} x_{2}\right) \tag{3.15.92}
\end{equation*}
$$
with domain the octant $\mathcal{B}_{\infty} x_{1} \geq 0, x_{2} \geq 0, x_{3} \geq 0$. These are the energy eigenvalues and eigenfunctions of case iii examined earlier (cf. eqs. (3.15.64), $(3.15 .65))$.

![](https://cdn.mathpix.com/cropped/2024_09_22_5d1e855547710648961eg-0341.jpg?height=389&width=1118&top_left_y=497&top_left_x=471)

Figure 3.15.6. As $\ell$ grows, the lattice of the rescaled quantum numbers $\boldsymbol{\kappa}_{\ell}$ becomes denser and denser and eventually fills $\mathcal{Q}$

Consider again the case of the bounded box where $l_{1}, l_{2}, l_{3}$ are all finite. When $l_{1}, l_{2}, l_{3}$ have rational ratios, degeneration can occur: different triples of quantum numbers $\boldsymbol{n}$ may yield equal energy eigenvalue $w_{\boldsymbol{n}}$. The degeneration $d_{w}$ of an energy eigenvalue $w$ is the number of quantum numbers triples $\boldsymbol{n}$ such that $w_{n}=w$.

Degeneration is particularly simple to analyze when the energy eigenvalue $w$ is large, that is when $w \gg \hbar^{2} / 2 m l^{2}$, where $l$ is the smallest of $l_{1}, l_{2}, l_{3}$. By (3.15.17), the quantum numbers $\boldsymbol{n}$ involved are then necessarily large and, so, can be treated as continuous variables, since the relative variation they suffer when they vary by one unit is negligibly small. Consequently, the energy eigenvalue $w$ can also be treated as a continuous variable. Being so, instead of concentrating on degeneration $d_{w}$ of an eigenvalue $w$ defined in the previous paragraph, it is more natural to study the differential degeneration $d(w)$ at an energy value $w$ defined as follows: $d(w) \Delta w$ is the number of quantum numbers triples $\boldsymbol{n}$ such that $w_{n}$ lies between $w$ and $w+\Delta w$.

The differential degeneration $d(w)$ is particularly simple to compute in the special case where $l_{1}=l_{2}=l_{3}=l$,
$$
\begin{equation*}
d(w) \Delta w=\frac{(2 m)^{3 / 2} V}{4 \pi^{2} \hbar^{3}} w^{1 / 2} \Delta w \tag{3.15.93}
\end{equation*}
$$
where $V=l^{3}$ is the volume of the box

Proof. When $l_{1}=l_{2}=l_{3}=l$, the energy eigenvalue $w_{\boldsymbol{n}}$ can be written as
$$
\begin{equation*}
w_{\boldsymbol{n}}=\frac{\hbar^{2} \pi^{2}|\boldsymbol{n}|^{2}}{2 m l^{2}} \tag{3.15.94}
\end{equation*}
$$
$d(w) \Delta w$ is thus the number of triples $\boldsymbol{n}$ such that one has $(2 m w)^{1 / 2} l / \pi \hbar \leq|\boldsymbol{n}|<$ $(2 m(w+\Delta w))^{1 / 2} l / \pi \hbar$, which equals the volume of the spherical layer in the the octant $n_{1}, n_{2}, n_{3} \geq 0$ of internal radius $(2 m w)^{1 / 2} l / \pi \hbar$ and external radius $(2 m w)^{1 / 2} l / \pi \hbar \cdot(1+$ $\Delta w / 2 w)$. This volume is given by
$$
\begin{equation*}
d(w) \Delta w=\frac{1}{8} \cdot 4 \pi\left(\frac{(2 m w)^{1 / 2} l}{\pi \hbar}\right)^{2} \cdot \frac{(2 m w)^{1 / 2} l}{\pi \hbar} \frac{\Delta w}{2 w} \tag{3.15.95}
\end{equation*}
$$
which simplifies into (3.15.93).

The particle in a box model provides an approximate solution of the Schroedinger problem of a particle trapped in a narrow region of low electric potential between two high potential barriers. Situations of this kind occur in optoelectronics, laser technology, photodetection etc.

\subsection*{3.16. A basic example: the free particle}

The free particle is a simple quantum system for which the Schroedinger problem can be solved exactly. It is the topic of this section. The free particle is physically interesting for two reasons. First, it is an instance of a quantum system in an unbounded configuration space domain. Second, a particle in a potential force field $U$ characterized by an energy scale $U_{0}$ is approximately free at energy $w \gg U_{0}$, since in this energy regime the field constitutes only a small perturbation of the free dynamics. Hence, the energy eigenvalues and eigenfunctions of the free particle are a good approximation of those of the particle in the potential $U$ at high energies $w \gg U_{0}$.

By definition, a free particle is subject to no external force. The potential energy $U$ thus vanishes identically
$$
\begin{equation*}
U=0 \tag{3.16.1}
\end{equation*}
$$

The time independent Schroedinger equation (3.14.12) thus reads
$$
\begin{equation*}
-\frac{\hbar^{2}}{2 m} \nabla^{2} \phi=w \phi \tag{3.16.2}
\end{equation*}
$$

The domain $\mathcal{D}$ accessible to the particle is the whole configuration space $\mathbb{E}^{3}$. As $\mathbb{E}^{3}$ has no boundary, the boundary condition (3.14.13) is empty. As discussed in sect. $3.14, \phi$ is further required to be bounded,
$$
\begin{equation*}
|\phi(\boldsymbol{x})| \leq C, \quad \text { as }|\boldsymbol{x}| \rightarrow \infty \tag{3.16.3}
\end{equation*}
$$

The general solution of (3.16.2) is parametrized by an arbitrary momentum space vector $\boldsymbol{y}$, which is to be viewed as a triplet of continuous quantum number similarly to the case of the unbounded box of sect. 3.15,
$$
\begin{align*}
& w_{\boldsymbol{y}}=\frac{\boldsymbol{y}^{2}}{2 m}  \tag{3.16.4}\\
& \epsilon_{\boldsymbol{y}}(\boldsymbol{x})=\frac{\exp (i \boldsymbol{y} \cdot \boldsymbol{x} / \hbar)}{(2 \pi \hbar)^{3 / 2}} \tag{3.16.5}
\end{align*}
$$

The free particle energy spectrum is therefore continuous.

Proof. By (3.16.4), (3.16.5), we have
$$
\begin{align*}
-\frac{\hbar^{2}}{2 m} \boldsymbol{\nabla}_{\boldsymbol{x}}^{2} \epsilon_{\boldsymbol{y}}(\boldsymbol{x})=- & \frac{\hbar^{2}}{2 m} \boldsymbol{\nabla}_{\boldsymbol{x}}^{2} \frac{\exp (i \boldsymbol{y} \cdot \boldsymbol{x} / \hbar)}{(2 \pi \hbar)^{3 / 2}}  \tag{3.16.6}\\
& =-\frac{\hbar^{2}}{2 m}\left(-\frac{\boldsymbol{y}^{2}}{\hbar^{2}}\right) \frac{\exp (i \boldsymbol{y} \cdot \boldsymbol{x} / \hbar)}{(2 \pi \hbar)^{3 / 2}}=w_{\boldsymbol{y}} \epsilon_{\boldsymbol{y}}(\boldsymbol{x})
\end{align*}
$$
so that (3.16.2) is fulfilled. As $\left|\epsilon_{\boldsymbol{y}}(\boldsymbol{x})\right|=1 /(2 \pi \hbar)^{3 / 2}, \epsilon_{\boldsymbol{y}}$ is bounded as required.

A typical form of the plot of the real and imaginary part of the free particle energy eigenfunction $\epsilon_{\boldsymbol{y}}$ is shown in fig. 3.16.1.

The eigenfunctions $\epsilon_{\boldsymbol{y}}$ with varying momentum $\boldsymbol{y}$ constitute a generalized orthonormal set. Indeed,
$$
\begin{equation*}
\int d^{3} x \epsilon_{\boldsymbol{y}}^{*} \epsilon_{\boldsymbol{y}^{\prime}}=\delta\left(\boldsymbol{y}^{\prime}-\boldsymbol{y}\right) \tag{3.16.7}
\end{equation*}
$$
where $\delta(\boldsymbol{y})$ is the momentum space Dirac delta function: $\delta(\boldsymbol{y})=\infty$ for $\boldsymbol{y}=\mathbf{0}$, $\delta(\boldsymbol{y})=0$ for $\boldsymbol{y} \neq \mathbf{0}$, with
$$
\begin{equation*}
\int d^{3} y^{\prime} \delta\left(\boldsymbol{y}-\boldsymbol{y}^{\prime}\right) f\left(\boldsymbol{y}^{\prime}\right)=f(\boldsymbol{y}) \tag{3.16.8}
\end{equation*}
$$
for any momentum space function $f$.

Proof. By (3.16.5), using relation (3.5.64), we have
$$
\begin{equation*}
\int d^{3} x \epsilon_{\boldsymbol{y}}{ }^{*} \epsilon_{\boldsymbol{y}^{\prime}}=\int \frac{d^{3} x}{(2 \pi \hbar)^{3}} \exp \left(\frac{i}{\hbar}\left(\boldsymbol{y}^{\prime}-\boldsymbol{y}\right) \cdot \boldsymbol{x}\right)=\delta\left(\boldsymbol{y}^{\prime}-\boldsymbol{y}\right) \tag{3.16.9}
\end{equation*}
$$
showing (3.16.8).

According to quantum theory, the generalized orthonormal set of eigenfunction $\epsilon_{\boldsymbol{y}}$ is complete. Hence, for any wave function $\phi$
$$
\begin{equation*}
\phi=\int d^{3} y c_{\boldsymbol{y}} \epsilon_{\boldsymbol{y}} \tag{3.16.10}
\end{equation*}
$$

![](https://cdn.mathpix.com/cropped/2024_09_22_5d1e855547710648961eg-0345.jpg?height=1140&width=744&top_left_y=506&top_left_x=688)

Figure 3.16.1. Typical form of the plot of the real and imaginary part of the free particle energy eigenfunction $\epsilon_{y_{1} y_{2}}$. Here again, we consider space as two dimensional to allow for a pictorial representation. Plane wave fronts are clearly recognizable.
$$
\begin{equation*}
c_{\boldsymbol{y}}=\int d^{3} x \epsilon_{\boldsymbol{y}}{ }^{*} \phi \tag{3.16.11}
\end{equation*}
$$

In this way, the $\epsilon_{\boldsymbol{y}}$ constitute a generalized othonormal basis.

Proof. By (3.16.5), using again relation (3.5.64), we have
$$
\begin{equation*}
\int d^{3} y \epsilon_{\boldsymbol{y}}(\boldsymbol{x}) \epsilon_{\boldsymbol{y}}^{*}\left(\boldsymbol{x}^{\prime}\right)=\int \frac{d^{3} y}{(2 \pi \hbar)^{3}} \exp \left(\frac{i}{\hbar} \boldsymbol{y} \cdot\left(\boldsymbol{x}-\boldsymbol{x}^{\prime}\right)\right)=\delta\left(\boldsymbol{x}-\boldsymbol{x}^{\prime}\right) \tag{3.16.12}
\end{equation*}
$$
verifying (3.16.10). From (3.16.12), for any wave function $\phi$, we have
$$
\begin{align*}
& \int d^{3} y \epsilon_{\boldsymbol{y}}(\boldsymbol{x})\left[\int d^{3} x^{\prime} \epsilon_{\boldsymbol{y}}\left(\boldsymbol{x}^{\prime}\right)^{*} \phi\left(\boldsymbol{x}^{\prime}\right)\right]  \tag{3.16.13}\\
& =\int d^{3} x^{\prime}\left[\int d^{3} y \epsilon_{\boldsymbol{y}}(\boldsymbol{x}) \epsilon_{\boldsymbol{y}}\left(\boldsymbol{x}^{\prime}\right)^{*}\right] \phi\left(\boldsymbol{x}^{\prime}\right)=\int d^{3} x^{\prime} \delta\left(\boldsymbol{x}-\boldsymbol{x}^{\prime}\right) \phi\left(\boldsymbol{x}^{\prime}\right)=\phi(\boldsymbol{x})
\end{align*}
$$
which shows (3.16.11).

\subsection*{3.17. Energy eigenvalues and eigenfunctions and their properties}

In this section, we shall discuss the properties of the energy eigenvalues and eigenfunctions of a particle enclosed in an arbitrarily shaped region $\mathcal{D}$ of configuration space and acted upon there by an arbitrary potential $U$.

In sect. 3.15 , we studied a particle in a suitably sized potential box as a soluble approximate model of a particle confined in some configuration space region and subject there to a certain potential. In spite of the simplicity of such representation, the overall structure of the energy spectrum and the organization of the eigenfunctions observed in the potential box case are expected to be discernible also in the generic case the box approximates. In fact, whilst the explicit expressions of the energy eigenvalues and eigenfunctions do depend on the form of the confinement region and the potential, their main qualitative features very likely do not. We can so reach a sound educated guess about these latter.

We shall consider first the case where $\mathcal{D}$ is bounded, since this is mathematically simpler than the general case, to be treated later in this section, and is physically interesting in its own. Based in our experience with the potential box model, we expect that for such a $\mathcal{D}$ the solutions of the Schroedinger problem can be classified by multiplets of discrete quantum numbers $r$, i.e. ordered sequences of one or more discrete labels. More specifically, we postulate that
the solution of the Schroedinger problem is a collection of energy eigenvalues $w_{r}$ and eigenfunctions $\phi_{r}$ indexed by discrete quantum number multiplets $r$ such that $\phi_{r}$ belongs to $w_{r}$ and that $\phi_{r} \neq \phi_{s}$ for $r \neq s$.

Recall that one may have $w_{r}=w_{s}$ for $r \neq s$ in the presence of degeneracy.
In $\mathcal{D}$, therefore, the energy spectrum, the collection of all energy eigenvalues, consists of a discrete sequence $w_{n}, n=1,2, \ldots$, of distinct energy eigenvalues and for this reason is called discrete. In fact, the Schroedinger equation (3.14.12) has
a non zero solution $\phi$ satisfying the boundary condition (3.14.13) only for certain values $w$, since condition (3.14.13) is a very restrictive one. One can get a physical intuition of this. With an appropriate time harmonic factor multiplied, an eigenfunction $\phi$ describes harmonic oscillations that are quenched at the boundary. These oscillations are akin to the oscillatory normal modes of the membrane of a drum, whose edge is fixed on the drum's frame. Normal oscillations are known to occur only for discrete values of the wave length $\lambda$ and so, by the relevant dispersion relation, of the frequency $\omega$. In quantum theory, something like that happens. By (3.14.11), $w$ is related to $\omega$ by the familiar Plank-Einstein relation (1.12.1), $w=\hbar \omega$. Hence, just as $\omega, w$ can take only discrete values.

Based again in our experience with the potential box model, we expect the eigenfunctions $\phi_{r}$ are, or can be chosen to be, an orthonormal basis of space $\mathcal{H}$ of all wave functions. More specifically, we postulate that the $\phi_{r}$ satisfy the orthonormality relation
$$
\begin{equation*}
\int_{\mathcal{D}} d^{3} x \phi_{r}{ }^{*} \phi_{s}=\delta_{r, s} \tag{3.17.1}
\end{equation*}
$$

Furthermore, we assume that the $\phi_{r}$ constitute a complete set, that is that every wave function $\phi$ (not necessarily an eigenfunction) has an expansion of the form
$$
\begin{equation*}
\phi=\sum_{r} c_{r} \phi_{r} \tag{3.17.2}
\end{equation*}
$$
where the $c_{r}$ are complex coefficients depending on $r$. The $c_{r}$ can actually be computed. They are given by the expression.
$$
\begin{equation*}
c_{r}=\int_{\mathcal{D}} d^{3} x \phi_{r}^{*} \phi \tag{3.17.3}
\end{equation*}
$$

Proof. Indeed, we have
$$
\begin{equation*}
\int_{\mathcal{D}} d^{3} x \phi_{r}^{*} \phi=\sum_{s} c_{s} \int_{\mathcal{D}} d^{3} x \phi_{r}^{*} \phi_{s}=\sum_{s} c_{s} \delta_{r, s}=c_{r} \tag{3.17.4}
\end{equation*}
$$
which shows (3.17.3).

The wave function space $\mathcal{H}$ is infinite dimensional. There are thus infinitely many eigenfunctions $\phi_{r}$. Else, the $\phi_{r}$ could not possibly constitute a complete set. Therefore, the number of values of the quantum number multiplet $r$ is also infinite.

Some properties of the energy eigenvalues and eigenfunctions for a bounded domain $\mathcal{D}$ can actually be shown as we discuss next.

Since the domain $\mathcal{D}$ is bounded, any energy eigenfunction $\phi$ is normalizable, that is satisfying the condition
$$
\begin{equation*}
\int_{\mathcal{D}} d^{3} x|\phi|^{2}<\infty \tag{3.17.5}
\end{equation*}
$$

The linearity of the Schroedinger problem allow us to fix the normalization of $\phi$ according to a convenient convention. Henceforth, we assume that
$$
\begin{equation*}
\int_{\mathcal{D}} d^{3} x|\phi|^{2}=1 \tag{3.17.6}
\end{equation*}
$$

The time independent Schroedinger equation (3.14.12) implies that the energy eigenvalues $w$ are real and fulfill the bound
$$
\begin{equation*}
w>\min _{\mathcal{D}} U \tag{3.17.7}
\end{equation*}
$$

Proof. Let $w$ be an energy eigenvalue and $\phi$ be an energy eigenfunction belonging to $w$. By the Schroedinger equation (3.14.12) and the normalization condition (3.17.6),
$$
\begin{align*}
w & =w \int_{\mathcal{D}} d^{3} x|\phi|^{2}=\int_{\mathcal{D}} d^{3} x \phi^{*} w \phi=\int_{\mathcal{D}} d^{3} x \phi^{*}\left(-\frac{\hbar^{2}}{2 m} \nabla^{2} \phi+U \phi\right)  \tag{3.17.8}\\
& =\int_{\mathcal{D}} d^{3} x\left(\frac{\hbar^{2}}{2 m}|\boldsymbol{\nabla} \phi|^{2}+U|\phi|^{2}\right)-\frac{\hbar^{2}}{2 m} \int_{\mathcal{D}} d^{3} x \boldsymbol{\nabla} \cdot\left(\phi^{*} \boldsymbol{\nabla} \phi\right)
\end{align*}
$$

By the boundary condition (3.14.13), we have further
$$
\begin{equation*}
\int_{\mathcal{D}} d^{3} x \boldsymbol{\nabla} \cdot\left(\phi^{*} \boldsymbol{\nabla} \phi\right)=\oint_{\mathcal{D} \mathcal{D}} d^{2} \boldsymbol{x} \cdot\left(\phi^{*} \boldsymbol{\nabla} \phi\right)=0 \tag{3.17.9}
\end{equation*}
$$
where we have applied Gauss' theorem. From (3.17.8), (3.17.9), we have so
$$
\begin{equation*}
w=\int_{\mathcal{D}} d^{3} x\left(\frac{\hbar^{2}}{2 m}|\nabla \phi|^{2}+U|\phi|^{2}\right) \tag{3.17.10}
\end{equation*}
$$

This relation implies immediately that $w$ is real.
In eq. (3.17.10), $\boldsymbol{\nabla} \phi$ cannot vanish identically. If it did, $\phi$ would be constant in $\mathcal{D}$. But then the boundary condition (3.14.13) would imply that $\phi=0$ identically. This however would not compatible be with the normalization condition (3.17.6). We find in this way that
$$
\begin{equation*}
w>\int_{\mathcal{D}} d^{3} x U|\phi|^{2} \geq \min _{\mathcal{D}} U \int_{\mathcal{D}} d^{3} x|\phi|^{2}=\min _{\mathcal{D}} U \tag{3.17.11}
\end{equation*}
$$

This shows (3.17.7).

Taking the bound (3.17.7) into account, if the potential energy $U$ is plotted and is viewed as a potential well of sort, the energy eigenvalues form a sequence of energy levels extending from slightly above the potential minimum all the way up to infinity as shown in fig. 3.17.1.

If $w, w^{\prime}$ are two energy eigenvalues with $w^{\prime} \neq w$ and $\phi, \phi^{\prime}$ are energy eigenfunctions belonging to $w, w^{\prime}$, respectively, then $\phi, \phi^{\prime}$ are orthogonal, that is
$$
\begin{equation*}
\int_{\mathcal{D}} d^{3} x \phi^{\prime *} \phi=0 \tag{3.17.12}
\end{equation*}
$$

Proof. Using the Schroedinger equation (3.14.12) and the boundary condition (3.14.13) obeyed by $\phi, \phi^{\prime}$, we have
$$
\begin{align*}
& \left(w^{\prime}-w\right) \int_{\mathcal{D}} d^{3} x \phi^{\prime *} \phi=\int_{\mathcal{D}} d^{3} x\left(w^{\prime} \phi^{\prime}\right)^{*} \phi-\int_{\mathcal{D}} d^{3} x \phi^{\prime *} w \phi  \tag{3.17.13}\\
& \quad=\int_{\mathcal{D}} d^{3} x\left(-\frac{\hbar^{2}}{2 m} \nabla^{2} \phi^{\prime}+U \phi^{\prime}\right)^{*} \phi-\int_{\mathcal{D}} d^{3} x \phi^{\prime *}\left(-\frac{\hbar^{2}}{2 m} \nabla^{2} \phi+U \phi\right) \\
& \quad=\frac{\hbar^{2}}{2 m} \int_{\mathcal{D}} d^{3} x\left(\phi^{\prime *} \nabla^{2} \phi-\nabla^{2} \phi^{\prime *} \phi\right)=\frac{\hbar^{2}}{2 m} \int_{\mathcal{D}} d^{3} x \boldsymbol{\nabla} \cdot\left(\phi^{\prime *} \boldsymbol{\nabla} \phi-\boldsymbol{\nabla} \phi^{\prime *} \phi\right) \\
& \quad=\frac{\hbar^{2}}{2 m} \oint_{\mathcal{D}} d^{2} \boldsymbol{x} \cdot\left(\phi^{\prime *} \boldsymbol{\nabla} \phi-\nabla \phi^{\prime *} \phi\right)=0
\end{align*}
$$
where we used Gauss' theorem in the last step. Since $w^{\prime}-w \neq 0$ by assumption,

![](https://cdn.mathpix.com/cropped/2024_09_22_5d1e855547710648961eg-0351.jpg?height=614&width=785&top_left_y=533&top_left_x=670)

Figure 3.17.1. In a bounded domain $\mathcal{D}$, the energy eigenvalues, represented by horizontal planes, lie all above the minimum of the potential energy and grow to infinity. Here, we consider space as two dimensional to allow for a pictorial representation.
relation (3.17.12) holds.

By (3.17.12), the orthonormality relation (3.17.1) necessarily holds for any pair of multiplets $r, s$ such that $w_{r} \neq w_{s}$. It is not difficult to show that (3.17.1) holds in general.

Proof. Suppose $\phi_{r}$ is an arbitrary basis of energy eigenfunctions labelled by the quantum number multiplets $r$ as indicated above. Let $w_{n}$ be an energy eigenvalue. Then, the eigenfunctions $\phi_{r}$ such that $w_{r}=w_{n}$ form a basis of the eigenspace $\mathcal{H}_{n}$ of $w_{n}$. In general, these $\phi_{r}$ do not satisfy (3.17.1). However, as we are free to choose the basis of $\mathcal{H}_{n}$ as we like, we may substitute the $\phi_{r}$ with an equal number of linearly independent linear combinations
$$
\begin{equation*}
\phi^{\prime}{ }_{r}=\sum_{p} a_{r p} \phi_{p} \tag{3.17.14}
\end{equation*}
$$
where $a_{r p}$ is a non singular $d_{n} \times d_{n}$ complex matrix and here and below all sum extend
to the indices $p, q, \ldots$, such that $w_{p}=w_{q}=\ldots=w_{n}$. By (3.17.14), we have
$$
\begin{equation*}
\int_{\mathcal{D}} d^{3} x \phi^{\prime}{ }_{r}^{*} \phi^{\prime}{ }_{s}=\sum_{p, q} a_{r p}^{*} a_{s q} \int_{\mathcal{D}} d^{3} x \phi_{p}^{*} \phi_{q} \tag{3.17.15}
\end{equation*}
$$

Next, we note that
$$
\begin{equation*}
\left[\int_{\mathcal{D}} d^{3} x \phi_{p}{ }^{*} \phi_{q}\right]^{*}=\int_{\mathcal{D}} d^{3} x \phi_{q}^{*} \phi_{p} \tag{3.17.16}
\end{equation*}
$$
and that
$$
\begin{equation*}
\sum_{p, q} c_{p}^{*} c_{q} \int_{\mathcal{D}} d^{3} x \phi_{p}^{*} \phi_{q}=\int_{\mathcal{D}} d^{3} x\left|\sum_{p} c_{p} \phi_{p}\right|^{2} \geq 0 \tag{3.17.17}
\end{equation*}
$$
where the equality holds only if $c_{p}=0$ for all $p$. By (3.17.16), (3.17.17), $\int_{\mathcal{D}} d^{3} x \phi_{q}{ }^{*} \phi_{p}$ is a strictly positive definite Hermitian matrix. By a theorem of basic linear algebra, it is possible to choose the matrix $a_{s q}$ such that
$$
\begin{equation*}
\sum_{p, q} a_{r p}{ }^{*} a_{s q} \int_{\mathcal{D}} d^{3} x \phi_{p}^{*} \phi_{q}=\delta_{r, s} \tag{3.17.18}
\end{equation*}
$$

Combining (3.17.15), (3.17.18), we find
$$
\begin{equation*}
\int_{\mathcal{D}} d^{3} x \phi^{\prime}{ }_{r}^{*} \phi^{\prime}{ }_{s}=\delta_{r, s} \tag{3.17.19}
\end{equation*}
$$

Redefining notationally $\phi^{\prime}{ }_{r}$ into $\phi_{r}$, we have shown (3.17.1) for quantum number multiplets $r, s$ such that $w_{r}=w_{s}=w_{n}$ for some $n$. As $n$ is arbitrary, we have shown (3.17.1) for multiplets $r$, $s$ such that $w_{r}=w_{s}$.

Consider next quantum number multiplets $r, s$ such that $w_{r} \neq w_{s}$. As the eigenfunctions $\phi_{r}, \phi_{s}$ belong to distinct eigenvalues, (3.17.1) holds by virtue of property (3.17.12).

Combining the results of the two previous paragraphs, relation (3.17.1) follows for arbitrary quantum number multiplets $r, s$.

The completeness property (3.17.2) cannot be proven in general, but only on a case by case basis. In general, showing completeness is a very hard problem of functional analysis, which is better left to the care of mathematicians. In quantum theory, one contents oneself with postulating completeness the way we did.

We shall consider next the case where the domain $\mathcal{D}$ where the particle is confined is not necessarily bounded. All the properties found in the bounded domain case keep holding in an appropriate sense.

In sect. 3.15 , we solved the Schroedinger problem also for a particle in a potential box with one or more infinitely long edges. Based on the results found in that special case, we expect that in a possibly unbounded region $\mathcal{D}$ the solutions of the Schroedinger problem can be classified through composite sequences $i \rho$ of quantum numbers, where $i, \rho$ are subsequences of discrete and continuous numbers, respectively. The structure of the splitting may vary along the range of variation of the multiplet in a complicated way. In a part of the range, $\rho$ may be empty and the multiplet may so reduce to its component $i$. In another part, $i$ may be absent and the multiplet may hence coincide with its component $\rho$. In another part again, both the components $i$ and $\rho$ may be simultaneously present. In analogy to the bounded domain case, we postulate that
the solution of the Schroedinger problem is a collection of energy eigenvalues $w_{i \rho}$ and eigenfunctions $\phi_{i \rho}$ indexed by multiplets of mixed quantum numbers $i \rho$ such that $\phi_{i \rho}$ belongs to $w_{i \rho}$ and that $\phi_{i \rho} \neq \phi_{j \sigma}$ for $i \rho \neq j \sigma$.

Again, as before, one may have $w_{i \rho}=w_{j \sigma}$ for $i \rho \neq j \sigma$ in the presence of degeneracy.

The energy eigenvalues $w_{i \rho}$ will form generally a partly discrete and partly continuous spectrum, depending on the nature of the multiplet $i \rho$. Degeneration does occur in general. Finite discrete as well as infinite continuous degeneration is possible.

Based again on the results found in the unbounded potential box model, we expect that in a possibly possibly unbounded region $\mathcal{D}$ the eigenfunctions $\phi_{i \rho}$ are, or can be chosen to be, a generalized orthonormal basis of space $\mathcal{H}$ of all wave functions $\phi$. More specifically, we postulate that the $\phi_{i \rho}$ satisfy the generalized orthonormality relation
$$
\begin{equation*}
\int_{\mathcal{D}} d^{3} x \phi_{i \rho}{ }^{*} \phi_{j \sigma}=N_{j \sigma} \delta_{i, j} \delta(\rho-\sigma) \tag{3.17.20}
\end{equation*}
$$
where $N_{j \sigma}$ is a normalization constant, generalizing (3.17.1). Furthermore, we assume that the $\phi_{r}$ constitute a generalized complete set, that is that every wave function $\phi$ has an expansion of the form
$$
\begin{equation*}
\phi=\sum_{i} \int \frac{d \rho}{N_{i \rho}} c_{i \rho} \phi_{i \rho} \tag{3.17.21}
\end{equation*}
$$
where the coefficient $c_{i \rho}$ is given by
$$
\begin{equation*}
c_{i \rho}=\int_{\mathcal{D}} d^{3} x \phi_{i \rho}^{*} \phi \tag{3.17.22}
\end{equation*}
$$
extending $(3.17 .2),(3.17 .3)$

Proof. Indeed, we have
$$
\begin{align*}
\int_{\mathcal{D}} d^{3} x \phi_{i \rho}{ }^{*} \phi=\sum_{j} & \int \frac{d \sigma}{N_{j \sigma}} c_{j \sigma} \int_{\mathcal{D}} d^{3} x \phi_{i \rho}{ }^{*} \phi_{j \sigma}  \tag{3.17.23}\\
& =\sum_{j} \int \frac{d \sigma}{N_{j \sigma}} c_{j \sigma} N_{j \sigma} \delta_{i, j} \delta(\rho-\sigma)=c_{i \rho}
\end{align*}
$$
which shows (3.17.22).

We distinguish three important particular cases. They do not exhaust all possible cases, but they are typical enough to deserve mention. In the first case, which we shall call $D$, the quantum number range is completely discrete and the multiplet $i \rho$ reduces to discrete submultiplet $i$, the continuous one $\rho$ being empty. The orthonormality relation $(3.17 .20)$ then reduces to
$$
\begin{equation*}
\int_{\mathcal{D}} d^{3} x \phi_{i}{ }^{*} \phi_{j}=\delta_{i, j} \tag{3.17.24}
\end{equation*}
$$

The completeness relation (3.17.21) becomes
$$
\begin{equation*}
\phi=\sum_{i} c_{i} \phi_{i} \tag{3.17.25}
\end{equation*}
$$
where, by (3.17.22), the coefficient $c_{i}$ is given by
$$
\begin{equation*}
c_{i}=\int_{\mathcal{D}} d^{3} x \phi_{i}^{*} \phi \tag{3.17.26}
\end{equation*}
$$

In the second case, which we shall call $C$, the quantum number range is completely continuous and the multiplet $i \rho$ reduces to the continuous submultiplet $\rho$, the discrete one $i$ being empty. The orthonormality relation (3.17.20) then gets
$$
\begin{equation*}
\int_{\mathcal{D}} d^{3} x \phi_{\rho}{ }^{*} \phi_{\sigma}=N_{\sigma} \delta(\rho-\sigma) \tag{3.17.27}
\end{equation*}
$$

The completeness relation $(3.17 .21)$ becomes
$$
\begin{equation*}
\phi=\int \frac{d \rho}{N_{\rho}} c_{\rho} \phi_{\rho} \tag{3.17.28}
\end{equation*}
$$
where, by (3.17.22), the coefficient $c_{\rho}$ is given by
$$
\begin{equation*}
c_{\rho}=\int_{\mathcal{D}} d^{3} x \phi_{\rho}{ }^{*} \phi \tag{3.17.29}
\end{equation*}
$$

In the third case, called $D C$, the quantum number range is formed by two parts, a completely discrete one, where the multiplet $i \rho$ reduces to the discrete submultiplet $i$ while the continuous one $\rho$ is empty, and a completely continuous one, where the multiplet $i \rho$ reduces to the continuous submultiplet $\rho$ while the discrete one $i$ is empty. The orthonormality relations (3.17.20) then take the more complicated form
$$
\begin{align*}
& \int_{\mathcal{D}} d^{3} x \phi_{i}^{*} \phi_{j}=\delta_{i, j}, \quad \int_{\mathcal{D}} d^{3} x \phi_{\rho}{ }^{*} \phi_{\sigma}=N_{\sigma} \delta(\rho-\sigma)  \tag{3.17.30}\\
& \int_{\mathcal{D}} d^{3} x \phi_{i}^{*} \phi_{\sigma}=\int_{\mathcal{D}} d^{3} x \phi_{\rho}{ }^{*} \phi_{j}=0
\end{align*}
$$

The completeness relation (3.17.21) becomes
$$
\begin{equation*}
\phi=\sum_{i} c_{i} \phi_{i}+\int \frac{d \rho}{N_{\rho}} c_{\rho} \phi_{\rho} \tag{3.17.31}
\end{equation*}
$$
where, by (3.17.22), the coefficients $c_{i}, c_{\rho}$ are given by
$$
\begin{equation*}
c_{i}=\int_{\mathcal{D}} d^{3} x \phi_{i}^{*} \phi, \quad c_{\rho}=\int_{\mathcal{D}} d^{3} x \phi_{\rho}^{*} \phi \tag{3.17.32}
\end{equation*}
$$

Some properties of the energy eigenvalues and eigenfunctions for a possibly
unbounded domain $\mathcal{D}$ can actually be proven on the same lines as in the bounded case. However, on an unbounded $\mathcal{D}$, the normalizability property (3.17.5) generally fails to hold and this renders all the analytical manipulations rather formal.

The bound (3.17.7) for the energy eigenvalues still holds. The proof we have given in the bounded domain case however uses relation (3.17.5) which in the unbounded one generally fails. To show (3.17.7), we have to consider particle confined in a bounded domain $\mathcal{D}_{\ell}$ of linear size $\ell$ subject to a possibly $\ell$-dependent potential $U_{\ell}$ such that in the limit $\ell \rightarrow \infty \mathcal{D}_{\ell}$ dilates to $\mathcal{D}$ and $U_{\ell}$ converges to $U$. Reasoning as in sect. 3.15 , we expect that the energy spectrum in $\mathcal{D}_{\ell}$ reduces in the appropriate sense explained there to the energy spectrum in $\mathcal{D}$. Since $w_{\ell}>\min _{\mathcal{D}_{\ell}} U_{\ell}$ for any energy eigenvalue $w_{\ell}$ in $\mathcal{D}_{\ell}$ at finite $\ell$ by (3.17.7), it is likely that $w>\min _{\mathcal{D}} U$ for any energy eigenvalue in $\mathcal{D}$.

\subsection*{3.18. Time evolution of the wave function}

As anticipated in sect. 3.14, the main use of the energy eigenvalues $w_{i \rho}$ and eigenfunctions $\phi_{i \rho}$ furnished by solution of the Schroedinger problem is the following. The solution $\psi$ of the Schroedinger equation (3.4.14) with initial condition (3.4.24), where $\psi_{0}$ is any wave function, can be expanded as
$$
\begin{equation*}
\psi_{t}=\sum_{i} \int \frac{d \rho}{N_{i \rho}} a_{i \rho} \exp \left(-\frac{i w_{i \rho} t}{\hbar}\right) \phi_{i \rho} \tag{3.18.1}
\end{equation*}
$$
where the coefficients $a_{i \rho}$ are given by the expressions
$$
\begin{equation*}
a_{i \rho}=\int_{\mathcal{D}} d^{3} x \phi_{i \rho}{ }^{*} \psi_{0} \tag{3.18.2}
\end{equation*}
$$

Proof. We show the statement by direct verification of the Schroedinger equation (3.4.14) and initial condition (3.4.24). Suppose that $\psi$ is given by eq. (3.18.1). Then, by (3.14.12), we have
$$
\begin{align*}
i \hbar \frac{\partial \psi_{t}}{\partial t}= & i \hbar \frac{\partial}{\partial t}\left[\sum_{i} \int \frac{d \rho}{N_{i \rho}} a_{i \rho} \exp \left(-\frac{i w_{i \rho} t}{\hbar}\right) \phi_{i \rho}\right]  \tag{3.18.3}\\
= & \sum_{i} \int \frac{d \rho}{N_{i \rho}} a_{i \rho} \exp \left(-\frac{i w_{i \rho} t}{\hbar}\right) w_{i \rho} \phi_{i \rho} \\
= & \sum_{i} \int \frac{d \rho}{N_{i \rho}} a_{i \rho} \exp \left(-\frac{i w_{i \rho} t}{\hbar}\right)\left(-\frac{\hbar^{2}}{2 m} \nabla^{2} \phi_{i \rho}+U \phi_{i \rho}\right) \\
= & -\frac{\hbar^{2}}{2 m} \nabla^{2}\left[\sum_{i} \int \frac{d \rho}{N_{i \rho}} a_{i \rho} \exp \left(-\frac{i w_{i \rho} t}{\hbar}\right) \phi_{i \rho}\right] \\
& \quad+U\left[\sum_{i} \int \frac{d \rho}{N_{i \rho}} a_{i \rho} \exp \left(-\frac{i w_{i \rho} t}{\hbar}\right) \phi_{i \rho}\right]=-\frac{\hbar^{2}}{2 m} \nabla^{2} \psi_{t}+U \psi_{t}
\end{align*}
$$

Further, taking (3.18.2) into account,
$$
\begin{equation*}
\left.\psi_{t}\right|_{t=0}=\sum_{i} \int \frac{d \rho}{N_{i \rho}} a_{i \rho} \phi_{i \rho}=\psi_{0} \tag{3.18.4}
\end{equation*}
$$
by $(3.17 .21),(3.17 .22)$. The statement is so shown.

The possibility of expanding $\psi$ with respect to a generalized orthonormal basis of energy eigenfunctions as in (3.18.1) is the main motivation for the solution of the Schroedinger problem.

In the three special cases considered in sect. 3.17, relations (3.18.1), (3.18.2) simplify as follows. In case $D$, we have
$$
\begin{equation*}
\psi_{t}=\sum_{i} a_{i} \exp \left(-\frac{i w_{i} t}{\hbar}\right) \phi_{i} \tag{3.18.5}
\end{equation*}
$$
with the coefficients $a_{i}$ given by
$$
\begin{equation*}
a_{i}=\int_{\mathcal{D}} d^{3} x \phi_{i}^{*} \psi_{0} \tag{3.18.6}
\end{equation*}
$$

In case $C$, we have instead
$$
\begin{equation*}
\psi_{t}=\int \frac{d \rho}{N_{\rho}} a_{\rho} \exp \left(-\frac{i w_{\rho} t}{\hbar}\right) \phi_{\rho} \tag{3.18.7}
\end{equation*}
$$
with the coefficients $a_{\rho}$ given by
$$
\begin{equation*}
a_{\rho}=\int_{\mathcal{D}} d^{3} x \phi_{\rho}{ }^{*} \psi_{0} \tag{3.18.8}
\end{equation*}
$$

Finally, in case $D C$, we have
$$
\begin{equation*}
\psi_{t}=\sum_{i} a_{i} \exp \left(-\frac{i w_{i} t}{\hbar}\right) \phi_{i}+\int \frac{d \rho}{N_{\rho}} a_{\rho} \exp \left(-\frac{i w_{\rho} t}{\hbar}\right) \phi_{\rho} \tag{3.18.9}
\end{equation*}
$$
with the coefficients $a_{i}, a_{\rho}$ given by
$$
\begin{equation*}
a_{i}=\int_{\mathcal{D}} d^{3} x \phi_{i}^{*} \psi_{0}, \quad a_{\rho}=\int_{\mathcal{D}} d^{3} x \phi_{\rho}{ }^{*} \psi_{0} \tag{3.18.10}
\end{equation*}
$$

In cases $D$ and $D C$, we can take $\psi_{0}=\phi_{i_{0}}$ for some value $i_{0}$ of $i$ as $\phi_{i_{0}}$ satisfies the normalization condition (3.5.32). By (3.18.1), (3.18.2), we have then
$$
\begin{equation*}
\psi_{t}=\exp \left(-\frac{i w_{i_{0}} t}{\hbar}\right) \phi_{i_{0}} \tag{3.18.11}
\end{equation*}
$$

Proof. We show the statement for case $D C$ only. For case $D$, the argument is basically the same. By (3.18.10), when $\psi_{0}=\phi_{i_{0}}$, we have
$$
\begin{equation*}
a_{i}=\int_{\mathcal{D}} d^{3} x \phi_{i}{ }^{*} \phi_{i_{0}}=\delta_{i, i_{0}}, \quad a_{\rho}=\int_{\mathcal{D}} d^{3} x \phi_{\rho}{ }^{*} \phi_{i_{0}}=0 \tag{3.18.12}
\end{equation*}
$$
in virtue of (3.17.30). By (3.18.9), then, we find
$$
\begin{equation*}
\psi_{t}=\sum_{i} \delta_{i, i_{0}} \exp \left(-\frac{i w_{i} t}{\hbar}\right) \phi_{i}+0=\exp \left(-\frac{i w_{i_{0}} t}{\hbar}\right) \phi_{i_{0}} \tag{3.18.13}
\end{equation*}
$$
which shows (3.18.11).

As we have seen (cf. sect. 3.11), a wave function $\psi$ with a time evolution of the form (3.18.11) describes a stationary state of of the ensemble of particle or, alternatively, of the particle itself, depending on the interpretation of the theory.

The energy eigenfunctions $\phi_{\rho}$ of cases $C$ and $D C$ are generalized orthonormal and thus not normalizable according to (3.13.23). Thus, strictly speaking, they cannot encode stationary states. However, if $\psi_{0}$ is a wave function normalized as in (3.13.23) whose coefficient function $a_{\rho}$ happens to be peaked around a certain value $\rho_{0}$ of the quantum number multiplet $\rho$, so that $a_{\rho}$ is appreciably different from zero only for $\left|\rho-\rho_{0}\right|<\eta$ for some appropriately small scale $\eta$, then we can approximate (3.18.7) as
$$
\begin{equation*}
\psi_{t} \simeq \exp \left(-\frac{i w_{\rho_{0}} t}{\hbar}\right) \int_{\left|\rho-\rho_{0}\right|<\eta} \frac{d \rho}{N_{\rho}} a_{\rho} \phi_{\rho}=\exp \left(-\frac{i w_{\rho_{0}} t}{\hbar}\right) \psi_{0} \tag{3.18.14}
\end{equation*}
$$
provided $t$ is not too large. Therefore, for a while the wave function $\psi$ evolves as that of a stationary state of energy $w_{\rho_{0}}$. We have already discussed this point in greater detail in sect. 3.12 , but here we can illustrate it in an alternative simpler way. To obtain relation (3.18.14), we replaced $\exp \left(-i w_{\rho} t / \hbar\right)$ with $\exp \left(-i w_{\rho_{0}} t / \hbar\right)$ in the $\rho$ integral. This approximation is justified as long as $\exp \left(-i\left(w_{\rho}-w_{\rho_{0}}\right) t / \hbar\right) \simeq 1$. Denote by $\Delta w$ the order of magnitude of the maximum energy difference $\left|w_{\rho}-w_{\rho_{0}}\right|$ for $\left|\rho-\rho_{0}\right|<\eta$. Then, the maximum time $\Delta t$ for which $(3.18 .14)$ provides an accurate expression of $\psi_{t}$ must satisfy
$$
\begin{equation*}
\Delta t \Delta w \simeq \hbar \tag{3.18.15}
\end{equation*}
$$

We recover in this way the time-energy uncertainty relation (cf. eq. (3.12.29)). The smaller $\Delta w$ is, the longer the time $\Delta t$ is during which the system appears to
be stationary and viceversa. $\Delta w$ can be made as small as on wishes but cannot be made vanish. $\Delta w$ could vanish only if $a_{\rho}$ were proportional to $\delta\left(\rho-\rho_{0}\right)$ making $\psi$ proportional to $\phi_{\rho_{0}}$. This however is not admissible because $\phi_{\rho_{0}}$ is not normalizable.

A wave function of the form (3.18.14) describes a quasi-stationary state of the ensemble of particle or, alternatively, of the particle itself, depending on the interpretation of the theory. One such state can appear stationary for relatively long times, but eventually it evolves, a process called decay.

\subsection*{3.19. Time evolution of simple quantum systems}

Using the general method described in sect. 3.18, we can write down an expression of the solution $\psi$ of the Schroedinger equation (3.4.14) with initial condition (3.4.24) with $\psi_{0}$ any wave function. In simple models, this expression can be evaluated explicitly providing additional insight into quantum theory.

We consider first the particle in a potential box model studied in detail in sect. 3.15. The energy eigenvalues $w_{n}$ and eigenfunctions $\phi_{\boldsymbol{n}}$ of the particle are labelled by a triple of positive integer quantum numbers $\boldsymbol{n}=\left(n_{1}, n_{2}, n_{3}\right)$ and are given explicitly by (3.15.17) and (3.15.18), respectively. Being $\boldsymbol{n}$ discrete, the model belongs to the case $D$ studied in sect. 3.17. By (3.18.5), (3.18.6), the time evolution of the wave function of the particle is thus given by
$$
\begin{equation*}
\psi(t, \boldsymbol{x})=\sum_{\boldsymbol{n}} a_{\boldsymbol{n}} \exp \left(-\frac{i w_{\boldsymbol{n}} t}{\hbar}\right) \phi_{\boldsymbol{n}}(\boldsymbol{x}) \tag{3.19.1}
\end{equation*}
$$
where the coefficients $a_{\boldsymbol{n}}$ is given by
$$
\begin{equation*}
a_{\boldsymbol{n}}=\int_{\mathcal{B}} d^{3} x \phi_{\boldsymbol{n}}{ }^{*} \psi_{0} \tag{3.19.2}
\end{equation*}
$$

If $\psi_{0}=\phi_{\boldsymbol{n}_{0}}$ for a certain triple of quantum numbers $\boldsymbol{n}_{0}$, then
$$
\begin{equation*}
\psi(t, \boldsymbol{x})=\exp \left(-\frac{i w_{\boldsymbol{n}_{0}} t}{\hbar}\right) \phi_{\boldsymbol{n}_{0}}(\boldsymbol{x}) \tag{3.19.3}
\end{equation*}
$$
by (3.18.11). As we have remarked in sect. 3.18, a $\psi$ of this form represents a stationary state of the ensemble of the particle or probabilistic state of the particle depending on interpretation.

In sect. 3.15 , the particle in a potential box model was studied also in the case where one or more of the box edges are infinitely long. We distinguished three cases: case $i$, where $l_{1}=\infty, l_{2}, l_{3}<\infty$; case $i i$, where $l_{1}, l_{2}=\infty, l_{3}<\infty$; case $i i i$, where $l_{1}, l_{2}, l_{3}=\infty$.

In case $i$, the energy eigenvalues $w_{\kappa_{1}, n_{2}, n_{3}}$ and eigenfunctions $\phi_{\kappa_{1}, n_{2}, n_{3}}$ of the particle are labelled by a positive wave number $\kappa_{1}$ and two positive integer quan-
tum numbers $n_{2}, n_{3}$ and are given explicitly by (3.15.54) and (3.15.55), respectively. Using the general relations (3.18.1), (3.18.2), we find that the time evolution of the wave function of the particle is given by
$$
\begin{equation*}
\psi(t, \boldsymbol{x})=\int_{0}^{\infty} \frac{d \kappa_{1}}{2 \pi} \sum_{n_{2}, n_{3}=1}^{\infty} a_{\kappa_{1}, n_{2}, n_{3}} \exp \left(-\frac{i w_{\kappa_{1}, n_{2}, n_{3}} t}{\hbar}\right) \phi_{\kappa_{1}, n_{2}, n_{3}}(\boldsymbol{x}) \tag{3.19.4}
\end{equation*}
$$
where the coefficients $a_{\kappa_{1}, n_{2}, n_{3}}$ is given by
$$
\begin{equation*}
a_{\kappa_{1}, n_{2}, n_{3}}=\int_{\mathcal{B}} d^{3} x \phi_{\kappa_{1}, n_{2}, n_{3}}^{*} \psi_{0} \tag{3.19.5}
\end{equation*}
$$

In case $i i$, the energy eigenvalues $w_{\kappa_{1}, \kappa_{2}, n_{3}}$ and eigenfunctions $\phi_{\kappa_{1}, \kappa_{2}, n_{3}}$ of the particle are labelled by two positive wave numbers $\kappa_{1}, \kappa_{2}$ and a positive integer quantum numbers $n_{3}$ and are given explicitly by (3.15.59) and (3.15.60), respectively. Using again the general relations (3.18.1), (3.18.2), we find that the time evolution of the wave function of the particle is given by
$$
\begin{align*}
& \psi(t, \boldsymbol{x})=\int_{0}^{\infty} \frac{d \kappa_{1}}{2 \pi} \int_{0}^{\infty} \frac{d \kappa_{2}}{2 \pi} \sum_{n_{3}=1}^{\infty} a_{\kappa_{1}, \kappa_{2}, n_{3}}  \tag{3.19.6}\\
& \quad \exp \left(-\frac{i w_{\kappa_{1}, \kappa_{2}, n_{3}} t}{\hbar}\right) \phi_{\kappa_{1}, \kappa_{2}, n_{3}}(\boldsymbol{x})
\end{align*}
$$
where the coefficients $a_{\kappa_{1}, \kappa_{2}, n_{3}}$ is given by
$$
\begin{equation*}
a_{\kappa_{1}, \kappa_{2}, n_{3}}=\int_{\mathcal{B}} d^{3} x \phi_{\kappa_{1}, \kappa_{2}, n_{3}}^{*} \psi_{0} \tag{3.19.7}
\end{equation*}
$$

In case $i i i$, the energy eigenvalues $w_{\boldsymbol{\kappa}}$ and eigenfunctions $\phi_{\boldsymbol{\kappa}}$ of the particle are labelled by a triple of positive wave numbers $\boldsymbol{\kappa}=\left(\kappa_{1}, \kappa_{2}, \kappa_{3}\right)$ and are given explicitly by (3.15.64) and (3.15.65), respectively. Being $\boldsymbol{\kappa}$ continuous, the model belongs to the case $C$ studied in sect. 3.17. By (3.18.7), (3.18.8), the time evolution of the wave function of the particle is thus given by
$$
\begin{equation*}
\psi(t, \boldsymbol{x})=\int_{\mathcal{K}} \frac{d^{3} \kappa}{(2 \pi)^{3}} a_{\kappa} \exp \left(-\frac{i w_{\boldsymbol{\kappa}} t}{\hbar}\right) \phi_{\boldsymbol{\kappa}}(\boldsymbol{x}) \tag{3.19.8}
\end{equation*}
$$
where the coefficients $a_{\kappa}$ is given by
$$
\begin{equation*}
a_{\kappa}=\int_{\mathcal{B}} d^{3} x \phi_{\kappa}{ }^{*} \psi_{0} \tag{3.19.9}
\end{equation*}
$$

We consider next the free particle model studied in detail in sect. 3.16. The
energy eigenvalues $w_{\boldsymbol{y}}$ and eigenfunctions $\epsilon_{\boldsymbol{y}}$ of the particle are labelled by a momentum space vector $\boldsymbol{y}$ and are given explicitly by (3.16.4) and (3.16.5), respectively. Being $\boldsymbol{y}$ continuous, the model belongs to the case $C$ studied in sect. 3.17. Using eqs. $(3.18 .7),(3.18 .8)$, we find then
$$
\begin{equation*}
\psi(t, \boldsymbol{x})=\int d^{3} y a_{\boldsymbol{y}} \exp \left(-\frac{i w_{\boldsymbol{y}} t}{\hbar}\right) \epsilon_{\boldsymbol{y}}(\boldsymbol{x}) \tag{3.19.10}
\end{equation*}
$$
where the coefficients $a_{\boldsymbol{y}}$ is given by the expression
$$
\begin{equation*}
a_{\boldsymbol{y}}=\int d^{3} x \epsilon_{\boldsymbol{y}}{ }^{*} \psi_{0} \tag{3.19.11}
\end{equation*}
$$

For a free particle, the energy eigenfunctions are generalized orthonormal and, so, as shown in sect. 3.18, they do not encode stationary states. However, when the coefficient function $a_{\boldsymbol{y}}$ is peaked around a certain value $\boldsymbol{y}_{0}$ of the momentum vector $\boldsymbol{y}$, so that $a_{\boldsymbol{y}}$ is appreciably different from zero only for $\left|\boldsymbol{y}-\boldsymbol{y}_{0}\right|<\eta$ for an appropriately small momentum scale $\eta$, then $\psi$ is approximated by
$$
\begin{align*}
\psi(t, \boldsymbol{x}) \simeq \exp \left(-\frac{i w_{\boldsymbol{y}_{0}} t}{\hbar}\right) \int_{\left|\boldsymbol{y}-\boldsymbol{y}_{0}\right|<\eta} & d^{3} y a_{\boldsymbol{y}} \epsilon_{\boldsymbol{y}}(\boldsymbol{x})  \tag{3.19.12}\\
& =\exp \left(-\frac{i w_{\boldsymbol{y}_{0}} t}{\hbar}\right) \psi_{0}(\boldsymbol{x})
\end{align*}
$$
by (3.18.14), if $t$ is not too large.
Consider the case where the initial wave function $\psi_{0}$ of the free particle is the Gaussian wave packet
$$
\begin{equation*}
\psi_{0}(\boldsymbol{x})=\frac{1}{(2 \pi)^{3 / 4} \xi^{3 / 2}} \exp \left(-\frac{\left(\boldsymbol{x}-\boldsymbol{q}_{0}\right)^{2}}{4 \xi^{2}}+\frac{i \boldsymbol{p}_{0} \cdot \boldsymbol{x}}{\hbar}\right) \tag{3.19.13}
\end{equation*}
$$
where $\xi$ is a length scale and $\boldsymbol{q}_{0}$ and $\boldsymbol{p}_{0}$ are a constant position and momentum vector, respectively. The real and imaginary part of $\psi_{0}$ are plotted in fig. 3.19.1. It is readily checked that $\psi_{0}$ is normalized according to (3.5.32) as required.

Proof. By (3.19.13), one has
$$
\begin{equation*}
\left|\psi_{0}(\boldsymbol{x})\right|^{2}=\frac{1}{(2 \pi)^{3 / 2} \xi^{3}} \exp \left(-\frac{\left(\boldsymbol{x}-\boldsymbol{q}_{0}\right)^{2}}{2 \xi^{2}}\right) \tag{3.19.14}
\end{equation*}
$$

Consequently,
$$
\begin{align*}
& \int d^{3} x\left|\psi_{0}(\boldsymbol{x})\right|^{2}=\int d^{3} x \frac{1}{(2 \pi)^{3 / 2} \xi^{3}} \exp \left(-\frac{\left(\boldsymbol{x}-\boldsymbol{q}_{0}\right)^{2}}{2 \xi^{2}}\right)  \tag{3.19.15}\\
&=\frac{1}{(2 \pi)^{3 / 2} \xi^{3}} \int d^{3} x^{\prime} \exp \left(-\frac{\boldsymbol{x}^{\prime 2}}{2 \xi^{2}}\right)=1
\end{align*}
$$
where $\boldsymbol{x}^{\prime}=\boldsymbol{x}-\boldsymbol{q}_{0}$, by the standard Gaussian integration formula. $\psi_{0}$ is thus normalized as claimed.

We want to study the time evolution of the above wave packet using the general results of sect. 3.18.
![](https://cdn.mathpix.com/cropped/2024_09_22_5d1e855547710648961eg-0364.jpg?height=1046&width=1031&top_left_y=1215&top_left_x=493)

Figure 3.19.1. The real and imaginary part of the wave function of a Gaussian wave packet

At a later time $t$, the wave function $\psi$ is given by
$$
\begin{align*}
\psi(t, \boldsymbol{x})= & \frac{1}{(2 \pi)^{3 / 4}\left(\xi\left(1+i t / t_{0}\right)\right)^{3 / 2}}  \tag{3.19.16}\\
& \quad \times \exp \left(-\frac{\left(\boldsymbol{x}-\boldsymbol{q}_{0}-t \boldsymbol{p}_{0} / m\right)^{2}}{4 \xi^{2}\left(1+i t / t_{0}\right)}-\frac{i t \boldsymbol{p}_{0}{ }^{2}}{2 m \hbar}+\frac{i \boldsymbol{p}_{0} \cdot \boldsymbol{x}}{\hbar}\right)
\end{align*}
$$
where $t_{0}$ is the time scale defined by
$$
\begin{equation*}
t_{0}=\frac{2 m \xi^{2}}{\hbar} \tag{3.19.17}
\end{equation*}
$$

Proof. To compute $a_{\boldsymbol{y}}$, we insert the expression (3.19.13) of $\psi_{0}$ and that (3.16.5) of $\epsilon_{\boldsymbol{y}}$ into (3.19.11) and express the result in the form $J(\boldsymbol{z}, \zeta)$ for appropriate values of $\boldsymbol{z}$, $\sigma$, where $J(\boldsymbol{z}, \zeta)$ is defined in (3.5.60) and given by (3.5.61). We find so
$$
\begin{align*}
a_{\boldsymbol{y}} & =\int \frac{d^{3} x}{(2 \pi \hbar)^{3 / 2}} \exp \left(-\frac{i \boldsymbol{y} \cdot \boldsymbol{x}}{\hbar}\right) \frac{1}{(2 \pi)^{3 / 4} \xi^{3 / 2}} \exp \left(-\frac{\left(\boldsymbol{x}-\boldsymbol{q}_{0}\right)^{2}}{4 \xi^{2}}+\frac{i \boldsymbol{p}_{0} \cdot \boldsymbol{x}}{\hbar}\right)  \tag{3.19.18}\\
& =\frac{(2 \pi)^{3 / 4}}{(\hbar \xi)^{3 / 2}} \int \frac{d^{3} x}{(2 \pi)^{3}} \exp \left(-\frac{\left(\boldsymbol{x}-\boldsymbol{q}_{0}\right)^{2}}{4 \xi^{2}}-\frac{i\left(\boldsymbol{y}-\boldsymbol{p}_{0}\right) \cdot \boldsymbol{x}}{\hbar}\right) \\
& =\frac{(2 \pi)^{3 / 4}}{(\hbar \xi)^{3 / 2}} \int \frac{d^{3} x^{\prime}}{(2 \pi)^{3}} \exp \left(-\frac{\boldsymbol{x}^{\prime 2}}{4 \xi^{2}}-\frac{i\left(\boldsymbol{y}-\boldsymbol{p}_{0}\right) \cdot\left(\boldsymbol{x}^{\prime}+\boldsymbol{q}_{0}\right)}{\hbar}\right) \\
& =\frac{(2 \pi)^{3 / 4}}{(\hbar \xi)^{3 / 2}} \exp \left(-\frac{i\left(\boldsymbol{y}-\boldsymbol{p}_{0}\right) \cdot \boldsymbol{q}_{0}}{\hbar}\right) J\left(-\frac{i\left(\boldsymbol{y}-\boldsymbol{p}_{0}\right)}{\hbar}, \frac{1}{2^{1 / 2} \xi}\right) \\
& =\frac{(2 \pi)^{3 / 4}}{(\hbar \xi)^{3 / 2}} \exp \left(-\frac{i\left(\boldsymbol{y}-\boldsymbol{p}_{0}\right) \cdot \boldsymbol{q}_{0}}{\hbar}\right) \frac{\xi^{3}}{\pi^{3 / 2}} \exp \left(-\frac{\xi^{2}\left(\boldsymbol{y}-\boldsymbol{p}_{0}\right)^{2}}{\hbar^{2}}\right) \\
& =\left(\frac{2}{\pi}\right)^{3 / 4}\left(\frac{\xi}{\hbar}\right)^{3 / 2} \exp \left(-\frac{\xi^{2}\left(\boldsymbol{y}-\boldsymbol{p}_{0}\right)^{2}}{\hbar^{2}}-\frac{i\left(\boldsymbol{y}-\boldsymbol{p}_{0}\right) \cdot \boldsymbol{q}_{0}}{\hbar}\right)
\end{align*}
$$
where we have set $\boldsymbol{x}^{\prime}=\boldsymbol{x}-\boldsymbol{q}_{0}$. Next, to compute $\psi$, we insert the expression (3.19.18) of $a_{\boldsymbol{y}}$ and those (3.16.4) and (3.16.5) of $w_{\boldsymbol{y}}$ and $\epsilon_{\boldsymbol{y}}$ into (3.19.10) and express again the result in the form $J(\boldsymbol{z}, \zeta)$ for appropriate values of $\boldsymbol{z}, \sigma$. (Here, the roles of $\boldsymbol{x}$ and $\boldsymbol{y}$ are exchanged, but mathematically there is no difference.) We find then
$$
\begin{align*}
\psi(t, \boldsymbol{x})=\int \frac{d^{3} y}{(2 \pi \hbar)^{3 / 2}}\left(\frac{2}{\pi}\right)^{3 / 4}\left(\frac{\xi}{\hbar}\right)^{3 / 2} & \exp \left(-\frac{\xi^{2}\left(\boldsymbol{y}-\boldsymbol{p}_{0}\right)^{2}}{\hbar^{2}}\right.  \tag{3.19.19}\\
& \left.-\frac{i\left(\boldsymbol{y}-\boldsymbol{p}_{0}\right) \cdot \boldsymbol{q}_{0}}{\hbar}\right) \exp \left(-\frac{i \boldsymbol{y}^{2} t}{2 m \hbar}+\frac{i \boldsymbol{y} \cdot \boldsymbol{x}}{\hbar}\right)
\end{align*}
$$
$$
\begin{aligned}
&= \frac{(8 \pi)^{3 / 4} \xi^{3 / 2}}{\hbar^{3}} \exp \left(-\frac{\xi^{2} \boldsymbol{p}_{0}{ }^{2}}{\hbar^{2}}+\frac{i \boldsymbol{p}_{0} \cdot \boldsymbol{q}_{0}}{\hbar}\right) \\
& \quad \times \int \frac{d^{3} y}{(2 \pi)^{3}} \exp \left[-\frac{\xi^{2}}{\hbar^{2}}\left(1+\frac{i t}{2 m \xi^{2} / \hbar}\right) \boldsymbol{y}^{2}+\frac{i}{\hbar} \boldsymbol{y} \cdot\left(\boldsymbol{x}-\boldsymbol{q}_{0}-\frac{2 i \xi^{2}}{\hbar} \boldsymbol{p}_{0}\right)\right] \\
&=\frac{(8 \pi)^{3 / 4} \xi^{3 / 2}}{\hbar^{3}} \exp \left(-\frac{\xi^{2} \boldsymbol{p}_{0}{ }^{2}}{\hbar^{2}}+\frac{i \boldsymbol{p}_{0} \cdot \boldsymbol{q}_{0}}{\hbar}\right) \\
& \times J\left(\frac{i}{\hbar}\left(\boldsymbol{x}-\boldsymbol{q}_{0}-\frac{i t_{0}}{m} \boldsymbol{p}_{0}\right), \frac{2^{1 / 2} \xi\left(1+i t / t_{0}\right)^{1 / 2}}{\hbar}\right) \\
&=\frac{(8 \pi)^{3 / 4} \xi^{3 / 2}}{\hbar^{3}} \exp \left(-\frac{\xi^{2} \boldsymbol{p}_{0}{ }^{2}}{\hbar^{2}}+\frac{i \boldsymbol{p}_{0} \cdot \boldsymbol{q}_{0}}{\hbar}\right) \\
& \quad \times \frac{\hbar^{3}}{(4 \pi)^{3 / 2} \xi^{3}\left(1+i t / t_{0}\right)^{3 / 2}} \exp \left[-\frac{1}{4 \xi^{2}\left(1+i t / t_{0}\right)}\left(\boldsymbol{x}-\boldsymbol{q}_{0}-\frac{i t_{0}}{m} \boldsymbol{p}_{0}\right)^{2}\right] \\
&=\left.-\frac{1}{4 \xi^{2}\left(1+i t / t_{0}\right)}\left(\boldsymbol{x}-\boldsymbol{q}_{0}-\frac{i t_{0}}{m} \boldsymbol{p}_{0}\right)^{2}\right]
\end{aligned}
$$

The next step will be to suitably reshape the argument of the exponential.
Separating real and imaginary part we have
$$
\begin{align*}
&- \frac{1}{4 \xi^{2}\left(1+i t / t_{0}\right)}\left(\boldsymbol{x}-\boldsymbol{q}_{0}-\frac{i t_{0}}{m} \boldsymbol{p}_{0}\right)^{2}  \tag{3.19.20}\\
&=-\frac{1-i t / t_{0}}{4 \xi^{2}\left(1+t^{2} / t_{0}^{2}\right)}\left[\left(\boldsymbol{x}-\boldsymbol{q}_{0}\right)^{2}-\frac{t_{0}{ }^{2}}{m^{2}} \boldsymbol{p}_{0}{ }^{2}-\frac{2 i t_{0}}{m} \boldsymbol{p}_{0} \cdot\left(\boldsymbol{x}-\boldsymbol{q}_{0}\right)\right] \\
&=-\frac{1}{4 \xi^{2}\left(1+t^{2} / t_{0}^{2}\right)}\left[\left(\boldsymbol{x}-\boldsymbol{q}_{0}\right)^{2}-\frac{t_{0}{ }^{2}}{m^{2}} \boldsymbol{p}_{0}{ }^{2}-\frac{2 t}{m} \boldsymbol{p}_{0} \cdot\left(\boldsymbol{x}-\boldsymbol{q}_{0}\right)\right. \\
&\left.\quad-\frac{i t}{t_{0}}\left(\left(\boldsymbol{x}-\boldsymbol{q}_{0}\right)^{2}-\frac{t_{0}{ }^{2}}{m^{2}} \boldsymbol{p}_{0}{ }^{2}\right)-\frac{2 i t_{0}}{m} \boldsymbol{p}_{0} \cdot\left(\boldsymbol{x}-\boldsymbol{q}_{0}\right)\right] \\
&=-\frac{1}{4 \xi^{2}\left(1+t^{2} / t_{0}^{2}\right)}\left[\left(1-i t / t_{0}\right)\left(\boldsymbol{x}-\boldsymbol{q}_{0}-\frac{t}{m} \boldsymbol{p}_{0}\right)^{2}\right. \\
&\left.\quad+\left(1+t^{2} / t_{0}{ }^{2}\right)\left(-\frac{t_{0}{ }^{2}}{m^{2}} \boldsymbol{p}_{0}{ }^{2}+\frac{i t_{0} t}{m^{2}} \boldsymbol{p}_{0}{ }^{2}-\frac{2 i t_{0}}{m} \boldsymbol{p}_{0} \cdot\left(\boldsymbol{x}-\boldsymbol{q}_{0}\right)\right)\right] \\
&=-\frac{1}{4 \xi^{2}\left(1+i t / t_{0}\right)}\left(\boldsymbol{x}-\boldsymbol{q}_{0}-\frac{t}{m} \boldsymbol{p}_{0}\right)^{2}+\frac{\xi^{2} \boldsymbol{p}_{0}{ }^{2}}{\hbar^{2}}-\frac{i t \boldsymbol{p}_{0}{ }^{2}}{2 m \hbar}+\frac{i \boldsymbol{p}_{0} \cdot\left(\boldsymbol{x}-\boldsymbol{q}_{0}\right)}{\hbar}
\end{align*}
$$

Substituting (3.19.20) into (3.19.19), we obtain the sought for relation (3.19.16).

From (3.19.16), the probability density of the free particle is thus given by
$$
\begin{equation*}
\rho(t, \boldsymbol{x})=|\psi(t, \boldsymbol{x})|^{2}=\frac{1}{(2 \pi)^{3 / 2} \xi(t)^{3}} \exp \left(-\frac{\left(\boldsymbol{x}-\boldsymbol{q}_{0}-t \boldsymbol{p}_{0} / m\right)^{2}}{2 \xi(t)^{2}}\right) \tag{3.19.21}
\end{equation*}
$$
where $\xi(t)$ is the time dependent length scale
$$
\begin{equation*}
\xi(t)=\xi\left[1+\left(t / t_{0}\right)^{2}\right]^{1 / 2} \tag{3.19.22}
\end{equation*}
$$
roughly equal to $\xi$ for $t \ll t_{0}$ and growing linearly in $t$ for $t \gg t_{0}$.
The quantum mean values of the position, momentum angular momentum and energy of the particle are found to be given by
$$
\begin{align*}
& \langle\boldsymbol{q}\rangle_{t}=\boldsymbol{q}_{0}+\frac{t \boldsymbol{p}_{0}}{m}  \tag{3.19.23}\\
& \langle\boldsymbol{p}\rangle_{t}=\boldsymbol{p}_{0}  \tag{3.19.24}\\
& \langle\boldsymbol{l}\rangle_{t}=\boldsymbol{q}_{0} \times \boldsymbol{p}_{0}  \tag{3.19.25}\\
& \langle H\rangle_{t}=\frac{\boldsymbol{p}_{0}^{2}}{2 m}+\frac{3 \hbar^{2}}{8 m \xi^{2}} \tag{3.19.26}
\end{align*}
$$

Proof. Inspecting (3.19.16), it is immediate that the wave function $\psi$ of our free particle can be cast as
$$
\begin{equation*}
\psi(t, \boldsymbol{x})=f\left(t, \boldsymbol{x}-\boldsymbol{q}_{0}-\frac{t \boldsymbol{p}_{0}}{m}\right) \exp \left(i S_{0}(t) / \hbar\right) \tag{3.19.27}
\end{equation*}
$$
where $f$ is the wave function
$$
\begin{equation*}
f(t, \boldsymbol{x})=\frac{1}{(2 \pi)^{3 / 4}\left(\xi\left(1+i t / t_{0}\right)\right)^{3 / 2}} \exp \left(-\frac{\boldsymbol{x}^{2}}{4 \xi^{2}\left(1+i t / t_{0}\right)}+\frac{i \boldsymbol{p}_{0} \cdot \boldsymbol{x}}{\hbar}\right) \tag{3.19.28}
\end{equation*}
$$
and $S_{0}(t)$ is the action
$$
\begin{equation*}
S_{0}(t)=\frac{t \boldsymbol{p}_{0}{ }^{2}}{2 m}+\boldsymbol{p}_{0} \cdot \boldsymbol{q}_{0} \tag{3.19.29}
\end{equation*}
$$

The calculations is best carried out expressing everything in terms of $f$. Aiming to this, we conveniently collect first a few basic relations satisfied by $f$. The square magnitude $|f|^{2}$ of $f$ is the Gaussian
$$
\begin{equation*}
|f(t, \boldsymbol{x})|^{2}=\frac{1}{(2 \pi)^{3 / 2} \xi(t)^{3}} \exp \left(-\frac{\boldsymbol{x}^{2}}{2 \xi(t)^{2}}\right) \tag{3.19.30}
\end{equation*}
$$
where $\xi(t)$ is given by (3.19.22). Further, the following integral identities
$$
\begin{align*}
& \int d^{3} x|f(t, \boldsymbol{x})|^{2}=1  \tag{3.19.31}\\
& \int d^{3} x|f(t, \boldsymbol{x})|^{2} \boldsymbol{x}=\mathbf{0}  \tag{3.19.32}\\
& \int d^{3} x|f(t, \boldsymbol{x})|^{2} \boldsymbol{x}^{2}=3 \xi(t)^{2} \tag{3.19.33}
\end{align*}
$$
hold. (3.19.31) is just the familiar Guassian integration formula. (3.19.32) follows immediately from tha fact that the integrand is an odd function and the integration domain, $\mathbb{E}^{3}$, is symmetrical. (3.19.33) is the result of the following calculation
$$
\begin{align*}
\int d^{3} x & \frac{1}{(2 \pi)^{3 / 2} \xi^{3}} \exp \left(-\frac{\boldsymbol{x}^{2}}{2 \xi^{2}}\right) x^{2}  \tag{3.19.34}\\
& =\frac{1}{(2 \pi)^{3 / 2} \xi^{3}} \xi^{3} \frac{d}{d \xi} \int d^{3} x \exp \left(-\frac{\boldsymbol{x}^{2}}{2 \xi^{2}}\right)=\frac{1}{(2 \pi)^{3 / 2}} \frac{d}{d \xi}(2 \pi)^{3 / 2} \xi^{3}=3 \xi^{2}
\end{align*}
$$

From (3.5.82), inserting (3.19.27) and using (3.19.31), (3.19.32), we have
$$
\begin{align*}
&\langle\boldsymbol{q}\rangle_{t}= \int d^{3} x\left[f\left(t, \boldsymbol{x}-\boldsymbol{q}_{0}-\frac{t \boldsymbol{p}_{0}}{m}\right) \exp \left(i S_{0}(t) / \hbar\right)\right]^{*}  \tag{3.19.35}\\
& \times \boldsymbol{x} f\left(t, \boldsymbol{x}-\boldsymbol{q}_{0}-\frac{t \boldsymbol{p}_{0}}{m}\right) \exp \left(i S_{0}(t) / \hbar\right) \\
&=\int d^{3} x^{\prime}\left|f\left(t, \boldsymbol{x}^{\prime}\right)\right|^{2}\left(\boldsymbol{x}^{\prime}+\boldsymbol{q}_{0}+\frac{t \boldsymbol{p}_{0}}{m}\right)=\boldsymbol{q}_{0}+\frac{t \boldsymbol{p}_{0}}{m}
\end{align*}
$$
where we set $\boldsymbol{x}^{\prime}=\boldsymbol{x}-\boldsymbol{q}_{0}-t \boldsymbol{p}_{0} / m$. This shows (3.19.23).
By (3.19.28), we have
$$
\begin{equation*}
\boldsymbol{\nabla}_{\boldsymbol{x}} f(t, \boldsymbol{x})=\left(-\frac{\boldsymbol{x}}{2 \xi^{2}\left(1+i t / t_{0}\right)}+\frac{i \boldsymbol{p}_{0}}{\hbar}\right) f(t, \boldsymbol{x}) . \tag{3.19.36}
\end{equation*}
$$

From (3.5.83), inserting (3.19.27) and using (3.19.31), (3.19.32) and (3.19.36), we have
$$
\begin{align*}
\langle\boldsymbol{p}\rangle_{t}= & \int d^{3} x\left[f\left(t, \boldsymbol{x}-\boldsymbol{q}_{0}-\frac{t \boldsymbol{p}_{0}}{m}\right) \exp \left(i S_{0}(t) / \hbar\right)\right]^{*}  \tag{3.19.37}\\
& \times(-i \hbar) \boldsymbol{\nabla}_{\boldsymbol{x}}\left[f\left(t, \boldsymbol{x}-\boldsymbol{q}_{0}-\frac{t \boldsymbol{p}_{0}}{m}\right) \exp \left(i S_{0}(t) / \hbar\right)\right] \\
= & \int d^{3} x^{\prime} f\left(t, \boldsymbol{x}^{\prime}\right)^{*}(-i \hbar) \nabla_{\boldsymbol{x}^{\prime}} f\left(t, \boldsymbol{x}^{\prime}\right) \\
= & \int d^{3} x^{\prime}\left|f\left(t, \boldsymbol{x}^{\prime}\right)\right|^{2}\left(\frac{i \hbar \boldsymbol{x}^{\prime}}{2 \xi^{2}\left(1+i t / t_{0}\right)}+\boldsymbol{p}_{0}\right)=\boldsymbol{p}_{0}
\end{align*}
$$
where $\boldsymbol{x}^{\prime}$ is defined as above. This shows (3.19.24).

From (3.5.84), inserting (3.19.27) and using (3.19.31), (3.19.32) and (3.19.36),
$$
\begin{align*}
\langle\boldsymbol{l}\rangle_{t}= & \int d^{3} x\left[f\left(t, \boldsymbol{x}-\boldsymbol{q}_{0}-\frac{t \boldsymbol{p}_{0}}{m}\right) \exp \left(i S_{0}(t) / \hbar\right)\right]^{*}  \tag{3.19.38}\\
& \times(-i \hbar) \boldsymbol{x} \times \boldsymbol{\nabla}_{\boldsymbol{x}}\left[f\left(t, \boldsymbol{x}-\boldsymbol{q}_{0}-\frac{t \boldsymbol{p}_{0}}{m}\right) \exp \left(i S_{0}(t) / \hbar\right)\right] \\
= & \int d^{3} x^{\prime} f\left(t, \boldsymbol{x}^{\prime}\right)^{*}(-i \hbar)\left(\boldsymbol{x}^{\prime}+\boldsymbol{q}_{0}+\frac{t \boldsymbol{p}_{0}}{m}\right) \times \boldsymbol{\nabla}_{\boldsymbol{x}^{\prime}} f\left(t, \boldsymbol{x}^{\prime}\right) \\
= & \int d^{3} x^{\prime}\left|f\left(t, \boldsymbol{x}^{\prime}\right)\right|^{2}\left(\boldsymbol{x}^{\prime}+\boldsymbol{q}_{0}+\frac{t \boldsymbol{p}_{0}}{m}\right) \times\left(\frac{i \hbar \boldsymbol{x}^{\prime}}{2 \xi^{2}\left(1+i t / t_{0}\right)}+\boldsymbol{p}_{0}\right) \\
= & \int d^{3} x^{\prime}\left|f\left(t, \boldsymbol{x}^{\prime}\right)\right|^{2}\left[\boldsymbol{x}^{\prime} \times\left(-\frac{i \hbar\left(\boldsymbol{q}_{0}+t \boldsymbol{p}_{0} / m\right)}{2 \xi^{2}\left(1+i t / t_{0}\right)}+\boldsymbol{p}_{0}\right)+\boldsymbol{q}_{0} \times \boldsymbol{p}_{0}\right]=\boldsymbol{q}_{0} \times \boldsymbol{p}_{0}
\end{align*}
$$
where $\boldsymbol{x}^{\prime}$ is defined as above. (3.19.25) is thus proven.
By (3.19.28), we have
$$
\begin{align*}
& \boldsymbol{\nabla}_{\boldsymbol{x}}^{2} f(t, \boldsymbol{x})=\left[-\frac{3}{2 \xi^{2}\left(1+i t / t_{0}\right)}+\left(-\frac{\boldsymbol{x}}{2 \xi^{2}\left(1+i t / t_{0}\right)}+\frac{i \boldsymbol{p}_{0}}{\hbar}\right)^{2}\right] f(t, \boldsymbol{x})  \tag{3.19.39}\\
& \quad=\left(-\frac{3}{2 \xi^{2}\left(1+i t / t_{0}\right)}+\frac{\boldsymbol{x}^{2}}{4 \xi^{4}\left(1+i t / t_{0}\right)^{2}}-\frac{\boldsymbol{p}_{0}^{2}}{\hbar^{2}}-\frac{i \boldsymbol{p}_{0} \cdot \boldsymbol{x}}{\hbar \xi^{2}\left(1+i t / t_{0}\right)}\right) f(t, \boldsymbol{x})
\end{align*}
$$

From (3.5.85), inserting (3.19.13) and using (3.19.31)-(3.19.33), we find
$$
\begin{align*}
\langle H\rangle_{t}= & \int d^{3} x\left[f\left(t, \boldsymbol{x}-\boldsymbol{q}_{0}-\frac{t \boldsymbol{p}_{0}}{m}\right) \exp \left(i S_{0}(t) / \hbar\right)\right]^{*}  \tag{3.19.40}\\
& \times \frac{-\hbar^{2}}{2 m} \boldsymbol{\nabla}_{\boldsymbol{x}}{ }^{2}\left[f\left(t, \boldsymbol{x}-\boldsymbol{q}_{0}-\frac{t \boldsymbol{p}_{0}}{m}\right) \exp \left(i S_{0}(t) / \hbar\right)\right] \\
= & \int d^{3} x^{\prime} f\left(t, \boldsymbol{x}^{\prime}\right)^{*} \frac{-\hbar^{2}}{2 m} \boldsymbol{\nabla}_{\boldsymbol{x}^{\prime}}{ }^{2} f\left(t, \boldsymbol{x}^{\prime}\right) \\
= & \int d^{3} x^{\prime}\left|f\left(t, \boldsymbol{x}^{\prime}\right)\right|^{2}\left(\frac{3 \hbar^{2}}{4 m \xi^{2}\left(1+i t / t_{0}\right)}\right. \\
& \left.-\frac{\hbar^{2} \boldsymbol{x}^{\prime 2}}{8 m \xi^{4}\left(1+i t / t_{0}\right)^{2}}+\frac{\boldsymbol{p}_{0}{ }^{2}}{2 m}+\frac{i \hbar \boldsymbol{p}_{0} \cdot \boldsymbol{x}^{\prime}}{2 m \xi^{2}\left(1+i t / t_{0}\right)}\right) \\
= & \frac{3 \hbar^{2}\left(1-i t / t_{0}\right)}{4 m \xi(t)^{2}}-\frac{3 \hbar^{2}\left(1-i t / t_{0}\right)^{2}}{8 m \xi(t)^{2}}+\frac{\boldsymbol{p}_{0}{ }^{2}}{2 m} \\
= & \frac{3 \hbar^{2}\left(1+t^{2} / t_{0}{ }^{2}\right)}{8 m \xi(t)^{2}}+\frac{\boldsymbol{p}_{0}{ }^{2}}{2 m}=\frac{3 \hbar^{2}}{8 m \xi^{2}}+\frac{\boldsymbol{p}_{0}{ }^{2}}{2 m}
\end{align*}
$$
where again $\boldsymbol{x}^{\prime}$ is defined in the same way. (3.19.26) is thus also shown.

Inspecting (3.19.21), we realize that the wave packet at time $t$ is peaked around the point $\boldsymbol{r}(t)=\boldsymbol{q}_{0}+t \boldsymbol{p}_{0} / m$ that is the position at time $t$ of a classical free particle of initial position $\boldsymbol{q}_{0}$ and momentum $\boldsymbol{p}_{0}$. From (3.19.23)-(3.19.25), it further emerges that the mean position $\langle\boldsymbol{q}\rangle_{t}$, momentum $\langle\boldsymbol{p}\rangle_{t}$ and angular momentum $\langle\boldsymbol{l}\rangle_{t}$ equal the position, momentum and angular momentum of the classical particle at $\boldsymbol{r}(t)$. This suggests a mechanism by which classical mechanics emerges in wave mechanics. This theory, however, is actually incorrect. By (3.19.22), the wave packet spreads out and ceases to be peaked around $\boldsymbol{r}(t)$, since its width $\xi(t)$ grows indefinitely as $t$ flows. As long as $\xi(t)$ is smaller then the spacial resolution of the experimental equipment used for observation, the packet exhibits the typical features a classical free particle, since the probability of finding the particle away

![](https://cdn.mathpix.com/cropped/2024_09_22_5d1e855547710648961eg-0370.jpg?height=762&width=1196&top_left_y=1370&top_left_x=432)

Figure 3.19.2. The spreading of a Gaussian wave packet as time flows. Here, the probability density $\rho$ has been plotted at three successive times. Configuration space has been reduced to two dimensions to allow for a graphical representation.
from $\boldsymbol{q}(t)$ a distance larger than $\xi(t)$ is negligible. But as soon as $\xi(t)$ gets so large that the packet collapses, this is no longer so, as shown in fig. 3.19.2. Furthermore, by (3.19.26), the energy of the classical free particle, $\boldsymbol{p}_{0}{ }^{2} / 2 m$, differs from the mean energy $\langle H\rangle_{t}$ by a non vanishing amount $3 \hbar^{2} / 8 m \xi^{2}$. This latter is negligible only if $\xi \gg \hbar / p_{0}$, the de Broglie wave length of the particle, which happens only for sufficiently large $p_{0}$.

We note that the collapse of the packet starts roughly when $\xi(t)$ gets appreciably larger that $\xi$. From (3.19.22), it appears that this happens for $t \gg t_{0}$ where $t_{0}$ is the time scale given in eq. (3.19.17).

Above, we computed the expansion coefficient $a_{\boldsymbol{y}}$ appearing in the expansion (3.19.10) for the Gaussian wave packet. From that calculation, it follows that
$$
\begin{equation*}
\left|a_{\boldsymbol{y}}\right| \simeq\left(\frac{\xi}{\hbar}\right)^{3 / 2} \exp \left(-\frac{\xi^{2}\left(\boldsymbol{y}-\boldsymbol{p}_{0}\right)^{2}}{\hbar^{2}}\right) \tag{3.19.41}
\end{equation*}
$$

Proof. This follows readily from the calculation (3.19.18).

Consequently, $a_{\boldsymbol{y}}$ is appreciably different from zero in units $(\xi / \hbar)^{3 / 2}$ for $\left|\boldsymbol{y}-\boldsymbol{p}_{0}\right|$ $<\hbar / \xi$. It follows that the uncertainty of momentum is $\Delta p \sim \hbar / \xi$ and therefore that the uncertainty of energy is
$$
\begin{equation*}
\Delta w \sim \frac{(\Delta p)^{2}}{2 m} \sim \frac{\hbar^{2}}{2 m \xi^{2}} \tag{3.19.42}
\end{equation*}
$$

By the energy-time uncertainty relation (3.18.15), the state of the particle is roughly stationary for a time duration
$$
\begin{equation*}
\Delta t \sim \frac{\hbar}{\Delta w} \sim \frac{2 m \xi^{2}}{\hbar}=t_{0} \tag{3.19.43}
\end{equation*}
$$
where relation (3.19.17) was used. As we found above, $t_{0}$ is also the time at which the collapse of the wave packet sets in. Thus, the collapse is directly related to the loss of stationarity.

\subsection*{3.20. The Schroedinger equation for systems of many particles}

Nuclei, atoms and molecules are systems formed by several quantum particles They constitute themselves quantum systems though of a composite instead than elementary nature. The natural question arises about whether it is possible to construct a suitable generalization of the Schroedinger equation (3.4.14) governing such many particle systems.

The basic idea underlying the solution of this problem is that a system formed by $n$ particles of positions $\boldsymbol{q}_{1}, \ldots, \boldsymbol{q}_{n}$ and momenta $\boldsymbol{p}_{1}, \ldots, \boldsymbol{p}_{n}$ is formally equivalent to a fictitious particle moving in a higher $3 n$-dimensional space of position $\left(\boldsymbol{q}_{1}, \ldots, \boldsymbol{q}_{n}\right)$ and momentum $\left(\boldsymbol{p}_{1}, \ldots, \boldsymbol{p}_{n}\right)$. One can construct a statistical ensemble for the fictitious particle, define the associated statistical mass density $\rho_{c}$ and Hamilton principal function $S_{c}$ and write down the Hamilton-Jacobi and statistical mass conservation equations they obey. Finally, one can show that these equations are the geometrical optical limit of a $3 n$-dimensional undulatory system described by a wave function $\psi$ obeying the appropriate $3 n$-dimensional generalization of the Schroedinger equation. We now spell out this analysis in greater detail avoiding however to repeat considerations which are essentially identical to those holding in the single particle case.

We consider a system of $n$ particles subject to and interacting through conservative force fields. A trajectory $\left(\boldsymbol{q}_{1}(t), \ldots, \boldsymbol{q}_{n}(t)\right)$ of the particle system is said physical if it satisfies the equation of motion
$$
\begin{equation*}
m_{i} \frac{d^{2} \boldsymbol{q}_{i}}{d t^{2}}=-\nabla_{i} U\left(\boldsymbol{q}_{1}, \ldots, \boldsymbol{q}_{n}\right) \tag{3.20.1}
\end{equation*}
$$
where $\boldsymbol{\nabla}_{i}=\boldsymbol{\nabla}_{\boldsymbol{q}_{i}}$ and $U\left(\boldsymbol{x}_{1}, \ldots, \boldsymbol{x}_{n}\right)$ is the total potential energy (cf. eq. (3.3.1)). A physical trajectory is called allowed if $\boldsymbol{q}_{i}(0)=\boldsymbol{x}_{i 0}$, where the $\boldsymbol{x}_{i 0}$ are assigned initial position of the constituent particles.

The statistical ensemble of the system of particles is a collection of a large number $N$ of independent copies of the system. The set of allowed trajectories
of each copy of the system is the same as that of this latter. Therefore, the trajectory $\left(\boldsymbol{q}_{1}, \ldots, \boldsymbol{q}_{n}\right)$ of each copy is such that $\boldsymbol{q}_{i}(0)=\boldsymbol{x}_{i 0}$, while the individual copies trace generally different trajectories. The instantaneous position of the copies in the $3 n$-dimensional space form a dense set of points which can be viewed as a statistical fluid (cf. 3.3). If we assign a statistical mass $1 / N$ to each copy, the statistical mass density $\rho_{c}\left(t, \boldsymbol{x}_{1}, \ldots, \boldsymbol{x}_{n}\right)$ is defined: $d^{3} x_{1} \cdots d^{3} x_{n} \rho_{c}\left(t, \boldsymbol{x}_{1}, \ldots, \boldsymbol{x}_{n}\right)$ is the statistical mass of the fluid contained in a $3 n$-dimensional spacial region around $\left(\boldsymbol{x}_{1}, \ldots, \boldsymbol{x}_{n}\right)$ of small volume $d^{3} x_{1} \cdots d^{3} x_{n}$ at time $t$.

For any position $\left(\boldsymbol{x}_{1}, \ldots, \boldsymbol{x}_{n}\right)$ of the system, there is just an allowed trajectory $\left(\boldsymbol{q}_{1}, \ldots, \boldsymbol{q}_{n}\right)$ such that $\boldsymbol{q}_{i}(t)=\boldsymbol{x}_{i}$ at time $t$. The statistical momentum field of the system is defined by the relation $\boldsymbol{p}_{c i}\left(t, \boldsymbol{x}_{1}, \ldots, \boldsymbol{x}_{n}\right)=\boldsymbol{p}_{i}(t)$, where $\boldsymbol{p}_{i}$ is given by
$$
\begin{equation*}
\boldsymbol{p}_{i}=m \frac{d \boldsymbol{q}_{i}}{d t} \tag{3.20.2}
\end{equation*}
$$
$\left(\boldsymbol{q}_{1}, \ldots, \boldsymbol{q}_{n}\right)$ being the unique allowed trajectory such that $\boldsymbol{q}_{i}(t)=\boldsymbol{x}_{i}$ (cf. eq. $(3.3 .8))$.

The Hamilton principal function is defined by
$$
\begin{equation*}
S_{c}\left(t, \boldsymbol{x}_{1}, \ldots, \boldsymbol{x}_{n}\right)=\int_{0}^{t} d t^{\prime}\left[\sum_{i} \boldsymbol{p}_{i} \cdot \frac{d \boldsymbol{q}_{i}}{d t}-\sum_{i} \frac{\boldsymbol{p}_{i}{ }^{2}}{2 m_{i}}-U\left(\boldsymbol{q}_{1}, \ldots, \boldsymbol{q}_{n}\right)\right]^{\prime} \tag{3.20.3}
\end{equation*}
$$
where in the right hand side the trajectory $\left(\boldsymbol{q}_{1}, \ldots, \boldsymbol{q}_{n}\right)$ is the unique allowed trajectory such that $\boldsymbol{q}_{i}(t)=\boldsymbol{x}_{i}$, the $\boldsymbol{p}_{i}$ is related to $\boldsymbol{q}_{i}$ according to eq. (3.20.2) and the prime ${ }^{\prime}$ denotes evaluation at $t^{\prime}$ (cf. eq. (3.3.11)). $S_{c}$ satisfies the Hamilton-Jacobi equation
$$
\begin{equation*}
\frac{\partial S_{c}}{\partial t}+\sum_{i} \frac{\left(\boldsymbol{\nabla}_{i} S_{c}\right)^{2}}{2 m_{i}}+U=0 \tag{3.20.4}
\end{equation*}
$$
(cf. eq. (3.3.12)). Further, $S_{c}$ acts as a potential for the momentum field $\boldsymbol{p}_{c}$
$$
\begin{equation*}
\boldsymbol{p}_{c i}=\boldsymbol{\nabla}_{i} S_{c} \tag{3.20.5}
\end{equation*}
$$
(cf. eq. (3.3.13)). Finally, the statistical mass conservation reads
$$
\begin{equation*}
\frac{\partial \rho_{c}}{\partial t}+\sum_{i} \frac{1}{m_{i}} \boldsymbol{\nabla}_{i} \cdot\left(\rho_{c} \boldsymbol{\nabla}_{i} S_{c}\right)=0 \tag{3.20.6}
\end{equation*}
$$
(cf. eq. $(3.3 .18))$.
Proceeding as in sect. 3.4, we can view the statistical ensemble of our many particle system as the approximate geometrical optical description of $3 n$-dimensional undulatory system. The wave function of this latter is a complex valued function $\psi\left(\boldsymbol{x}_{1}, \ldots, \boldsymbol{x}_{n}\right)$. Just as in single particle case, $\psi$ can be written in the amplitude-phase decomposition
$$
\begin{equation*}
\psi=\rho^{1 / 2} \exp (i S / \hbar) \tag{3.20.7}
\end{equation*}
$$
(cf. eq. (3.4.13)), where the the real fields $\rho$ and $S$ reduce to $\rho_{c}$ and $S_{c}$ in the geometrical optical limit $\hbar \rightarrow 0$. The first order in time linear differential equation yielding (3.20.4)-(3.20.6) as $\hbar \rightarrow 0$ is the many particle Schroedinger equation
$$
\begin{equation*}
i \hbar \frac{\partial \psi}{\partial t}=-\sum_{i} \frac{\hbar^{2}}{2 m_{i}} \nabla_{i}^{2} \psi+U \psi \tag{3.20.8}
\end{equation*}
$$
(cf. eq. (3.4.14)). The many particle wave function has a statistical interpretation in line with that it does in the single particle case. Proceeding as in sect. 3.5, we identify statistical mass and probability. We find furthermore
(1) The probability density of the quantum ensemble is
$$
\begin{equation*}
\rho=|\psi|^{2} \tag{3.20.9}
\end{equation*}
$$
(cf. eq. (3.5.26)). So, for any $3 n$-dimensional space region $\mathcal{V}$,
$$
\begin{equation*}
p(\mathcal{V})=\int_{\mathcal{V}} d^{3} x_{1} \cdots d^{3} x_{n}|\psi|^{2} \tag{3.20.10}
\end{equation*}
$$
is the probability of finding a randomly chosen copy of the system located in $\mathcal{V}$
(cf. eq. (3.5.30)). More concretely, if $\mathcal{V}$ is the Cartesian product of the ordinary space regions $\mathcal{V}_{1}, \ldots, \mathcal{V}_{n}, p(\mathcal{V})$ is the probability that for a copy the system the constituent particles $1, \ldots, n$ are found in $\mathcal{V}_{1}, \ldots, \mathcal{V}_{n}$, respectively. However, $\mathcal{V}$ is not of such factorized form in general.
(2) The probability current density of the quantum ensemble is $\left(\boldsymbol{j}_{1}, \ldots, \boldsymbol{j}_{n}\right)$, where $\boldsymbol{j}_{i}$ is given by the formula
$$
\begin{equation*}
\boldsymbol{j}_{i}=\frac{\hbar}{m_{i}} \operatorname{Im}\left(\psi^{*} \nabla_{i} \psi\right) \tag{3.20.11}
\end{equation*}
$$
(cf. eq. (3.20.11)). It follows that, for any oriented $3 n-1$-dimensional space hypersurface $\mathcal{A}$,
$$
\begin{equation*}
\Phi(\mathcal{A})=\sum_{i} \frac{\hbar}{m_{i}} \int_{\mathcal{A}} d^{3} x_{1} \cdots d^{3} x_{i-1} d^{2} \boldsymbol{x}_{i} d^{3} x_{i+1} \cdots d^{3} x_{n} \cdot \operatorname{Im}\left(\psi^{*} \nabla_{i} \psi\right) \tag{3.20.12}
\end{equation*}
$$
is the probability of finding a randomly chosen copy of the system flowing through $\mathcal{A}$ per unit time (cf. eq. (3.5.31)). More concretely, if $\mathcal{A}$ is the Cartesian product of the ordinary space regions $\mathcal{V}_{1}, \ldots, \mathcal{V}_{i-1}$, an ordinary space surface $\mathcal{A}_{i}$ and the ordinary space regions $\mathcal{V}_{i+1}, \ldots, \mathcal{V}_{n}$, then $\Phi(\mathcal{A})$ is the probability that particles $1, \ldots, i-1$ are found in $\mathcal{V}_{1}, \ldots, \mathcal{V}_{i-1}$, particle $i$ flows through $\mathcal{A}_{i}$ and particles $i+1, \ldots, n$ are found in $\mathcal{V}_{i+1}, \ldots, \mathcal{V}_{n}$ per unit time. However, $\mathcal{A}$ is not of such factorized form in general.

From (3.20.10), it follows then that the wave function $\psi$ must satisfy the normalization condition
$$
\begin{equation*}
\int d^{3} x_{1} \cdots d^{3} x_{n}|\psi|^{2}=1 \tag{3.20.13}
\end{equation*}
$$
since we obviously have $p(\mathcal{V})=1$ when $\mathcal{V}$ is the whole $3 n$-dimensional space $\mathbb{E}^{3 n}$ (cf. eq. $(3.5 .32)$ ).

For a multiparticle system, theoretical expressions of the mean values of the main mechanical observables can be obtained using an appropriate generalization
of the Weyl-Wigner theory of sect. 3.5.
The quantum mean value of the quantity $Q_{f}$ corresponding to a a generic phase function $f$ has an operator expression analogous to (3.5.78), (3.5.79)
$$
\begin{equation*}
\left\langle Q_{f}\right\rangle=\int d^{3} x_{1} \cdots d^{3} x_{n} \psi^{*} \mathrm{Q}_{f} \psi \tag{3.20.14}
\end{equation*}
$$

Here, the Hermitian linear operator $\mathrm{Q}_{f}$ acts on the wave function $\psi$ as
$$
\begin{align*}
& \mathrm{Q}_{f} \psi\left(t, \boldsymbol{x}_{1}, \ldots, \boldsymbol{x}_{n}\right)  \tag{3.20.15}\\
& \qquad=\int d^{3} u_{1} \cdots d^{3} u_{n} K_{f}\left(\boldsymbol{x}_{1}, \ldots, \boldsymbol{x}_{n}, \boldsymbol{u}_{1}, \ldots, \boldsymbol{u}_{n}\right) \psi\left(t, \boldsymbol{u}_{1}, \ldots, \boldsymbol{u}_{n}\right)
\end{align*}
$$
where the integral kernel $K_{f}$ is given by the Weyl-Wigner transform
$$
\begin{align*}
& K_{f}\left(\boldsymbol{x}_{1}, \ldots, \boldsymbol{x}_{n}, \boldsymbol{u}_{1}, \ldots, \boldsymbol{u}_{n}\right)  \tag{3.20.16}\\
& \qquad=\int \frac{d^{3} y_{1} \cdots d^{3} y_{n}}{(2 \pi \hbar)^{3 n}} \exp \left(i\left(\left(\boldsymbol{x}_{1}-\boldsymbol{u}_{1}\right) \cdot \boldsymbol{y}_{1}+\ldots+\left(\boldsymbol{x}_{n}-\boldsymbol{u}_{n}\right) \cdot \boldsymbol{y}_{n}\right) / \hbar\right) \\
& \quad f\left(\left(\boldsymbol{x}_{1}+\boldsymbol{u}_{1}\right) / 2, \ldots,\left(\boldsymbol{x}_{n}+\boldsymbol{u}_{n}\right) / 2, \boldsymbol{y}_{1}, \ldots, \boldsymbol{y}_{n}\right)
\end{align*}
$$
in keeping with (3.5.80). In particular, the quantum mean position, momentum, angular momentum of the $i$-th particle and the total energy of a copy of the system $\left\langle\boldsymbol{q}_{i}\right\rangle,\left\langle\boldsymbol{p}_{i}\right\rangle,\left\langle\boldsymbol{l}_{i}\right\rangle,\langle H\rangle$ are given by
$$
\begin{align*}
\left\langle\boldsymbol{q}_{i}\right\rangle & =\int d^{3} x_{1} \cdots d^{3} x_{n} \psi^{*} \boldsymbol{x}_{i} \psi  \tag{3.20.17}\\
\left\langle\boldsymbol{p}_{i}\right\rangle & =\int d^{3} x_{1} \cdots d^{3} x_{n} \psi^{*}\left(-i \hbar \boldsymbol{\nabla}_{i} \psi\right)  \tag{3.20.18}\\
\left\langle\boldsymbol{l}_{i}\right\rangle & =\int d^{3} x_{1} \cdots d^{3} x_{n} \psi^{*}\left(-i \hbar \boldsymbol{x}_{i} \times \boldsymbol{\nabla}_{i} \psi\right)  \tag{3.20.19}\\
\langle H\rangle & =\int d^{3} x_{1} \cdots d^{3} x_{n} \psi^{*}\left(-\sum_{i} \frac{\hbar^{2}}{2 m_{i}} \boldsymbol{\nabla}_{i}^{2} \psi+U \psi\right) \tag{3.20.20}
\end{align*}
$$
(cf. eqs. $(3.5 .82)-(3.5 .85))$.
The wave function $\psi$ determines the probability density and current density
$\rho,\left(\boldsymbol{j}_{1}, \ldots, \boldsymbol{j}_{n}\right)$ of the ensemble of the many particle system as well all the mean values $\left\langle\boldsymbol{q}_{i}\right\rangle,\left\langle\boldsymbol{p}_{i}\right\rangle,\left\langle\boldsymbol{l}_{i}\right\rangle,\langle H\rangle$ through (3.20.9), (3.20.11) and (3.20.17)-(3.20.20), respectively. $\psi$ encodes the state of the ensemble (cf. sect. 3.6). Alternatively, it codifies the probabilistic state of the system (cf. sect. 3.8).

Starting from (3.20.8) and proceeding as in sect. 3.14, it is possible to obtain also the time independent many particle Schroedinger equation
$$
\begin{equation*}
-\sum_{i} \frac{\hbar^{2}}{2 m_{i}} \nabla_{i}^{2} \phi+U \phi=w \phi \tag{3.20.21}
\end{equation*}
$$
(cf. eq. (3.14.12)). The analysis of the energy spectrum and eigenfunctions is totally analogous to that of the single particle case.

We have built the Schroedinger theory of a many particle system as a straightforward higher dimensional generalization of the ordinary Schroedinger theory of a single particle. However, there are novel properties characterizing systems of many particles that do not show up when only one particle is involved, as we explain next.

Let a system of $n$ particles be composed of two non interacting subsystems $I$ and $I I$ of $n_{I}$ and $n_{I I}$ particles respectively, where $n=n_{I}+n_{I I}$. The particles are numbered in such a way that if $1, \ldots, n$ are the particles of the system, then $1, \ldots, n_{I}$ and and $n_{I}+1, \ldots, n_{I}+n_{I I}$ are are those of subsystems $I$ and $I I . A s$ the subsystems are non interacting, there must be states of the systems in which the subsystems are statistically independent. For these, the probability density $\rho$ of the system factorizes as the product of those $\rho_{I}, \rho_{I I}$ of the subsystems,
$$
\begin{equation*}
\rho=\rho_{I} \rho_{I I} \tag{3.20.22}
\end{equation*}
$$

Above, $\rho$ depends on all space variables $\boldsymbol{x}_{1}, \ldots, \boldsymbol{x}_{n}$, while $\rho_{I}$ and $\rho_{I I}$ depend on the space variables $\boldsymbol{x}_{1}, \ldots, \boldsymbol{x}_{n_{I}}$ and $\boldsymbol{x}_{n_{I}+1}, \ldots, \boldsymbol{x}_{n_{I}+n_{I I}}$ only, respectively. Similarly, the probability current density $\left(\boldsymbol{j}_{1}, \ldots, \boldsymbol{j}_{n_{I}+n_{I I}}\right)$ of the system and those $\left(\boldsymbol{j}_{I 1}, \ldots, \boldsymbol{j}_{I n_{I}}\right),\left(\boldsymbol{j}_{I I n_{I}+1}, \ldots, \boldsymbol{j}_{I I n_{I}+n_{I I}}\right)$ of the subsystems are related as
$$
\begin{align*}
\boldsymbol{j}_{i}=\rho_{I I} \boldsymbol{j}_{I i}, & i=1, \ldots, n_{I}  \tag{3.20.23}\\
\boldsymbol{j}_{i}=\rho_{I} \boldsymbol{j}_{I I}, & i=n_{I}+1, \ldots, n_{I}+n_{I I}
\end{align*}
$$

Above, the $\boldsymbol{j}_{i}$ depends on all space variables $\boldsymbol{x}_{1}, \ldots, \boldsymbol{x}_{n}$, while $\boldsymbol{j}_{I i}$ and $\boldsymbol{j}_{I I i}$ depend on the space variables $\boldsymbol{x}_{1}, \ldots, \boldsymbol{x}_{n_{I}}$ and $\boldsymbol{x}_{n_{I}+1}, \ldots, \boldsymbol{x}_{n_{I}+n_{I I}}$ only, respectively.

These properties can be explained if the wave function $\psi$ of the system factorizes in those $\psi_{I}, \psi_{I I}$ of the subsystems
$$
\begin{equation*}
\psi=\psi_{I} \psi_{I I} \tag{3.20.24}
\end{equation*}
$$
where again $\psi$ depends on all space variables $\boldsymbol{x}_{1}, \ldots, \boldsymbol{x}_{n}$, while $\psi_{I}$ and $\psi_{I I}$ depend on the space variables $\boldsymbol{x}_{1}, \ldots, \boldsymbol{x}_{n_{I}}$ and $\boldsymbol{x}_{n_{I}+1}, \ldots, \boldsymbol{x}_{n_{I}+n_{I I}}$ only, respectively.

Proof. Inserting (3.20.24) into (3.20.9), we find
$$
\begin{equation*}
\rho=|\psi|^{2}=\left|\psi_{I} \psi_{I I}\right|^{2}=\rho_{I} \rho_{I I} \tag{3.20.25}
\end{equation*}
$$
showing (3.20.22). Next, inserting (3.20.24) into (3.20.11), we find
$$
\begin{align*}
& \boldsymbol{j}_{i}=\frac{\hbar}{m_{i}} \operatorname{Im}\left(\psi^{*} \nabla_{i} \psi\right)=\frac{\hbar}{m_{i}} \operatorname{Im}\left(\left(\psi_{I} \psi_{I I}\right)^{*} \nabla_{i}\left(\psi_{I} \psi_{I I}\right)\right)  \tag{3.20.26}\\
& =\frac{\hbar}{m_{i}}\left|\psi_{I I}\right|^{2} \operatorname{Im}\left(\psi_{I}{ }^{*} \nabla_{i} \psi_{I}\right)=\rho_{I I} \boldsymbol{j}_{I i}, \quad i=1, \ldots, n_{I}, \\
& \boldsymbol{j}_{i}=\frac{\hbar}{m_{i}} \operatorname{Im}\left(\psi^{*} \nabla_{i} \psi\right)=\frac{\hbar}{m_{i}} \operatorname{Im}\left(\left(\psi_{I} \psi_{I I}\right)^{*} \nabla_{i}\left(\psi_{I} \psi_{I I}\right)\right) \\
& =\frac{\hbar}{m_{i}}\left|\psi_{I}\right|^{2} \operatorname{Im}\left(\psi_{I I}{ }^{*} \nabla_{i} \psi_{I I}\right)=\rho_{I} \boldsymbol{j}_{I I}, \quad i=n_{I}+1, \ldots, n_{I}+n_{I I} .
\end{align*}
$$
showing (3.20.23).

The factorization (3.20.24) is compatible with dynamics in the following sense. If the wave functions $\psi_{I}, \psi_{I I}$ satisfy the many particle Schroedinger equation (3.20.8) for the subsystems $I$ and $I I$, then the wave function $\psi$ does for the whole system. This follows from the fact that, by the non interaction of the subsystems,
the total potential energy of the systems splits as
$$
\begin{equation*}
U=U_{I}+U_{I I} \tag{3.20.27}
\end{equation*}
$$
where $U_{I}, U_{I I}$ are the potential energies of the subsytems $I$ and $I I$ depending on the space variables $\boldsymbol{x}_{1}, \ldots, \boldsymbol{x}_{n_{I}}$ and $\boldsymbol{x}_{n_{I}+1}, \ldots, \boldsymbol{x}_{n_{I}+n_{I I}}$ only, respectively.

Proof. Since $\psi_{I}, \psi_{I I}$ satisfy the Schroedinger equation (3.20.8),
$$
\begin{align*}
& i \hbar \frac{\partial \psi}{\partial t}=i \hbar \frac{\partial\left(\psi_{I} \psi_{I I}\right)}{\partial t}=i \hbar \frac{\partial \psi_{I}}{\partial t} \psi_{I I}+\psi_{I} i \hbar \frac{\partial \psi_{I I}}{\partial t}  \tag{3.20.28}\\
& =\left(-\sum_{i \text { in } I} \frac{\hbar^{2}}{2 m_{i}} \nabla_{i}^{2} \psi_{I}+U_{I} \psi_{I}\right) \psi_{I I}+\psi_{I}\left(-\sum_{i \text { in } I I} \frac{\hbar^{2}}{2 m_{i}} \nabla_{i}^{2} \psi_{I I}+U_{I I} \psi_{I I}\right) \\
& =-\sum_{i \text { in } I} \frac{\hbar^{2}}{2 m_{i}} \nabla_{i}^{2}\left(\psi_{I} \psi_{I I}\right)-\sum_{i \text { in } I I} \frac{\hbar^{2}}{2 m_{i}} \nabla_{i}^{2}\left(\psi_{I} \psi_{I I}\right)+\left(U_{I}+U_{I I}\right)\left(\psi_{I} \psi_{I I}\right) \\
& =-\sum_{i} \frac{\hbar^{2}}{2 m_{i}} \nabla_{i}^{2} \psi+U \psi
\end{align*}
$$

Thus, $\psi$ solves (3.20.8).

Likewise, if the wave functions $\phi_{I}, \phi_{I I}$ are energy eigenfunctions of the subsystems $I$ and $I I$ belonging to the energy eigenvalues $w_{I}, w_{I I}$, then the wave function $\phi=\phi_{I} \phi_{I I}$ is one of the system belonging to the energy eigenvalue $w=w_{I}+w_{I I}$.

Proof. By (3.20.21), since $\phi_{I}, \phi_{I I}$ are energy eigenfunctions belonging to $w_{I}, w_{I I}$
$$
\begin{align*}
& =-\sum_{i} \frac{\hbar^{2}}{2 m_{i}} \nabla_{i}^{2} \phi+U \phi  \tag{3.20.29}\\
& =-\sum_{i \text { in } I} \frac{\hbar^{2}}{2 m_{i}} \nabla_{i}^{2}\left(\phi_{I} \phi_{I I}\right)-\sum_{i \text { in II }} \frac{\hbar^{2}}{2 m_{i}} \nabla_{i}^{2}\left(\phi_{I} \phi_{I I}\right)+\left(U_{I}+U_{I I}\right)\left(\phi_{I} \phi_{I I}\right) \\
& =\left(-\sum_{i \text { in } I} \frac{\hbar^{2}}{2 m_{i}} \nabla_{i}^{2} \phi_{I}+U_{I} \phi_{I}\right) \phi_{I I}+\phi_{I}\left(-\sum_{i \text { in II }} \frac{\hbar^{2}}{2 m_{i}} \nabla_{i}^{2} \phi_{I I}+U_{I I} \phi_{I I}\right) \\
& =w_{I} \phi_{I} \phi_{I I}+\phi_{I} w_{I I} \phi_{I I}=\left(w_{I}+w_{I I}\right)\left(\phi_{I} \phi_{I I}\right)=w \phi .
\end{align*}
$$

Thus, $\phi=\phi_{I} \phi_{I I}$ is an energy eigenfunction belonging to $w=w_{I}+w_{I I}$.

The above considerations generalize in obvious fashion to case of system composed of any number of non interacting systems. In particular, for a system of $n$ non interacting particles, there are states of the system for which the wave function factorizes as the product of the wave functions of the single particles
$$
\begin{equation*}
\psi=\psi_{1} \cdots \psi_{n} \tag{3.20.30}
\end{equation*}
$$

States with factorized wave functions as those described above are not the most general ones. There are entangled states, whose wave functions does not factorize but can expressed a sum of factorized terms.

\subsection*{3.21. The Schroedinger equation for a spinning particle}

A spinning particle has a two-valued spinorial degree of freedom in addition to the orbital ones (cf. sect. 2.10). For this reason, the particle is not described by a single wave function $\psi$ as for a spinless particle, but by two wave functions $\psi_{ \pm \frac{1}{2}}$, corresponding to the two values $\pm \hbar / 2$ of the spin projection along the $\boldsymbol{e}_{3}$ axis.

The Schroedinger equation of a spinning particle is a natural generalization of that of a spinless one, eq. (3.4.14). It reads
$$
\begin{equation*}
i \hbar \frac{\partial \psi_{ \pm \frac{1}{2}}}{\partial t}=-\frac{\hbar^{2}}{2 m} \nabla^{2} \psi_{ \pm \frac{1}{2}}+U_{ \pm \frac{1}{2} \pm \frac{1}{2}} \psi_{ \pm \frac{1}{2}}+U_{ \pm \frac{1}{2} \mp \frac{1}{2}} \psi_{\mp \frac{1}{2}} \tag{3.21.1}
\end{equation*}
$$
where $U_{ \pm \frac{1}{2} \pm \frac{1}{2}}, U_{ \pm \frac{1}{2} \mp \frac{1}{2}}$ are spinorial energy potential functions.
The interpretation of a spinning particle wave function components $\psi_{ \pm \frac{1}{2}}$ is an adaptation of that of a spinless particle wave function $\psi$ (cf. sect. 3.5):
$$
\begin{equation*}
\rho_{ \pm \frac{1}{2}}=\left|\psi_{ \pm \frac{1}{2}}\right|^{2} \tag{3.21.2}
\end{equation*}
$$
is the probability density of the particle when its spin projection is $\pm \hbar / 2$ (cf. eq. (3.5.26)). This means that the probability that the spinning particle is found with spin projection $\pm \hbar / 2$ in a space region $\mathcal{V}$ is given by
$$
\begin{equation*}
p_{ \pm \frac{1}{2}}(\mathcal{V})=\int_{\mathcal{V}} d^{3} x\left|\psi_{ \pm \frac{1}{2}}\right|^{2} \tag{3.21.3}
\end{equation*}
$$
(cf. eq. (3.5.30)). When $\mathcal{V}=\mathbb{E}^{3}$ is the whole space, this yields the probability $p_{ \pm \frac{1}{2}}$ that the spinning particle is found anywhere in space with spin $\pm \hbar / 2$. So,
$$
\begin{equation*}
\int d^{3} x\left|\psi_{ \pm \frac{1}{2}}\right|^{2}=p_{ \pm \frac{1}{2}} \tag{3.21.4}
\end{equation*}
$$

The total probability density of the spinning particle is
$$
\begin{equation*}
\rho=\rho_{+\frac{1}{2}}+\rho_{-\frac{1}{2}}=\left|\psi_{+\frac{1}{2}}\right|^{2}+\left|\psi_{-\frac{1}{2}}\right|^{2} \tag{3.21.5}
\end{equation*}
$$

The probability that the spinning particle is found with any spin projection in a space region $\mathcal{V}$ is therefore given by
$$
\begin{equation*}
p(\mathcal{V})=\int_{\mathcal{V}} d^{3} x\left(\left|\psi_{+\frac{1}{2}}\right|^{2}+\left|\psi_{-\frac{1}{2}}\right|^{2}\right) \tag{3.21.6}
\end{equation*}
$$

When $\mathcal{V}=\mathbb{E}^{3}$, this furnishes the normalization condition
$$
\begin{equation*}
\int d^{3} x\left(\left|\psi_{+\frac{1}{2}}\right|^{2}+\left|\psi_{-\frac{1}{2}}\right|^{2}\right)=p_{+\frac{1}{2}}+p_{-\frac{1}{2}}=1 \tag{3.21.7}
\end{equation*}
$$

In analogy to the spinless case,
$$
\begin{equation*}
\boldsymbol{j}_{ \pm \frac{1}{2}}=\frac{\hbar}{m} \operatorname{Im}\left(\psi_{ \pm \frac{1}{2}} * \nabla \psi_{ \pm \frac{1}{2}}\right) \tag{3.21.8}
\end{equation*}
$$
is the probability current density of the spinning particle when its spin projection is $\pm \hbar / 2$ (cf. eq. (3.5.27)). Thus, for any oriented surface $\mathcal{A}$
$$
\begin{equation*}
\Phi_{ \pm \frac{1}{2}}(\mathcal{A})=\frac{\hbar}{m} \int_{\mathcal{A}} d^{2} \boldsymbol{x} \cdot \operatorname{Im}\left(\psi_{ \pm \frac{1}{2}} * \nabla \psi_{ \pm \frac{1}{2}}\right) \tag{3.21.9}
\end{equation*}
$$
is the probability that the particle crosses $\mathcal{A}$ with spin projection $\pm \hbar / 2$ per unit time (cf. eq. (3.5.31)). The particle's total probability current density is
$$
\begin{equation*}
\boldsymbol{j}=\boldsymbol{j}_{+\frac{1}{2}}+\boldsymbol{j}_{-\frac{1}{2}}=\frac{\hbar}{m}\left[\operatorname{Im}\left(\psi_{+\frac{1}{2}} * \boldsymbol{\nabla} \psi_{+\frac{1}{2}}\right)+\operatorname{Im}\left(\psi_{-\frac{1}{2}} * \boldsymbol{\nabla} \psi_{-\frac{1}{2}}\right)\right] \tag{3.21.10}
\end{equation*}
$$

The probability that the particle crosses $\mathcal{A}$ with any spin projection per unit time is therefore
$$
\begin{equation*}
\Phi(\mathcal{A})=\frac{\hbar}{m} \int_{\mathcal{A}} d^{2} \boldsymbol{x} \cdot\left(\operatorname{Im}\left(\psi_{+\frac{1}{2}} * \boldsymbol{\nabla} \psi_{+\frac{1}{2}}\right)+\operatorname{Im}\left(\psi_{-\frac{1}{2}} * \boldsymbol{\nabla} \psi_{-\frac{1}{2}}\right)\right) \tag{3.21.11}
\end{equation*}
$$

The Weyl-Wigner theory of sect. 3.5 is based on the assumption that every
classical observable has a quantum analogue and viceversa. Therefore, the classical observable $Q_{f}$ corresponding to phase function $f$ is also a quantum observable and any quantum observable is of this form. Quantically, the statistical mean of $Q_{f}$ is determined by the ensemble's wave function $\psi$ through a Hermitian operator $\mathrm{Q}_{f}$ (cf. eq. (3.5.78)). This latter, in turn, is expressed in term of $f$ via an integral kernel $K_{f}$ (cf. eqs. (3.5.79), (3.5.80)).

Weyl-Wigner theory properly applies only to a non spinning particle and has no immediate generalization to a spinning one. Spin is in fact a purely quantum observable with no classical analogue. In the semiclassical regime $\hbar \rightarrow 0$ spin vanishes identically, since it can take only the values $\pm \hbar / 2$. This notwithstanding, the Weyl-Wigner theory does have a meaningful extension to the spinning case.

In the non spinning case, the potential $U$ is an instance of phase function. Above, we found that the description of the dynamics a spinning particle requires four potentials, viz $U_{ \pm \frac{1}{2} \pm \frac{1}{2}}, U_{ \pm \frac{1}{2} \mp \frac{1}{2}}$. This suggests the counterpart of classical phase functions for a spinning particle are spinorial phase functions $f_{ \pm \frac{1}{2} \pm \frac{1}{2}}, f_{ \pm \frac{1}{2} \mp \frac{1}{2}}$, quadruplets of ordinary phase functions. The simplest extension of Weyl-Wigner theory to a spinning particle consists then in hypothesizing that there exists a classical observable $Q_{f}$ associated with each such quadruplet, that $Q_{f}$ is also a quantum observable and that each quantum observable is of this form.

The general expression of a quantum mean $\left\langle Q_{f}\right\rangle$ of a spinning particle observable $Q_{f}$ takes now presumably the following spinor operatorial form
$$
\begin{array}{r}
\left\langle Q_{f}\right\rangle=\int d^{3} x\left[\psi_{+\frac{1}{2}}{ }^{*} \mathrm{Q}_{f+\frac{1}{2}+\frac{1}{2}} \psi_{+\frac{1}{2}}+\psi_{+\frac{1}{2}}{ }^{*} \mathrm{Q}_{f+\frac{1}{2}-\frac{1}{2}} \psi_{-\frac{1}{2}}\right.  \tag{3.21.12}\\
\left.+\psi_{-\frac{1}{2}} * \mathrm{Q}_{f-\frac{1}{2}+\frac{1}{2}} \psi_{+\frac{1}{2}}+\psi_{-\frac{1}{2}} * \mathrm{Q}_{f-\frac{1}{2}-\frac{1}{2}} \psi_{-\frac{1}{2}}\right]
\end{array}
$$
where $\mathrm{Q}_{f \pm \frac{1}{2} \pm \frac{1}{2}}, \mathrm{Q}_{f \pm \frac{1}{2} \mp \frac{1}{2}}$ act on $\psi_{ \pm \frac{1}{2}}$ as
$$
\begin{equation*}
\mathrm{Q}_{f \pm \frac{1}{2} \pm \frac{1}{2}} \psi_{ \pm \frac{1}{2}}(t, \boldsymbol{x})=\int d^{3} u K_{f \pm \frac{1}{2} \pm \frac{1}{2}}(\boldsymbol{x}, \boldsymbol{u}) \psi_{ \pm \frac{1}{2}}(t, \boldsymbol{u}) \tag{3.21.13a}
\end{equation*}
$$
$$
\begin{equation*}
\mathrm{Q}_{f \pm \frac{1}{2} \mp \frac{1}{2}} \psi_{\mp \frac{1}{2}}(t, \boldsymbol{x})=\int d^{3} u K_{f \pm \frac{1}{2} \mp \frac{1}{2}}(\boldsymbol{x}, \boldsymbol{u}) \psi_{\mp \frac{1}{2}}(t, \boldsymbol{u}) \tag{3.21.13b}
\end{equation*}
$$
the integral kernels $K_{f \pm \frac{1}{2} \pm \frac{1}{2}}, K_{f \pm \frac{1}{2} \mp \frac{1}{2}}$ in the right hand side being given by
$$
\begin{align*}
K_{f \pm \frac{1}{2} \pm \frac{1}{2}}(\boldsymbol{x}, \boldsymbol{u})=\int \frac{d^{3} y}{(2 \pi \hbar)^{3}} \exp (i(\boldsymbol{x}-\boldsymbol{u}) & \cdot \boldsymbol{y} / \hbar)  \tag{3.21.14a}\\
& \times f_{ \pm \frac{1}{2} \pm \frac{1}{2}}((\boldsymbol{x}+\boldsymbol{u}) / 2, \boldsymbol{y}) \\
K_{f \pm \frac{1}{2} \mp \frac{1}{2}}(\boldsymbol{x}, \boldsymbol{u})=\int \frac{d^{3} y}{(2 \pi \hbar)^{3}} \exp (i(\boldsymbol{x}-\boldsymbol{u}) & \cdot \boldsymbol{y} / \hbar)  \tag{3.21.14b}\\
& \times f_{ \pm \frac{1}{2} \mp \frac{1}{2}}((\boldsymbol{x}+\boldsymbol{u}) / 2, \boldsymbol{y})
\end{align*}
$$
$K_{f \pm \frac{1}{2} \pm \frac{1}{2}}, K_{f \pm \frac{1}{2} \mp \frac{1}{2}}$ are the spinorial Weyl-Wigner transforms of the spinorial phase function $f_{ \pm \frac{1}{2} \pm \frac{1}{2}}$, $f_{ \pm \frac{1}{2} \mp \frac{1}{2}}$ (cf. eqs. (3.5.79)-(3.5.80)).

The expressions of the quantum mean values of the position $\boldsymbol{q}$, momentum $\boldsymbol{p}$, angular momentum $\boldsymbol{l}$ and energy $H$ of the spinning particle, extending those those of the spinless case (cf. eqs. (3.5.82)-(3.5.85)), are now easily obtained. Roughly, the former are just the spin projection summed version of the latter,
$$
\begin{align*}
& \langle\boldsymbol{q}\rangle=\int d^{3} x\left(\psi_{+\frac{1}{2}}{ }^{*} \boldsymbol{x} \psi_{+\frac{1}{2}}+\psi_{-\frac{1}{2}}{ }^{*} \boldsymbol{x} \psi_{-\frac{1}{2}}\right),  \tag{3.21.15}\\
& \langle\boldsymbol{p}\rangle=\int d^{3} x\left({\psi_{+\frac{1}{2}}}^{*}\left(-i \hbar \boldsymbol{\nabla} \psi_{+\frac{1}{2}}\right)+\psi_{-\frac{1}{2}}{ }^{*}\left(-i \hbar \boldsymbol{\nabla} \psi_{-\frac{1}{2}}\right)\right),  \tag{3.21.16}\\
& \langle\boldsymbol{l}\rangle=\int d^{3} x\left(\psi_{+\frac{1}{2}}{ }^{*}\left(-i \hbar \boldsymbol{x} \times \boldsymbol{\nabla} \psi_{+\frac{1}{2}}\right)+\psi_{-\frac{1}{2}}{ }^{*}\left(-i \hbar \boldsymbol{x} \times \boldsymbol{\nabla} \psi_{-\frac{1}{2}}\right)\right),  \tag{3.21.17}\\
& \langle H\rangle=\int d^{3} x\left[\psi_{+\frac{1}{2}} *\left(-\frac{\hbar^{2}}{2 m} \nabla^{2} \psi_{+\frac{1}{2}}+U_{+\frac{1}{2}+\frac{1}{2}} \psi_{+\frac{1}{2}}+U_{+\frac{1}{2}-\frac{1}{2}} \psi_{-\frac{1}{2}}\right)\right.  \tag{3.21.18}\\
& \left.+\psi_{-\frac{1}{2}} *\left(-\frac{\hbar^{2}}{2 m} \nabla^{2} \psi_{-\frac{1}{2}}+U_{-\frac{1}{2}-\frac{1}{2}} \psi_{-\frac{1}{2}}+U_{-\frac{1}{2}+\frac{1}{2}} \psi_{+\frac{1}{2}}\right)\right] .
\end{align*}
$$

Proof. The above relations rest on reasonable hypotheses about the form of the spinorial phase functions $f_{ \pm \frac{1}{2} \pm \frac{1}{2}}, f_{ \pm \frac{1}{2} \mp \frac{1}{2}}$ corresponding to the observables under consideration. For non spinorial orbital observables, the functions $f_{ \pm \frac{1}{2} \pm \frac{1}{2}}$ should have the
same form as those of the spinless case while the functions $f_{ \pm \frac{1}{2} \mp \frac{1}{2}}$ presumably vanish, since it would be difficult to explain how contributions relating opposite spin projections can arise for such observables. The observables $\boldsymbol{q}, \boldsymbol{p}, \boldsymbol{l}$ have a non spinorial orbital nature. Thus, for $\boldsymbol{q}, \boldsymbol{p}, \boldsymbol{l}$, the $f_{ \pm \frac{1}{2} \pm \frac{1}{2}}(\boldsymbol{x}, \boldsymbol{y})$ are a component of $\boldsymbol{x}, \boldsymbol{y}, \boldsymbol{x} \times \boldsymbol{y}$, respectively, while the $f_{ \pm \frac{1}{2} \mp \frac{1}{2}}(\boldsymbol{x}, \boldsymbol{y})$ vanish. This yields immediately relations (3.21.15)(3.21.17) by a calculation totally analogous to that leading to expressions (3.5.82)(3.5.84) The observable $H$, conversely, is spinorial since the interactions can couple opposite spin projections. For $H$, the most reasonable assumptions that can be made gives $f_{ \pm \frac{1}{2} \pm \frac{1}{2}}(\boldsymbol{x}, \boldsymbol{y})=\boldsymbol{y}^{2} / 2 m+U_{ \pm \frac{1}{2} \pm \frac{1}{2}}(\boldsymbol{x}), f_{ \pm \frac{1}{2} \mp \frac{1}{2}}(\boldsymbol{x}, \boldsymbol{y})=U_{ \pm \frac{1}{2} \mp \frac{1}{2}}(\boldsymbol{x})$. This yields relation (3.21.18) by a calculation similar in its development to that leading to expressions (3.5.85).

All the above expressions can be written in a more compact and suggestive form in terms of the spinor wave function
$$
\psi=\left[\begin{array}{l}
\psi_{+\frac{1}{2}}  \tag{3.21.19}\\
\psi_{-\frac{1}{2}}
\end{array}\right]
$$

Introduce the spinor matrix potential energy function
$$
U=\left[\begin{array}{ll}
U_{+\frac{1}{2}+\frac{1}{2}}, & U_{+\frac{1}{2}-\frac{1}{2}}  \tag{3.21.20}\\
U_{-\frac{1}{2}+\frac{1}{2}}, & U_{-\frac{1}{2}-\frac{1}{2}}
\end{array}\right]
$$

Pointwise, $U$ is Hermitian,
$$
\begin{equation*}
U=U^{+} \tag{3.21.21}
\end{equation*}
$$
where ${ }^{+}$denotes matrix adjunction, i. e. transposition and complex conjugation. In fact, the contribution of the potential energy to the energy mean value $\langle H\rangle$ can be cast as $\int d^{3} x \psi^{+} U \psi$. Since this must be real, $\psi^{+} U \psi$ must be real for any $\psi$ and so $U$ must be Hermitian as stated.

The spinor matrix potential energy function $U$ can conveniently be expanded
as a linear combination of the identity matrix
$$
1_{2}=\left[\begin{array}{ll}
1 & 0  \tag{3.21.22}\\
0 & 1
\end{array}\right]
$$
and the Pauli matrices
$$
\sigma_{1}=\left[\begin{array}{ll}
0 & 1  \tag{3.21.23}\\
1 & 0
\end{array}\right], \quad \sigma_{2}=\left[\begin{array}{rr}
0 & -i \\
i & 0
\end{array}\right], \quad \sigma_{3}=\left[\begin{array}{rr}
1 & 0 \\
0 & -1
\end{array}\right]
$$

As these matrices are linearly independent and Hermitian, $1_{2}{ }^{+}=1_{2}$ and $\sigma_{1}{ }^{+}=\sigma_{1}$, $\sigma_{2}{ }^{+}=\sigma_{2}, \sigma_{3}{ }^{+}=\sigma_{3}, U$ can be expressed as
$$
\begin{equation*}
U=U_{0} 1_{2}+U_{1} \sigma_{1}+U_{2} \sigma_{2}+U_{3} \sigma_{3} \tag{3.21.24}
\end{equation*}
$$

Define the vector matrix $\boldsymbol{\sigma}=\left(\sigma_{1}, \sigma_{2}, \sigma_{3}\right)$. Define further the energy potential vector $\boldsymbol{U}=\left(U_{1}, U_{2}, U_{3}\right)$. Then, (3.21.24) can be cast more succinctly as
$$
\begin{equation*}
U=U_{0} 1_{2}+\boldsymbol{U} \cdot \boldsymbol{\sigma} \tag{3.21.25}
\end{equation*}
$$

The the Schroedinger equations (3.21.1) can now be written compactly as
$$
\begin{equation*}
i \hbar \frac{\partial \psi}{\partial t}=-\frac{\hbar^{2}}{2 m} \nabla^{2} \psi+U_{0} \psi+\boldsymbol{U} \cdot \boldsymbol{\sigma} \psi \tag{3.21.26}
\end{equation*}
$$

In this form, the analogy with the spinless Schroedinger equation (3.4.14) is more evident.

From (3.21.5), the total probability density can be expressed as
$$
\begin{equation*}
\rho=|\psi|^{2} \tag{3.21.27}
\end{equation*}
$$
where for a spinor vector $u,|u|^{2}=u^{+} u$. By (3.21.6), The probability that the spinning particle is found with any spin projection in a space region $\mathcal{V}$ is so
$$
\begin{equation*}
p(\mathcal{V})=\int_{\mathcal{V}} d^{3} x|\psi|^{2} \tag{3.21.28}
\end{equation*}
$$

The normalization condition (3.21.7) now takes the form
$$
\begin{equation*}
\int d^{3} x|\psi|^{2}=1 \tag{3.21.29}
\end{equation*}
$$

By (3.21.10), the total probability current density of the spinning particle is
$$
\begin{equation*}
\boldsymbol{j}=\frac{\hbar}{m} \operatorname{Im}\left(\psi^{+} \boldsymbol{\nabla} \psi\right) \tag{3.21.30}
\end{equation*}
$$

By (3.21.11), the probability that the particle crosses an oriented surface $\mathcal{A}$ with any spin projection per unit time is
$$
\begin{equation*}
\Phi(\mathcal{A})=\frac{\hbar}{m} \int_{\mathcal{A}} d^{2} \boldsymbol{x} \cdot \operatorname{Im}\left(\psi^{+} \boldsymbol{\nabla} \psi\right) \tag{3.21.31}
\end{equation*}
$$

Again, the analogy with spinless case is evident.
The spinor formalism allow also to write a particularly simple expression of the quantum means $\left\langle Q_{f}\right\rangle$ of an observable $Q_{f}$ of the spinning particle. In terms of the spinor operator
$$
\mathrm{Q}_{f}=\left[\begin{array}{ll}
\mathrm{Q}_{f+\frac{1}{2}+\frac{1}{2}}, & \mathrm{Q}_{f+\frac{1}{2}-\frac{1}{2}}  \tag{3.21.32}\\
\mathrm{Q}_{f-\frac{1}{2}+\frac{1}{2}}, & \mathrm{Q}_{f-\frac{1}{2}-\frac{1}{2}}
\end{array}\right]
$$
(3.21.12) can be cast compactly as
$$
\begin{equation*}
\left\langle Q_{f}\right\rangle=\int d^{3} x \psi^{+} \mathrm{Q}_{f} \psi \tag{3.21.33}
\end{equation*}
$$
a relation formally analogous to (3.5.78). By (3.21.13), further, the action of the spinor operator $\mathrm{Q}_{f}$ on the spinor wave function $\psi$ can be expressed through the spinor integral kernel given by
$$
K_{f}=\left[\begin{array}{ll}
K_{f+\frac{1}{2}+\frac{1}{2}}, & K_{f+\frac{1}{2}-\frac{1}{2}}  \tag{3.21.34}\\
K_{f-\frac{1}{2}+\frac{1}{2}}, & K_{f-\frac{1}{2}-\frac{1}{2}}
\end{array}\right]
$$
in the concise form
$$
\begin{equation*}
\mathrm{Q}_{f} \psi(t, \boldsymbol{x})=\int d^{3} u K_{f}(\boldsymbol{x}, \boldsymbol{u}) \psi(t, \boldsymbol{u}) \tag{3.21.35}
\end{equation*}
$$
in keeping with relation (3.5.79). Writing the spinor phase function as
$$
f=\left[\begin{array}{ll}
f_{+\frac{1}{2}+\frac{1}{2}}, & f_{+\frac{1}{2}-\frac{1}{2}}  \tag{3.21.36}\\
f_{-\frac{1}{2}+\frac{1}{2}}, & f_{-\frac{1}{2}-\frac{1}{2}}
\end{array}\right]
$$
the spinor Weyl-Wigner relations (3.21.14) can be written concisely as
$$
\begin{equation*}
K_{f}(\boldsymbol{x}, \boldsymbol{u})=\int \frac{d^{3} y}{(2 \pi \hbar)^{3}} \exp (i(\boldsymbol{x}-\boldsymbol{u}) \cdot \boldsymbol{y} / \hbar) f((\boldsymbol{x}+\boldsymbol{u}) / 2, \boldsymbol{y}) \tag{3.21.37}
\end{equation*}
$$
in analogy to $(3.5 .80)$.
The expressions (3.21.15)-(3.21.18) of the quantum mean values of $\boldsymbol{q}, \boldsymbol{p}, \boldsymbol{l}$ and $H$ take the form
$$
\begin{align*}
& \langle\boldsymbol{q}\rangle=\int d^{3} x \psi^{+} \boldsymbol{x} \psi  \tag{3.21.38}\\
& \langle\boldsymbol{p}\rangle=\int d^{3} x \psi^{+}(-i \hbar \boldsymbol{\nabla} \psi)  \tag{3.21.39}\\
& \langle\boldsymbol{l}\rangle=\int d^{3} x \psi^{+}(-i \hbar \boldsymbol{x} \times \boldsymbol{\nabla} \psi)  \tag{3.21.40}\\
& \langle H\rangle=\int d^{3} x \psi^{+}\left(-\frac{\hbar^{2}}{2 m} \boldsymbol{\nabla}^{2} \psi+U_{0} \psi+\boldsymbol{U} \cdot \boldsymbol{\sigma} \psi\right) \tag{3.21.41}
\end{align*}
$$

The spin projection is itself an observable. Its quantum mean value is
$$
\begin{equation*}
\left\langle s_{3}\right\rangle=\frac{\hbar}{2} p_{+\frac{1}{2}}-\frac{\hbar}{2} p_{-\frac{1}{2}} \tag{3.21.42}
\end{equation*}
$$

Using the relations (3.21.3), this can be cast as
$$
\begin{equation*}
\left\langle s_{3}\right\rangle=\frac{\hbar}{2} \int d^{3} x\left(\left|\psi_{+\frac{1}{2}}\right|^{2}-\left|\psi_{-\frac{1}{2}}\right|^{2}\right) \tag{3.21.43}
\end{equation*}
$$

This can be cast in terms of the spinor wave function
$$
\begin{equation*}
\left\langle s_{3}\right\rangle=\frac{\hbar}{2} \int d^{3} x \psi^{+} \sigma_{3} \psi \tag{3.21.44}
\end{equation*}
$$

Now, spin, as a form of angular momentum, is a vector quantity $\boldsymbol{s}$. Above, thus, $s_{3}=\boldsymbol{s} \cdot \boldsymbol{e}_{3} \cdot(3.21 .44)$ so should be a particular case of the more general relation
$$
\begin{equation*}
\langle s \cdot \boldsymbol{n}\rangle=\frac{\hbar}{2} \int d^{3} x \psi^{+} \boldsymbol{\sigma} \cdot \boldsymbol{n} \psi \tag{3.21.45}
\end{equation*}
$$
where $\boldsymbol{n}$ is a unit vector, from which it follows that
$$
\begin{equation*}
\langle\boldsymbol{s}\rangle=\frac{\hbar}{2} \int d^{3} x \psi^{+} \boldsymbol{\sigma} \psi \tag{3.21.46}
\end{equation*}
$$

From this relation, we conclude that spin, on a par with the other observables, is represented by an operator, namely
$$
\begin{equation*}
\mathrm{s}=\frac{\hbar}{2} \sigma \tag{3.21.47}
\end{equation*}
$$

The spinning particle's total angular momentum is
$$
\begin{equation*}
k=\boldsymbol{l}+\boldsymbol{s} \tag{3.21.48}
\end{equation*}
$$

Its mean value is
$$
\begin{equation*}
\langle\boldsymbol{k}\rangle=\langle\boldsymbol{l}\rangle+\langle\boldsymbol{s}\rangle=\int d^{3} x \psi^{+}\left(-i \hbar \boldsymbol{x} \times \boldsymbol{\nabla} \psi+\frac{\hbar}{2} \boldsymbol{\sigma} \psi\right) \tag{3.21.49}
\end{equation*}
$$

The solution of the spinning particle time dependent Schroedinger equation (3.21.26) can be reduced to that of corresponding time independent equation
$$
\begin{equation*}
-\frac{\hbar^{2}}{2 m} \boldsymbol{\nabla}^{2} \psi+U_{0} \psi+\boldsymbol{U} \cdot \boldsymbol{\sigma} \psi=w \psi \tag{3.21.50}
\end{equation*}
$$
with the obvious spinor generalization usual regularity and auxiliary conditions. The spinning particle Schroedinger problem is in general more complicated than the spinless particle one, because the potential can couple the two components of the spinor wave function.

\subsection*{3.22. Spinning particle in a magnetic field}

As an application, we solve the Schroedinger problem for a spinning electron subject to an external constant and uniform magnetic field $\mathcal{H}$.

The interaction potential is
$$
\begin{equation*}
U=\mu_{B} \boldsymbol{\sigma} \cdot \mathcal{H} \tag{3.22.1}
\end{equation*}
$$
where $\mu_{B}$ is the Bohr magneton.

Proof. Combining relations $(2.10 .16),(2.10 .23)$ and (3.21.47)
$$
\begin{equation*}
U=-\boldsymbol{\mu} \cdot \mathcal{H}=\frac{\mu_{B} g_{e}}{\hbar} \boldsymbol{s} \cdot \boldsymbol{\mathcal { H }}=\frac{\mu_{B} g_{e}}{\hbar} \frac{\hbar}{2} \boldsymbol{\sigma} \cdot \boldsymbol{\mathcal { H }}=\frac{\mu_{B} g_{e}}{2} \boldsymbol{\sigma} \cdot \mathcal{H} \tag{3.22.2}
\end{equation*}
$$

The electron gyromagnetic ratio $g_{e}=2$.(3.22.1) follows

The Schroedinger problem consists in finding the energy values $w$ such that there exists a non vanishing spinor wave function $\phi$ satisfying the spinning electron Schroedinger equation $(3.21 .50)$
$$
\begin{equation*}
-\frac{\hbar^{2}}{2 m} \boldsymbol{\nabla}^{2} \phi+\mu_{B} \boldsymbol{\sigma} \cdot \boldsymbol{\mathcal { H }} \phi=w \phi \tag{3.22.3}
\end{equation*}
$$
and the boundedness condition at spacial infinity.
The magnetic field $\mathcal{H}$ couples to the spin degrees of freedom only. The orbital motion of the particle is therefore free. We thus expect that the orbital part of the energy eigenfuction is just a plane wave of arbitrary momentum, as in the spinless case, and that the spinorial part is an eigenspinor of the matrix $\mu_{B} \boldsymbol{\sigma} \cdot \mathcal{H}$. This is indeed the case.

Suppose for simplicity that the magnetic field is directed along the 3 -axis, $\mathcal{H}=\mathcal{H} e_{3}$. Then, the energy eigenvalues and eigenfunctions are labelled by the momentum vector $\boldsymbol{y}$ and the spin projection $\pm 1 / 2$. The eigenvlaues are
$$
\begin{equation*}
w_{\boldsymbol{y}+1 / 2}=\frac{\boldsymbol{y}^{2}}{2 m}+\mu_{B} \mathcal{H}, \quad w_{\boldsymbol{y}-1 / 2}=\frac{\boldsymbol{y}^{2}}{2 m}-\mu_{B} \mathcal{H} \tag{3.22.4}
\end{equation*}
$$
while the corresponding eigenfunctions are given by
$$
\phi_{\boldsymbol{y}+1 / 2}(\boldsymbol{x})=\frac{\exp (i \boldsymbol{y} \cdot \boldsymbol{x} / \hbar)}{(2 \pi \hbar)^{3 / 2}}\left[\begin{array}{l}
1  \tag{3.22.5}\\
0
\end{array}\right], \quad \phi_{\boldsymbol{y}-1 / 2}(\boldsymbol{x})=\frac{\exp (i \boldsymbol{y} \cdot \boldsymbol{x} / \hbar)}{(2 \pi \hbar)^{3 / 2}}\left[\begin{array}{l}
0 \\
1
\end{array}\right]
$$

Proof. Inspired by the considerations made above, we try to solve the Schroedinger equation through the anstatz
$$
\begin{equation*}
\phi(\boldsymbol{x})=\frac{\exp (i \boldsymbol{y} \cdot \boldsymbol{x} / \hbar)}{(2 \pi \hbar)^{3 / 2}} \phi_{s} \tag{3.22.6}
\end{equation*}
$$
where $\phi_{s}$ is a constant spinor. Then, $\phi$ satisfies eq. (3.22.3) with energy value $w$, provided $\phi_{s}$ satisfies the eigenvalue equation
$$
\begin{equation*}
\mu_{B} \boldsymbol{\sigma} \cdot \mathcal{H} \phi_{s}=w_{s} \phi_{s} \tag{3.22.7}
\end{equation*}
$$
where $w_{s}$ is given by
$$
\begin{equation*}
w_{s}=w-\frac{\boldsymbol{y}^{2}}{2 m} \tag{3.22.8}
\end{equation*}
$$

Indeed, we have
$$
\begin{align*}
-\frac{\hbar^{2}}{2 m} \boldsymbol{\nabla}^{2} \phi(\boldsymbol{x})+ & \mu_{B} \boldsymbol{\sigma} \cdot \boldsymbol{\mathcal { H }} \phi(\boldsymbol{x})-w \phi(\boldsymbol{x})  \tag{3.22.9}\\
& =\frac{\exp (i \boldsymbol{y} \cdot \boldsymbol{x} / \hbar)}{(2 \pi \hbar)^{3 / 2}}\left[\frac{\boldsymbol{y}^{2}}{2 m} \phi_{s}+\mu_{B} \boldsymbol{\sigma} \cdot \boldsymbol{\mathcal { H }} \phi_{s}-w \phi_{s}\right]
\end{align*}
$$

Since $\mathcal{H}=\mathcal{H} e_{3}$ by assumption
$$
\mu_{B} \boldsymbol{\sigma} \cdot \boldsymbol{\mathcal { H }}=\mu_{B} \mathcal{H}\left[\begin{array}{rr}
1 & 0  \tag{3.22.10}\\
0 & -1
\end{array}\right]
$$

Thus, the eigenvalue equation (3.22.7) has as solution
$$
\begin{equation*}
w_{s+1 / 2}=\mu_{B} \mathcal{H}, \quad w_{s-1 / 2}=-\mu_{B} \mathcal{H} \tag{3.22.11}
\end{equation*}
$$
the corresponding eigenspinors being
$$
\phi_{s+1 / 2}=\left[\begin{array}{l}
1  \tag{3.22.12}\\
0
\end{array}\right], \quad \phi_{s-1 / 2}=\left[\begin{array}{l}
0 \\
1
\end{array}\right]
$$

This shows (3.22.4), (3.22.5).

\end{document}